\documentclass[cjk,dvipdfmx,10pt,compress,%
hyperref={bookmarks=true,bookmarksnumbered=true,bookmarksopen=false,%
colorlinks=false,%
pdftitle={第 87 回 関西 Debian 勉強会},%
pdfauthor={倉敷・のがた・佐々木・かわだ・八津尾},%
%pdfinstitute={関西 Debian 勉強会},%
pdfsubject={資料},%
}]{beamer}

\title{第 87 回 関西 Debian 勉強会}
\subtitle{$\sim$発表資料$\sim$}
\author[かわだ てつたろう]{{\large\bf 倉敷・のがた・佐々木・かわだ・八津尾}}
\institute[Debian JP]{{\normalsize\tt 関西 Debian 勉強会}}
\date{{\small 2014 年 8 月 24 日}}

%\usepackage{amsmath}
%\usepackage{amssymb}
\usepackage{graphicx}
\usepackage{moreverb}
\usepackage[varg]{txfonts}
\AtBeginDvi{\special{pdf:tounicode EUC-UCS2}}
\usetheme{Kyoto}
\def\museincludegraphics{%
  \begingroup
  \catcode`\|=0
  \catcode`\\=12
  \catcode`\#=12
  \includegraphics[width=0.9\textwidth]}
%\renewcommand{\familydefault}{\sfdefault}
%\renewcommand{\kanjifamilydefault}{\sfdefault}
\begin{document}
\settitleslide
\begin{frame}
\titlepage
\end{frame}
\setdefaultslide

\begin{frame}[fragile]
  \frametitle{Disclaimer}
  \begin{itemize}
  \item 疑問、質問、ツッコミ、茶々、\alert{大歓迎}
  \item その場でインタラクティブにどうぞ
  \item ハッシュタグ \#kansaidebian
\end{itemize}
\end{frame}

\begin{frame}[fragile]
\frametitle{Agenda}

\tableofcontents

\end{frame}

\section{最近の Debian 関係のイベント}

\takahashi[40]{最近の Debian\\関係のイベント}

\begin{frame}[fragile]
  \frametitle{第85回関西Debian勉強会}
  \begin{itemize}
  \item 日時: 6月22日(日)
  \item 場所: 福島区民センター
  \end{itemize}
  \begin{block}{内容}
    \begin{itemize}
    \item 「Linuxのドライバメンテナになった体験記」
    \item 「Debian での systemd とのつきあい方」
    \item もくもくの会
    \end{itemize}
  \end{block}
\end{frame}

\begin{frame}[fragile]
  \frametitle{第86回関西Debian勉強会\\OSC2014 Kansai@Kyoto}
  \begin{itemize}
  \item 日時: 8月2日(土)
  \item 場所: 京都リサーチパーク
  \end{itemize}
  \begin{block}{内容}
    \begin{itemize}
    \item ブース展示
      \begin{itemize}
      \item 実機展示
      \item Tシャツ
      \end{itemize}
    \item 「Debian Project の最近の動向について」
    \end{itemize}
  \end{block}
\end{frame}

\begin{frame}[fragile]
  \frametitle{第115回東京エリアDebian勉強会\\with第2回Debianパッケージング道場}
  \begin{itemize}
  \item 日時: 7月19日(土)
  \item 場所: 株式会社サイバーエージェント 東京オフィス
  \end{itemize}
  \begin{block}{内容}
    \begin{itemize}
    \item パッケージング道場
    \end{itemize}
  \end{block}
\end{frame}

\begin{frame}[fragile]
  \frametitle{第116回東京エリアDebian勉強会}
  \begin{itemize}
  \item 日時: 8月23日(土)
  \item 場所: 株式会社スクウェア・エニックス セミナールーム
  \end{itemize}
  \begin{block}{内容}
    \begin{itemize}
    \item 「Debianでタイルマップサービス作ってみた」
    \item もくもくの会
    \end{itemize}
  \end{block}
\end{frame}

\begin{frame}[fragile]
  \frametitle{Debian Project}
  \begin{itemize}
  \item Debian Day 2014
  \item Linux kernel version for jessie
  \item Debian Installer Jessie Beta 1 release
  \item First steps towards source-only uploads
  \item systemd
  \item RFS: ircii/20131230-1 [NMU]
  \item DebConf14

    Portland, Oregon

    August 23 - 31, 2014
  \end{itemize}
\end{frame}

\begin{frame}[fragile]
  \frametitle{DebConf14}
  \begin{block}{}
    \begin{itemize}
    \item Meet the Technical Committee
    \item Quit logging! (or, data minimization in Debian)
    \item Debian Long Term Support
    \item Removing obsolete packages for fun and profit
    \item Validation and Continuous Integration BoF
    \item New Network Interface Manager for Debian: ifupdown2
    \item ACC for abi breaks
    \item A glimpse into a systemd future
    \item What's new in the Linux kernel
    \item debdry - Debian Don't Repeat Yourself
    \end{itemize}
  \end{block}
\end{frame}

\takahashi[50]{そんな\\こんなで}
\takahashi[120]{次}

\section{事前課題発表}

\takahashi[50]{事前課題}

\begin{frame}[fragile]
  \frametitle{事前課題}
  \begin{block}{今回の事前課題}
    \begin{description}
    \item[事前課題1]
      もくもくの会で行なう作業、質問などの課題を用意して教えてください。
    \item[事前課題2]
      前回(第85回)の勉強会に参加された方は、前回の作業や課題がその後どう
      なったか結果を教えてください。
    \item[事前課題3]
      LT(ライトニングトーク) 歓迎です。何かお話したい方はタイトルを下さい。
    \end{description}
  \end{block}
\end{frame}

\takahashi[50]{事前課題\\発表}

\begin{frame}
  \frametitle{ takata (1/2)}
  \begin{enumerate}
  \item Docker.ioで クロスツールビルド環境(32ビット・コンテナイメージ)を整備

    参考URL
    \begin{itemize}
    \item 32-bit container on a 64-bit system \#611

      \url{https://github.com/docker/docker/issues/611}
    \item docker で 32bit コンテナイメージを作成する

      \url{http://d.hatena.ne.jp/defiant/20130704/1372947160}
    \end{itemize}
  \end{enumerate}
\end{frame}

\begin{frame}
  \frametitle{ takata (2/2)}
  \begin{enumerate}
    \setcounter{enumi}{1}
  \item Redmineデータベースの復旧
    wheezyで運用していたときの postgresqlデータベースを jessieの redmine\_{}2.5.2-1に移行。
    \begin{enumerate}
    \item redmine\_{}2.5.2-1は passengerで起動
    \item postgresqlデータベースのリストア

      \begin{block}{}
        \${} pg\_{}restore -c -Upostgres --dbname=redmine\_{}default redmine\_{}0.sqlc
      \end{block}
      redmine\_{}0.sqlcは、次の操作で以前に作成したデータベースのバックアップ
      \begin{block}{}
        \${} pg\_{}dump -Upostgres --format=c --file=redmine\_{}0.sqlc redmine\_{}default
      \end{block}
    \item ownerの変更: redmine -$>$ redmine\_{}default
    \item データベースのマイグレーション(db:migrate)
      \begin{block}{}
        \# RAILS\_{}ENV=production rake db:migrate
      \end{block}
      このとき、queries\_{}roles, custom\_{}fields\_{}rolesが既に存在するためエラーとなるが、
      テーブルを dropして再試行することでマイグレーションに成功。
    \end{enumerate}
  \end{enumerate}
\end{frame}

\begin{frame}
  \frametitle{ koedoyoshida }
  DDTSS翻訳や自分で作成しているソフトのupdate,ボランティアしているOSS関連のイベント作業など。

  遅れて参加になると思います。
\end{frame}

\begin{frame}
  \frametitle{ かわだてつたろう }
  \begin{enumerate}
  \item 溜っているメールを読む。
  \item gnucacheでかな入力できなかったのはGTK\_{}IM\_{}MODULEがximとなっていたからでした。

    im-configはuim-gtk2.0とuim-gtk3の両方がインストールされていないとGTK\_{}IM\_{}MODULEをuimに設定しない。
  \item ありません。
  \end{enumerate}
\end{frame}

\begin{frame}
  \frametitle{ kozo2 }
  \begin{enumerate}
  \item
    \url{https://bitbucket.org/anekos/iron-maiden-cl}のDebian packageの作成

    上記作業でわからないことがでたらそれを質問する。

    また\url{https://github.com/spotify/dh-virtualenv}利用経験者がいればそれにまつわる話を聞く。
  \item 前回(第85回)の勉強会に参加していませんでした。
  \item cmakeで.deb作成
  \end{enumerate}
\end{frame}

\begin{frame}
  \frametitle{ Mitsutoshi NAKANO $<$bkbin005@rinku.zaq.ne.jp$>$ }
  \begin{enumerate}
  \item a) b)のうちどちらか、または双方。
    \begin{itemize}
    \item [a] tamagoのupstreamの準備

      詳細: \url{http://lists.debian.or.jp/debian-devel/201408/msg00007.html}
    \item [b]Debianのパッケージビルドの方法についての勉強

      詳細: \url{http://lists.debian.or.jp/debian-users/201408/msg00027.html}
    \end{itemize}
  \end{enumerate}
\end{frame}

\begin{frame}
  \frametitle{ 坂本 貴史 }
  \begin{enumerate}
  \item 開発中のALSAのドライバを書きます。
  \item 作業や課題とは関係ないのですが、前回受けた質問への回答を、前回分資料に掲載してもらいました。
  \item ネタの用意がありません。
  \end{enumerate}
\end{frame}

\begin{frame}
  \frametitle{  清野陽一 }
  \begin{enumerate}
  \item 久しぶりなので最近のDebian界隈の状況を知る。
  \item 不参加
  \item 今のところ無し
  \end{enumerate}
\end{frame}

\begin{frame}
  \frametitle{ lurdan }
  \begin{enumerate}
  \item serverspec と e2wm のパッケージ更新
  \item webwml はまだ稼働にこぎつけてないです。これもやらないと……
  \end{enumerate}
\end{frame}

\begin{frame}
  \frametitle{ 佐々木洋平 }
  \begin{enumerate}
  \item tDiary の更新
  \item systemd-sysv$<$-$>$sysvinit-core をウロウロしてます。さいきんずっとコレばっか。
  \item 余裕ないです。
  \end{enumerate}
\end{frame}

\begin{frame}
  \frametitle{ 川江 }
  \begin{enumerate}
  \item emacsを使って、HTML5のCSS、Javascriptのコードの生成。
  \item 同上(苦戦中)
  \item
  \end{enumerate}
\end{frame}

\takahashi[50]{そんな\\こんなで}
\takahashi[120]{次}

\section{もくもくの会}
\takahashi[30]{もくもくの会}

\takahashi[50]{そんな\\こんなで}
\takahashi[120]{次}

\section{今後の予定}
\begin{frame}[fragile]
\frametitle{今後の予定}

\begin{block}{第88回関西Debian勉強会}
  \begin{itemize}
  \item 日時: 9月28日(日) 13:30 -
  \item 場所: 福島区民センター
  \end{itemize}
\end{block}

\begin{block}{第117回東京エリアDebian勉強会}
  \begin{itemize}
  \item 日時: 9月16日(土)
  \item 場所: 未定
  \item 内容: 未定
  \end{itemize}
\end{block}

\end{frame}

\takahashi[50]{  }

\end{document}
%%% Local Variables:
%%% mode: japanese-latex
%%% TeX-master: t
%%% End:
