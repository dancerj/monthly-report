%; whizzy-master ../debianmeetingresume201311.tex
% 以上の設定をしているため、このファイルで M-x whizzytex すると、whizzytexが利用できます。
%

\santaku
{2014/11月中旬にてDebconf15のスポンサーとなった企業は何社あるでしょう?}
{3社}
{9社}
{11社}
{C}
{スポンサー企業一覧:credativ, sipgate, Matanel Foundation, Google,
Farsight Security, Martin Alfke / Buero 2.0, Ubuntu, Mirantis, Logilab,
Netways,Hetzner。さあ!日本企業の皆様も是非ご一緒に!}

\santaku
{2014/11/18に遂にグラフ表示機能付き電卓でDebianを動かした強者が現れた事が話題になっていました。使われた電卓のメーカはどれ?}
{Texus Instruments社}
{CASIO社}
{Hewllet Packard社}
{A}
{詳しくは:http://hackaday.com/2014/11/18/ running-debian-on-a-graphing-calculator/。使われた電卓の製品名は:TI-NSpire CXというシリーズの機種だそうです。}

\santaku
{Debianプロジェクトにてinitシステムのアプリケーションの依存度に関するGeneral Resolutionの投票が行われました。投票の結果は?}
{パッケージは特定のinitシステムのみに依存してはならない}
{パッケージメンテナが望めば、特定のinitシステムに依存してもOK}
{General Resoultionは不要だ}
{C}
{Debianでもinit systemとパッケージとの調整が引き続き一生懸命行われています。}

\santaku
{Debian Medメタパッケージのバージョンが1.99へ一気にジャンプする予定であることがアナウンスされました。この時初めて搭載される予定の医療関係者向けシステムは何?}
{遺伝子解析システム}
{外科手術用ロボットシステム}
{病院情報システム(HIS)}
{C}
{Debian Medチームの今回の頑張りは素晴らしく、PHYLPのライセンスを長年の交渉の末、DFSG準拠のものに変更してもらうなど、医療関係者ら・生物学関係者らに様々にポジティブな影響を与えています。この活動は医療分野の学術系の情報サイトBioMedCentralに''Community-driven development for computational biology at Sprints, Hackathons and Codefests''ということで取り上げられたそうです。}

\santaku
{2014/11/14にDebianのBTSに、Debianのパッケージ開発初心者へ修正作業が良い演習となるようなバグに関するtagが設けられました。以下のどれ}
{newcomer}
{gift}
{easeofcake}
{A}
{今までパッケージ開発初心者向けに割り当てられていたBTSのtagはgiftでしたが、newcomerへ変わるそうです。以前のgiftというtag名だとちょっと判りにくいとのことで、今回名前変更となったようです。}

\santaku
{Mysqlに関する仮想パッケージに定義されたMysqlDB互換DBソフトとしてPXCというのがあるが、こちらは何の略?}
{Protected eXchange Controler}
{Posgresql eXtra Controler}
{Percona XtraDB Cluster}
{C}
{Debianにて、Mysql系のDBとみなされたものは、3つあり、MysqlDB,MariaDB,Percona XtraDB Cluster(PCX)の3つとなります。}

\santaku
{2014/11/20にて、systemd用の定義ファイル内部で使えるセキュリティを大幅に向上出来るオプションのうち、ホームディレクトリの不正アクセスを抑止するのは次のどれ}
{ProtectHome}
{ProtectSystem}
{PrivateTmp}
{A}
{systemdの定義ファイルに用意されているセキュリティに関するオプションをうまく使いこなすと非常に強固なシステムにすることが出来るそうです。ノウハウと紹介はhttp://0pointer.net/public/systemd-nluug-2014.pdf}














