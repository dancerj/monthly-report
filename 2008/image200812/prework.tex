%; whizzy-master ../debianmeetingresume200812-presentation.tex
%%; whizzy-master ../debianmeetingresume200812.tex

\begin{prework}{上川純一}

今年やって来年やりたいことを考えてみました。

\preworksection{今年やったこと}

\begin{itemize}
 \item 一年の最初に計画をたててみた。最終的にはそのまま実施はしなかった
       けど、なんらかのアイデアの源泉にはなった。
 \item \LaTeX{} のハンズオンをやった。
 \item Debian パッケージ作成のための一連の講座をやってみて、
       新しく講師役をいろいろな人たちにしてもらった。
\end{itemize}
というのがありました。

\preworksection{来年やりたいなぁと思っているのは}

いろいろとハンズオン的なものを増やしていきたくて

\begin{itemize}
 \item DocBookのハンズオン
 \item パッケージ作成のハンズオン
 \item avahi の活用講座
 \item \LaTeX{}関連のもろもろのバグを直す。
 \item screencast / ビデオカンファレンス / ストリーミング関連の何か
\end{itemize}
あたりを考えています。

\end{prework}

\begin{prework}{平澤俊継}

\preworksection{Debian勉強会で年やったこと、来年したいこと}
今回で3回めの参加です。
デビアンもそうですが、しらないことだらけなので、
来年は皆勤賞をねらいつつ、色々とできるようになるといいなぁ

\preworksection{\LaTeX{}+Gitの事前課題ができなかった、こんなハマり方しました体験
       記}
\LaTeX{}はまぁ、あつかいなれている方なのでストレス無ですが、
ビバviなわたしにはemacsがなんともいえず、はぁぁぁ
ってかんじでとても苦労しています。
これだけの文章たたくのに10分ぐらいかかっているんですが

今回と前回の課題を通してまだよくわかっていないなぁと
感じるところは
\begin{itemize}
 \item gitのしくみ
 \item 自作マクロ(たとえばsantakuとか)
 \item そしていーまっくす
\end{itemize}

こんなもので勘弁してくだせぇ

\end{prework}
\begin{prework}{藤沢理聡}

\preworksection{「Debian勉強会で今年やったこと、来年したいこと」}

今年は、3月までは関西Debian勉強会、4月からは東京エリアDebian勉強会に参加
しました(何度かスケジュールの都合で参加できませんでしたが)。
毎回目新しい知識に触れられるという点で、勉強会という場は自分にとって魅力
的なものですが、すべてを一度に吸収するのは難しく、とり逃したものもありま
す。その点、過去の資料が見れることがありがたいです。

私的に今年の反省として、
\begin{itemize}
\item 勉強会で扱われた内容に対して、十分なtry and errorの時間を確保でき
      なかった
\item 勉強会やDebian Projectに対して何も貢献していない
\end{itemize}
という2点を来年は改善していきたいと考えています。

\preworksection{来年は}

上記を踏まえて \textbf{「1ユーザからの脱却」}を目指して頑張ろう
と思います。そのためにまず、
\url{http://www.debian.or.jp/community/devel/}に書いてあることは年末年始
を利用して理解しておきたいと思う次第です。

\end{prework}
\begin{prework}{前田耕平}
\preworksection{Debian 勉強会で今年やったこと。}

今年は、パッケージの作成で発表を行いました。当日せっかくツッコミをいろい
ろ頂いたのに、資料を修正していなかった、ということに今更ながら気づきまし
た。復習大事ですね。
あとは、Debian の ``で''は出会い系の``で''に貢献したこと。(わら

\preworksection{来年への決意表明。}
\begin{itemize}
 \item ネタの発表と勉強会の運営にもっと積極的に参画していきます、じゃなくて、\textbf{``しま
す''}。
 \item Debian 勉強会への参加者、Debian の開発者を増やすための方法を考え
       ます。
 \item ヨメを Debian ユーザにして、勉強会へ連れてきます。
\end{itemize}

\end{prework}
\begin{prework}{じつかた}

\preworksection{「Debian勉強会で今年やったこと」}

勉強会に参加して、

\begin{itemize}
 \item パッケージ作成関連(OSCでのハンズオン含む)
 \item \LaTeX{}ハンズオン
\end{itemize}
が勉強になった。
なんとなくパッケージ作成できるようになってきたので、もう一段ステップアップ
したい。

\preworksection{「Debian勉強会で来年したいこと」}

目標としては、
\begin{itemize}
 \item 勉強会の参加回数を増やす
 \item 何かの形で勉強会、Debian Projectに貢献したい
\end{itemize}

勉強会のネタとしては、
\begin{itemize}
 \item パッケージ作成
 \item バグレポートハンズオン
 \item 大規模な環境での、Debianサーバ管理方法。パッケージの一斉適用とか
       設定ファイルの管理とか、実際運用している方の話を聞いてみたいです。
\end{itemize}
といった感じでしょうか。

\end{prework}
\begin{prework}{やまだたくま}

\textbf{Debian勉強会で今年やったこと、来年したいこと}

勉強会には11月から参加し始めたばかりの Debian 初心者なので、今年やったことは
残念ながら一つしかないのですが...

\textbf{今年やったこと}

\begin{itemize}
\item avahi の準備の指示を見落としていたので、会場内に DHCP server がな
      いため途方にくれました(Windows の VMware 上で Debian を動かしていたため)
\item \TeX 懐かしい
\end{itemize}

\preworksection{来年したいこと}

\begin{itemize}
 \item git のハンズオン
 \item パッケージ作成のハンズオン
 \item 翻訳のハンズオン
\end{itemize}
あたりができるといいなあと思っています。

\end{prework}
\begin{prework}{小室 文}
\begin{itemize}
\item Debian勉強会で今年やったこと\\
今年は殆んど勉強会は参加しませんでした。。。。
とりあえず将来のDDは産みました。

\item  来年したいこと\\
基本的な事ですが,もっとCとkernelを勉強する事です。
んでserverのTuningをもっと出来るようになりたい。
それと似た境遇の人を探す事。女性でママでシステムやっている人いないかなあ。。。

 \item \LaTeX{}+Gitでこんなハマり方しました体験記\\
anthy+scimの入力が初めてなのでかなり時間がかかりました。
\end{itemize}

\end{prework}
\begin{prework}{小林 儀匡}

\preworksection{Debian勉強会で今年やったこと}
今年は、debhelper+CDBS+quilt/dpatchでのパッケージングに関する話と、
Po4aを用いた翻訳メンテナンスについての話をしました。

\preworksection{Debian勉強会で来年したいこと}
来年は、builddを用いた自動ビルドについて話ができたらと思います。
また、ちょっと手を動かしてみていることがいくつかあるので、
それらをもう少しきちんとしたかたちにして発表したいです。

最後に、来年こそはDebian Developerになりたいです。

\end{prework}
\begin{prework}{キタハラ}
\begin{itemize}
\item Debian勉強会で今年やったこと\\
序盤から中盤は健康上の理由で、あまり参加できませんでしたねぇ。
終盤では、万年stable利用者にもかかわらずsidを導入したり、
ん十年ぶりにTeXいじったり、怒涛の展開になっています。(今も・・・)

\item  来年したいこと\\
勉強会の事前課題が、次回以降もこの形式になるならば、emacsを覚えないと
いけないのかなぁ? 基本的に*nix環境では、vi派なのですが。
(・・・と言うより、emacsまったく使えません!)
\end{itemize}

\end{prework}
\begin{prework}{山本 浩之}
今年もあまり大したことをなさずに過ぎ去ってしまいました。

\textbf{今年やったこと}
\begin{itemize}
\item 昔、岩松さんが講演した live-helper の使い方を実践し、OSC にて Live
      DVD を配布しました。
\item 小林さんの講演に触発され、CDBS にてパッケージを作りました。
\end{itemize}

\textbf{来年したいこと}
\begin{itemize}
\item 英語を勉強し、翻訳関係 (査読とか) でも貢献したい。
\item emacs をもっとうまく使えるようになりたいです (笑)
\end{itemize}

\end{prework}
\begin{prework}{岩松 信洋}
\preworksection{今年}

今年は以下のようなことを行いました。
\begin{itemize}
 \item Debianパッケージングハンズオンをやった。
 \item Git にどっぷりとつかった。
 \item git-buildpackage / VCS と Debianパッケージについて考えてみた。
 \item カーネル側からのパッケージに関するアプローチをしてみた。\\
   Linux kernel patch / kernel module について話した。
 \item 合宿をやった。
 \item Ustream を使ったストリーミングを行った。
 \item 新しい言語への挑戦(Perl/Ruby/Lua)、チームへの参加を行った。
 \item 他の勉強会とのコラボをした。
 \item 勉強会運営を行った。
 \item サポートするパッケージを増やした。
\end{itemize}

\preworksection{来年の目標}

\begin{itemize}
 \item DDになる
 \item SH ポーティング、wanna-build/buuildd 関係
 \item 上位レイヤの方との交流
 \item ビデオ関係、ストリーミングのサポート強化
 \item カーネル関係での Debianへの貢献
\end{itemize}

\end{prework}


\begin{prework}{日比野 啓}

\preworksection{今年やったこと}
\begin{itemize}
\item 温泉に行ってBigloo(Schemeの処理系)をpackagingしてみました
\item Debian勉強会の会場として大学の部屋を借りる計画をしました
\end{itemize}

\preworksection{来年したいこと}
\begin{itemize}
\item 手元でいじっているコードを公開する
\item OCamlの布教活動(笑)
\end{itemize}

\end{prework}
