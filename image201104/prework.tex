%; whizzy-master ../debianmeetingresume201101.tex
% 以上の設定をしているため、このファイルで M-x whizzytex すると、whizzytexが利用できます。


\begin{prework}{ 吉野(yy\_y\_ja\_jp) }

普段通りデスクトップ環境として使ってます.
\end{prework}

\begin{prework}{ henrich }

活用は特段していません。停電のために家サーバと知人の会社に置いてあるサーバの両方が止まって難儀しました。
\end{prework}

\begin{prework}{ dictoss(杉本 典充) }

 \begin{itemize}
  \item 停電でPCを起動できないときもあり、むしろDebianを使えないときもありました。
  \item 使い方はメールしたり、Webブラウズをしたりと普段と変わらない感じで
	す。
 \end{itemize}
\end{prework}

\begin{prework}{ 岩松 信洋 }

 \begin{itemize}
  \item chromium の拡張プログラムを書いて、地震が来たらウィンドウを出すようなプログラムを書いてみました。
  \item 停電用にVPSを契約して、Debianにした。
  \item Debian上でOSMをいじったりした。
 \end{itemize}
 
\end{prework}

\begin{prework}{ 野島 貴英 }

震災で大変な中、非常につまらない例で恥ずかしいのですが、

 \begin{enumerate}
  \item 震災情報集めは主にdebain+ephipanyの組み合わせが多いです
  \item 震災に関する諸々の情報集めでNIKKEINET(有料)を利用する際はdebian+iceweaselばかりです。
  \item USBブート可能なパソコンさえあれば有事の際どこでも対応できるよう
	に自分のdebian環境を入れたブート可能なUSBメモリを持ち歩いていた
	りします。自由な使い方できる自分の環境をポケットにしのばせて持ち
	歩けるのはよいかも。地震あったときはこのUSBを胸ポケットに入れて
	避難しました。(たまたま入っていたともいう。でも自分の環境はここ
	にあるからとちょっと安心できたのは確か。)
 \end{enumerate}
 ※別にdebianでなくてもよいじゃんというツッコミはナシの方向で。
\end{prework}

\begin{prework}{ 山田 }

震災関係だとあまり活躍していないです…

今回はPCを持って被災したもののほとんど出番がなく、
携帯電話だけが活躍しました。

そういうわけで、早急にAndroid携帯にDebian/Ubuntuを
入れたいと思います(iDroidで通話できるのだろうか?)。
\end{prework}

\begin{prework}{ 村田信人 }

震災に対して直接は活用できていません。ただ、Sahana Edenの日本チームのLaunchpad, Bazaarに対する疑問に答えることで間接的に貢献できればと。
\end{prework}

\begin{prework}{ Kiwamu Okabe }

sakura VPS上にDebianをインストールしています。
仕事でNetBSDを使った開発をしているのですがネットワークが止まってもメンバーが開発継続できるように公式cvsリポジトリのtar玉を取るようにしました。

netbsd\_cvs =rsync=$>$ sakura\_vps(debian) =tar玉=$>$ 会社のメンツPC

http://www.masterq.net/files/
毎朝4時ごろ更新しています。
\end{prework}

\begin{prework}{ Osamu MATSUMOTO }

未定だったので回答なしで。
もう10年ぐらいDebian使ってばかりなので、貢献したいと思っている。
\end{prework}

\begin{prework}{ yamamoto }

今回の震災で帰宅時間が早くなった分、マシンへのお触りが増えました。
今もおいらの横でウンウン唸っています。
\end{prework}

\begin{prework}{ まえだこうへい }

我が家のDebianマシンたち。
\begin{itemize}
 \item 地震の当日、ヨメとディズニーシーから帰れなく、翌日帰宅するまで
       こまめが心配で仕方なかったので、OpenBlockS 600にWebカメ
       ラをつけて、こまちゃん監視システムを作りました。普段の様子や、停
       電の確認、大きな余震があった際の自宅の様子を見てます。
 \item 一方、OpenBlockS 266でハードディスクで稼働していた普通のサーバたち(Web,
       DNS, CouchDB, Gitリモートリポジトリ用, 監視など)は、停電に備えす
       べて停止しました。
 \item 地震の直後はしばらくMacBookの使用は控えて、iPadやAndroid使ってま
       した。
 \item 今後は外に出せるサーバやデータは、基本クラウドサービスを利用して、
       OpenBlockS等はディスクレス化して、センサーでの監視専用にしていく
       予定です。
\end{itemize}

\end{prework}
