

\begin{prework}{ 岩松 信洋 }

(1) erlang とHaskell を少々。

(2) Real World Haskell。プログラミングErlang。

(3) Erlang の ビルドシステムである rebar をもうちょっと理解したい。


\end{prework}

\begin{prework}{ yamamoto }

(1) 関数型プログラミング言語の利用経験 − なし
(2) お勧め書籍 − 読んだこと、ありません
(3) 使ってみたいソフト − 特になし

でも、haskell はなんか面白そうだし、学んでみようかな?とか考えています。
\end{prework}

\begin{prework}{ alice.ferrazzi }


\end{prework}

\begin{prework}{ 鈴木崇文 }

(1)Debian での関数型プログラミング言語の利用経験とその時に感じた事柄(Lisp, Emacs Lisp, OCaml, Haskell 等)
Erlang ・・・さわり程度ですが、Riakなどの実用レベルのソフトウェアで使用されている点や、分散環境や無停止に言語レベルで対応されている点が面白かったです。Riakから情報を取ったり入れたりするだけならば比較的簡単に操作できました。
Haskell ・・・さわり程度ですが、Haskell好きな人が多いため学ぶには良い言語だと感じました。

(2) 関数型プログラミング言語初心者に向けたお勧め書籍とそのウリの紹介
Erlangはなかなか本がなかったです。Haskellは「すごいHaskellたのしく学ぼう!」が読みやすいように感じますが、まだ理解できてません。

(3) 関数型言語で実装されている、使ってみたいソフトをあげてください
Riak
xmonad

\end{prework}

\begin{prework}{ 吉野(yy\_y\_ja\_jp) }

(1) 最近Haskellを触っています.cabal-debianをもう少し知りたいです.
\end{prework}

\begin{prework}{ キタハラ }

(1) Haskell 入門書のサンプルを動かした程度、
    apt-getで簡単に導入できて感動したような記憶が・・・。
(2) Haskellの入門書を2冊ほど読みましたが、共にmonadで
    挫折した、最近の本は読んでいない。
(3) 特になし。

\end{prework}

\begin{prework}{ dictoss(杉本 典充) }

(1) emacs.elを書くくらいでなんとなく使っている感じです。シングルクォーテーションは閉じなくていい場合があるのでそれに戸惑うことがあります。
\end{prework}

\begin{prework}{ 野首 }

elispでちょっとmajor modeとshinbum moduleを書いてみたことがあるぐらいです。
関数型プログラミングというレベルに至りませんでした。

\end{prework}

\begin{prework}{ @Lost\_dog\_ }

 (1) Haskell:型安全のありがたみが分かった
 (2) 『すごいHaskellたのしく学ぼう!』は関数型の雰囲気を俯瞰できる。本気で勉強するなら、もっと王道のテキストを選んだほうがよいかも。
 (3) yi-editor
\end{prework}

\begin{prework}{ 日比野 啓 }

(1) Haskell, OCaml ともに Debian には多数のパッケージがあってすばらしいです。

(2)
\begin{itemize}
\item {\bf プログラミングHaskell\\ - Graham Hutton (著), 山本 和彦 (翻訳)]}\\
最初に読むならこれです。関数プログラミングのトピックを平易に解説しならがHaskellを試していきます。
\item {\bf すごいHaskellたのしく学ぼう!\\ - Miran Lipovaa (著), 田中 英行 (翻訳), 村主 崇行 (翻訳)}\\
「プログラミングHaskell」の次はこれだと思います。より複雑な型の機能をも含めてHaskellの解説が進んでいきます。
\item {\bf Real World Haskell 実戦で学ぶ関数型言語プログラミング\\
 - Bryan O'Sullivan (著), John Goerzen (著), Don Stewart (著), 山下 伸夫 (翻訳), 伊東 勝利 (翻訳), 株式会社タイムインターメディア (翻訳)}\\
実際にHaskellを現場で利用してる人たちが書いた本としての価値がある本です。
Haskellにもいろいろなライブラリがあり、その利用例として参考になると思います。
少し古いのが問題点です。
\end{itemize}

(3) OCamlで作られている定理証明器Coqをうまく使えるようになりたいです。あとHaskellでいろいろ作る側にまわりたい。

\end{prework}
