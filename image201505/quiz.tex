%; whizzy-master ../debianmeetingresume201311.tex
% 以上の設定をしているため、このファイルで M-x whizzytex すると、whizzytexが利用できます。
%

\santaku
{Jessieが無事リリースされました。いつだったでしょうか?}
{2015/4/11}
{2015/4/18}
{2015/4/25}
{C}
{無事4/25にJessie(Debian 8)がリリースされました。Linux kernelは3.16.7、GNU Compiler Collection 4.9.2、GNOME 3.14、Apache 2.4.10、PHP 5.6.7などのバージョンのものが搭載されました。搭載されているパッケージ数は全部で43,000パッケージ以上もあります。}

\santaku
{Jessieで初めて追加されたものではないものが混ざっています。どれ?}
{Debian Games Blend}
{OpenJDK}
{androidsdk-tools}
{B}
{Jessieでの変更点はリリースノート(https://www.debian.org/releases/ jessie/amd64/release-notes/index.ja.html)にあります。Debian Gamesチームよりゲームの詰め合わせのDebian Blendが追加されたり、androidの開発ツールの一部も追加されたりしています。}

\santaku
{Debian GNU/Hurd 2015も2015/4/30にリリースされました。心臓部のGNU Machのバージョンはいくつ?}
{1.6}
{1.5}
{1.4}
{B}
{2015/4/10〜2015/4/15周辺でGNU/Hurd(upstream側)のアップデートが行われ、GNU Hurdはversion 0.6に、GNU Machは1.5になりました。未だにKVM/QEMUなどの仮想環境での動作が推奨の32bitシステムですが、多くのOSがクラウド環境で動作している世の中ですので、ホビー用途の利用でそろそろ注目されても良いかも?と思う次第です。}

\santaku
{2015年のGSoCに採択された、Debian MIPS portsについての開発内容は次のどれ?}
{多数のビルド出来ないパッケージを、ちゃんとビルドできるようにする。}
{新しいMIPS CPUへの対応}
{新しいMIPS CPU搭載製品への対応}
{A}
{Debian MIPS portsの多数のパッケージは再ビルドが出来ない、インストールに失敗するものが多数ある状況です。こちらを再ビルドできるように修正し、インストール出来るようにする事が採択されました。GSoCに限らず、いつでも本件の協力者募集中ですので、MIPS portsに興味ある方はパッケージの修正に是非ご協力くださいませー。}

\santaku
{遂にhttp.debian.netがdebian.orgのインフラに移動となりました。新しいURLはどれ?}
{http://http.debian.org/debian}
{http://httpredir.debian.org/debian}
{http://www.debian.org/}
{B}
{http.debian.netはHTTPのRedirectを活用し、ユーザに最も適切な場所にある、パッケージのミラー先を教えてくれるサービスです。遂に、実験的サービスの扱いであるdebian.netからDebian公式サービスであるdebian.orgへ移動となりました。http.debian.netをsource.listに設定している人は、httpredir.debian.orgへ修正をおねがいします。また、不具合は"mirrors" pseudo-packageでBTSするようです。参照:http://bugs.debian.org/mirrors }

\santaku
{debianへのlibdvdcss/ZFSパッケージ搭載の件について、2015/5のBit From DPLで報告された状況は以下のどれ?}
{DPLがいろいろ議論して回っている状況}
{搭載日確定した}
{搭載を諦めた}
{A}
{2015/5/19現在、FTPMasters, SFLC, FSFと議論中とのこと。さらに近々Linux Foundationとも議論予定。現在、当初想定したよりも、実現に関する状況が複雑とのことで、本件いつ決着するか見えなくなったとの事です。}

