%; whizzy chapter
% -initex iniptex -latex platex -format platex -bibtex jbibtex -fmt fmt
% 以上 whizzytex を使用する場合の設定。

%     Kansai Debian Meeting resources
%     Copyright (C) 2007 Takaya Yamashita
%     Thank you for Tokyo Debian Meeting resources

%     This program is free software; you can redistribute it and/or modify
%     it under the terms of the GNU General Public License as published by
%     the Free Software Foundation; either version 2 of the License, or
%     (at your option) any later version.

%     This program is distributed in the hope that it will be useful,
%     but WITHOUT ANY WARRANTY; without even the implied warranty of
%     MERCHANTABILITY or FITNESS FOR A PARTICULAR PURPOSE.  See the
%     GNU General Public License for more details.

%     You should have received a copy of the GNU General Public License
%     along with this program; if not, write to the Free Software
%     Foundation, Inc., 51 Franklin St, Fifth Floor, Boston, MA  02110-1301 USA

%  preview (shell-command (concat "evince " (replace-regexp-in-string "tex$" "pdf"(buffer-file-name)) "&"))
% 画像ファイルを処理するためにはebbを利用してboundingboxを作成。
%(shell-command "cd image200708; ebb *.png")

%%ここからヘッダ開始。

\documentclass[mingoth,a4paper]{jsarticle}
\usepackage{kansaimonthlyreport}
\usepackage[dvipdfmx]{xy}
\usepackage{etex}
\usepackage{ulem}

% 日付を定義する、毎月変わります。
\newcommand{\debmtgyear}{2014}
\newcommand{\debmtgdate}{28}
\newcommand{\debmtgmonth}{12}
\newcommand{\debmtgnumber}{92}

\def\fixme#1{{\color{red}{#1}}}

\begin{document}

\begin{titlepage}

% 毎月変更する部分、本文の末尾も修正することをわすれずに

 第\debmtgnumber{}回 関西 Debian 勉強会資料

\vspace{2cm}

\begin{center}
\includegraphics{image200802/kansaidebianlogo.png}
\end{center}

\begin{flushright}
\hfill{}関西 Debian 勉強会担当者 佐々木・倉敷・のがた・かわだ \\
\hfill{}\debmtgyear{}年\debmtgmonth{}月\debmtgdate{}日
\end{flushright}

\thispagestyle{empty}
\end{titlepage}

\dancersection{Introduction}{Debian JP}

\vspace{1em}

 関西Debian勉強会はDebian GNU/Linuxのさまざまなトピック
 (新しいパッケージ、Debian特有の機能の仕組、Debian界隈で起こった出来事、
 などなど)について話し合う会です。

 目的として次の三つを考えています。
 \begin{itemize}
  \item MLや掲示板ではなく、直接顔を合わせる事での情報交換の促進
  \item 定期的に集まれる場所
  \item 資料の作成
 \end{itemize}

 それでは、楽しい一時をお過ごしください。

\newpage

\begin{minipage}[b]{0.2\hsize}
 {\rotatebox{90}{\fontsize{80}{80}
{\gt 関西 Debian 勉強会}}}
\end{minipage}
\begin{minipage}[b]{0.8\hsize}
\hrule
\vspace{2mm}
\hrule
\setcounter{tocdepth}{1}
\tableofcontents
\vspace{2mm}
\hrule
\end{minipage}

\dancersection{最近のDebian関係のイベント報告}{Debian JP}

\subsection{第91回関西Debian勉強会}

91回目の関西Debian勉強会は11月23日(日)に、港区民センターで行なわれました。

Jessieのインストーラテストを中心とした内容での開催でした。

\subsection{第120回東京エリアDebian勉強会}

120回目の東京エリアDebian勉強会は11月29日(土)に株式会社スクウェア・エニッ
クス セミナールームで開催されました。

野島さんによる「DebianからみたArch Linux」ともくもくの会で開催されました。

\subsection{第121回東京エリアDebian勉強会}

121回目の東京エリアDebian勉強会は12月20日(土)に株式会社スクウェア・エニッ
クス セミナールームで開催されました。

林さんによるGroongaのDebianパッケージ化とCutterをFedoraパッケージ化した
「DebianとFedoraでパッケージをリリースするまでの話」と、野島さんによる
「DebianとLinux Mint」の二本立てでした。

\dancersection{事前課題}{Debian JP}

今回の課題は以下の通りです。
\begin{screen}
  \begin{enumerate}

  \item %
    Debian 界隈で今年印象に残っていること、話題を教えてください。
    
  \end{enumerate}
\end{screen}

参加者の皆さんの解答は以下の通りです:

\begin{prework}{ 木下聖士 }
  \begin{enumerate}
  \item ここ最近Debianな環境から離れてしまっていますので今は御座いませんが、
     プライベートWebサーバのDiskが最近お亡くなりになったので
     これを機に最新のDebianにてWebサーバの立ち上げにチャレンジしてみます。
  \end{enumerate}
\end{prework}

\begin{prework}{ murase\_syuka }
\end{prework}

\begin{prework}{ Kozo Nishida }
  \begin{enumerate}
  \item uim-tcodeの不要なemacs依存をdaiさんが取り除いてくれたこと
  \end{enumerate}
\end{prework}

\begin{prework}{ Yukiharu YABUKI }
\end{prework}

\begin{prework}{ 榎真治 }
\end{prework}

\begin{prework}{ かわだてつたろう }
  \begin{enumerate}
  \item Code of Conduct
  \end{enumerate}
\end{prework}

\begin{prework}{ ItSANgo }
  \begin{enumerate}
  \item
    \begin{enumerate}
    \item systemdがDebianにもやってきたこと。
    \item Egg(Tamago)をgniibeさんが更新したこと。

      \url{http://anonscm.debian.org/cgit/pkg-anthy/egg.git}

      Egg(Tamago)の実装が分裂してしまいそうなこと。

      \url{https://sourceforge.jp/projects/tamago-tsunagi/}

      (個人的に気がかりです。)
    \end{enumerate}
  \end{enumerate}
\end{prework}

\begin{prework}{ nogajun }
\end{prework}

\begin{prework}{ 川江 浩 }
\end{prework}

\dancersection{DPNで振り返る2014}{かわだ てつたろう}

2014年に発行されたDPN(Debian Project News\footnote{\url{https://www.debian.org/News/weekly/}})
からDebian Project界隈の動向をざっと振り返ってみたいと思います。

2014年は1月6日号から12月1日号まで計16通発行されました。

\subsection{1st quarter}

1月31日、IA64アーキテクチャがJessieから取り除かれました。

2月11日、CTTEはJessieのLinuxアーキテクチャのinitシステムをsystemdとする決定を行な
いました。

3月19日、Jessieのインストーラのアルファ1をリリースされました。

\subsection{2nd quarter}
4月17日、DPL選挙の結果再選されたLucas NussbaumさんのDPL体制がスタートしました。

4月28日、Code of Conductが批准されました。これは先のCTTEのsystemdの決定を受けて
debian-develメーリングリストが荒れに荒れたことに端を発しています。

4月26日、SPARCアーキテクチャがJessieから取り除かれました。

5月22日、Debian GNU/Hurdの活動報告があり、80\%のパッケージをカバーしSysVinitが動作す
るとのことです。

5月31日、旧stableのSqueezeはサポート終了となりました。しかし、新な試みとして限
定ながら5年の長期サポートを目指して「squeeze-lts」が始まりました。

6月3日、MATE1.8の全パッケージングが完了したとのアナウンスがありました。

6月18日、libcがeglibcからglibcに戻りました。

\subsection{3rd quarter}

8月16日、Debian Dayを迎えDebianは21歳になりました。

8月22日から8月31日にかけてDebConf14がアメリカのオレゴン州ポートランドで開催され
ました。

9月4日、Cinnamonの全パッケージがJessieに入ったと報告がありました。

9月19日、再評価の結果Jessieのデフォルトデスクトップ環境がGNOMEに決まりました。

\subsection{4th quarter}

10月22日、Debian Multimedia Maintainersからの活動報告があり、Codecのアップデート、
新しいアプリの紹介などがありました。

11月5日、Jessieのフリーズが行なわれました。

11月9日、Debian 9のコードネームは"Stretch"でDebian 10のコードネームは"Buster"であ
るとリリースチームより報告がありました。

11月18日、「init system coupling」の投票結果が「General Resolution is not required」
となりました。

\dancersection{2014年の振り返りと2015年の企画}{Debian JP}

2014年も今月で最後の関西Debian勉強会になります。
2007年3月からはじまった勉強会も今年で8年目になり、来年には100回目を迎えようとしています。

\subsection{勉強会全体について}

\subsubsection{もくもくの会}

今年からは内容をもくもくの会を主体として開催するようにしました。
各自抱えておられる作業を持ち寄って進めていくよい場となりました。しかし、参加され
たことのない新規の方々にとっては、内容がよくわからないようで、障壁となってしまっ
た面もあるようです。

\subsubsection{セッション、定例ネタ}

毎回のセッションは設定しませんでしたが、「Debian で楽しむ 3D プリンティング」、
「自宅サーバにkvmを導入してみよう」、「Notmuch Mail」、「Debian での systemd と
  のつきあい方」、「Linuxのドライバメンテナになった体験記」の5本の発表をしていただ
きました。

いわゆる定番のネタについてのセッションはできませんでした。
来年はjessieのリリースにあわせて定番のネタ(Debianの入門的なお話、ライセンス、パッ
ケージ作成、BTS)のセッションができたら良いかなと考えています。

\subsection{運営}

運営に関しては、毎月の継続した開催は行なってきていました。

申し込みのシステムとしてDoorkeeperを試験的に使用しました。
connpassなど他の手段も試し、いわゆる今風の申し込みやすい環境を整えていきます。

昨年に続き、事前の準備、告知が不十分な場合がありましたので、今後は、定型の告知を
先に投げるなど、改善していきたいと考えています。

また、運営に参加していただいていた八津尾さんが、関西から離れられることになり、運
営から抜けられました。

\subsection{イベント/NM申請}

イベント参加については、OSC Kansai@Kyoto、KOF2014に参加しました。
今後も継続して参加する予定です。
%
今年は合宿や「大統一Debian勉強会」などのイベントの実施はありませんでした。

Debian開発者への道、として一昨年から倉敷さんがNMプロセスの申請を進められています。
今後もNMプロセスに挑む参加者やDebian Maintainerとなって精力的にパッケージを開発/
更新する人が増えると良いですね。

\subsection{開催実績}
関西Debian勉強会の出席状況を確認してみましょう。グラフで見る
と\fgref{fig:kansaipeoplechart}になります。また、毎回の参加者、アンケート
回答者の人数とその際のトピックを \tbref{tab:count2014kansai} にまとめまし
た。グラフ中の黒線は参加人数、赤線は1年の移動平均、青線はアンケートの回答
人数です。参加人数が$0$となっているところは人数が集計されていないor開催さ
れなかった月です。アンケートの回答人数が$0$となっているところはアンケート
が実施されなかった月です。

事前課題は毎回設定しましたが、アンケートシステムは1回しか活用しませんでし
た。

%
\begin{figure}[h]
  \begin{center}
    \includegraphics[width=.6\hsize]{image201412/memberanalysis/kansai.png}
  \end{center}
  \caption{関西の参加人数推移(参加人数と6ヶ月移動平均、アンケート回答人数)}
  \label{fig:kansaipeoplechart}
\end{figure}

%\pagebreak

\begin{table}
  \begin{minipage}{.5\linewidth}
    \caption{関西Debian勉強会の参加人数とトピック(2007-2008)}
    \begin{center}
      \begin{tabular}{|l|c|p{10em}|}
        \hline
        開催年月   & 参加人数 & 内容 \\
        \hline
        2007年3月  & 19       & 開催にあたり \\
        2007年4月  & 25       & goodbye、youtube、プロジェクトトラッカー\\
        2007年6月  & 23       & 社会契約、テーマ、debian/rules、bugreport\\
        2007年7月  & 20前後   & OSC-Kansai \\
        2007年8月  & 20       & Inkscape、patch、dpatch\\
        2007年9月  & 16       & ライブラリ、翻訳、debtorrent\\
        2007年10月 & 22       & 日本語入力、SPAMフィルタ\\
        2007年11月 & 20前後   & KOF \\
        2007年12月 & 15       & 忘年会、iPod touch\\
        \hline
        \hline
        開催年月   & 参加人数 & 内容 \\
        \hline
        2008年2月  & 20       & PC Cluster, GIS, \TeX \\
        2008年3月  & 23       & bug report, developer corner, GPG \\
        2008年4月  & 24       & coLinux, Debian GNU/kFreeBSD, sid \\
        2008年5月  & 25       & ipv6, emacs, ustream.tv\\
        2008年6月  & 20       & pbuilder, hotplug, ssl\\
        2008年8月  & 13       & coLinux \\
        2008年9月  & 17       & debian mentors, ubiquity, DFSG\\
        2008年10月 & 11       & cdbs,cdn.debian.or.jp \\
        2008年11月 & 35       & KOF \\
        2008年12月 & ?        & TeX資料作成ハンズオン\\
        \hline
      \end{tabular}
    \end{center}
  \end{minipage}
  \pagebreak
  \begin{minipage}{.5\linewidth}
    \begin{center}
      \caption{関西Debian勉強会の参加人数とトピック(2009-2010)}
      \begin{tabular}{|l|c|p{10em}|}
        \hline
        開催年月   & 参加人数 & 内容 \\
        \hline
        2009年1月  & 18       & DMCK, LT \\
        2009年3月  & 12       & Git \\
        2009年4月  & 13       & Installing sid, Mancoosi, keysign \\
p        2009年6月  & 18       & Debian Live, bash\\
        2009年7月  & 30?      & OSC2009Kansai \\
        2009年8月  & 14       & DDTSS, lintian \\
        2009年9月  & 14       & reportbug, debian mentors\\
        2009年10月 & 16       & gdb, packaging \\
        2009年11月 & 35       & KOF2009 \\
        2009年12月 & 16       & GPS program, OpenStreetMap \\
        \hline
        \hline
        開催年月   & 参加人数 & 内容 \\
        \hline
        2010年1月  & 16       & Xen, 2010年企画 \\
        2010年2月  & 16       & レンタルサーバでの利用, GAE \\
        2010年3月  & 30?      & OSC2010Kobe \\
        2010年4月  & 12       & デスクトップ環境, 正規表現 \\
        2010年5月  & 11       & ubuntu, squeeze \\
        2010年6月  & 11       & debhelper7, cdbs, puppet \\
        2010年7月  & 40?      & OSC2010Kyoto \\
        2010年8月  & 17       & emdebian, kFreeBSD \\pp
        2010年9月  & 17       & タイルWM \\
        2010年10月 & 12       & initramfs, debian live \\
        2010年11月 & 33       & KOF2010 \\
        2010年12月 & 14       & Proxmox, annual review \\
        \hline
      \end{tabular}
    \end{center}
  \end{minipage}
\end{table}

\begin{table}
  \begin{minipage}{.5\linewidth}
    \caption{関西Debian勉強会の参加人数とトピック(2011-2012)}
    \begin{center}
      \begin{tabular}{|l|c|c|p{10em}|}
        \hline
        開催年月  & 参加 & 回答 & 内容 \\
        \hline
        2011年1月 &10    & 0    & BTS, kFreeBSD\\
        2011年2月 &15    & 0    & pbuilder, Squeezeリリースパーティ\\
        2011年3月 &17    & 0    & ライセンス, ドキュメント\\
        2011年4月 &25    & 0    & OSC Kansai@Kobe \\
        2011年5月 &20    &12    & vi, dpkg \\
        2011年6月 &17    & 0    & vcs-buildpackage\{svn, git\}, IPv6\\
        2011年7月 &17    & 0    & OSC Kansai@Kyoto \\
        2011年8月 &20    & 9    & パッケージ作成ハンズオン\\
        2011年9月 &11    & 0    & vcs-buildpackage\{bzr, git\}\\
        2011年10月&11    & 0    & Emacs, vim 拡張のDebianパッケージ, 翻訳\\
        2011年11月&23    & 0    & KOF 2011\\
        2011年12月&13    & 5    & NMプロセス, BTS\\
        \hline
        \hline
        開催年月  & 参加 & 回答 & 内容 \\
        \hline
        2012年1月 & 7    &0     & Debian温泉合宿 \\
        2012年2月 &14    &0     & autofs+pam\_chroot, t-code, Debian Policy \\
        2012年3月 &12    &0     & Konoha, t-code, Debian Policy \\
        2012年4月 &12    &0     & フリーソフトウェアと著作権, Konoha, Debian Policy \\
        2012年5月 &13    &0     & DebianとLDAP(頓挫), ITP入門, Debian Policy \\
        2012年6月 & -    &0     & 大統一Debian勉強会 \\
        2012年7月 &10    &0     & DebianとLDAP, Debian Policy \\
        2012年8月 &28    &2     & OSC Kansai@Kyoto \\
        2012年8月 &16    &0     & DebianとKerberos, News from EDOS \\
        2012年9月 & 8    &0     & clang, Debian Policy \\
        2012年10月&14    &3     & 翻訳環境構築, DSAの舞台裏\\
        2012年11月&34    &0     & KOF 2012\\
        2012年12月&12    &0     & Debian on Android, Debian Policy \\
        \hline
      \end{tabular}
    \end{center}
  \end{minipage}
  \pagebreak
  \begin{minipage}{.5\linewidth}
    \caption{関西Debian勉強会の参加人数とトピック(2013)}
    \begin{center}
      \begin{tabular}{|l|c|c|p{10em}|}
        \hline
        開催年月  & 参加 & 回答 & 内容 \\
        \hline
        2013年1月 & 8    &0     & Using Drupal on Debian, 月刊Debian Policy その8 \\
        2013年2月 &11    &6     & Debian Installerトラブルシューティング, Ruby In Wheezy \\
        2013年3月 &12    &6     & UbuntuとGNOME Shellと私, 管理者視点からのGNOMEの大規模な配置 \\
        2013年4月 &10    &0     & リリースノートを読んでみよう。, \\
                  &      &      & クラウド初心者がAWSにDebianをのっけて翻訳サービスの試行に挑戦してみた \\
        2013年5月 &17    &0     & DebianとUbuntuの違いを知ろう, Debianの歩き方 \\
        2013年6月 & -    &0     & 大統一Debian勉強会 \\
        2013年7月 &      &0     & OSC 2013 Kansai @ Kyoto, GPG キーサインパーティ\\
        2013年8月 & 8    &3     & puppetによる構成管理の実践 \\
        2013年9月 &11    &2     & Linuxとサウンドシステム, dgitでソースパッケージを触ってみる \\
        2013年10月&11    &0     & ALSAのユーザーランド解説, git-buildpackage入門again \\
        2013年11月&20    &0     & KOF 2013 \\
        2013年12月& 6    &0     & 2013年の振り返りと2014年の企画, 忘年会 \\
        \hline
      \end{tabular}
    \end{center}
  \end{minipage}
\end{table}

\begin{table}
    \caption{関西Debian勉強会の参加人数とトピック(2014)}
    \label{tab:count2014kansai}
    \begin{center}
%      \begin{tabular}{|l|c|c|p{10em}|}
      \begin{tabular}{|l|c|c|l|}
        \hline
        開催年月  & 参加人数 & 回答人数 & 内容 \\
        \hline
        2014年1月 &12        &0         & LT, もくもくの会 \\
        2014年2月 &10        &0         & もくもくの会 \\
        2014年3月 &10        &0         & Debian で楽しむ 3D プリンティング, もくもくの会 \\
        2014年4月 &11        &0         & 自宅サーバにkvmを導入してみよう, Notmuch Mail, もくもくの会 \\
        2014年5月 & 8        &0         & もくもくの会 \\
        2014年6月 &11        &2         & Debian での systemd とのつきあい方, Linuxのドライバメンテナになった体験記, \\
                  &          &          & キーサイン, もくもくの会 \\
        2014年8月 &30        &0         & OSC 2014 Kansai @ Kyoto \\
                  &11        &0         & もくもくの会 \\
        2014年9月 & 7        &0         & もくもくの会 \\
        2014年10月& 7        &0         & もくもくの会 \\
        2014年11月&30        &0         & KOF 2014 \\
                  & 4        &0         & インストーラテスト, もくもくの会 \\
        2014年12月& 9        &0         & 2014年の振り返りと2015年の企画, 忘年会 \\
        \hline
      \end{tabular}
    \end{center}
\end{table}


\clearpage

% 冊子都合で「今後の予定を削除」
%% \dancersection{今後の予定}{Debian JP}
%% \subsection{関西Debian勉強会}
%% 次回、第93回関西Debian勉強会は1月25日(日)に港区民センター 楓で
%% 開催予定です。
%% \subsection{東京エリアDebian勉強会}
%% 第122回東京エリアDebian勉強会は1月おそらく17日(土)に開催予定です。

%
% 冊子にするために、4の倍数にする必要がある。
% そのための調整
%% \dancersection{メモ}{}
%% \mbox{}\newpage
%% \mbox{}\newpage
%% \mbox{}\newpage

\pagebreak

\begin{center}
本資料のライセンスについて
\end{center}

本資料はフリー・ソフトウェアです。あなたは、Free Software
Foundation が公表したGNU GENERAL PUBLIC LICENSEの "バージョン2"もしくはそれ以降
が定める条項に従って本プログラムを再頒布または変更することができ
ます。

本プログラムは有用とは思いますが、頒布にあたっては、市場性及び特
定目的適合性についての暗黙の保証を含めて、いかなる保証も行ないま
せん。詳細についてはGNU GENERAL PUBLIC LICENSE をお読みください。

\begin{multicols}{2}
 \begin{fontsize}{6}{6}
 \begin{verbatim}
		    GNU GENERAL PUBLIC LICENSE
		       Version 2, June 1991

 Copyright (C) 1989, 1991 Free Software Foundation, Inc.
	51 Franklin St, Fifth Floor, Boston, MA  02110-1301  USA
 Everyone is permitted to copy and distribute verbatim copies
 of this license document, but changing it is not allowed.

			    Preamble

  The licenses for most software are designed to take away your
freedom to share and change it.  By contrast, the GNU General Public
License is intended to guarantee your freedom to share and change free
software--to make sure the software is free for all its users.  This
General Public License applies to most of the Free Software
Foundation's software and to any other program whose authors commit to
using it.  (Some other Free Software Foundation software is covered by
the GNU Library General Public License instead.)  You can apply it to
your programs, too.

  When we speak of free software, we are referring to freedom, not
price.  Our General Public Licenses are designed to make sure that you
have the freedom to distribute copies of free software (and charge for
this service if you wish), that you receive source code or can get it
if you want it, that you can change the software or use pieces of it
in new free programs; and that you know you can do these things.

  To protect your rights, we need to make restrictions that forbid
anyone to deny you these rights or to ask you to surrender the rights.
These restrictions translate to certain responsibilities for you if you
distribute copies of the software, or if you modify it.

  For example, if you distribute copies of such a program, whether
gratis or for a fee, you must give the recipients all the rights that
you have.  You must make sure that they, too, receive or can get the
source code.  And you must show them these terms so they know their
rights.

  We protect your rights with two steps: (1) copyright the software, and
(2) offer you this license which gives you legal permission to copy,
distribute and/or modify the software.

  Also, for each author's protection and ours, we want to make certain
that everyone understands that there is no warranty for this free
software.  If the software is modified by someone else and passed on, we
want its recipients to know that what they have is not the original, so
that any problems introduced by others will not reflect on the original
authors' reputations.

  Finally, any free program is threatened constantly by software
patents.  We wish to avoid the danger that redistributors of a free
program will individually obtain patent licenses, in effect making the
program proprietary.  To prevent this, we have made it clear that any
patent must be licensed for everyone's free use or not licensed at all.

  The precise terms and conditions for copying, distribution and
modification follow.

		    GNU GENERAL PUBLIC LICENSE
   TERMS AND CONDITIONS FOR COPYING, DISTRIBUTION AND MODIFICATION

  0. This License applies to any program or other work which contains
a notice placed by the copyright holder saying it may be distributed
under the terms of this General Public License.  The "Program", below,
refers to any such program or work, and a "work based on the Program"
means either the Program or any derivative work under copyright law:
that is to say, a work containing the Program or a portion of it,
either verbatim or with modifications and/or translated into another
language.  (Hereinafter, translation is included without limitation in
the term "modification".)  Each licensee is addressed as "you".

Activities other than copying, distribution and modification are not
covered by this License; they are outside its scope.  The act of
running the Program is not restricted, and the output from the Program
is covered only if its contents constitute a work based on the
Program (independent of having been made by running the Program).
Whether that is true depends on what the Program does.

  1. You may copy and distribute verbatim copies of the Program's
source code as you receive it, in any medium, provided that you
conspicuously and appropriately publish on each copy an appropriate
copyright notice and disclaimer of warranty; keep intact all the
notices that refer to this License and to the absence of any warranty;
and give any other recipients of the Program a copy of this License
along with the Program.

You may charge a fee for the physical act of transferring a copy, and
you may at your option offer warranty protection in exchange for a fee.

  2. You may modify your copy or copies of the Program or any portion
of it, thus forming a work based on the Program, and copy and
distribute such modifications or work under the terms of Section 1
above, provided that you also meet all of these conditions:

    a) You must cause the modified files to carry prominent notices
    stating that you changed the files and the date of any change.

    b) You must cause any work that you distribute or publish, that in
    whole or in part contains or is derived from the Program or any
    part thereof, to be licensed as a whole at no charge to all third
    parties under the terms of this License.

    c) If the modified program normally reads commands interactively
    when run, you must cause it, when started running for such
    interactive use in the most ordinary way, to print or display an
    announcement including an appropriate copyright notice and a
    notice that there is no warranty (or else, saying that you provide
    a warranty) and that users may redistribute the program under
    these conditions, and telling the user how to view a copy of this
    License.  (Exception: if the Program itself is interactive but
    does not normally print such an announcement, your work based on
    the Program is not required to print an announcement.)

These requirements apply to the modified work as a whole.  If
identifiable sections of that work are not derived from the Program,
and can be reasonably considered independent and separate works in
themselves, then this License, and its terms, do not apply to those
sections when you distribute them as separate works.  But when you
distribute the same sections as part of a whole which is a work based
on the Program, the distribution of the whole must be on the terms of
this License, whose permissions for other licensees extend to the
entire whole, and thus to each and every part regardless of who wrote it.

Thus, it is not the intent of this section to claim rights or contest
your rights to work written entirely by you; rather, the intent is to
exercise the right to control the distribution of derivative or
collective works based on the Program.

In addition, mere aggregation of another work not based on the Program
with the Program (or with a work based on the Program) on a volume of
a storage or distribution medium does not bring the other work under
the scope of this License.

  3. You may copy and distribute the Program (or a work based on it,
under Section 2) in object code or executable form under the terms of
Sections 1 and 2 above provided that you also do one of the following:

    a) Accompany it with the complete corresponding machine-readable
    source code, which must be distributed under the terms of Sections
    1 and 2 above on a medium customarily used for software interchange; or,

    b) Accompany it with a written offer, valid for at least three
    years, to give any third party, for a charge no more than your
    cost of physically performing source distribution, a complete
    machine-readable copy of the corresponding source code, to be
    distributed under the terms of Sections 1 and 2 above on a medium
    customarily used for software interchange; or,

    c) Accompany it with the information you received as to the offer
    to distribute corresponding source code.  (This alternative is
    allowed only for noncommercial distribution and only if you
    received the program in object code or executable form with such
    an offer, in accord with Subsection b above.)

The source code for a work means the preferred form of the work for
making modifications to it.  For an executable work, complete source
code means all the source code for all modules it contains, plus any
associated interface definition files, plus the scripts used to
control compilation and installation of the executable.  However, as a
special exception, the source code distributed need not include
anything that is normally distributed (in either source or binary
form) with the major components (compiler, kernel, and so on) of the
operating system on which the executable runs, unless that component
itself accompanies the executable.

If distribution of executable or object code is made by offering
access to copy from a designated place, then offering equivalent
access to copy the source code from the same place counts as
distribution of the source code, even though third parties are not
compelled to copy the source along with the object code.

  4. You may not copy, modify, sublicense, or distribute the Program
except as expressly provided under this License.  Any attempt
otherwise to copy, modify, sublicense or distribute the Program is
void, and will automatically terminate your rights under this License.
However, parties who have received copies, or rights, from you under
this License will not have their licenses terminated so long as such
parties remain in full compliance.

  5. You are not required to accept this License, since you have not
signed it.  However, nothing else grants you permission to modify or
distribute the Program or its derivative works.  These actions are
prohibited by law if you do not accept this License.  Therefore, by
modifying or distributing the Program (or any work based on the
Program), you indicate your acceptance of this License to do so, and
all its terms and conditions for copying, distributing or modifying
the Program or works based on it.

  6. Each time you redistribute the Program (or any work based on the
Program), the recipient automatically receives a license from the
original licensor to copy, distribute or modify the Program subject to
these terms and conditions.  You may not impose any further
restrictions on the recipients' exercise of the rights granted herein.
You are not responsible for enforcing compliance by third parties to
this License.

  7. If, as a consequence of a court judgment or allegation of patent
infringement or for any other reason (not limited to patent issues),
conditions are imposed on you (whether by court order, agreement or
otherwise) that contradict the conditions of this License, they do not
excuse you from the conditions of this License.  If you cannot
distribute so as to satisfy simultaneously your obligations under this
License and any other pertinent obligations, then as a consequence you
may not distribute the Program at all.  For example, if a patent
license would not permit royalty-free redistribution of the Program by
all those who receive copies directly or indirectly through you, then
the only way you could satisfy both it and this License would be to
refrain entirely from distribution of the Program.

If any portion of this section is held invalid or unenforceable under
any particular circumstance, the balance of the section is intended to
apply and the section as a whole is intended to apply in other
circumstances.

It is not the purpose of this section to induce you to infringe any
patents or other property right claims or to contest validity of any
such claims; this section has the sole purpose of protecting the
integrity of the free software distribution system, which is
implemented by public license practices.  Many people have made
generous contributions to the wide range of software distributed
through that system in reliance on consistent application of that
system; it is up to the author/donor to decide if he or she is willing
to distribute software through any other system and a licensee cannot
impose that choice.

This section is intended to make thoroughly clear what is believed to
be a consequence of the rest of this License.

  8. If the distribution and/or use of the Program is restricted in
certain countries either by patents or by copyrighted interfaces, the
original copyright holder who places the Program under this License
may add an explicit geographical distribution limitation excluding
those countries, so that distribution is permitted only in or among
countries not thus excluded.  In such case, this License incorporates
the limitation as if written in the body of this License.

  9. The Free Software Foundation may publish revised and/or new versions
of the General Public License from time to time.  Such new versions will
be similar in spirit to the present version, but may differ in detail to
address new problems or concerns.

Each version is given a distinguishing version number.  If the Program
specifies a version number of this License which applies to it and "any
later version", you have the option of following the terms and conditions
either of that version or of any later version published by the Free
Software Foundation.  If the Program does not specify a version number of
this License, you may choose any version ever published by the Free Software
Foundation.

  10. If you wish to incorporate parts of the Program into other free
programs whose distribution conditions are different, write to the author
to ask for permission.  For software which is copyrighted by the Free
Software Foundation, write to the Free Software Foundation; we sometimes
make exceptions for this.  Our decision will be guided by the two goals
of preserving the free status of all derivatives of our free software and
of promoting the sharing and reuse of software generally.

			    NO WARRANTY

  11. BECAUSE THE PROGRAM IS LICENSED FREE OF CHARGE, THERE IS NO WARRANTY
FOR THE PROGRAM, TO THE EXTENT PERMITTED BY APPLICABLE LAW.  EXCEPT WHEN
OTHERWISE STATED IN WRITING THE COPYRIGHT HOLDERS AND/OR OTHER PARTIES
PROVIDE THE PROGRAM "AS IS" WITHOUT WARRANTY OF ANY KIND, EITHER EXPRESSED
OR IMPLIED, INCLUDING, BUT NOT LIMITED TO, THE IMPLIED WARRANTIES OF
MERCHANTABILITY AND FITNESS FOR A PARTICULAR PURPOSE.  THE ENTIRE RISK AS
TO THE QUALITY AND PERFORMANCE OF THE PROGRAM IS WITH YOU.  SHOULD THE
PROGRAM PROVE DEFECTIVE, YOU ASSUME THE COST OF ALL NECESSARY SERVICING,
REPAIR OR CORRECTION.

  12. IN NO EVENT UNLESS REQUIRED BY APPLICABLE LAW OR AGREED TO IN WRITING
WILL ANY COPYRIGHT HOLDER, OR ANY OTHER PARTY WHO MAY MODIFY AND/OR
REDISTRIBUTE THE PROGRAM AS PERMITTED ABOVE, BE LIABLE TO YOU FOR DAMAGES,
INCLUDING ANY GENERAL, SPECIAL, INCIDENTAL OR CONSEQUENTIAL DAMAGES ARISING
OUT OF THE USE OR INABILITY TO USE THE PROGRAM (INCLUDING BUT NOT LIMITED
TO LOSS OF DATA OR DATA BEING RENDERED INACCURATE OR LOSSES SUSTAINED BY
YOU OR THIRD PARTIES OR A FAILURE OF THE PROGRAM TO OPERATE WITH ANY OTHER
PROGRAMS), EVEN IF SUCH HOLDER OR OTHER PARTY HAS BEEN ADVISED OF THE
POSSIBILITY OF SUCH DAMAGES.

		     END OF TERMS AND CONDITIONS

	    How to Apply These Terms to Your New Programs

  If you develop a new program, and you want it to be of the greatest
possible use to the public, the best way to achieve this is to make it
free software which everyone can redistribute and change under these terms.

  To do so, attach the following notices to the program.  It is safest
to attach them to the start of each source file to most effectively
convey the exclusion of warranty; and each file should have at least
the "copyright" line and a pointer to where the full notice is found.

    <one line to give the program's name and a brief idea of what it does.>
    Copyright (C) <year>  <name of author>

    This program is free software; you can redistribute it and/or modify
    it under the terms of the GNU General Public License as published by
    the Free Software Foundation; either version 2 of the License, or
    (at your option) any later version.

    This program is distributed in the hope that it will be useful,
    but WITHOUT ANY WARRANTY; without even the implied warranty of
    MERCHANTABILITY or FITNESS FOR A PARTICULAR PURPOSE.  See the
    GNU General Public License for more details.

    You should have received a copy of the GNU General Public License
    along with this program; if not, write to the Free Software
    Foundation, Inc., 51 Franklin St, Fifth Floor, Boston, MA  02110-1301 USA


Also add information on how to contact you by electronic and paper mail.

If the program is interactive, make it output a short notice like this
when it starts in an interactive mode:

    Gnomovision version 69, Copyright (C) year  name of author
    Gnomovision comes with ABSOLUTELY NO WARRANTY; for details type `show w'.
    This is free software, and you are welcome to redistribute it
    under certain conditions; type `show c' for details.

The hypothetical commands `show w' and `show c' should show the appropriate
parts of the General Public License.  Of course, the commands you use may
be called something other than `show w' and `show c'; they could even be
mouse-clicks or menu items--whatever suits your program.

You should also get your employer (if you work as a programmer) or your
school, if any, to sign a "copyright disclaimer" for the program, if
necessary.  Here is a sample; alter the names:

  Yoyodyne, Inc., hereby disclaims all copyright interest in the program
  `Gnomovision' (which makes passes at compilers) written by James Hacker.

  <signature of Ty Coon>, 1 April 1989
  Ty Coon, President of Vice

This General Public License does not permit incorporating your program into
proprietary programs.  If your program is a subroutine library, you may
consider it more useful to permit linking proprietary applications with the
library.  If this is what you want to do, use the GNU Library General
Public License instead of this License.
 \end{verbatim}
 \end{fontsize}
\end{multicols}

\begin{center}
ソースコードについて
\end{center}

このプログラムは tex で記述されたものです。ソースコードは
\begin{center}
  \url{git://anonscm.debian.org/tokyodebian/monthly-report.git}
\end{center}
から取得できます。

\begin{center}
Debian オープンユーズロゴ ライセンス
\end{center}

\begin{multicols}{2}
 \begin{fontsize}{6}{6}
 \begin{verbatim}

Copyright (c) 1999 Software in the Public Interest
Permission is hereby granted, free of charge, to any person
obtaining a copy of this software and associated documentation
files (the "Software"), to deal in the Software without restriction,
including without limitation the rights to use, copy, modify, merge,
publish, distribute, sublicense, and/or sell copies of the Software,
and to permit persons to whom the Software is furnished to do so,
subject to the following conditions:

The above copyright notice and this permission notice shall be
included in all copies or substantial portions of the Software.

THE SOFTWARE IS PROVIDED "AS IS", WITHOUT WARRANTY OF ANY
KIND, EXPRESS OR IMPLIED, INCLUDING BUT NOT LIMITED TO THE
WARRANTIES OF MERCHANTABILITY, FITNESS FOR A PARTICULAR PURPOSE AND
NONINFRINGEMENT. IN NO EVENT SHALL THE AUTHORS OR COPYRIGHT HOLDERS
BE LIABLE FOR ANY CLAIM, DAMAGES OR OTHER LIABILITY, WHETHER IN
AN ACTION OF CONTRACT, TORT OR OTHERWISE, ARISING FROM, OUT OF OR
IN CONNECTION WITH THE SOFTWARE OR THE USE OR OTHER DEALINGS IN
THE SOFTWARE.
 \end{verbatim}
 \end{fontsize}
\end{multicols}

\printindex
%\cleartooddpage

 \begin{minipage}[b]{0.2\hsize}
  \rotatebox{90}{\fontsize{80}{80} {\gt 関西 Debian 勉強会} }
 \end{minipage}
 \begin{minipage}[b]{0.8\hsize}

 \vspace*{15cm}
 \rule{\hsize}{1mm}
 \vspace{2mm}
 \includegraphics[width=2cm]{image200502/openlogo-nd.eps}
 \noindent \Large \bfseries{Debian 勉強会資料}\\ \\
 \noindent \normalfont \debmtgyear{}年\debmtgmonth{}月\debmtgdate{}日 \hspace{5mm}  初版第1刷発行\\
 \noindent \normalfont 関西 Debian 勉強会 (編集・印刷・発行)\\
 \rule{\hsize}{1mm}
 \end{minipage}

\end{document}
