\documentclass[cjk,dvipdfmx,12pt,%
hyperref={bookmarks=true,bookmarksnumbered=true,bookmarksopen=false,%
colorlinks=false,%
pdftitle={第38回関西Debian勉強会},%
pdfauthor={倉敷・のがた・佐々木},%
%pdfinstitute={関西Debian勉強会},%
pdfsubject={資料},%
}]{beamer}

\title{第38回関西Debian勉強会}
\subtitle{{\scriptsize 資料}}
\author[佐々木 洋平]{{\large\bf 倉敷・のがた・佐々木}}
\institute[Debian JP]{{\normalsize\tt 関西Debian勉強会}}
\date{{\small 2010 年 8 月 22 日}}

%\usepackage{amsmath}
%\usepackage{amssymb}
\usepackage{graphicx}
\usepackage{moreverb}
\usepackage[varg]{txfonts}
\AtBeginDvi{\special{pdf:tounicode EUC-UCS2}}
\usetheme{Kyoto}
\def\museincludegraphics{%
  \begingroup
  \catcode`\|=0
  \catcode`\\=12
  \catcode`\#=12
  \includegraphics[width=0.9\textwidth]}
%\renewcommand{\familydefault}{\sfdefault}
%\renewcommand{\kanjifamilydefault}{\sfdefault}
\begin{document}
\settitleslide
\begin{frame}
\titlepage
\end{frame}
\setdefaultslide


\begin{frame}[fragile]
\frametitle{Agenda}

\tableofcontents

\end{frame}

\section{最近の Debian 関係のイベント}


\takahashi[20]{最近のDebian関係のイベント}
\takahashi[80]{(1)}
\takahashi[50]{DebConf10}


\begin{frame}[fragile]
\frametitle{DebConf10}

\begin{itemize}
\item 08/01〜07@ニューヨーク
\item やまねさんや岩松さんが参加

\begin{itemize}
\item そのうち東京エリア勉強会で参加報告がある(と思う)
\end{itemize}
\end{itemize}

\end{frame}


\takahashi[80]{(2)}
\takahashi[20]{Debian 6.0 ``squeeze'' frozen!!!!}


\begin{frame}[fragile]
\frametitle{Debian 6.0 ``squeeze'' frozen!!!!}

\begin{itemize}
\item 08/06 にリースチームより宣言アリ

\begin{itemize}
\item これにより squeeze は frozen へ
\end{itemize}
\item リリースへ向けて

\begin{itemize}
\item RC バグ潰し
\item マニュアルの翻訳/更新
\vspace{2em}
 
\begin{center}
みなさん頑張りましょうね。
\end{center}
\end{itemize}
\end{itemize}

\end{frame}


\takahashi[80]{(3)}
\takahashi[20]{前回の関西Debian勉強会}


\begin{frame}[fragile]
\frametitle{OSC 2010 Kansai@Kyoto}


\begin{itemize}
\item 「野良ビルドから始めるDebianパッケージ作成」

\begin{itemize}
\item 周知不徹底. ハンズオンとしては要改善
\item 講義自体は楽しんで貰えました\dots{}でしょうか?
\end{itemize}
\item 「GPG キーサインパーティ」

\begin{itemize}
\item 好評だった(?)
\item 今後も実施するため, 水面下で動きがある
\end{itemize}
\item ブース

\begin{itemize}
\item Debian 機材展示
\item 「あんどきゅめんてっどでびあん」と T シャツ物販
\end{itemize}
\end{itemize}

\end{frame}

\begin{frame}[fragile]
\frametitle{OSC 2010 Kansai@Kyoto(2)}

\begin{figure}[h]
\centering\museincludegraphics{./image201008/osckyoto1.jpg}|endgroup
\end{figure}

\end{frame}

\begin{frame}[fragile]
\frametitle{OSC 2010 Kansai@Kyoto(3)}

\begin{figure}[h]
\centering\museincludegraphics{./image201008/osckyoto2.jpg}|endgroup
\end{figure}

\end{frame}


\takahashi[80]{(4)}
\takahashi[20]{前々回の関西Debian勉強会}


\begin{frame}[fragile]
\frametitle{第36回関西 Debian 勉強会}


\begin{itemize}
\item OSC直前 06/27@福島区民会館

\begin{itemize}
\item 久しぶりに ustream 中継
\end{itemize}

\item 「debhelper7 と cdbs の深追い」 by 佐々木さん

\begin{itemize}
\item タイトルに偽りアリ: 深追い→初歩
\item 応用編をリベンジ予定(?)
\end{itemize}

\item 「puppet に \textbackslash{}\$HOME を整理させよう」by 倉敷さん

\begin{itemize}
\item ハンズオンの予定が時間切れ
\item またいずれ, 活用編をやる予定(?)
\end{itemize}
\end{itemize}


\end{frame}


\takahashi[20]{...}
\takahashi[20]{時系列がぐちゃぐちゃな気もするが}
\takahashi[40]{気にしたら負け}
\takahashi[40]{そんな\\こんなで}
\takahashi[80]{次}



\section{事前課題発表}

\takahashi[40]{事前課題}
\takahashi[80]{兼}
\takahashi[40]{自己紹介}

\begin{frame}[fragile]
\frametitle{今回の事前課題}


\begin{block}{以下の三つでした}
    \begin{enumerate}
          \item
        Debian が動作する CPU、ターゲット機器について予習をしておいて下さい。
          \item
        それらのうち、
        これまでに使用したことがあるアーキテクチャを教えてください。
        \begin{itemize}
        \item (例:i386、amd64、arm、powerpcなど)。
        \end{itemize}
          \item
        組込み機器の OS として
        Debian を使う場合のメリット/デメリットについて、
        思う所を書いてください。
    \end{enumerate}
\end{block}


\end{frame}


\takahashi[40]{事前課題\\発表}

\begin{frame}[fragile]
\frametitle{ IPv6waterstar }

\begin{verbatim}
個人的には、arm以外のアーキテクチャでDebianをインストールもしくは使用したことがあります。一度、動いてしまえばどのアーキテクチャでもDeianである以上「不便」はありませんでした。
組込み機器のOSとしてDebianを使ってもいいと思うのですが、OSの基本的なレスポンスとして「遅い」のではないかと思います。
\end{verbatim}
\end{frame}



\begin{frame}[fragile]
\frametitle{ 佐藤誠 }


\begin{verbatimtab}
1. http://www.debian.org/releases/stable/
   見てみました。

2, i386, amd64, powerpc

3. (たぶん)環境を早く整備できる。(ひょっとすると)痒いところに気づきにくいかもしれない。
\end{verbatimtab}


\end{frame}



\begin{frame}[fragile]
\frametitle{ yabuki@netfort.gr.jp }



\begin{verbatimtab}
(1) 予習ということは、ハードを買ってハックせよと? ;-)

(2) i386,amd64,arm は所持している。さわったことがあるとか、sshしたとなるともうちょっと増えるけど。あとqemuでも作れますよね。環境は。

(3) メリット:Debianで得た有形無形の資産が、組込み機器でもつかえる。デメリット:組み込み機器は、Debianを使うことが目的ではなく、アプリケーションの動作をさせることが目的なので、Debianを使うことによるオーバヘッドが問題になるかどうか検討が必要だ。
\end{verbatimtab}


\end{frame}



\begin{frame}[fragile]
\frametitle{ 古川竜雄 (frkwtto@gmail.com) }



\begin{verbatimtab}
1. これからします

2. i486とi686はあります

3. メリットはオープンソースとパッケージの多さ、ユーザーが多いことによる十分な情報。デメリットは商用利用? (知識がないので間違いかもしれません)
\end{verbatimtab}


\end{frame}



\begin{frame}[fragile]
\frametitle{ 山田 洋平 }



\begin{verbatimtab}
2. Debian では i386 と amd64 だけですね。

3. ROM の関係で、apt-get が実機で使えないことと最小インストールに必要な容量が大きいことが解決できれば....
\end{verbatimtab}


\end{frame}



\begin{frame}[fragile]
\frametitle{ dictoss(杉本 典充) }


{\small
\begin{verbatimtab}

1.予習しておきます。

2. i386, amd64, ppc (玄箱 HG), kfreebsd-i386,
   kfreebsd-amd64

3.メリットは、開発するものをドライバ・専用アプリに絞り込めるため開発量を減らせること、パッケージ形式になっているため不必要なファイルはアンインストールが容易であり小さいrootfsが作りやすいこと。
  デメリットはLinuxゆえにLinuxが動くハード性能が要求されること、開発したプログラムもDebianパッケージ化するはずなのでDebianパッケージについて知っている人にビルドの仕事が集中しそうなこと。
\end{verbatimtab}
}

\end{frame}


\begin{frame}[fragile]
\frametitle{ 八津尾 雄介 }



\begin{verbatimtab}
2, i386, amd64, powerpc

3,【メリット】
 アプリケーション・ミドルウェアが豊富,オープンソース
 【デメリット】 
 起動・終了が遅い, リソースの大量消費
\end{verbatimtab}

\end{frame}



\begin{frame}[fragile]
\frametitle{ SKINO }



\begin{verbatimtab}
2. i386、Alpha、UltraSparc(microSparc)
  その他、MIPSは悪戦苦闘中
3. 経験はございませんが、素人感覚的に
  メリット:
  ・サポートデバイスが豊富
  デメリット:
  ・パッケージの縛り
  ・カスタマイズは大変そう
\end{verbatimtab}


\end{frame}



\begin{frame}[fragile]
\frametitle{ かわだてつたろう }



\begin{verbatimtab}
2. i386 と amd64 と sh4

3. 多くのパッケージ, セキュリティサポート, パッケージ管理システムが使用
 できることはメリットだとおもう。
\end{verbatimtab}


\end{frame}



\begin{frame}[fragile]
\frametitle{ のがたじゅん }



\begin{verbatimtab}
1. はい。

2. 実際に使ったことがあるのはi386とamd64だけです。Digoo A320というゲーム機やNano NoteがJZ4740というmipsアーキテキクチャマシンなので、いじりたいなと思いつつそのまま放置になっています。

3. メリットは使い慣れたDebianの環境で使えることでしょうか。デメリットは、それを感じるぐらいまで使ったことがないのでよくわかりません。
\end{verbatimtab}


\end{frame}



\begin{frame}[fragile]
\frametitle{ 山下康成 }


{\small
\begin{verbatimtab}
1. (予習って何をすればいいのでしょう)

2. arm, armel

3. 広いアーキテクチャをサポートしていて、第三世代
LinkStation では Debian GNU/Linux 以外の選択肢はありませんでした。カーネルさえ動けばあとは Debian の豊富なパッケージがそのまま(ちょっと言いすぎ)利用させていただけるので、苦労なく Debian 化させていただくことが可能でした。
 組込みでは X を使うことが少ないのに X 関連パッケージに依存するパッケージが多く、実質不要なパッケージまでインストールしなければならないのがデメリットでしょうか。
\end{verbatimtab}
}

\end{frame}



\begin{frame}[fragile]
\frametitle{ ”まさ”こと”甲斐正三”です(1) }


{\small
\begin{verbatimtab}
1.
  (1) Debian正式サポートCPU:
    [alpha][amd64][arm][armel][hppa][i386][ia64]
    [mips][mipsel][powerpc][sparc]
  (2) Debianがサポートする訳ではないが、
    Debianの走るCPU:
    [m68k][SH3/4]
  (3) Debianがサポートするターゲット機器
    わかりません。
\end{verbatimtab}
}


\end{frame}
\begin{frame}[fragile]
\frametitle{ ”まさ”こと”甲斐正三”です(2) }


{\small
\begin{verbatimtab}

2.
    [armel]
    armadillo9、ワンボード。 debianです。
    [i386]
    msi,intelマザーボードの手作りPC
    [powerpc]
    iMac(ボンダイブルー)'linux for ppc'です。その当時DebianにPPCがあることを知りませんでした。
    [SH3]
    T-SH7706LAN rev.2.0 SH3 w/LAN/SD ボード、Debianではありません。

\end{verbatimtab}
}


\end{frame}
\begin{frame}[fragile]
\frametitle{ ”まさ”こと”甲斐正三”です(3) }


{\small
\begin{verbatimtab}

3.
    組込みOSを云々できるほど経験がありませんが、
    希望としては、
      ・開発環境の構築が簡単にできる。(DebianでサポートされていないCPUであっても)
      ・きめ細かいインストールが可能。
      ・ブートも含めて組込みに関するドキュメントが豊富。
    などです。
以上。
\end{verbatimtab}
}


\end{frame}

\begin{frame}[fragile]
\frametitle{  清野陽一 }



\begin{verbatimtab}
1.はーい

2.i386,amd64,ppc,arm

3.余計なパッケージを入れずに済むこと…かなぁ…。
\end{verbatimtab}


\end{frame}

\begin{frame}[fragile]
\frametitle{ lurdan }


{\small
\begin{verbatimtab}
2. i386 (常用), amd64 (常用),  arm (Zaurus), sh4 (リビルドごっこ), powerpc (玄箱、初代TeraStation), alpha (入れただけ), kfreebsd-amd64 (ビルドテスト用)

3. メリット:組み込み Linux (のディストリビューション) としては事例が豊富っぽいのと、ユニバーサルオペレーティングシステムとしての debian 自体が eglibc や emdebian などで組み込みを意識しているあたり、後は最低限動作に必要な構成を作るのが超簡単とか。組み込み向け skkserv もあるよ!
   デメリット:日本の組み込み業界的には、メーカーからポンとドキュメントもらえた方が嬉しそう。
\end{verbatimtab}
}

\end{frame}

\begin{frame}[fragile]
\frametitle{ 坂本 敏久(サカモト トシヒサ) }



\begin{verbatimtab}
はじめまして、坂本といいます。
2週間前にDebianを使いはじめた初心者です。
(ちなみにLinuxもほぼ初心者です。)
よろしく、お願いします。

今回の課題は、無回答でお願いします。
\end{verbatimtab}

\end{frame}


\takahashi[40]{そんな\\こんなで}
\takahashi[80]{次}
\takahashi[50]{御題1}



\section{Debian GNU/kFreeBSDで暮らせる環境を構築してみる。}


\takahashi[20]{Debian GNU/kFreeBSDで暮らせる環境を構築してみる。\\ \\杉本 典充 }
\takahashi[80]{次}
\takahashi[50]{御題2}



\section{Emdebian について -関西 Debian 勉強会参加者中間報告-}


\takahashi[20]{Emdebian について\\ {\normalsize{-関西 Debian 勉強会参加者中間報告-}} \\ \\たなかとしひさ}
\takahashi[80]{次}
\takahashi[50]{今後の予定}


\begin{frame}[fragile]
\frametitle{今後の予定(1)}


\begin{block}{次回の関西Debian勉強会}
\begin{itemize}
\item 9月26日(日)
\item (財)京都高度技術研究所 (ASTEM) \\ 8F コミュニケーションルーム
   \begin{itemize}
    \item いつもと会場が違うので注意!
    \item ネタは未定.
    \item ネットワークが使えるので、いろいろ画策中
   \end{itemize}
\end{itemize}
\end{block}


\end{frame}

\begin{frame}[fragile]
\frametitle{今後の予定(2)}


\begin{block}{関西オープンソース2010 + 関西コミュニティ大決戦}
  \begin{itemize}
    \item 11月5日(金)$\sim$6日(土)
    \item 大阪南港ATC
       \begin{itemize}
         \item 関西Debian勉強として参加予定
         \item ksp-ja も参加するんじゃないかと思います(by 佐々木)
       \end{itemize}
  \end{itemize}
\end{block}


\end{frame}


\takahashi[50]{ }


\end{document}
%%% Local Variables: 
%%% mode: japanese-latex
%%% TeX-master: t
%%% End: 
