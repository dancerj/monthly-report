%; whizzy-master ../debianmeetingresume201003.tex
% 以上の設定をしているため、このファイルで M-x whizzytex すると、whizzytexが利用できます。

\begin{prework}{ 日比野 啓 }

好きな日本語manページは accept(2) です。
Linusのsocklen\_tとBSDソケット層についてのコメントが引用されているのが興味深いです。

\end{prework}


\begin{prework}{ 岩松 信洋 }

1。 好きな日本語Manページ
日本語だと情報古いからみんな嫌いよ!
2。 ニューラルネットワークで解決できる問題

\end{prework}

\begin{prework}{ 吉野 }

\preworksection{好きな日本語Manページ}

sed(1)の一番最後でしょうか.関連項目の
regex(5) [うーん、書かないとダメですねえ] 
がいつも気になります.ただ,原文では既に削除されています.

\end{prework}



\begin{prework}{ 本庄 }

\preworksection{好きな日本語Manページ}

ぼくはman headが大好きです。特にありません。


\preworksection{ニューラルネットワークで解決できる問題}

イアン・エアーズ著『その数学が戦略を決める』という本にニューラル
ネットを活用した事例がいくつか載っていました。ひとつ引用します。

\begin{quotation}
たとえばアリゾナ大学の研究者たちは、ツーソン・グレイハウンド
公園でのグレイハウンド・ドッグレースで勝ち犬を予測するニュー
ラル・ネットワークを構築した。 (中略) 最高の予想屋でも60ドル
d すっていたが、ニューラル・ネットワークは125ドル儲けた。
\end{quotation}

お金が儲かります。


\end{prework}



\begin{prework}{ henrich }

1. 好きなページは特になく、原文との差に心を痛めています。結構手作業が発生しそうで、とても何かの合間にやるというのも現実的ではない様子。スポンサーがいて作業するならいいんだけど…

\end{prework}

\begin{prework}{ akedon }

\preworksection{好きな日本語Manページ}
 ED,マニュアルページの DESCRIPTION がチュートリアルっぽい点が気に入っています。昔、emacs どころか vi も無かった頃のエディタとして基本でした。
 TCPDUMP, IPTABLES のマニュアルページもネットワークの勉強ではお世話になりました。(なっています?)


\end{prework}



\begin{prework}{ mkouhei }
\begin{itemize}
 
 \item  考えたことはとくにありませんでしたが、あえて挙げるならbashですね。3172行って、どんだけ膨大なんだか。

 \item  家計(特に光熱費)の分析に使ってみて、取り合えずどんなものかをまず見てみようと思います。
\end{itemize}

\end{prework}



\begin{prework}{ yasuhiro }

\begin{itemize}
 \item  なんでしょうね..まずはlsですかね.あんな豊富なのをよく書いて訳すよなあ,というかんじ.つぎはman 7 ipv6っですかね.

 \item  ニューラルネットワークっていちおうなんでもできるんじゃないのすかね.
\end{itemize}

\end{prework}

\begin{prework}{ koedoyoshida }

\preworksection{好きな日本語Manページ}

openssh-jman
\url{http://www.unixuser.org/~euske/doc/openssh/jman/}

基本的にDebianはサーバ使用なのでsshのマニュアルはよく見ます。
Debianには日本語マニュアルはパッケージになっていないようなので、ローカルにインストールしたり、Webページを見ることが多いです。
過去にdebパッケージも作っていたのですが、結構頻繁に更新されるので結局上記になりました。


\end{prework}



\begin{prework}{ なかお }

好きな日本語manページ:
section 7 url

\end{prework}



\begin{prework}{ 上川純一 }

好きな日本語のManページはありません。
英語版を更新しても日本語版が更新されないという仕組みがよくない気がしています。


\end{prework}
