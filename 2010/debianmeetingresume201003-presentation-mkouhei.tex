%; whizzy paragraph -pdf xpdf -latex ./whizzypdfptex.sh
%; whizzy-paragraph "^\\\\begin{frame}"
% latex beamer presentation.
% platex, latex-beamer でコンパイルすることを想定。 

%     Tokyo Debian Meeting resources
%     Copyright (C) 2009 Junichi Uekawa
%     Copyright (C) 2010 Nobuhiro Iwamatsu

%     This program is free software; you can redistribute it and/or modify
%     it under the terms of the GNU General Public License as published by
%     the Free Software Foundation; either version 2 of the License, or
%     (at your option) any later version.

%     This program is distributed in the hope that it will be useful,
%     but WITHOUT ANY WARRANTY; without even the implied warranty of
%     MERCHANTABILITY or FITNESS FOR A PARTICULAR PURPOSE.  See the
%     GNU General Public License for more details.

%     You should have received a copy of the GNU General Public License
%     along with this program; if not, write to the Free Software
%     Foundation, Inc., 51 Franklin St, Fifth Floor, Boston, MA  02110-1301 USA

\documentclass[cjk,dvipdfmx,12pt]{beamer}
\usetheme{Tokyo}
\usepackage{monthlypresentation}
%  preview (shell-command (concat "evince " (replace-regexp-in-string "tex$" "pdf"(buffer-file-name)) "&"))
%  presentation (shell-command (concat "xpdf -fullscreen " (replace-regexp-in-string "tex$" "pdf"(buffer-file-name)) "&"))
%  presentation (shell-command (concat "evince " (replace-regexp-in-string "tex$" "pdf"(buffer-file-name)) "&"))

%http://www.naney.org/diki/dk/hyperref.html
%日本語EUC系環境の時
\AtBeginDvi{\special{pdf:tounicode EUC-UCS2}}
%シフトJIS系環境の時
%\AtBeginDvi{\special{pdf:tounicode 90ms-RKSJ-UCS2}}

\title{Weka を使ってみた}
\subtitle{}
\author{前田 耕平 mkouhei@debian.or.jp\\IRC nick: mkouhei}
\date{2010年3月20日}
\logo{\includegraphics[width=8cm]{image200607/openlogo-light.eps}}

\begin{document}

\frame{\titlepage{}}

\begin{frame}
 \frametitle{はじめに}

  \begin{itemize}
   \item ニューラルネットワークって何それ、おいしい?
   \item データマイニングって聞いたことはあるけど。
   \item 統計学とは違うの?
  \end{itemize}
という人でも取り合えず使ってみよう、という内容です。

理論は後から、気が向いたら勉強しませう。

というか本庄さんのセッションにリダイレクト。
\end{frame}

\section{}
\begin{frame}
 \frametitle{Wekaとは}
   \begin{itemize}
   \item Waikato Environment for Knowledge Analysisの略。
   \item Waikato 大学というニュージーランドの国立大で開発。
   \item GPL で公開されているフリーソフトウェア。
   \item いわゆるデータマイニングツールらしい。
  \end{itemize}
 \end{frame}

\begin{frame}[containsverbatim]{インストールしてみる}
\begin{commandline}
$ sudo apt-get install weka
\end{commandline}
\end{frame}

\begin{frame}[containsverbatim]{起動してみる}
\begin{commandline}
$ weka &
\end{commandline}
\end{frame}

\begin{frame}[containsverbatim]{Wekaで扱えるデータ}
\begin{commandline}
@relation 名前
@attribute 属性名 属性の型
@attribute 属性名 属性の型
:
:
@data
データ,データ,…,データ
\end{commandline}
@dataの次の行からはCSVな。
\end{frame}

\emtext{デモる。}

\end{document}

;;; Local Variables: ***
;;; outline-regexp: "\\([ 	]*\\\\\\(documentstyle\\|documentclass\\|emtext\\|section\\|begin{frame}\\)\\*?[ 	]*[[{]\\|[]+\\)" ***
;;; End: ***
