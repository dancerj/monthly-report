\begin{prework}{ matsuu }

やったこと・おきたこと
\begin{itemize}
 \item Debian勉強会に参加するようになった。
 \item VPSでDebianを使い始めた。
 \item Debianのパッケージを参考にGentooパッケージを作った。
\end{itemize}
予想・やりたいこと
\begin{itemize}
 \item DebianとGentooは徐々に衰退していき、ALL YOUR DISTRIBUTION ARE
       BELONG TO CHROMEOS.
\end{itemize}
\end{prework}

\begin{prework}{ あらきやすひろ }

予想
\begin{itemize}
 \item 無事リリースされることによるDebian派生ディストロの大崩壊と回帰。
\end{itemize}
やりたいこと
\begin{itemize}
 \item Debianで仕事!
\end{itemize}
\end{prework}

\begin{prework}{ koedoyoshida }

やったこと
\begin{itemize}
 \item ddtssでレビュー(最近お休み中)
 \item 関西KOFでの関西Debianブースのお留守番
 \item 夏冬のイベントでの書籍の頒布
\end{itemize}
おきたこと
 \begin{itemize}
  \item Squeeze freeze
\end{itemize}
予想
\begin{itemize}
 \item  Squeeze release?
\end{itemize}
やりたいこと
\begin{itemize}
 \item 予定は未定
\end{itemize}
\end{prework}

\begin{prework}{ キタハラ }

やったこと
\begin{itemize}
\item 今年は一番何もしていない年ではないだろうか?ネットサーフィン用寝床PCも死んだままだし。
\end{itemize}
やりたいこと
\begin{itemize}
\item 来年はこれを復活し、あと会社にDebianマシンを復活させたいですね。
\end{itemize}
\end{prework}

\begin{prework}{ yamamoto }

やったこと
\begin{itemize}
 \item なんかポチポチと勝手に ppc64 ポートを、マズいラーメン屋の頑固オヤ
       ジのごとく、細く長く続けていた。
\end{itemize}
おきたこと
\begin{itemize}
 \item debian-ports に、sparc64 やら powerpcspe やら armhf やらがいきな
       り現れて、いっきに抜かれて行った。
\end{itemize}
予想
\begin{itemize}
 \item 次は arm64 かな?
\end{itemize}
やりたいこと
\begin{itemize}
 \item 頑固オヤジの迷惑なラーメンを、ひたすら生み出していく。
 \item arm64 ポートに参加できるといいな。
\end{itemize}
\end{prework}

\begin{prework}{ 本庄 }

やったこと
\begin{itemize}
\item 温泉行きました。
\end{itemize}
やりたいこと
\begin{itemize}
\item また行きたいです。
\end{itemize}
\end{prework}

\begin{prework}{ yyuu }

私事ではありますが、2009 年くらいから仕事で Debian を使えるようになりま
 した。2010 年はその環境の整備に費やしているうちにあっという間に過ぎてし
 まいました。

やったこと
\begin{itemize}
 \item lenny のシステムを 100 台以上の単位で扱うことができた
 \item 自分で個人的な apt のレポジトリを作って運用できるようになった
\end{itemize}
おきたこと
\begin{itemize}
 \item 特に思いつきませんでした...
\end{itemize}

予想
 \begin{itemize}
  \item squeeze がリリースされる (2010年?)
 \end{itemize}
やりたいこと
\begin{itemize}
 \item 一部の既存ホストを lenny から squeeze へ移行したい
 \item 新興プロジェクトの Debian パッケージ化にできるだけコミットしてい
       きたい。今年は Apache Thrift などにパッチを提供できたが、今後はも
       う少し手を広げたい
\end{itemize}
\end{prework}

\begin{prework}{ 野島 貴英 }

やったこと
\begin{itemize}
 \item debian-sidのKVM上でOpensolarisを稼働させた事。
 \item debian-sidの様々なソースコードいっぱい読んだこと。
       \footnote{\url{http://d.hatena.ne.jp/nozzy123nozzy/}}
\end{itemize}
予想
\begin{itemize}
\item いわゆる携帯型tabletPC等へdebian-sidを搭載するHackがそこいら中で起きる。
\end{itemize}
やりたいこと
\begin{itemize}
 \item Contribution \& Hack!Hack!Hack!(ソフトもハードも)
\end{itemize}
\end{prework}

\begin{prework}{ henrich }

やったこと
\begin{itemize}
 \item 5月: NMプロセスを進めてDDになった
 \item 7月: スイスからDebian傘買ってみた :)
 \item 8月: Debconf に参加した
 \item パッケージのアップデートを大体継続できた
 \item その他国内のイベントに参加した
 \item RC バグ潰しに参加した
 \item 多少ではあるが翻訳作業に参加した
 \item 手が付いていないことは以下
       \begin{itemize}
	\item netbeans のパッケージ
	\item eclipse-l10n のパッケージ(ソースを見つけられない…)
	\item Knoppix-Math を Debian ベースに誘う
	\item lenny インストール記事を JP ページに載せる
       \end{itemize}
\end{itemize}
やりたいこと
\begin{itemize}
 \item 開発者リファレンス訳の完了
 \item 初歩のプログラミングができるように(pythonあたり)
 \item debconf で何かしらの成果を出せるように事前準備
 \item l10n をもっと進められるようにもっとステップと成果を明確にしたい
\end{itemize}
\end{prework}

\begin{prework}{ 岩松 信洋 }

やったこと
\begin{itemize}
 \item 組み込み関係のサポート。
 \item SH4のemdebianサポート。
 \item Macbook 関係のサポート。
 \item DDになってからの始めてのリリース作業。
 \item Debconf への参加
\end{itemize}
おきたこと
\begin{itemize}
 \item Non-packaging contributors
 \item backports が正式サービスになった。 
 \item Squeeze フリーズ
 \item Miniconf が多く行われた。
 \item fjp の急逝
\end{itemize}
予想
\begin{itemize}
 \item Debian では SH4 unstable 入り
 \item 4G ネットワークがじわじわと浸透
 \item Android を使ったタブレットデバイス祭り
\end{itemize}
やりたいこと
\begin{itemize}
 \item web 系の開発
 \item 関数型言語の習得
\end{itemize}
\end{prework}

\begin{prework}{ 野首 }

2010年はほとんど何もしませんでした...
さすがにsqueezeは2011年にはリリースされていると思いたい。
\end{prework}

\begin{prework}{ まえだこうへい }

やったこと
\begin{itemize}
 \item Debian勉強会の運営(いつもどおり)
 \item 山田さんのJPへの勧誘・加入(よくやった)
 \item 他の勉強会での勧誘(Webアプリ開発者に交友関係は増えたがDebianへの
       直接的な成果はなし)
\end{itemize}
おきたこと(予想?)
\begin{itemize}
 \item SqueezeにはCouchDB 1.0.1が入らなさそう。
\end{itemize}
やりたいこと
\begin{itemize}
 \item いろいろpendingなものを再開する。
\end{itemize}
\end{prework}

\begin{prework}{ 上川純一 }

やったこと
\begin{itemize}
 \item 子供がうまれた、Debian活動レベルが低下した。
 \item 携帯電話はAndroid携帯電話がメインになって、パソコンをあまり起動し
       なくなった。
 \item 勉強会にくるようなメンバーで、Smart Phone (iPhone or Android)をもっ
       ていない人に出会わなくなってきた。
\end{itemize}
予想
\begin{itemize}
 \item さらなるスマートフォンとタブレットの普及とそれに伴うより自由で便
       利な環境への渇望。
 \item Debianとしてはその環境との相互運用性の向上と、そのデバイス自体で
       動くシステムとしてのすすみかたが二つ考えられる。
\end{itemize}
やりたいこと
\begin{itemize}
 \item 相互運用性は少なくとも向上したいと考えている。
\end{itemize}
\end{prework}

