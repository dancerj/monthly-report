\begin{prework}{ yamamoto }
ext2 一点勝負。
理由は recover の呪文が使えるから。

確認せずによくファイルを消してしまうのよ。
だから inode からデータブロックへの参照をクリアしてしまう fs には移行できない。

最近、テレビの録画ファイルの増加で、やっと lvm を使うようになったけど、フォーマットは相変わらず ext2 です。
\end{prework}

\begin{prework}{ 吉野(yy\_y\_ja\_jp) }
あまり面白設定ではないですが... 手元のラップトップでは /boot には ext2
 その他には ext3,最近使っているデスクトップでは /boot には ext3 その他
 には LVM 上に ext4 を使っています。
\end{prework}

\begin{prework}{ キタハラ }
デフォルトのext3一択。
設定もデフォルトのまま。
(NTFS上のVMイメージ上のext3もあるけど)
\end{prework}

\begin{prework}{高橋斉大}
ext3/4を主に使ってます。ext3のデータサルベージについて日経Linux2011年1月号に執筆しました。
\end{prework}

\begin{prework}{ MATOHARA }
以前inode 枯渇に遭遇してからinode が動的に割り当てられるXFS を選択するこ
 とが多いです.NILFS は少し試してみたのですが,mount 時に
 \ref{sec:try-nilfs}節のWarningメッセージが出てまだ怖いなと思いま
 した。

その他NotePC ではdm-crypt の上にファイルシステムを置いて暗号化したり、
 eCryptfs で暗号化したりしています。書き込み時にCPU をかなり消費します…。
\end{prework}

\begin{prework}{ やまだ }
過去、ext2-$>$reiserfs-$>$jfs-$>$xfs-$>$reiserfs-$>$ext3と使ってきました。

雑感:
\begin{itemize}
 \item reiserfs: 小ファイルや多数ファイルのアクセスがキビキビしてて良かっ
       た!しかしふとした浮気心がその後の七転八倒ロードに・・・
 \item jfs: 当時v1.0を名乗ったIBMアリエナサス!翌月にはxfsに逃避
 \item xfs: fsck==trueに感動。しかし数年使ったもののマシン不調時の0byte
       ファイル問題の嵐で耐えられず逃げた
\end{itemize}
そしてreiser4に超期待するうちにあれがああなって、結局ext3に固定化。htreeも入ったし、もう鉄板なら何でもいいです・・・といいつつnilfsなどに手を出しています。ext3の設定はnoatime程度ですが、これにaufsを被せて自分自身だとか *strap 環境をクローニングしています。実験・テストに便利でお勧めです。USBメモリ稼動でも有用。

後はLVMではなくMDを使って冗長化+バックアップをしています。
 MD(sda,sdb,sdc)で構築し、通常はMD(2台)で稼動。バックアップの時は
 attach/detachをする。ブロックレベルなのでリカバリはFS任せですが、容量が
 増えると他に方法がない・・。
\end{prework}

\begin{prework}{ henrich }
使ってる時間はNTFSが長いんじゃないですかね@職場。
先日ノート用の新しいディスクをext4でフォーマットしましたが全く違いが判らない使い方しかしていません。
\end{prework}

\begin{prework}{ emasaka }
つるしのFSを使ってます
\end{prework}

\begin{prework}{ 本庄 }
ext3を使用しています。特に変わったことはしていません。
FSじゃないですが、最近Lennyで2TBのHDDを使ったらpartedっての使われて驚きました。
\end{prework}

\begin{prework}{ 吉田@小江戸 }
自宅で日常的に活用しているファイルシステムはReiserFSです。
仕事で使ってる環境はext3ですが、ext3は電源断等でジャーナルが
壊れて壊滅した過去の経験(古いカーネルですが...)があり、
あまり信用していません。
自宅のReiserFS環境ではこれまでの所いきなり電源を切ったり、
夏にHDDが壊れかけたりしても被害を被った経験が無いので。
継続的に使っています。
数年前はReiserFSのメイン開発者(Hans Reiser)が逮捕されてしまい、メンテナンスを心配していました。
しかし、その後もReiserFSは他の開発者により継続して保守されているので、安心しました。
\end{prework}

\begin{prework}{ nozzy123nozzy }
\begin{enumerate}
 \item LVMについては、CentOS5.5が導入するものそのままに利用してます。た
       だ、システムが乗っているVolume名はデフォルトから手作業(実際には
       kickstartにて)で変更して使ってます。(障害時のサルベージに困るた
       め)
\item ext3については、debian-sidがそのまま指定しているものをそのまま使っ
      ていたりします。relatime, noatime ぐらいは将来追加してみたいなーと
      は思ってます。
\end{enumerate}
\end{prework}

\begin{prework}{ まえだこうへい }
\begin{itemize}
 \item Debianでは特に凝ったことせず、ext3を使ってます。LVMは仮想マシンでqcow2
       イメージディスク利用時のみ使ってます。
 \item うちで一番特殊なのは、自宅のDHCPサーバ用のArmadillo-Jで使ってい
       るJFFSです。デフォルトのファームウェアではリブートすると設定は全て初期化されてし
       まうので、RAM領域に書きこみ、電源切っても消えない点が便利です。
       Debianのudhcpのソースパッケージからビルドして使ってます
       \footnote{\url{http://d.hatena.ne.jp/mkouhei/20080601/1212330630}}
 \item Debian絡みで、今自分にとって一番ホットなのはpalm webOSです。
       Ubuntuをカスタマイズしたものらしいです。マウントポイントは
       /etc/mtabを見ると35行もある、かなり変態構成です。
\end{itemize}
\end{prework}

\begin{prework}{ 荒木靖宏 }
\subsubsection{日常で活用しているのはext3とxfs}

LVMは使いますね。
典型的なパターンとしては、

\begin{enumerate}
\item vgdisplay、lvdisplay、lvcreateをキメる
\item 
  \begin{itemize} 
    \item KVM用に切りだし(あとは知らん)
    \item iSCSIとして外だし(あとは知らん)
    \item NFS用にmkfs.xfs
    \item ローカル用に意識して使っているものは、ext3かxfs
  \end{itemize}
\end{enumerate}

こんなことをやらない日はないとは言いませんが、やらない週はありません。
やる日だと延々とやっては壊しの繰り返しております。

\begin{itemize}
\item 最近はFUSEを使わなくなりました。isoのloopbackもしなくなりました。
\item Cephはたのしーぜ!って思ってたのですが、ベンチかけてへこんでdisられたので最近は見ていません。
\item POSIXのAPIがないfsはバケツだと思ってるので個人的にはあまり語りたくない。
\item PVFS2は未着手..これもけっこうおもしろげな分散fsと思うんですが。
\item petardfsはマゾ的でいいですよ。
\end{itemize}

NILFS、BtrFS、Cephいずれも発表楽しみにしています。

\end{prework}
