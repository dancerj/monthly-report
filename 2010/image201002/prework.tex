\begin{prework}{ キタハラ }
\begin{enumerate}
\item 創始者の Ian さんとその妻 Debra さんの名前から命名された。 
\item いっぱいありすぎて、回答欄におさまらない。
\item 一番の理由は、自分があまのじゃくだからかなぁ?
\end{enumerate}
\end{prework}
\begin{prework}{ emasaka }
\begin{enumerate}
\setcounter{enumi}{2}
\item とにかくたくさんのパッケージから構成を選んで、あるいは必要に応じて
      パッケージを追加して、なおかつある程度安心して、用途に合ったシステ
      ムを組める点。
\end{enumerate}
\end{prework}
\begin{prework}{ henrich }
\begin{enumerate}
\item disられながらもがんばる健気なプロジェクトです。
\item プロジェクトリーダーの活動!一体何してるの??
\item 「縁があったから」です :)
\end{enumerate}
\end{prework}
\begin{prework}{ mkouhei }
\begin{enumerate}
\setcounter{enumi}{2}
\item 複数のアーキテクチャのマシンを、その違いを気にせずに利用できるパッ
      ケージシステムが便利すぎるためです。使い始めたきっかけでもあります。
      うちで使っているのだけでも5種類です。
\end{enumerate}
\end{prework}
\begin{prework}{ 吉野(yy\_y\_ja\_jp) }
\begin{enumerate}
\setcounter{enumi}{2}
\item DFSGがあるから.\&パッケージングシステムが優れているから.ですね.
\end{enumerate}
\end{prework}
\begin{prework}{ 日比野 }
\begin{enumerate}
\setcounter{enumi}{2}
\item 学生時代に理想のパッケージ管理システムに出会ったと思って以来、ずっと使ってます。
\end{enumerate}
\end{prework}
\begin{prework}{なかおけいすけ}
 \begin{enumerate}
  \setcounter{enumi}{2}
  \item  aptの存在に気づいた時、他のディストロを使う理由が無くなったなぁと
        感じたことを覚えています。
        
         次に、stable releaseが保守的なことで
        す。potatoからdebianを使い始めたのでこういう印象があるのかもしれま
        せんが、長期間使い続けることが前提の用途が多いので、stableは変わら
        ないで欲しいのです。debianのstableは、その点都合が良い。
        
         あと、X-Windowをむやみやたらに入れたがらないところも良いです。最
        近のディストロは、やけにXを入れたがるような印象があります。
        
         最後に、debianが、ある日突然有料になったり、無くなったりしそうに
        ないディストロであるということです。Debian Social Contractがある限
        りdebianはdebianでありつづけるのかなと思っています。
 \end{enumerate}
\end{prework}
