\documentclass[cjk,dvipdfmx,10pt,compress,%
hyperref={bookmarks=true,bookmarksnumbered=true,bookmarksopen=false,%
colorlinks=false,%
pdftitle={第 111 回 関西 Debian 勉強会},%
pdfauthor={倉敷・のがた・佐々木・かわだ},%
%pdfinstitute={関西 Debian 勉強会},%
pdfsubject={資料},%
}]{beamer}

\title{第 111 回 関西 Debian 勉強会}
\subtitle{$\sim$発表資料$\sim$}
\author[かわだ てつたろう]{{\large\bf 倉敷・のがた・佐々木・かわだ}}
\institute[Debian JP]{{\normalsize\tt 関西 Debian 勉強会}}
\date{{\small 2016 年 6 月 26 日}}

%\usepackage{amsmath}
%\usepackage{amssymb}
\usepackage{graphicx}
\usepackage{moreverb}
\usepackage[varg]{txfonts}
\AtBeginDvi{\special{pdf:tounicode EUC-UCS2}}
\usetheme{Kyoto}
\def\museincludegraphics{%
  \begingroup
  \catcode`\|=0
  \catcode`\\=12
  \catcode`\#=12
  \includegraphics[width=0.9\textwidth]}
%\renewcommand{\familydefault}{\sfdefault}
%\renewcommand{\kanjifamilydefault}{\sfdefault}
\begin{document}
\settitleslide
\begin{frame}
\titlepage
\end{frame}
\setdefaultslide

\begin{frame}[fragile]
  \frametitle{Disclaimer}
  \begin{itemize}
  \item 疑問、質問、ツッコミ、茶々、\alert{大歓迎}
  \item その場でインタラクティブにどうぞ
  \item ハッシュタグ \#kansaidebian
  \end{itemize}
\end{frame}

\begin{frame}[fragile]
\frametitle{Agenda}

\tableofcontents

\end{frame}

\section{最近の Debian 関係のイベント}
\takahashi[40]{最近の Debian\\関係のイベント}

\begin{frame}[fragile]
  \frametitle{第110回関西Debian勉強会}
  \begin{itemize}
  \item 日時: 5月28日(土)
  \item 場所: 福島区民センター
  \end{itemize}
\end{frame}

\begin{frame}[fragile]
  \frametitle{OSC 2016 Hokkaido}
  \begin{itemize}
  \item 日時: 6月18日(土)
  \item 場所: 札幌コンベンションセンター
  \end{itemize}
  \begin{block}{内容}
    \begin{itemize}
    \item ブース
    \item セミナー 「Debian updates」
    \end{itemize}
  \end{block}
\end{frame}

\begin{frame}[fragile]
  \frametitle{第140回東京エリアDebian勉強会}
  \begin{itemize}
  \item 日時: 6月25日(土)
  \item 場所: イベント&コミュニティスペース dots.
  \end{itemize}
  \begin{block}{内容}
    \begin{itemize}
    \item 「debexpoをハックするには」
    \end{itemize}
  \end{block}
\end{frame}

\takahashi[50]{そんな\\こんなで}
\takahashi[120]{次}

\section{事前課題}
\takahashi[50]{事前課題}

\begin{frame}[fragile]
  \frametitle{事前課題}
  \begin{block}{今回の事前課題}
    本日の作業予定を教えてください。
  \end{block}
\end{frame}

\takahashi[50]{事前課題\\発表}

\begin{frame}
  \frametitle{ Yukiharu YABUKI }
  \begin{itemize}
  \item ibus-skk NICOLA(jessie)が動かないbugレポートの検証
  \item パッケージングのベストプラクティス
  \end{itemize}
\end{frame}

\begin{frame}
  \frametitle{ t3rkwd }
  \begin{itemize}
  \item PostgreSQL 関連のツール/パッケージを調べてみる
  \item OSC 2016 Kyoto の資料作り
  \item 佐々木さんのお手伝い tdiary-theme のライセンス
  \end{itemize}
\end{frame}

\begin{frame}
  \frametitle{ むんくさん }
  \begin{itemize}
  \item cqrlog
  \item perl スクリプトのパッケージング
  \end{itemize}
\end{frame}

\begin{frame}
  \frametitle{ Tomoyasu Suzuki }
\end{frame}

\begin{frame}
  \frametitle{ Masahiro Yamada }
  \begin{itemize}
  \item stableからsidにdist-upgradeしたら起動しなかった(i386ならOK)
  \item パッケージングのチュートリアルを知りたい
  \item patman をパッケージ化してみたい
  \end{itemize}
\end{frame}

\begin{frame}
  \frametitle{ yy\_y\_ja\_jp }
  \begin{itemize}
  \item manpages-ja
  \item キーサイン
  \end{itemize}
\end{frame}

\begin{frame}
  \frametitle{ Katsuki Kobayashi }
  \begin{itemize}
  \item 昔作ったパッケージを作りなおしてみる
  \item Qiita clone
  \end{itemize}
\end{frame}

\begin{frame}
  \frametitle{ 佐々木洋平 }
  \begin{itemize}
  \item tDiary-5.0.0
  \end{itemize}
\end{frame}

\begin{frame}
  \frametitle{ 古野智士 }
  \begin{itemize}
  \item 社内用に運用系のパッケージを作成しているので公式パッケージに
  \item control.ja
  \item rules
  \item copyright
  \item 他パッケージのファイルを更新
  \end{itemize}
\end{frame}

\begin{frame}
  \frametitle{ 川江 浩 }
\end{frame}

\begin{frame}
  \frametitle{ Yosuke OTSUKI }
  \begin{itemize}
  \item Bugs \#827705 を踏んだ
  \item \#660508 ITA: rc -- an implementation of the AT\&T Plan 9 shell
  \end{itemize}
\end{frame}

\takahashi[50]{そんな\\こんなで}
\takahashi[120]{次}

\takahashi[50]{成果発表}

\begin{frame}
  \frametitle{ Yukiharu YABUKI }
  \begin{itemize}
  \item サポート
  \item ruby-extra以下をclone中
  \item libskk パッケージング中
  \end{itemize}
\end{frame}

\begin{frame}
  \frametitle{ t3rkwd }
  \begin{itemize}
  \item \#828214 rabbit: missing debian theme
  \end{itemize}
\end{frame}

\begin{frame}
  \frametitle{ むんくさん }
  \begin{itemize}
  \item dh-make-perlが通るところまで
  \item fpm2 のITAを進める
  \end{itemize}
\end{frame}

\begin{frame}
  \frametitle{ Tomoyasu Suzuki }
  \begin{itemize}
  \item ぷらっとホーム 株式会社
  \end{itemize}
\end{frame}

\begin{frame}
  \frametitle{ Masahiro Yamada }
  \begin{itemize}
  \item ubootのソースパッケージからpatmanのバイナリパッケージを作ってもらう方向
  \item チュートリアルを進めた
  \end{itemize}
\end{frame}

\begin{frame}
  \frametitle{ yy\_y\_ja\_jp }
  \begin{itemize}
  \item manpages-ja アップストリームの変更点の確認
  \item DDTP
  \item Python のパッケージング
  \end{itemize}
\end{frame}

\begin{frame}
  \frametitle{ Katsuki Kobayashi }
  \begin{itemize}
  \item アップストリームが止まっておりビルドできなかった
  \item 最近のパッケージング方法の取得
  \end{itemize}
\end{frame}

\begin{frame}
  \frametitle{ 佐々木洋平 }
  \begin{itemize}
  \item tdiary-5.0.0パッケージ作成
  \item tdiary-themeのライセンスを修正
  \item tdiary-style-rdのITP
  \end{itemize}
\end{frame}

\begin{frame}
  \frametitle{ 古野智士 }
  \begin{itemize}
  \item control.jaは廃止された
  \item rules更新中
  \item copyrightのフォーマット
  \item パッケージビルド中はネットワーク遮断
  \item debとchefの設定ファイル更新の違いを調べてみる
  \end{itemize}
\end{frame}

\begin{frame}
  \frametitle{ 川江 浩 }
\end{frame}

\begin{frame}
  \frametitle{ Yosuke OTSUKI }
  \begin{itemize}
  \item Bugs \#827705 を踏んだn
  \item \#660508 ITA: rc -- an implementation of the AT\&T Plan 9 shell
  \end{itemize}
\end{frame}


\section{今後の予定}
\begin{frame}[fragile]
  \frametitle{今後の予定}

  \begin{block}{第112回関西Debian勉強会}
    \begin{itemize}
    \item 日時: 7月30日(土)
    \item 場所: KRP
    \item 内容: OSC 2016 Kyoto
    \end{itemize}
  \end{block}

  \begin{block}{第141回東京エリアDebian勉強会}
    \begin{itemize}
    \item 日時: 7月16日(土)
    \item 場所:
    \end{itemize}
  \end{block}

\end{frame}

\takahashi[50]{  }

\end{document}
%%% Local Variables:
%%% mode: japanese-latex
%%% TeX-master: t
%%% End:
