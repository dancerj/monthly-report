
\begin{prework}{ koedoyoshida }

\begin{enumerate}
\item お使いのマシンにARMはありますか?もしあるのでしたら、どのように使っ
ているか教えてください。\\
Raspberry Piを使ってます。
とりあえずリモートアクセス用がメインですが、
I2Cとかを使って外部IOをやってみたいと思っています。
\item  Debian のARM に期待していること、お願いしたいことがあれば教えてくだ
さい。\\
省電力な環境を生かしてReadOnlyboot(でも定期的にセキュリティupdateはし
たい)で放置できる環境とかが作れるとよいですね。
\end{enumerate}

\end{prework}

\begin{prework}{ 吉野 }
\begin{enumerate}
\item  地図マシンです
\item なし 
\end{enumerate}
\end{prework}

\begin{prework}{ dictoss}
\begin{enumerate}
\item  CuBoxを持っている。eSATA端子があるのでファイルサーバのバックアップ
をするマシンとして使おうとしたが、USBポートから電源供給量が不足のため、
HDDが動かず使えていない。そのため単なるarmelバイナリのお試しマシンになっ
ている。

\item  新Nexus7でDebianが動いてほしい。
\end{enumerate}
\end{prework}

\begin{prework}{ mtoshi.g }
\begin{enumerate}
\item  お使いのマシンにARMはありますか?\\
ないっす

\item  Debian のARM に期待していること、お願いしたいことがあれば教えてくだ
さい。\\
今のところないっす
\end{enumerate}
\end{prework}

\begin{prework}{ 野島 貴英 }
\begin{enumerate}
\item  昔売っていたB\&NのNookColorをARM機材実験ボードとして未だに使ってます。
この電子書籍端末は2012/4の東京エリアdebian勉強会,2012/6の大統一debian
勉強会で喋ったとおり、特定ブートフォーマットのSDカードを用意してSDカー
ドスロットへ入れちゃうと、うっかりそのままブートしちゃうという隠れ(?)
機能が大変便利です。おまけに連続8回起動失敗するとリカバリーが始まると
いう非文鎮化機能まで搭載しています。そのままでもpdfリーダ、ポータブル
mp4ビデオ/mp3鑑賞機材として、また、debian ARMのnativeブートの可能性を
秘めたhack機材としても楽しくお使いいただけます。
\item  NookColor用のdebianネイティブブートが可能なイメージ(というかインス
  トーラ)が欲しいといってみるテスト。
\end{enumerate}
\end{prework}

\begin{prework}{ 岩松 信洋 }
\begin{enumerate}
\item  お使いのマシンにARMはありますか?もしあるのでしたら、どのように使っ
ているか教えてください。\\
たくさんある。開発用とかビルドマシンとして使っています。
Raspberry Pi は家のゲートウェイマシンとして動いています。
\item  Debian のARM に期待していること、お願いしたいことがあれば教えてくだ
さい。\\
マルチメディア系が弱いので整備して欲しい。
\end{enumerate}

\end{prework}

\begin{prework}{ 上川純一 }
なし
\end{prework}

\begin{prework}{ まえだこうへい }
\begin{enumerate}
\item  Armadillo Jで自宅内のDHCPサーバとして使っています(not Debian)。
OpenBlockS AX3を、昨年夏ごろにカッとなって作ったioriというツール(最近
話題のdockerみたいなの)の開発環境として使っています。
\item  今は特にないですが、今後ARMサーバの製品版が出てきた時にd-iでインス
トールできることでしょうか。
\end{enumerate}
\end{prework}
