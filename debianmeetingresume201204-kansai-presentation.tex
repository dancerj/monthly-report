\documentclass[cjk,dvipdfmx,10pt,%
hyperref={bookmarks=true,bookmarksnumbered=true,bookmarksopen=false,%
colorlinks=false,%
pdftitle={第 58 回 関西 Debian 勉強会},%
pdfauthor={倉敷・のがた・河田・佐々木},%
%pdfinstitute={関西 Debian 勉強会},%
pdfsubject={資料},%
}]{beamer}

\title{第 58 回 関西 Debian 勉強会}
\subtitle{{\scriptsize 資料}}
\author[佐々木 洋平]{{\large\bf 倉敷・のがた・河田・佐々木}}
\institute[Debian JP]{{\normalsize\tt 関西 Debian 勉強会}}
\date{{\small 2012 年 4 月 22 日}}

%\usepackage{amsmath}
%\usepackage{amssymb}
\usepackage{graphicx}
\usepackage{moreverb}
\usepackage[varg]{txfonts}
\AtBeginDvi{\special{pdf:tounicode EUC-UCS2}}
\usetheme{Kyoto}
\def\museincludegraphics{%
  \begingroup
  \catcode`\|=0
  \catcode`\\=12
  \catcode`\#=12
  \includegraphics[width=0.9\textwidth]}
%\renewcommand{\familydefault}{\sfdefault}
%\renewcommand{\kanjifamilydefault}{\sfdefault}
\begin{document}
\settitleslide
\begin{frame}
\titlepage
\end{frame}
\setdefaultslide

\begin{frame}[fragile]
\frametitle{Agenda}

\tableofcontents

\end{frame}

\section{最近の Debian 関係のイベント}

\takahashi[40]{最近の Debian\\関係のイベント}

\begin{frame}[fragile]
\frametitle{第 57 回関西 Debian 勉強会}

\begin{itemize}
\item 日時: 3 月 25 日
\end{itemize}

\begin{block}{内容}
  \begin{itemize}
  \item 月刊 Debian Policy
  \item 月刊 t-code
  \item 月刊(?) Konoha
  \end{itemize}
\end{block}
ネタ出しは随時行なっております! 皆様よろしく!!
\end{frame}

\begin{frame}[fragile]
  \frametitle{第 87 回 東京エリア Debian 勉強会}
  \begin{itemize}
  \item  日時: 4 月 21 日(昨日!)
  \end{itemize}
  \begin{block}{内容}
    \begin{itemize}
    \item Debian での node.js 入門
    \item Debian on Android
    \item 月刊Debhelper: dh\_md5sums, dh\_strip
    \end{itemize}
  \end{block}
\end{frame}

\takahashi[50]{そんな\\こんなで}
\takahashi[120]{次}

\section{事前課題発表}

\takahashi[50]{事前課題}

\begin{frame}[fragile]
\frametitle{事前課題}

\begin{block}{今回の事前課題}
  \begin{description}
  \item[事前課題1] Debian Policy の第2章を読んで、アーカイブエリアのmain、contrib、non-freeの違いについて説明してください。
  \item[事前課題2] 次のうち、著作権の対象になりそうなものを選んでみよう。
    \\
    パスワード、Linux のカーネルイメージ、Linux カーネルのソースコード、Emacs Lisp の言語仕様、動画の圧縮方式
  \end{description}
\end{block}

\end{frame}

\takahashi[50]{事前課題\\発表}

\begin{frame}
\frametitle{ よしだともひろ }
  \begin{enumerate}
  \item
    \begin{description}
    \item [main] DFSGに準拠していて、コンパイル、実行時にmainの範囲外のパッケージを必要としないパッケージ
    \item [contrib] DFSGに準拠していて、contribまたはnon-freeに属するパッケージをコンパイル、実行時に必要とするパッケージ
    \item [non-free] DFSG に準拠していないパッケージ、または特許やその他の法的問題のために配布に問題のあるパッケージ
    \end{description}
  \item Linux のカーネルイメージ、Linux カーネルのソースコード
  \end{enumerate}
\end{frame}

\begin{frame}
\frametitle{ かわだてつたろう }
  \begin{enumerate}
  \item のちほど
  \item 「Linux のカーネルイメージ」と「Linux カーネルのソースコード」「Emacs Lisp の言語仕様」「動画の圧縮方式」かな。
  \end{enumerate}
\end{frame}

\begin{frame}
\frametitle{ 山城の国の住人 久保博 }
  \begin{enumerate}
  \item 回答1.
    \begin{description}
    \item [non-free]は、DFSG 非準拠だったり、配布に問題のあるパッケージを収める
    \item [contrib]は、DFSG 準拠で main 以外のパッケージやそれ以外のソフトウェアに依存するものを収める。
    \item [main]は、DFSG 準拠で、 main のパッケージへのみ依存するものを収める
    \end{description}
  \item 回答2\\
    会場で披露します
  \end{enumerate}
\end{frame}

\begin{frame}
\frametitle{ yoda }
  (無回答)
\end{frame}

\begin{frame}
\frametitle{ 酒井 忠紀 }
  \begin{enumerate}
  \item
    \begin{description}
    \item [main] 該当パッケージと依存パッケージ共にDFSG に準拠している
    \item [contrib] 該当パッケージは DFSG に準拠しているが、依存パッケージが準拠していない
    \item [non-free] 該当パッケージが DFSG に準拠していない
    \end{description}
  \item 著作権の対象は「思想または感情の創作的な表現」とういうことなので、以下の4つでしょうか?
    \begin{itemize}
    \item Linuxのカーネルイメージ
    \item Linuxカーネルのソースコード
    \item Emacs Lispの言語仕様
    \item 動画の圧縮方式
    \end{itemize}
        「創作的な」が何を含むのか、いまいち理解できていません。
  \end{enumerate}
\end{frame}

\begin{frame}
\frametitle{ yyatsuo }
  \begin{enumerate}
  \item
    \begin{description}
    \item [main] DSFG準拠で依存も含め完全に free なもの
    \item [contrib] DSFG準拠でそれ自体は free だけど non-free に依存するもの
    \item [non-free] DSFGに準拠せず、free ではないもの
    \end{description}
  \item
    \begin{description}
    \item [パスワード] "創作"かどうか微妙。性質上、個人と関連付けるものではない = ただの文字の羅列であると考えると創作ではないので著作権対象外。
    \item [カーネルイメージ] ソースコードは著作物でバイナリはその派生的著作物である、と考えれば著作権の対象。
    \item [ソースコード] (日本では)著作権保護対象
    \item [言語仕様] 仕様は著作権対象外
    \item [圧縮方式] アイディアそのものであれば著作権対象外。別の知財権で保護されるのでは?
    \end{description}
  \end{enumerate}
\end{frame}

\begin{frame}
\frametitle{ のがたじゅん }
  \begin{enumerate}
  \item
    \begin{description}
    \item [main] DFSGに合致したソフトウェアが収められている。
    \item [contrib] DFSGに合致しているが、DFSGに合致していないソフトウェアに依存しているソフトウェア、もしくはnon-freeなソフトウェアのラッパーなどが収められる。
    \item [non-free] DFSGに合致しない、もしくは特許や法的に問題のあるソフトウェアが収められる。
  \end{description}
  \item Linuxカーネルイメージとカーネルのソースコードかな?\\
    パスワードや言語仕様は思想や感情を創作的にあらわしたものではないし、圧縮方式は特許で守られるものだと思う。
  \end{enumerate}
\end{frame}

\begin{frame}
\frametitle{ 佐々木洋平 }
\begin{enumerate}
  \item 課題1の回答
  \begin{description}
    \item[main] DFSG に準拠するソフトウェア
    \item[contrib] それ自体は DFSG に準拠しているが、動作に non-free のソフトウェアが必要なソフトウェア
    \item[non-free] DFSG に準拠しないソフトウェア。
  \end{description}
  \item 課題2の回答
  \begin{description}
    \item 主張するかはともかく、これらのモンがどこからともなく湧くわけはないので、
          著作者は存在するような気がするんですがね.
  \end{description}
\end{enumerate}
\end{frame}

\begin{frame}
\frametitle{ 西山和広 }
  \begin{enumerate}
  \item
    \begin{description}
    \item [main] DFSG に準拠していて main だけで完結出来るパッケージを収録
    \item [contrib] DFSG に準拠しているが main 以外のものに依存しているパッケージを収録
    \item [non-free] DFSG に準拠していないパッケージやその他の配布に問題のあるパッケージを収録
    \end{description}
  \item 直感で選ぶと Linux のカーネルイメージ、Linux カーネルのソースコード だと思いました。
  \end{enumerate}
\end{frame}

\begin{frame}
\frametitle{ lurdan }
  \begin{enumerate}
  \item main には DFSG-Free なソフトウェア\\
non-free には DFSG-Free ではないソフトウェア\\
contrib にはそれ自体が DFSG-Free であっても、動作するために non-free なソフトウェアに依存しているもの\\
がそれぞれ収録されます
\item Linux カーネルイメージ\\
Linux カーネルのソースコード
  \end{enumerate}
\end{frame}

\begin{frame}
\frametitle{清野陽一}
  \begin{enumerate}
  \item
    \begin{description}
    \item [main]はDFSGに準拠していて、コンパイル及び実行時にmain以外のパッケージを必要としないもの。
    \item [contrib]はDFSG準拠であるが、mainと違い、コンパイル及び実行時にmain以外に属しているパッケージに依存していても良い。または非フリーなプログラムのラッパーパッケージなど。例えばgoogleearth-packageなど。
    \item [non-free]はDFSGに準拠していない、もしくは特許や法的な問題のために配布に問題のあるもの。
    \end{description}
    いずれにしてもあまりにバグだらけでサポートが拒否されるようなものは拒絶される。また、mainやcontribはマニュアルのポリシー要求に完全適合、non-freeも可能な限り準拠することが求められる。
  \item Linux カーネルのソースコード
  \end{enumerate}
\end{frame}

\begin{frame}
\frametitle{ 川江 }
  \begin{enumerate}
  \item
    \begin{description}
    \item [main] フリー
    \item [contrib] non-freeに依存する部分がある
    \item [non-free] フリーでない
    \end{description}
  \item パスワード Linuxのカーネルイメージ 動画の圧縮方式
  \end{enumerate}
\end{frame}


\takahashi[50]{そんな\\こんなで}
\takahashi[120]{次}

\section{フリーソフトウェアと戯れるための著作権入門  by 山城国の住人 /\  久保博}
\takahashi[25]{フリーソフトウェアと戯れるための著作権入門 \newline by 山城国の住人 /\  久保博}

\takahashi[50]{そんな\\こんなで}
\takahashi[120]{次}

\section{スクリプティング言語 Konoha の Debian パッケージ化について by 酒井 忠紀}
\takahashi[25]{スクリプティング言語 Konoha の Debian パッケージ化について\\by\\酒井 忠紀}

\takahashi[50]{そんな\\こんなで}
\takahashi[120]{次}

\section{月刊 Debian Policy 第3回 「Debian アーカイブ」by かわだ てつたろう}

\takahashi[25]{月刊 Debian Policy 第3回 「Debian アーカイブ」by かわだ てつたろう}

\takahashi[50]{そんな\\こんなで}
\takahashi[120]{次}

\section{今後の予定}
\begin{frame}[fragile]
\frametitle{今後の予定}

\begin{block}{第 59 回関西 Debian 勉強会}
\begin{itemize}
  \item 日時: 5 月 27 日
  \item 会場: 福島区民センター
  \item 内容: 未定
\end{itemize}
\end{block}

\end{frame}

\begin{frame}[fragile]
\frametitle{今後の予定}

\begin{block}{大統一Debian勉強会}
\begin{itemize}
  \item 日時: 6 月 23 日
  \item 会場: 京都大学理学部3号館 108, 109, 110
  \item 内容: CFP(〆切は今日です) \\
    \url{http://gum.debian.or.jp/}
    \begin{itemize}
    \item 一般参加申し込み/懇親会に関しては, 後日告知します.
    \end{itemize}
\end{itemize}
\end{block}

\end{frame}



\takahashi[50]{  }


\end{document}
%%% Local Variables:
%%% mode: japanese-latex
%%% TeX-master: t
%%% End:
