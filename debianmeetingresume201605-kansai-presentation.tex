\documentclass[cjk,dvipdfmx,10pt,compress,%
hyperref={bookmarks=true,bookmarksnumbered=true,bookmarksopen=false,%
colorlinks=false,%
pdftitle={第 109 回 関西 Debian 勉強会},%
pdfauthor={倉敷・のがた・佐々木・かわだ},%
%pdfinstitute={関西 Debian 勉強会},%
pdfsubject={資料},%
}]{beamer}

\title{第 110 回 関西 Debian 勉強会}
\subtitle{$\sim$発表資料$\sim$}
\author[かわだ てつたろう]{{\large\bf 倉敷・のがた・佐々木・かわだ}}
\institute[Debian JP]{{\normalsize\tt 関西 Debian 勉強会}}
\date{{\small 2016 年 5 月 21 日}}

%\usepackage{amsmath}
%\usepackage{amssymb}
\usepackage{graphicx}
\usepackage{moreverb}
\usepackage[varg]{txfonts}
\AtBeginDvi{\special{pdf:tounicode EUC-UCS2}}
\usetheme{Kyoto}
\def\museincludegraphics{%
  \begingroup
  \catcode`\|=0
  \catcode`\\=12
  \catcode`\#=12
  \includegraphics[width=0.9\textwidth]}
%\renewcommand{\familydefault}{\sfdefault}
%\renewcommand{\kanjifamilydefault}{\sfdefault}
\begin{document}
\settitleslide
\begin{frame}
\titlepage
\end{frame}
\setdefaultslide

\begin{frame}[fragile]
  \frametitle{Disclaimer}
  \begin{itemize}
  \item 疑問、質問、ツッコミ、茶々、\alert{大歓迎}
  \item その場でインタラクティブにどうぞ
  \item ハッシュタグ \#kansaidebian
  \end{itemize}
\end{frame}

\begin{frame}[fragile]
\frametitle{Agenda}

\tableofcontents

\end{frame}

\section{最近の Debian 関係のイベント}
\takahashi[40]{最近の Debian\\関係のイベント}

\begin{frame}[fragile]
  \frametitle{第109回関西Debian勉強会}
  \begin{itemize}
  \item 日時: 4月24日(日)
  \item 場所: 福島区民センター
  \end{itemize}
  \begin{block}{内容}
    \begin{itemize}
    \item 「OpenFOAM で数値流体解析」
    \end{itemize}
  \end{block}
\end{frame}

\begin{frame}[fragile]
  \frametitle{OSC 2016 Gunma}
  \begin{itemize}
  \item 日時: 5月14日(土)
  \item 場所: LABI1 高崎 4F イベントホール
  \end{itemize}
  \begin{block}{内容}
    \begin{itemize}
    \item ブース
    \item セミナー 「Debian updates」
    \end{itemize}
  \end{block}
\end{frame}

\begin{frame}[fragile]
  \frametitle{第139回東京エリアDebian勉強会}
  \begin{itemize}
  \item 日時: 5月21日(土)
  \item 場所: 株式会社朝日ネット
  \end{itemize}
  \begin{block}{内容}
    \begin{itemize}
    \item 「Buffalo Linkstation向けDebian Installer」
    \end{itemize}
  \end{block}
\end{frame}

\begin{frame}[fragile]
  \frametitle{Debian Project}
  \begin{itemize}
  \item Security support for Wheezy handed over to the LTS team
  \item Debian i386 architecture now requires a 686-class processor
  \item Bug\#824057: ITP: bitkeeper -- source code management system
  \item Debian now with ZFS on Linux included
  \end{itemize}
\end{frame}

\takahashi[50]{そんな\\こんなで}
\takahashi[120]{次}

\section{事前課題}
\takahashi[50]{事前課題}

\begin{frame}[fragile]
  \frametitle{事前課題}
  \begin{block}{今回の事前課題}
    事前課題はありませんでした。
  \end{block}
\end{frame}

\takahashi[50]{事前課題\\発表}

\begin{frame}
  \frametitle{ t3rkwd }
\end{frame}

\begin{frame}
  \frametitle{ Yosuke OTSUKI }
\end{frame}

\begin{frame}
  \frametitle{ 川江 浩 }
\end{frame}


\takahashi[50]{そんな\\こんなで}
\takahashi[120]{次}


\section{今後の予定}
\begin{frame}[fragile]
  \frametitle{今後の予定}

  \begin{block}{第111回関西Debian勉強会}
    \begin{itemize}
    \item 日時: 6月25日(土) or 26日(日)
    \item 場所: 福島区民センター
    \item 内容: パッケージング道場
    \end{itemize}
  \end{block}

  \begin{block}{第140回東京エリアDebian勉強会}
    \begin{itemize}
    \item 日時: 6月18日(土)?
    \item 場所:
    \end{itemize}
  \end{block}

\end{frame}

\takahashi[50]{  }

\end{document}
%%% Local Variables:
%%% mode: japanese-latex
%%% TeX-master: t
%%% End:
