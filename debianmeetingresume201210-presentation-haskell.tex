\documentclass[cjk,dvipdfmx,12pt]{beamer}
\usetheme{Tokyo}
\usepackage{monthlypresentation}

\setbeamertemplate{footline}{\hskip1mm\insertshortdate\hfill\hbox{\insertpagenumber /\insertdocumentendpage }\hspace*{1mm}\vskip1mm}

\newcommand<>{\fullsizegraphic}[1]{
  \begin{textblock*}{0cm}(-1cm,-3.78cm)
  \includegraphics[width=\paperwidth]{#1}
  \end{textblock*}
}

\newcommand<>{\minisectiontilte}[2]{
  {\fontsize{40pt}{40pt}\selectfont\color{#1}#2}
}

\AtBeginDvi{\special{pdf:tounicode EUC-UCS2}}

\title{Haskell周辺のDebian packaging}
\subtitle{-- 月刊 DebHelper --}
\author{日比野 啓}
\date{2012年10月20日}

\begin{document}

\frame{\titlepage{}}

\begin{frame}[fragile]
\frametitle{HackageDB}
http://hackage.haskell.org/

\begin{itemize}
\item Haskellのパッケージ群HackageDBの配布元
\item それぞれのパッケージをHackageと言ったりする
\end{itemize}

\end{frame}

\begin{frame}[fragile]
\frametitle{HackageDB}

\begin{commandline}
http://hackage.haskell.org/package/<hackageの名前>
\end{commandline}

たとえば

\begin{commandline}
http://hackage.haskell.org/package/language-objc
\end{commandline}

\begin{itemize}
\item package description
\end{itemize}

\end{frame}

\begin{frame}[fragile]
\frametitle{HackageDB}

\begin{commandline}
Name:          language-objc
Version:       0.4.2.5
...
Library
...
    Build-Depends: base       >= 3 && < 5,
                   filepath   >= 1.1 && < 1.4,
                   process    == 1.1.*,
                   directory  >= 1.1 && < 1.3,
                   array      == 0.4.*,
                   containers >= 0.4     && < 0.6,
                   newtype    == 0.2.*,
                   pretty     == 1.1.*
...
    Build-Tools:    happy, alex
\end{commandline}

\end{frame}

\begin{frame}[fragile]
\frametitle{Hackage - Cabal, cabal-install}

\begin{description}
\item[Cabal] 依存関係にしたがってパッケージをHackageのサイトから取ってきたり、
パッケージのソースツリーと手元の環境からパッケージをbuildしたりする
\item[cabal-install] Cabalライブラリをコマンドラインから呼び出すためのツール
\end{description}

\end{frame}

\begin{frame}[fragile]
\frametitle{Hackage - Cabal, cabal-install}

\begin{commandline}
% cabal unpack language-objc
   ## hackage.haskell.orgからの取得と展開
% cd language-objc-0.4.2.5
% cabal configure
   ## 依存関係の検査
% cabal build
   ## コンパイル
% cabal copy
   ## インストール
% cabal register
   ## インストール管理情報を処理系に登録
\end{commandline}

\end{frame}

\begin{frame}[fragile]
\frametitle{Hackage - Cabal, cabal-install}

\begin{commandline}
% cabal install language-objc
  ## 再帰的にインストールを全て実行
\end{commandline}

\end{frame}

\begin{frame}[fragile]
\frametitle{Hackage - debian, cabal-debian}

\begin{description}
\item[debian] Debianソースパッケージの情報をハンドリングするためのライブラリ
\item[cabal-debian] debianライブラリを使って作られている。HackageのソースをDebianのソースパッケージに変換
\end{description}

\end{frame}

\begin{frame}[fragile]
\frametitle{Hackage - debian, cabal-debian}

\begin{commandline}
% cabal unpack language-objc
% cd language-objc-0.4.2.5
% cabal-debian --debianize
\end{commandline}

\end{frame}

\begin{frame}[fragile]
\frametitle{Hackage - debian, cabal-debian}

\begin{commandline}
% cat debian/control
Source: haskell-language-objc
Priority: extra
Section: haskell
...
Build-Depends: debhelper (>= 7.0),
 haskell-devscripts (>= 0.8),
 cdbs,
 ghc,
 ghc-prof,
 libghc-newtype-dev (>= 0.2)
  | libghc-newtype-dev (<< 0.3),
...
 happy,
 alex
Build-Depends-Indep: ghc-doc,
 libghc-newtype-doc (>= 0.2)
  | libghc-newtype-doc (<< 0.3),
...
\end{commandline}

\end{frame}

\begin{frame}[fragile]

\begin{commandline}
    Build-Depends: base       >= 3 && < 5,
                   filepath   >= 1.1 && < 1.4,
                   process    == 1.1.*,
                   directory  >= 1.1 && < 1.3,
                   array      == 0.4.*,
                   containers >= 0.4     && < 0.6,
                   newtype    == 0.2.*,
                   pretty     == 1.1.*
...
    Build-Tools:    happy, alex
\end{commandline}

\end{frame}

\begin{frame}[fragile]
\frametitle{Hackage - debian, cabal-debian}

\begin{itemize}
\item ライブラリのパッケージだとそのまま利用できる程度のものが自動的に生成される
\item 実行ファイルがあるものについてはrulesにインストール先を書く必要あり
\item \verb|libghc-<hackageの名前>-{dev,prof,doc}| という名前でDebianパッケージが作られる
 \begin{itemize}
 \item haddockというツールでソースコード内の特定の形式のコメントがドキュメントに変換される(libghc-*-doc)
 \end{itemize}
\end{itemize}

\end{frame}

\begin{frame}[fragile]
\frametitle{haskell-devscripts}

cabal-debian で作られたDebianのソースパッケージは
haskell-devscripts に含まれているcdbs用のMakefile (hlibrary.mk)を利用する

\begin{commandline}
% cat debian/rules
#!/usr/bin/make -f
include /usr/share/cdbs/1/rules/debhelper.mk
include /usr/share/cdbs/1/class/hlibrary.mk

# How to install an extra file into ...
#binary-fixup/libghc-language-objc-doc::
#	echo "Some informative text" > ...
\end{commandline}

\end{frame}

\begin{frame}[fragile]
\frametitle{haskell-devscripts}
haskell-devscriptsは他にも dh\_haskell\_* コマンド群を含んでいる
\begin{itemize}
\item 基本的にはGHCのインストール情報管理用のファイル
(debian/*/var/lib/ghc/package.conf.d/*.conf)を利用しつつ、*.substvarsを出力
\end{itemize}

\end{frame}


\begin{frame}[fragile]
\frametitle{haskell-devscripts}

\begin{description}
\item[dh\_haskell\_depends]
Haskellのライブラリの依存関係をDebianの依存関係に変換
\item[dh\_haskell\_extra\_depends]
データのパッケージや実行プログラムのパッケージといった、
ライブラリの依存関係では解決できない依存関係をDebianの依存関係に変換
\item[dh\_haskell\_provides]
HaskellのライブラリのDebian仮想パッケージも含めた提供情報を計算
\item[dh\_haskell\_shlibdeps]
Haskellのライブラリが依存している
ライブラリアーカイブ(*.a)を提供しているパッケージを
検索しDebianの依存関係として出力
\end{description}

\end{frame}

\begin{frame}[fragile]
\frametitle{まとめ}
HackageのDebian化は簡単なので気になるライブラリがあったらやってみましょう
\end{frame}

\begin{frame}[fragile]
\frametitle{Discussion}
{\Huge Discussion}
\end{frame}

\end{document}
