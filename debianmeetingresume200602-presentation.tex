%; whizzy document
% latex beamer presentation.
% platex, latex-beamer でコンパイルすることを想定。 


%     Tokyo Debian Meeting resources
%     Copyright (C) 2006 Junichi Uekawa

%     This program is free software; you can redistribute it and/or modify
%     it under the terms of the GNU General Public License as published by
%     the Free Software Foundation; either version 2 of the License, or
%     (at your option) any later version.

%     This program is distributed in the hope that it will be useful,
%     but WITHOUT ANY WARRANTY; without even the implied warranty of
%     MERCHANTABILITY or FITNESS FOR A PARTICULAR PURPOSE.  See the
%     GNU General Public License for more details.

%     You should have received a copy of the GNU General Public License
%     along with this program; if not, write to the Free Software
%     Foundation, Inc., 51 Franklin St, Fifth Floor, Boston, MA  02110-1301 USA


\documentclass[cjk,dvipdfmx]{beamer}
\usetheme{Warsaw}
%  preview (shell-command (concat "xpdf " (replace-regexp-in-string "tex$" "pdf"(buffer-file-name)) "&"))
%  presentation (shell-command (concat "xpdf -fullscreen " (replace-regexp-in-string "tex$" "pdf"(buffer-file-name)) "&"))

%http://www.naney.org/diki/dk/hyperref.html
%日本語EUC系環境の時
\AtBeginDvi{\special{pdf:tounicode EUC-UCS2}}
%シフトJIS系環境の時
%\AtBeginDvi{\special{pdf:tounicode 90ms-RKSJ-UCS2}}


\title[Debian 勉強会クイズ問題]{Debian勉強会クイズ}
\subtitle{2006年2月18日版}
\author{上川}
\date{2006年2月18日}

% 三択問題用
\newcounter{santakucounter}
\newcommand{\santaku}[5]{%
\addtocounter{santakucounter}{1}
\frame{\frametitle{問題\arabic{santakucounter}. #1}
%問題\arabic{santakucounter}. #1
\begin{itemize}
\item □ A #2\\
\item □ B #3\\
\item □ C #4\\
\end{itemize}
}
\frame{\frametitle{問題\arabic{santakucounter}. #1}
%問題\arabic{santakucounter}. #1
\begin{itemize}
\item □ A #2\\
\item □ B #3\\
\item □ C #4\\
\end{itemize}
\vfill{}
#5
}
}


\begin{document}
\frame{\titlepage{}}

\section{DWNQuiz}
%% debianmeetingresume200602.texから以下コピー
\subsection{2006年4号}
\url{http://www.debian.org/News/weekly/2006/04/}
にある1月24日版です。

\santaku
{Debian のフォーラムをつくろうという提案に関して却下した理由は}
{フォーラムにはフォーラムの主がいついてしまうからダメだ}
{フォーラムは2ch化してしまうから不適切}
{メーリングリストの参加者とフォーラムの参加者は質的に違う}
{C}

\santaku
{1月1日にAnthony Townsが発表したDebianのリリース対象のアーキテクチャは}
{alpha、amd64、hppa、i386、ia64、mips、mipsel、powerpc }
{i386, powerpc, amd64}
{i386, m68k}
{A}

\santaku
{今後の標準でkaffeで利用するjavaコンパイラはどれか}
{jikes}
{gcj}
{ecj}
{C}

\subsection{2006年5号}
\url{http://www.debian.org/News/weekly/2006/05/}
にある1月31日版です。

\santaku
{2006年のDebian Dayが開催されるのはいつか}
{5月13日}
{6月13日}
{7月13日}
{A}


\santaku
{/var/run/ 以下のサブディレクトリをつかう場合はどうするべきか}
{パッケージに含める}
{サブディレクトリは使わない}
{起動時に毎回作成する}
{C}

\santaku
{launchpadをDebianの開発に使おうという提案に対して出た反論は}
{ubuntuの成果なんてつかえない}
{non-freeであるため、それに依存するのはよくない}
{ウェブインタフェースなんて使いたくない}
{B}

\subsection{2006年6号}
\url{http://www.debian.org/News/weekly/2006/06/}
にある2月7日版です。

\santaku
{Extremaduraのハッキングセッションで何ができたか。}
{政治的な思想の熟成}
{D-IのGUI版}
{日焼け}
{B}


\santaku
{2006年のDebian Project Leader選挙に最初に立候補したのは誰か} 
{Branden Robinson}
{Junichi Uekawa}
{Lars Wirzenius}
{C}

\santaku
{FLUGは何か}
{Finland の Linux Users Group}
{Finland の Lost Users Group}
{Finland の Legal Users Group}
{A}

\subsection{2006年7号}
\url{http://www.debian.org/News/weekly/2006/07/}
にある2月14日版です。

\santaku
{iBook G4の無線がどうなったか}
{仕様が公開された}
{あいかわらず動かない}
{動くようになった}
{C}


\santaku
{商標についてDebianはどういう立場をとっているか}
{変更と配布の邪魔にならないようであれば問題ない}
{商標は自由の思想に反するので許せない}
{なにそれ、おいしい?}
{A}

\santaku
{http://lists.debian.org/msgid-search/はどうやって使うのか}
{http://lists.debian.org/msgid-search/メールのメッセージID}
{http://lists.debian.org/msgid-search/送信者の名前}
{http://lists.debian.org/msgid-search/メールのサブジェクト}
{A}




\end{document}