\begin{prework}{やまだ}
\preworksection{今回の BSP への意気込みを熱く?語ってみる}

PrimerとRC-bugリスト読みながら圧倒されつつも、

\begin{itemize}
\item もう「いつ出るのか本当に出るのか」とは言わせない!
\item やっぱり「マダー?」より自分で何かして出る方がいいよね!
\item やるならトコトン集中的にやってメンテナスキル向上もしよう!
\end{itemize}

ということで、10個位直すことを目標に参加します。

実は「初参加がいきなりBSPって大丈夫か?>自分」なので
越えられない壁をいま築いてしまった感が無きにしも非ずですが、
野望達成のためせっせと仕込んでゆきますです・・・

\end{prework}

\begin{prework}{岡部 究}
\preworksection{今回の BSP への意気込みを熱く語ってください}
まずは手持ちのパッケージであるuimのlintianバグを0にします。
そのあと、uimのビルダからのエラー報告の解 癆 呂鬚笋蠅燭い隼廚い泙后?

それでも余力があったらmltermのlintianバグやその他のリリースクリティカル
バグを直そうと思います。
\end{prework}

\begin{prework}{ 日比野 啓 }

groff のバグを取るのに挑戦してみようと思います。

あと関数型まわりで何かあれば協力します。

\end{prework}



\begin{prework}{ 藤沢理聡(risou) }

何をしたら良いのかも全然わかってませんが、この機会を利用して、Debianに少
 しでも貢献できれば、と思っています。よろしくお願いします。

\end{prework}



\begin{prework}{ kmuto }

自分のパッケージのバグ修正、RCに限らず使ってて困るパッケージのdelay
 upload、ほかの人の作業のスポンサー、といったあたりを。
時期的にちょっと読めないのでキャンセルするかもしれません。あと、参加して
 も途中退席になると思います。


\end{prework}



\begin{prework}{ henrich }

手持ちの作業を少しでも減らしておこうと思います。

\end{prework}



\begin{prework}{ iwamatsu }
RCとQAを中心にバグを潰す予定です。


\end{prework}



\begin{prework}{ 上川純一 }

ガンガンバグをつぶす予定です。
あと、Debian勉強会予約システムについて情報共有したいです。

\end{prework}
