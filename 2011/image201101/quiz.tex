%; whizzy-master ../debianmeetingresume201101.tex
% 以上の設定をしているため、このファイルで M-x whizzytex すると、whizzytexが利用できます。
%
% ちなみに、クイズは別ブランチで作成し、のちにマージします。逆にマージし
% ないようにしましょう。
% (shell-command "git checkout quiz-prepare")

\santaku
{RCバグの現状ははどこで確認できるか}
{\url{http://bugs.debian.org/release-critical/}}
{\url{http://localhost/}}
{\url{http://debianmeeting.appspot.com/}}
{A}
{}

\santaku
{Debian勉強会予約システムのURLはどれか}
{\url{http://www.2ch.net/}}
{\url{http://atnd.org/events/}}
{\url{http://debianmeeting.appspot.com/}}
{C}
{}

\santaku
{events@debian.orgはどこと統合されたか}
{merchants@debian.org}
{hoge@debian.org}
{fuga@debin.org}
{A}
{}

\santaku
{antiharassment@debian.org のうらにいないのは誰か}
{Amaya Rodrigo Sastre}
{Patty Langasek}
{Kouhei Maeda}
{C}
{}

\santaku
{現在いくつかのSprintが開催され、企画されている。現在企画すらされていな
いsprintはどれか}
{-www sprint}
{security sprint}
{tokyo sprint}
{C}
{}

\santaku
{DACAはどこを見ればよいか}
{\url{http://qa.debian.org/daca/}}
{\url{http://daca.debian.org/}}
{\url{file:/tmp}}
{A}
{}

\santaku
{DEP は何の略か}
{Debian Enhancement Proposal}
{Device Enhancement Protocol}
{でっぷ}
{A}
{}

\santaku
{DEP5で提案されているdebian/copyrightの機械可読形式はどういうものか}
{S式}
{RFC822風}
{XML}
{B}
{}

