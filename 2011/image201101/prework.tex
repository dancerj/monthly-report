%; whizzy-master ../debianmeetingresume201101.tex
% 以上の設定をしているため、このファイルで M-x whizzytex すると、whizzytexが利用できます。

\begin{prework}{上川 純一}

\preworksection{2011年の勉強会でなしとげたいことを宣言してください }

今年はLAMP スタック、KVSなど、クラウドインフラをDebianで一通り構築する仕
組みを実現する方法について一通り勉強して関連したパッケージングにまつわる
問題について整理したい。

\preworksection{2015年のDebianがどうなっているか大胆に妄想してください。}
 
Debianはまだ生き残っていて、Squeeze+2のリリース準備で忙しい。
デバイスは携帯電話とタブレットデバイスのサポートが追加されており、
メインストリームのCPUアーキテクチャはx86\_64 と arm。

デバッグ用の情報がデフォルトで全部のパッケージについて提供される。

全VCSを共通で利用できるようなインタフェースが整備され、
Debianのソースパッケージを標準的なVCSとして提供されるようになる。

DebianのサーバリソースはP2Pオーバレイネットワーク経由で提供され、
共有のビルドサーバなどが利用できるようになっている。

\end{prework}

\begin{prework}{まえだこうへい}

\preworksection{2011年の勉強会でなしとげたいことを宣言してください。}

 スマートフォン関連のWebアプリ用ライブラリとネイティブアプリ変換ツールな
 どの開発環境をDebianで整備する方法、何がどう優れているかを勉強したい。
 Debianで環境を整える為にはどれがDebianパッケージ
 になっているのか、どれは対応可能で、どれが不可能なのか。Debian上で生成
 したアプリをDebianから配信する、あるいは既存の配信の仕組みと連動するた
 めの仕組みはできないか。

\preworksection{2015年のDebianがどうなっているか大胆に妄想してください。}

 意外とWheezyの次のリリースが出てしまっている。スマートフォン向けのイン
 ストーラが登場し、jail breakやrootedしなくても、簡単にDebianをインストー
 ルできてしまう。webOSがipkgをやめて、Debianパッケージを使うように先祖返
 りしている(webOSには起死回生ホームランで生き残っていてほしい)。そもそも
 2015年にはスマートフォンとかクラウドとかいう言葉は 消えて、また別の新し
 いバズワードに変わっているに違いない。

\end{prework}


\begin{prework}{ 日比野 啓 }

\preworksection{2011年の勉強会でなしとげたいことを宣言してください。}


 Debian勉強会の常連に関数型言語の愛好者を一人でも多く増やす。関数型言語を使ったハンズオンとかやれたら楽しそう。

\preworksection{2015年のDebianがどうなっているか大胆に妄想してください。}

 Debian 8.0 リリース?

\end{prework}

\begin{prework}{ キタハラ }

\preworksection{2011年の勉強会でなしとげたいことを宣言してください。}

(東京エリアでの)勉強会に皆勤。もちろん無遅刻。

\preworksection{2015年のDebianがどうなっているか大胆に妄想してください。}

自らカーネルを作るようになっていたり、ATコンパチのPCを
飛び出し、省電力やクラスタ等の特別な機能を実現するハードウェア
リファレンスを展開してたり、サポート請負会社が出来てたり、
妄想は幾つか考えたのですが、実際は今とあまり変わらず淡々と
開発が続けられているように思います。

\end{prework}

\begin{prework}{ henrich }

\preworksection{2011年の勉強会でなしとげたいことを宣言してください }

成し遂げる目標は今のところありません…多少なりとも刺激が受けられればいい
 な、と考えています。

\preworksection{2015年のDebianがどうなっているか大胆に妄想してください。}

Hurdがついに\{リリースされる,終了宣言が出される\}

\end{prework}

\begin{prework}{ 村田信人 }

\preworksection{2011年の勉強会でなしとげたいことを宣言してください }

 パッケージへの理解を深める。Debianへのアップロードの流れを知る。

\preworksection{2015年のDebianがどうなっているか大胆に妄想してください。}

 サーバ用のディストリビューションと言ったらDebian!になっている。

\end{prework}



\begin{prework}{ 山田 }

\preworksection{2011年の勉強会でなしとげたいことを宣言してください }

\begin{itemize}
 \item MiniConfの実現(勉強会的に最大のトピック?)
 \item パッケージ作成の再勉強会をして、自己流を改める
 \item 手がけるパッケージを増やす&自分で書いたのを配ってみる
\end{itemize}

\preworksection{2015年のDebianがどうなっているか大胆に妄想してください。}

\begin{itemize}
 \item /var/lib/dpkg/availableが行数換算で50万行を超え(てさすがに重くてdpkgが超改良される(といいな))
 \item 仮想化コンテナが超普及しつつ公式の超高機能unionfsが遂に完成し、apt-getでconflictしたら即分岐して両方の環境を両立しつつbind統合できるようになる(といいな)
 \item ファイルシステムのスナップショット・タイムマシン機能が普及して、誰もsidを使うことに懸念を感じなくなる(といいな)
\end{itemize}


\end{prework}



\begin{prework}{ 本庄 }

\preworksection{2011年の勉強会でなしとげたいことを宣言してください }

ずいぶん前ですが、依頼されたような気のするlibsvmに関して発表したいです。

\preworksection{2015年のDebianがどうなっているか大胆に妄想してください。}

ちょっと想像つかないですね。


\end{prework}



\begin{prework}{ emasaka }

\preworksection{2011年の勉強会でなしとげたいことを宣言してください }

ここは発表とかいうといいんでしょうか。

\preworksection{2015年のDebianがどうなっているか大胆に妄想してください。}

TVがCPUばんばん積むようになって、そいつをhackしてDebianが動いたりするとすごいですね。

\end{prework}



\begin{prework}{ higashiyama }

\preworksection{2011年の勉強会でなしとげたいことを宣言してください }

今年は毎回参加して、ただDebianを使う人からステップアップしたい

\preworksection{2015年のDebianがどうなっているか大胆に妄想してください。}

Xがひっそりと消えていく

\end{prework}



\begin{prework}{ yamamoto }

\preworksection{2011年の勉強会でなしとげたいことを宣言してください }

今年こそハッカーになる。

\preworksection{2015年のDebianがどうなっているか大胆に妄想してください。}

Debian が sid だけの、ローリングリリースになっており、
stable のリリースは派生のディストリビューションに任せている。

\end{prework}


\begin{prework}{ kmuto }


\preworksection{2011年の勉強会でなしとげたいことを宣言してください }

 ますます忙しいんだけど、数回は出席したい。また講師できるようなテーマも作れるといいな。

\preworksection{2015年のDebianがどうなっているか大胆に妄想してください。}

 (wheezyではなく)wheezy+1のリリースエンジニアリング中、だよね?? DebianをUbuntuにマージすべきだというGRを出すかどうかでdebian-voteがフレームに。

当日はCAcert assurerの仕事をやる予定。

\end{prework}



\begin{prework}{ 野島 貴英 }

\preworksection{2011年の勉強会でなしとげたいことを宣言してください }

 3つぐらいは何か発表してみたいと思いました。

\preworksection{2015年のDebianがどうなっているか大胆に妄想してください。}

\begin{enumerate}
 \item  Oracle Client ライブラリのサポートOSリストにDebian stableが入っている(本当に欲しい...)
 \item  X.org以外のWindow Systemが標準採用になっている。
 \item  起動画面(grub/gdm)や、ログインメニューがフルアニメーションしており、待機状態では普通にBGMが鳴っている。(ゲームのメニュー画面みたいな凝った演出が標準)
 \item  通常のモニタは3Dモニタで利用されており、kinectのような身体を使って操作するようなデバイスが入力デバイスとして標準対応。
 \item  Blue-Ray Disk4枚組みの配布となる。
 \item  Mysqlがupstreamの最新版と同等となる。
\end{enumerate}

\end{prework}



