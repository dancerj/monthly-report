\begin{prework}{ 吉野(yy\_y\_ja\_jp) }
TwitterクライアントPollyをUbuntu PPAが微妙だったのでパッケージ化してみました.

アイコンが Creative Commons Attribution-NonCommercial-ShareAlike 3.0
Unported だったので削除しました.自前で描くべきか微妙です。
コードの多くが GPL-3+ で、一部アイコンセットが GPL-2 となっていて矛盾しているので
そのアイコンセットを削除しました。
Pythonで書かれており、PythonライブラリをBuild-Depends\(-Indep\)と
Dependsに手書きで追加しました。Dependsは自動生成できたらうれしいかもしれません。
\end{prework}

\begin{prework}{ dictoss(杉本 典充) }
パッケージ化したものはないので、使い込んだDebianパッケージだけです。
\begin{itemize}
\item kfreebsd-image-8-amd64\\
FreeBSD本家ではビルドされるのに、Debianパッケージでは付属していないものが多く、その違いに悩んだ。
\item gftp\\
sshコマンドみたくデフォルトのssh鍵認証→パスワード認証でログイン処理が進んでくれず、ssh鍵認証でログインに失敗すると画面が進んでくれない。アップストリームの問題かは不明。
\item icewm
\end{itemize}
\end{prework}



\begin{prework}{岩松 信洋}

\begin{itemize}
\item 今年パッケージ化した Debian パッケージ:\\
bluez-tools, fonts-ipamj-mincho, mozc, mtdev, xf86-input-mtrack, xf86-input-multitouch
\item 今年使い込んだ Debian パッケージ:\\
libpng, opencv, bluez, ruby, mozc, buildd, pbuilder
\item 出会った課題: 
release チームとのやりとり
\end{itemize}
\end{prework}

\begin{prework}{ なかおけいすけ }
\begin{itemize}
\item パッケージ化しているパッケージ\\
stm32flash
\item 課題\\
manがないとか、ライセンスファイルの書き方がおかしいとか、パッケージングポリシーに従うこと、意外に大変なこと。
\end{itemize}
\end{prework}

\begin{prework}{ やまだ }
Debian構成や固有部分が絡んでるもので挙げてみました:
\begin{itemize}
\item busybox\\
OK:ビルド設定が不足してjob controlできない
\item initramfs-tools\\
??:rootwaitがあるからrootdelayは不要と思ったらMD+USBで起動障害
\begin{itemize}
\item Linux的には上は正だが、udevも見てタイミング調整するというクイックハック状態なのだった
\end{itemize}
\item grub-pc\\
OK:grub2でモニタとシリアルの同時有効化機能が一時落ちて、制御不能になった
\end{itemize}
\preworknewpage{やまだ}
\begin{itemize}
\item extlinux\\
OK:設定バラバラ化のみならず、標準ではシリアル有効化の設定を仕込む箇所が存在せず、制御不能に
\item dropbear\\
??:initramfs内でdropbearを起動する荒業により、固定IPのDNSサーバがDHCPアドレスになりNW死亡
\item debhelper\\
??:CFLAGS(や相当設定)の指定方法がわからず、gcc -m32 や -f... の渡し方に悩む
\item kexec\\
OK:デフォで再起動が単なるkexec実行に切り替わり、BIOSが取れず悩んだ
\item apt\\
??:実はsecurity.d.oが120ms-300msの彼方で遠い・・・
\end{itemize}
\end{prework}

\begin{prework}{ Kazuo Ishii }
emacs\\常にカスタマイズを続けています。これをどれだけ使いこなすかが、課題です。
\end{prework}

\begin{prework}{ Aru }
postfixのソースコードをいじってパッケージ化させてインストールしなおしました。

○CNの回線上でSMTPサーバを構築する際はOP25Bの関係上OCNのSMTPサーバを通さなければならないのですが、そのSMTPサーバが特殊な仕様のようでpostfixの設定を変えるだけではサーバに弾かれてしまいます。そこで、○CNのサーバ向けにソースコードをいじりました。
\end{prework}

\begin{prework}{ koedoyoshida }

\begin{itemize}
\item Unbound:\\
仕事で評価用DNSを建てるのに使ったり、
dnstudy等で発表の危険なLTネタとして使用。
BINDに比べて良くできてるので特に課題等はなし
\item zabbix:\\
基本stableを使っていること、かつ発展途上のソフトであることから豊富なはまりポイントが有った。
が基本的にぐぐって解決したり、表示上の問題なので無視したりしてます。
\end{itemize}
\end{prework}

\begin{prework}{ henrich }

今年はあまりパッケージをいじくってないです…パッケージ化したものといえばIRCクライアントのloquiとかぐらいでしょうか。それもupstreamの方が、ほとんど雛形作っていただいていたので細かい所を直しただけでした。

DEP5への対応や他のDEPへの対応などが今後パッケージクオリティを挙げる上で課題でしょうか(主に面倒くさい意味で)
\end{prework}

\begin{prework}{ yamamoto }

今年パッケージ化したDebianパッケージ: 無し
うーむ、自分専用のパッチをあてたカーネルパッケージぐらいっすね。
貢献はできてないな。

\begin{itemize}
\item 今年使い込んだDebianパッケージ: devscripts pbuilder\\
devscripts でポチポチと sbuild 用にパッケージを作り、pbuilder で再ビルドしてました。
今年は FTBFS を直してもらう BTS をたまにするのと、自分専用リポジトリを最新に保つだけで、いっぱいいっぱいでした。

\item 来年使い込む予定のDebianパッケージ: sbuild buildd\\
さて、弾ができたし、突撃じゃー。

\item 出会った課題: \\
base.tgz とかを最初に作る時、arch=all なのはオフィシャルから、アーキテクチャ依存なパッケージはオレオレリポジトリから、という使い方ができなかった。ヘタレなおいらは、build-essential なパッケージに依存する all なのは、全部オレオレリポジトリにつっこんで解決。
\end{itemize}
\end{prework}

\begin{prework}{ まえだこうへい }

12/8時点の状況

\begin{itemize}
\item パッケージング済み
\begin{itemize}
\item python-funcparserlib : テストでコケる問題
\item python-webcolors
\end{itemize}
\item パッケージング途中
\begin{itemize}
\item python-ordereddict:Python2.6のみ
\item python-blockdiag : Python 2.6はpython-ordereddictを、2.7は組み込みのordereddictを使うようにパッケージングするにはdebian/controlどう書けばいいんだ?
\item python-\{sec,act,nw\}diagおよびpython-sphinxcontrib.\{block,seq,act,nw\}diag : python-blockdiag待ち
\item python-tomahawk : nodetestsでコケる問題
\end{itemize}
\end{itemize}


\end{prework}

\begin{prework}{野島 貴英}
\begin{itemize}
\item パッケージング途中
\begin{itemize}
\item xmris\\
scoreのothersにrw付与を止めさせたい。古いゲームは特に。
\item tracef\\
dynamic loadingをhookして、symbol再ロードして欲しい。
\end{itemize}
\item 使い込んだパッケージ
\begin{itemize}
\item totem,gstreamer-tools\\
いくつか機能の動作が不完全な気が。例:字幕ファイル利用とか。
\item gnome-shell\\
いろいろドキュメントなさすぎ。
\item po-mode.el\\
改行全部に\textbackslash{}nを入れられてしまうので、po4a製poファイルの翻訳作業がきっつー。
\end{itemize}
\end{itemize}
\end{prework}
