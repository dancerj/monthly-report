\documentclass[cjk,dvipdfmx,12pt,%
hyperref={bookmarks=true,bookmarksnumbered=true,bookmarksopen=false,%
colorlinks=false,%
pdftitle={第 47 回 関西 Debian 勉強会},%
pdfauthor={倉敷・のがた・佐々木},%
%pdfinstitute={関西 Debian 勉強会},%
pdfsubject={資料},%
}]{beamer}

\title{第 47 回 関西 Debian 勉強会}
\subtitle{{\scriptsize 資料}}
\author[倉敷 悟]{{\large\bf 倉敷・のがた・佐々木}}
\institute[Debian JP]{{\normalsize\tt 関西 Debian 勉強会}}
\date{{\small 2011 年 5 月 22 日}}

%\usepackage{amsmath}
%\usepackage{amssymb}
\usepackage{graphicx}
\usepackage{moreverb}
\usepackage[varg]{txfonts}
\AtBeginDvi{\special{pdf:tounicode EUC-UCS2}}
\AtBeginSection[]{\begin{frame}<beamer>\frametitle{Agenda}\tableofcontents[currentsection]\end{frame}}
\usetheme{Kyoto}
\def\museincludegraphics{%
  \begingroup
  \catcode`\|=0
  \catcode`\\=12
  \catcode`\#=12
  \includegraphics[width=0.9\textwidth]}
%\renewcommand{\familydefault}{\sfdefault}
%\renewcommand{\kanjifamilydefault}{\sfdefault}
\begin{document}
\settitleslide
\begin{frame}
\titlepage
\end{frame}
\setdefaultslide

\begin{frame}[fragile]
\frametitle{Agenda}
\tableofcontents
\end{frame}

\section{最近の Debian 関係のイベント}

\takahashi[40]{最近の Debian\\関係のイベント}

\begin{frame}[fragile]
\frametitle{第 46 回関西 Debian 勉強会}

\begin{itemize}
\item 日時: 4 月 16 日
\item 於: OSC 2011 Kobe
\end{itemize}

\begin{block}{内容}
  \begin{itemize}
  \item 遂にリリースされた(?) Squeeze について by 佐々木さん
  \item GnuPG クイックスタート by 佐々木さん
  \end{itemize}
\end{block}
\end{frame}

\begin{frame}[fragile]
  \frametitle{東京エリア Debian 勉強会($+\alpha$)}
  \begin{itemize}
  \item 第 76 回: 2011/05/21 開催.
  \item 於:
  \end{itemize}
  \begin{block}{内容}
    \begin{itemize}
    \item Apache モジュールを作ってみた by 上川さん
    \item などなど
    \end{itemize}
  \end{block}
\end{frame}

\takahashi[50]{そんな\\こんなで}
\takahashi[120]{次}

\takahashi[50]{事前課題発表}

\takahashi[50]{ murase\_syuka }

\takahashi[50]{ 西山和広 }

\takahashi[50]{ 佐藤誠 }

\takahashi[50]{ yabuki\\@netfort.gr.jp }

\takahashi[50]{ よしだともひろ }

\takahashi[50]{ かわだ\\てつたろう }

\takahashi[50]{ kino }

\takahashi[50]{ 甲斐 正三 }

\takahashi[50]{ 木下 }

\takahashi[50]{ Kozo.Nishida }

\takahashi[50]{ 松澤二郎 }

\takahashi[50]{ 山田 洋平 }

\takahashi[50]{ 山下 康成 }

\takahashi[50]{ 八津尾 }

\takahashi[50]{ 小山 }

\takahashi[50]{ 川江 }

\takahashi[50]{ tanaka jun }

\takahashi[50]{ lurdan }

\takahashi[50]{ oJIN }

\takahashi[50]{ ゴーインズ・ポール }

\takahashi[50]{そんな\\こんなで}
\takahashi[120]{次}

\section{ハッカーに一歩近づく Tips:vi 編}
\takahashi[40]{ハッカーに一歩近づく Tips:vi 編\\by\\山下康成}

\takahashi[50]{そんな\\こんなで}
\takahashi[120]{次}

\section{dpkg のおさらい}

\takahashi[30]{dpkg のおさらい\\by\\倉敷悟}

\takahashi[50] { 続きは Term で! }

\takahashi[50]{ 解法1 }
\begin{frame}[fragile]
\frametitle{--ignore-depends を使う}
一番お手軽ですが、APT に禍根が残ります

\end{frame}



\takahashi[50]{ 解法2 }
\begin{frame}[fragile]
\frametitle{インストールに失敗した状態で /var/lib/dpkg/status をいじる}
比較的お手軽ですが、失敗するとシステムが壊れる場合があります
\end{frame}

\takahashi[50]{ 解法3 }
\begin{frame}[fragile]
\frametitle{バイナリパッケージをいじってから導入する (ar, tar)}
ちなみに、今回例題に使っている dropbox ではソース版も配布されているので、debian 的にはそちらをコンパイルする方がベターかも知れません。
\end{frame}

\takahashi[50]{ 解法4 }
\begin{frame}[fragile]
\frametitle{バイナリパッケージをいじってから導入する (dpkg-deb)}
慣れてきたら、解法3 よりはこちらの方がやりやすいでしょう。
\end{frame}

\takahashi[50]{ 解法5 }
\begin{frame}[fragile]
\frametitle{ソースからパッケージを作る}
がんばれ
\end{frame}

\takahashi[50]{そんな\\こんなで}
\takahashi[120]{次}

\begin{frame}[fragile]
\frametitle{今後の予定}

\begin{block}{第 48 回関西 Debian 勉強会}
  \begin{itemize}
  \item 日時: 6 月 26 日(日)
  \item 会場: 福島区民センター
  \item 内容: 「git-buildpackage を使ったパッケージメンテナンス」
    \begin{itemize}
    \item 講演者: 佐々木
    \end{itemize}
  \item 他 1 本
  \end{itemize}
\end{block}

\end{frame}

\takahashi[50]{  }


\end{document}
%%% Local Variables:
%%% mode: japanese-latex
%%% TeX-master: t
%%% End:
