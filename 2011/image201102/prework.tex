%; whizzy-master ../debianmeetingresume201101.tex
% 以上の設定をしているため、このファイルで M-x whizzytex すると、whizzytexが利用できます。

\begin{prework}{ キタハラ }

実家にリリース前のSqueezeを置いてきたが、特に問題なく使用されている様子。
何も設定しなくても、サスペンド・ハイバネーションできたような、プリンタの設定に少々悩んだ記憶があるが、詳細は忘れた。
昔に比べると格段に楽になりました。
\end{prework}

\begin{prework}{ 吉野(yy\_y\_ja\_jp) }

poppler-data,cmap-adobe-*,gs-cjk-resource が main に入ったことですね,Thanks to Adobe.
\end{prework}

\begin{prework}{ d123.itouch.com(dai) }

 \begin{itemize}
  \item 宇宙がテーマでかっこいい!
  \item 新しくDebianを覚えることになったので、丁度いい機会です。Debian初
	心者!一年生です!
 \end{itemize}

 2年ほど前に東京の某専門学校で開催された貴会のブースでご丁寧に
 ご紹介いただいたことがあります。すばらしい会を開催いただき
 ありがとうございます。今回初の参加です。どうぞよろしくお願いします。
\end{prework}

\begin{prework}{ hakase }

すいません、Squeezeの名前始めて聞きました。
勉強します。
\end{prework}

\begin{prework}{ henrich }

すいません、特段個人的にうれしいところが見当たりませんでした…
\end{prework}

\begin{prework}{ 山田 }

\preworksection{うれしいこと}
\begin{itemize}
 \item /bin/shがdashになった!スクリプト起動が明らかに軽い
 \item 依存関係ベースで起動処理がされるようになった(早くなった)
 \item パッケージがさらに充実した(気づいてなかっただけかもしれないものの、「こんなのまで入ってる」度が上昇)
 \item リリーススケジュールがしっかりしてきた
\end{itemize}

\preworksection{変わったこと(気になること)}
\begin{itemize}
 \item kFreeBSDが入った!すぐ使うわけではないけど、気になる
 \item arm(eb)が落っこちたけど、世の中大体armelだからOKなのかな?
 \item Debian Live。LiveCDからnetbootまでカバーということで要チェック
\end{itemize}
リリースの前から実際には使えていたので「先日からいきなりうれしくなった」事は実はないものの、着実なリリースという感想です。
\end{prework}

\begin{prework}{ dictoss(杉本 典充) }

起動が早くなった。
Debian GNU/kFreeBSDの名が広まってきた。
\end{prework}

\begin{prework}{ yamamoto }

\preworksection{うれしいこと}
Squeeze だから、というわけではないですが、debian-multimedia の mythtv が
 0.24 系になって、mythtv-backend が突然死することが無くなりました。

\preworksection{変わったこと}
KDE を常用してますが、Klipper の挙動が変わった。なんか時々、わざわざ Klipper のチェックをつけないと、中ボタンで選択文字列をペーストできない。
原因調査はまだだけど…。
「クリップボードのアクションを有効にする」あたりがあやしい!?
\end{prework}

\begin{prework}{ 野島 貴英 }

うはっ、普段からunstable(sid)しか使ってない...

取り急ぎlennyとの違いで嬉しい事。
\begin{enumerate}
 \item kernelが2.6.32系列になった。(kvmのバージョンが上がった!)
 \item libvirtが0.8.3になった。(いろんなパラメタがいろいろつけ放題に...)
 \item gnomeが2.30系列になった。freedesktop-sound-themeが標準装備されたので音がにぎやかになった。
 \item upstartが搭載された。(sid使っているのですが、upstartになってなく
       てがっくり。今から変えてみます...)
\end{enumerate}
\end{prework}

\begin{prework}{ taitioooo }

初めてDebianに触ったのがsqueezeでした。
今までCentos使っていました。
初めての印象は、「あ、ロケットがある。」でした。

あと、apt-getの設定が簡単でよかったです。(yumと比べてですが)
\end{prework}

\begin{prework}{ Hasegawa }

インストールでrootでログインするか聞いてきたり
設定項目が増えたようだし、かなり変わった印象を持った
64bit/32bitが選択できるインストーラーになった

\end{prework}

\begin{prework}{ ayakomuro }

まだアップグレードしてないので、週末にアップグレードしたら詳細書きますm(..)m
\end{prework}

\begin{prework}{ まえだこうへい }

\preworksection{うれしいこと}
\begin{itemize}
 \item SqueezeがリリースされたことでSidの開発が再開されたので、CouchAppや
       xserver-xorg-input-multitouchがDebianパッケージになった。
 \item Sargeで稼働中の組合サーバをハードウェア&システム老朽化により、
       KVM/Squeeze/HP MicroServerで構築中なので、リリースされてちょうど
       良かった。
 \item 今までSidでしか使えなかったパッケージがstableで使えるようになった
       ものが増えた。(lxc, scala, xz-utils, libapache-mod-security(これは復活し
       た、というのが正しいか))
 \item 起動スクリプトがbashからdashに変わった。VMでも起動が速くなった気
       がする。(体感速度。)
\end{itemize}
\preworksection{変わったこと}
\begin{itemize}
 \item libvirtが0.8.3になった。nwfilterの機能が入った
       (/etc/libvirt/nwfilter/*.xml)。Netfilterのルールはすべて自分で書
       いているので、自分には必要ないけれど、普段書かないNetfilterのルー
       ルを書かない人に取っては良いのでは。
 \item クラウド関連のツールがパッケージで増えた。時代ですねー。
\end{itemize}
\end{prework}

\begin{prework}{ 岩松信洋 }

sid 使っているのであまり変化はないです。
心残りのほうが多いです。
\end{prework}


