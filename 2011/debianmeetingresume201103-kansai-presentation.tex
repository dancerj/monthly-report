\documentclass[cjk,dvipdfmx,12pt,%
hyperref={bookmarks=true,bookmarksnumbered=true,bookmarksopen=false,%
colorlinks=false,%
pdftitle={第 45 回 関西 Debian 勉強会},%
pdfauthor={倉敷・のがた・佐々木},%
%pdfinstitute={関西 Debian 勉強会},%
pdfsubject={資料},%
}]{beamer}

\title{第 45 回 関西 Debian 勉強会}
\subtitle{{\scriptsize 資料}}
\author[佐々木 洋平]{{\large\bf 倉敷・のがた・佐々木}}
\institute[Debian JP]{{\normalsize\tt 関西 Debian 勉強会}}
\date{{\small 2011 年 3 月 27 日}}

%\usepackage{amsmath}
%\usepackage{amssymb}
\usepackage{graphicx}
\usepackage{moreverb}
\usepackage[varg]{txfonts}
\AtBeginDvi{\special{pdf:tounicode EUC-UCS2}}
\AtBeginSection[]{\begin{frame}<beamer>\frametitle{Agenda}\tableofcontents[currentsection]\end{frame}}
\usetheme{Kyoto}
\def\museincludegraphics{%
  \begingroup
  \catcode`\|=0
  \catcode`\\=12
  \catcode`\#=12
  \includegraphics[width=0.9\textwidth]}
%\renewcommand{\familydefault}{\sfdefault}
%\renewcommand{\kanjifamilydefault}{\sfdefault}
\begin{document}
\settitleslide
\begin{frame}
\titlepage
\end{frame}
\setdefaultslide

\begin{frame}[fragile]
\frametitle{Agenda}
\tableofcontents
\end{frame}

\section{最近の Debian 関係のイベント}

\takahashi[40]{最近の Debian\\関係のイベント}

\begin{frame}[fragile]
\frametitle{第 44 回関西 Debian 勉強会}

\begin{itemize}
\item 日時: 2 月 27 日
\item 於: 大阪港区民センター
\end{itemize}

\begin{block}{内容}
  \begin{itemize}
  \item pbuilder を使ってみよう by 水野さん
  \item squeeze の変更点をみんなで見てみよう by のがたさん
  \end{itemize}
\end{block}
\end{frame}

\begin{frame}[fragile]
  \frametitle{東京エリア Debian 勉強会($+\alpha$)}
  \begin{itemize}
  \item 第 74 回: 2010/02/19 開催.
  \item 於: OSC 2010 Tokyo/Spring
  \end{itemize}
  \begin{block}{内容}
    \begin{itemize}
    \item Squeeze の紹介と今後の開発について by 岩松さん
    \item GPG キーサインパーティ
    \item 日本初! CAcert公式トレーニング
    \end{itemize}
  \end{block}
\end{frame}

\takahashi[50]{そんな\\こんなで}
\takahashi[120]{次}

\takahashi[50]{事前課題発表}
\takahashi[50]{...}
\takahashi[50]{間違い}
\takahashi[50]{近況紹介}

\takahashi[50]{ 佐々木 洋平 }

\takahashi[50]{ 山下 尊也 }

\takahashi[50]{ かわだ\\てつたろう }

\takahashi[50]{ 川江 }

\takahashi[50]{ 木下 達也 }

\takahashi[50]{ 木下 }

\takahashi[50]{ kazken3 }

\takahashi[50]{ 山下 康成 }

\takahashi[50]{ 甲斐 正三 }

\takahashi[50]{ 山田 洋平 }

\takahashi[50]{ gdevmjc }

\takahashi[50]{ lurdan }

\takahashi[50]{ 清野陽一 }

\takahashi[50]{ 八津尾 }

\takahashi[50]{ HoriuchI Yasuhiko }

\takahashi[50]{ 古川竜雄 }

\takahashi[20]{ steven.mcintire.allen@originlaw.net }


\takahashi[50]{そんな\\こんなで}
\takahashi[120]{次}

\section{Debian のドキュメントをみてみよう}

\takahashi[30]{Debian のドキュメントをみてみよう\\by\\かわだてつたろう}

\takahashi[50]{そんな\\こんなで}
\takahashi[120]{次}

\section{「フリー」って、どういうこと?}
\takahashi[40]{「フリー」って、どういうこと?\\by\\木下 達也}

\takahashi[50]{そんな\\こんなで}
\takahashi[120]{次}

\begin{frame}[fragile]
\frametitle{今後の予定}

「神戸 IT フェスティバル + オープンソースカンファレンス 2011Kansai@Kobe」
に(毎度)出張します.

\begin{block}{第 46 回関西 Debian 勉強会}
  \begin{itemize}
  \item 日時: 4 月 16 日(土)
  \item 会場: 神戸市産業振興センター
  \item 内容: 「- 遂にリリースされた(?) Squeeze について - 」
    \begin{itemize}
    \item 講演者: 佐々木
    \end{itemize}
  \item 実機展示(?), 「Deb専」販売
  \end{itemize}
\end{block}

\end{frame}

\takahashi[50]{  }


\end{document}
%%% Local Variables:
%%% mode: japanese-latex
%%% TeX-master: t
%%% End:
