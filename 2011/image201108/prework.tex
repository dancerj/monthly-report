%; whizzy-master ../debianmeetingresume201108.tex
% 以上の設定をしているため、このファイルで M-x whizzytex すると、
% whizzytexが利用できます

\begin{prework}{ 鈴木崇文 }

aptのリポジトリ作成ってどうやるのでしょうか。
ググったらなんかページが出てきてわかりそうですが、やったことないので書きました。
あと、以前誰かが話していたaptのキャッシュサーバーって最近の動向はどうなのでしょうか。

\end{prework}

\begin{prework}{ キタハラ }

Debian固有のものだと、今のところないですね。
1コマンド(又はGUI操作)でUSBメモリにLive環境を作るツールとか?
(探すとあったりして…、調べていません。)

\end{prework}

\begin{prework}{ やまだ }

今後調べたい(解説歓迎)
\begin{itemize}
\item libapt-pkg-perl/python-aptなどのパッケージデータベースAPI
\item debhelper8(簡単になったが奥行きがさらに増した、ような…)
\item *.d.oなサイトを改良したくなったらどうすればいいか
\end{itemize}

緩募
\begin{itemize}
\item apt-buildの./configureの引数なども即いじれる版
\item apt-get changelog風の入れてないマニュアル等を読むツール
\end{itemize}

\end{prework}

\begin{prework}{ koedoyoshida }

dpkg -i(apt-get installも同様)がエラーになったときの原因追及で困ったことがあったので、そのようなときの調査法を知りたい。
\url{http://www.flcl.org/~takasugi/tdiary-org/?date=20061023}
に同現象があったのでとりあえずワークアラウンドは分かりましたが...

\end{prework}

\begin{prework}{ dictoss(杉本 典充) }

debootstrapのおもしろい使い方ってあるのかなぁ。(amd64上でi386環境が必要、常用環境とテスト環境の分離、くらいしか使ったことないです。)
\end{prework}

\begin{prework}{ 上川純一 }
とくになし。

\end{prework}

\begin{prework}{ Osamu MATSUMOTO }

ユーザーとしては特に困ってないっす。aptitudeでしかできない事ってあるのかわかりません。開発者ツールまだまだ勉強中。
\end{prework}

\begin{prework}{ なかおけいすけ }

\begin{itemize}
\item aptのオプションの解説

apt-get update, upgrade, install, remove, clean 位しか使ってないのでその他のオプションについて

\item 推奨パッケージもインストールしてくれるオプション

これはきっとある

\item apt.confの解説

debianのページを見ていると、時々apt.confを修正してみたいな記述があるけれども、何をやっているのか、何ができるのか、解説が欲しい。

\item apt-getとaptitudeの違い
\end{itemize}

\end{prework}

\begin{prework}{ やまねひでき }

debootstrapでxzがサポートされていると良いのかも。
\end{prework}

\begin{prework}{ 岩松 信洋 }

ぱっとは思いつかないです。
\end{prework}

\begin{prework}{ 吉野(yy\_y\_ja\_jp) }

こんにちは。
\end{prework}

\begin{prework}{ yamamoto }

最近は apt-get autoremove で、既に削除されたパッケージの、依存関係解決のために入れられたパッケージがごっそり削除できますが、apt-get build-dep ホゲホゲで入れられたパッケージも同じようにビルド終了後に消せるといいよね。
\end{prework}

\begin{prework}{ taitioooo }

課題は未定ですが。
debianでkaresansuiが使えるようになるといいですね〜

あ、やればいいのか!
\end{prework}
