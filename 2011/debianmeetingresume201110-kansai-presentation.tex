\documentclass[cjk,dvipdfmx,12pt,%
hyperref={bookmarks=true,bookmarksnumbered=true,bookmarksopen=false,%
colorlinks=false,%
pdftitle={第 52 回 関西 Debian 勉強会},%
pdfauthor={倉敷・のがた・佐々木},%
%pdfinstitute={関西 Debian 勉強会},%
pdfsubject={資料},%
}]{beamer}

\title{第 52 回 関西 Debian 勉強会}
\subtitle{{\small 資料}}
\author[佐々木 洋平]{{\large\bf かわだ}}
\institute[Debian JP]{{\normalsize\tt 関西 Debian 勉強会}}
\date{{\small 2011 年 10 月 23 日}}

%\usepackage{amsmath}
%\usepackage{amssymb}
\usepackage{graphicx}
\usepackage{moreverb}
\usepackage[varg]{txfonts}
\AtBeginDvi{\special{pdf:tounicode EUC-UCS2}}
\AtBeginSection[]{\begin{frame}<beamer>\frametitle{Agenda}\tableofcontents[currentsection]\end{frame}}
\usetheme{Kyoto}
\def\museincludegraphics{%
  \begingroup
  \catcode`\|=0
  \catcode`\\=12
  \catcode`\#=12
  \includegraphics[width=0.9\textwidth]}
%\renewcommand{\familydefault}{\sfdefault}
%\renewcommand{\kanjifamilydefault}{\sfdefault}
\begin{document}
\settitleslide
\begin{frame}
\titlepage
\end{frame}
\setdefaultslide

\begin{frame}[fragile]
\frametitle{Agenda}
\tableofcontents
\end{frame}

\section{最近の Debian 関係のイベント}

\takahashi[40]{最近の Debian\\関係のイベント}

\begin{frame}[fragile]
\frametitle{第 51 回関西 Debian 勉強会}

\begin{itemize}
\item 日時:9 月 25 日
\item 於: 福島区民センター
\end{itemize}

\begin{block}{内容}
  \begin{itemize}
  \item vcs-buildpackage その1、その2 (山下、佐々木)
  \end{itemize}
\end{block}
\end{frame}


\begin{frame}[fragile]
  \frametitle{第81回東京エリアDebian勉強会}
  \begin{itemize}
  \item 日時: 10 月 22(土)
  \item 於:筑波大学
  \end{itemize}
  \begin{block}{内容}
    つくらぐ(筑波大学 Linux User Group)と合同でおこなわれたとのこと。
  \end{block}
\end{frame}

\takahashi[50]{そんな\\こんなで}
\takahashi[120]{次}

\takahashi[50]{事前課題発表}

\begin{frame}[fragile]
\frametitle{事前課題}

\begin{block}{今回の事前課題}
\begin{enumerate}
\item EmacsもしくはVimの拡張機能のDebianパッケージを挙げてください(何個でも)。また、それらに含まれるファイル一覧を見ておいてください(これは記入不要)
\item OmegaT が動作する環境を用意し, お手軽スタートガイドに目を通してきて下さい。翻訳をしたことがある人は作業環境を教えて下さい
\end{enumerate}

\end{block}
\end{frame}

\takahashi[50]{参加者の回答\newline 兼自己紹介}

\begin{frame}{ kozo2 }
\begin{enumerate}
\item Emacs: auto-install-el, twittering-mode, auto-complete-el, anything-el\\
Vim: vim-rails
\item Sphinx
\end{enumerate}
\end{frame}

\begin{frame}{ Y.YATSUO }
\begin{enumerate}
\item vim-scripts ... vim にベルとホイッスル機能を追加するプラグイン\\こんなのあるんですね…
\item ちょっとした文章なら vim で翻訳します.
\end{enumerate}
\end{frame}

\begin{frame}{ 山下康成 }
会社で仕事ちう(藁
\end{frame}

\begin{frame}{ 川江 }
\begin{enumerate}
\item 拡張機能は使った事がないので勉強しておきます。
\item 見ておきます。
\end{enumerate}
\end{frame}

\begin{frame}{ gdevmjc }
お世話になります。
\begin{enumerate}
\item egg c-sig mew-beta
\item とりあえず、 squeeze の omegat 1.8.1 を入れました
\end{enumerate}
\end{frame}

\begin{frame}{ かわだてつたろう }
\begin{enumerate}
\item apel, flim, semi, auto-install-el, ddskk, easypg, elscreen, howm, lookup-el, sdic, wl-beta
\item 用意して目を通しました。
\end{enumerate}
\end{frame}

\begin{frame}{ のがたじゅん }
\begin{enumerate}
\item psgml css-mode org-mode gettext-el muse-el yatex
\item 翻訳するときgettextのファイルが多いので、emacs上のgettext-elを使っています。
\end{enumerate}
\end{frame}

\begin{frame}{ shuttaholic }
vim-syntax-go vim-puppet supercollider-vim vim-athena vim-dbg vim-lesstif vim-addon-manager vim-latexsuite vim-rails vim-syntax-gtk vimhelp-de vim-vimoutliner vim-scripts
\end{frame}

\begin{frame}{ 松澤二郎 }
\begin{enumerate}
\item vim-scripts
\item Debianの翻訳に携わったことはなく、GNOMEプロジェクトでの作業になりますが、PO編集が
作業の中心で、テキストエディター(vim, geditなど)とtranslate-toolkitで作業することが
多いです。バージョン管理にはgitを使っています。OmegaTは私にはちょっと難しくて疎遠になって
いましたが、これを機にがんばって覚えたいと思います。
\end{enumerate}
\end{frame}

\begin{frame}{ よしだともひろ }
\begin{enumerate}
\item Emacs: mew, howm, twittering-mode, autocompleteなど\\
  Vim: すみません、よくわかりません
\item OmegaT バージョン2.3.0 アップデート1をインストールしました。
  お手軽スタートガイドを読みました。
\end{enumerate}
\end{frame}

\begin{frame}{ lurdan }
\begin{enumerate}
\item ddskk, org-mode, howm, ack-grep, lookup-el, anything-el, autoinstall-el, auto-complete-el, emacs-calfw, apel, flim, semi, wl-beta, riece, w3m-el, puppet-el, vim-puppet
\item emacs + skk + lookup-el + po-mode
\end{enumerate}
\end{frame}

\takahashi[50]{そんな\\こんなで}
\takahashi[120]{次}

\section{Emacs, Vimの拡張機能で学ぶDebianパッケージ}
\takahashi[40]{Emacs, Vimの拡張機能で学ぶDebianパッケージ\\by 西田さん}

\takahashi[50]{そんな\\こんなで}
\takahashi[120]{次}

\section{翻訳で Debian に貢献しよう}
\takahashi[40]{翻訳で Debian に貢献しよう\\by やつおさん}

\takahashi[50]{そんな\\こんなで}
\takahashi[120]{次}


\begin{frame}[fragile]
\frametitle{今後の予定(1)}

\begin{block}{第 53 回関西 Debian 勉強会}
  \begin{itemize}
  \item 日時: 11 月 12 日(土)
  \item 会場: 大阪南港ATC ITM棟 10F
  \item 内容: なれる! Debian 開発者 ― 45 分でわかる?メンテナ入門
    \begin{itemize}
    \item 講演者: やまねひでき
    \end{itemize}
  \end{itemize}
\end{block}

\end{frame}

\takahashi[50]{  }


\end{document}
%%% Local Variables:
%%% mode: japanese-latex
%%% TeX-master: t
%%% End:
