%; whizzy-master ../debianmeetingresume201101.tex
% 以上の設定をしているため、このファイルで M-x whizzytex すると、whizzytexが利用できます。
%
% ちなみに、クイズは別ブランチで作成し、のちにマージします。逆にマージし
% ないようにしましょう。
% (shell-command "git checkout quiz-prepare")

\santaku
{2011年度 Debian JP 会長は誰でしょうか?}
{前田 耕平}
{岩松 信洋}
{荒木 靖宏}
{A}
{}

\santaku
{2011年度 DPL は誰でしょうか?}
{Kurt Roeckx}
{Kenshi Muto}
{Stefano Zacchiroli}
{C}
{}

\santaku
{Debian Policy 3.9.2 で追加されていない項目はどれか}
{Debianアカウントをメンテナアドレスに追加する必要があります。}
{アーキテクチャ依存のライブラリ等は DEB\_HOST\_MULTIARCH で取得した値を利
用する必要があります。}
{全てのパッケージはVCSで管理する必要があります。}
{C}
{}

\santaku
{DebConf chairsに指名されたのは誰?}
{Gunnar Wolf}
{Maeda Kouhei}
{Junichi Uekawa}
{A}
{}

\santaku
{ries.debian.org で取得できるようになったデータは?}
{debian.org の稼働状況データ}
{debian.org のMLデータをgzipで固めた物}
{dak のデータ}
{C}
{}

\santaku
{/run が消された理由は?}
{/go のほうがよくね?という人が現れた。}
{initscripts がまだサポートしてない!}
{/run を入れたのはパッケージングミスです。}
{B}
{}
