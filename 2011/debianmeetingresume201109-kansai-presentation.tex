\documentclass[cjk,dvipdfmx,12pt,%
hyperref={bookmarks=true,bookmarksnumbered=true,bookmarksopen=false,%
colorlinks=false,%
pdftitle={第 51 回 関西 Debian 勉強会},%
pdfauthor={倉敷・のがた・佐々木},%
%pdfinstitute={関西 Debian 勉強会},%
pdfsubject={資料},%
}]{beamer}

\title{第 51 回 関西 Debian 勉強会}
\subtitle{{\small 資料}}
%\author[佐々木 洋平]{{\large\bf 倉敷・のがた・佐々木}}
\author[佐々木 洋平]{{\large\bf かわだ}}
\institute[Debian JP]{{\normalsize\tt 関西 Debian 勉強会}}
\date{{\small 2011 年 9 月 25 日}}

%\usepackage{amsmath}
%\usepackage{amssymb}
\usepackage{graphicx}
\usepackage{moreverb}
\usepackage[varg]{txfonts}
\AtBeginDvi{\special{pdf:tounicode EUC-UCS2}}
\AtBeginSection[]{\begin{frame}<beamer>\frametitle{Agenda}\tableofcontents[currentsection]\end{frame}}
\usetheme{Kyoto}
\def\museincludegraphics{%
  \begingroup
  \catcode`\|=0
  \catcode`\\=12
  \catcode`\#=12
  \includegraphics[width=0.9\textwidth]}
%\renewcommand{\familydefault}{\sfdefault}
%\renewcommand{\kanjifamilydefault}{\sfdefault}
\begin{document}
\settitleslide
\begin{frame}
\titlepage
\end{frame}
\setdefaultslide

\begin{frame}[fragile]
\frametitle{Agenda}
\tableofcontents
\end{frame}

\begin{frame}[fragile]
\frametitle{ところで}

今日の宴会ってドコなの?

\end{frame}

\section{最近の Debian 関係のイベント}

\takahashi[40]{最近の Debian\\関係のイベント}

\begin{frame}[fragile]
\frametitle{第 50 回関西Debian勉強会}

\begin{itemize}
\item 日時: 8 月 28 日
\item 於: オムロン京都センタービル啓真館
\end{itemize}

\begin{block}{内容}
  \begin{itemize}
  \item モダンな Debian パッケージ作成入門 by 佐々木
  \end{itemize}
  ...掘り炬燵(・∀・)イイ!!
\end{block}
\end{frame}



\begin{frame}[fragile]
  \frametitle{第80回東京エリアDebian勉強会}
  \begin{itemize}
  \item 日時: 9 月 17(土), 18(日)
  \item 温泉合宿(Debian 温泉)
  \end{itemize}
  \begin{center}
    黙黙と開発をするストイックな開発合宿だったようです。
  \end{center}
  \centering

\end{frame}

\takahashi[50]{そんな\\こんなで}
\takahashi[120]{次}

\takahashi[50]{事前課題発表}

\begin{frame}[fragile]
\frametitle{事前課題}

\begin{block}{今回の事前課題}
\begin{enumerate}
\item Debian パッケージの作成手順を復習しておいてください。先月(2011年08月)の勉強会資料が参考になるでしょう。
\item {bzr,git}-buildpackage パッケージがインストールされた環境を用意しておいてください。
\end{enumerate}

\end{block}
\end{frame}

\takahashi[50]{参加者の回答\newline 兼自己紹介}

\begin{frame}{ 山下尊也 }
(無回答)
\end{frame}

\begin{frame}{ Takuspeed83 }
(無回答)
\end{frame}

\begin{frame}{ Y.YATSUO }
(無回答)
\end{frame}

\begin{frame}{ 木下 }
\begin{enumerate}
\item ここしばらく立て込んでいまして・・・\\
すみません。時間取れていません。
\item 遅いノートですが入れときます。\\
自宅マシンにログインしようかな・・・
\end{enumerate}
\end{frame}

\begin{frame}{  清野陽一 }
(無回答)
\end{frame}

\begin{frame}{ 山田 洋平 }
\begin{enumerate}
\item 勉強しておきます。先月は出来たので、大丈夫な、はず。
\item 完了。あ、最初 sid 環境に chroot するの忘れて親環境にインストールしてました...
\end{enumerate}
\end{frame}

\begin{frame}[fragile]{ kozo2 }
\begin{enumerate}
\item 先月の勉強会資料をやり直すことで復習しました。
\item 以下VirtualBoxゲストOSsidの端末出力です。
\begin{commandline}
root@debian:~# lsb_release -a |grep Description
No LSB modules are available.
Description: Debian GNU/Linux unstable (sid)
root@debian:~# aptitude show bzr-buildpackage
No current or candidate version found for bzr-buildpackage
Package: bzr-buildpackage
State: not a real package
Provided by: bzr-builddeb
root@debian:~# aptitude show bzr-builddeb git-buildpackage |egrep 'Package|State'
Package: bzr-builddeb
State: installed
Package: git-buildpackage
State: installed
\end{commandline}
\end{enumerate}
\end{frame}

\begin{frame}{ yabuki@netfort.gr.jp }
ちょっと、遅くなるかも知れませんが行きますので。
\end{frame}

\begin{frame}{ かわだてつたろう }
資料読み直して用意しておきます。
\end{frame}

\begin{frame}{ 川江 }
\begin{enumerate}
\item これはしときます。
\item できるかな?
\end{enumerate}
\end{frame}

\begin{frame}{ 佐々木洋平 }
というわけで、先々月(?) の git-buildpackage 編のおさらいから始めてみます。
\end{frame}

\begin{frame}{ よしだともひろ }
\begin{enumerate}
\item 先月頂いた資料読んでおきます。
\item bzr-buildpackageとgit-buildpackageは入れました。
\end{enumerate}
\end{frame}

\takahashi[50]{そんな\\こんなで}
\takahashi[120]{次}

\section{vcs-buildpackage $\sim$bzrの場合$\sim$}
\takahashi[40]{vcs-buildpackage\\$\sim$bzrの場合$\sim$\\by 山下尊也}

\takahashi[50]{そんな\\こんなで}
\takahashi[120]{次}

\section{vcs-buildpackage $\sim$Gitの場合$\sim$}
\takahashi[40]{vcs-buildpackage\\$\sim$gitの場合$\sim$\\by 佐々木洋平}

\takahashi[50]{そんな\\こんなで}
\takahashi[120]{次}


\begin{frame}[fragile]
\frametitle{今後の予定(1)}

\begin{block}{第 52 回関西 Debian 勉強会}
  \begin{itemize}
  \item 日時: 10 月 23 日(日)
  \item 会場: 福島区民センター
  \item 内容: T.B.D.
    \begin{itemize}
    \item 講演者: T.B.D.
    \end{itemize}
  \end{itemize}
\end{block}

\end{frame}

\begin{frame}[fragile]
\frametitle{今後の予定(2)}

\begin{block}{KOF 2011}
  \begin{itemize}
  \item 日時: 11 月 11 日(金) -- 12 (土)
  \item 会場: 大阪南港 ATC
  \item 内容: T.B.D.
    \begin{itemize}
    \item 講演者: T.B.D.
    \end{itemize}
  \end{itemize}
\end{block}

\end{frame}


\takahashi[50]{  }


\end{document}
%%% Local Variables:
%%% mode: japanese-latex
%%% TeX-master: t
%%% End:
