%; whizzy-master ../debianmeetingresume201101.tex
% 以上の設定をしているため、このファイルで M-x whizzytex すると、whizzytexが利用できます。
%
% ちなみに、クイズは別ブランチで作成し、のちにマージします。逆にマージし
% ないようにしましょう。
% (shell-command "git checkout quiz-prepare")

\santaku
{Debian温泉2011の1日目はいつでしょうか?}
{9/17}
{9/18}
{9/19}
{A}
{さっきの話を聞いていればわかって当然ですね}

\santaku
{8月にDebianは誕生日を迎えました。何周年でしたでしょうか?}
{17}
{18}
{19}
{B}
{今年もお祝いしましたよね}

\santaku
{最新のDebian Newsはいつ発行されたでしょうか?}
{9/17}
{9/18}
{9/19}
{C}
{購読していれば知っていて当然ですね}

\santaku
{10/17の''delegation for the DSA team''で代表団に任命されなかったのは誰でしょう?}
{Faidon Liambotis}
{Luca Filipozzi}
{Nobuhiro Iwamatsu}
{C}
{他に任命されたのは、の全部で合計名です。}

\santaku
{Wheezyフリーズの予定はいつでしょう?}
{2012年4月}
{2012年6月}
{2012年8月}
{B}
{あと6ヶ月ですよ!}

\santaku
{1.16.1がリリースされたdpkgに該当するのはどれ?}
{\texttt{dpkg-buildpackage}コマンドでは \texttt{CFLAGS, CXXFLAGS, LDFLAGS, CPPFLAGS, FFLAGS}の\texttt{export}が必須になった}
{\texttt{dpkg-deb}コマンドに\texttt{--verbose}オプションが追加された}
{Multi-Archフィールドがサポートされた}
{B}
{\texttt{dpkg-buildpackage}ではこれらのオプションが不要になりました。Multi-Archは1.16.2からサポートされる予定です。\texttt{dpkg-deb -x/--extract -v/--verbose}で\texttt{dpkg-deb -X/--xextract}と同じ動きをするようになりました。}
