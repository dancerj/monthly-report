\begin{prework}{ koedoyoshida }

\begin{enumerate}
\item 仕事はさておくとして、自宅環境ではサーバ系はDebian、クライアント系はWindowsにしています。サーバにはXも入れていないのでGUI系操作はすべてWindowsということになります。クライアントでやっていることがDebianで出来るかですが...
  \begin{itemize}
  \item メール:Beckey:以前(Thunderbird 2.X)移行を検討したけれど振り分けルール(日本語(である/でない)のメール判別、複合条件のand/or,fromおよび宛先(to/cc/bcc)判別)が複雑でTBで再現ができず移行困難で断念
  \item ブラウザ:メインブラウザがOpera:速度面でFirefox(iceweasel)より快適。サブはFirefoxとChromeなのでiceweaselとChromiumに移行するのはもっさりすること以外は特に問題なし
  \item エディタ:秀丸:Emacsへ移行すれば問題なし
  \item デバイスコントロール:
    \begin{itemize}
    \item X Video Station:TV録画サーバ:対応ソフトウェアがWinowsXPのみ、アナログ放送が終了したのでお役御免かと思いきや、xvproxy(\footnote{\url{http://xvproxy.local.io/c5/}})とデジアナ変換で寿命が延びた。Garapon TV等へ移行予定。動作監視(EPGチェック)はDebianにて実施中
    \item Garapon TV:ワンセグ録画サーバ:Firefoxで閲覧しているのでiceweaselへ移行可能。動作監視(ハングチェック等、異常時自動再起動)についてはDebianにて実施中
    \item Scansnap:自炊PDFを管理:移行可能らしいが詳細未調査
    \item iPhone\&iPod:音楽および動画ライブラリ:移行可能らしいが詳細未調査
    \end{itemize}
  \end{itemize}
\end{enumerate}
\end{prework}

\begin{prework}{ sheeta38 }

\begin{enumerate}
\item 授業(プログラミング入門1)のレポート作成。 gccと \LaTeX を用いている
\item プログラミング、およびLinuxにもっと触れる。現状、コンピュータに関して知っていることがあまりに少ないと感じており、これらに触れていくことで、少しずつ肩をならしていきたい。

C言語も\LaTeX、あるいは他の言語(Ruby, Python等)も、Windowsで環境を整えようとすれば可能だ。ただしLinuxの方が環境構築が容易であると思う。
他に具体的なLinux(Debian)の利点があるならば知りたい。
\end{enumerate}

\end{prework}

\begin{prework}{ r.matsumiya@syntaxerror.biz }

開発メイン、レポート作成。両方ともDebian使ってます。
\end{prework}

\begin{prework}{ 吉野(yy\_y\_ja\_jp) }

某社製オフィススイートのファイルをレイアウトを壊さず表示・編集したいです。

\end{prework}

\begin{prework}{ キタハラ }

時間ギリギリなので、将来やりたいことで、Linuxだと
やりにくそうなものをあげてみる。(よい方法があれば教えてください。)
\begin{enumerate}
\item 地デジの録画 \\
色々方法があるようですが、コンテンツ保護とかで面倒そう
\item 自炊コンテンツの作成\\
スキャンはできそうだが、OCRとかスキャナ付属のおまけに負けて、Windowsを使いそうな気が……。
\end{enumerate}

\end{prework}

\begin{prework}{ 木村 陽一 }

DebianでApache+SQLite+PHPなWebアプリ開発をしています。そろそろC\#やVB.NETもやりたいと思っていますが、MonoのVB.NETコンパイラは不安定で機能が十分でないのでWindows環境をVMに用意して開発しようと思っています。
\end{prework}

\begin{prework}{ dictoss(杉本 典充) }

\begin{itemize}
\item web閲覧、メール、プログラム開発は仕事も自宅もDebian GNU/Linuxで行っている
\item MIDIキーボード練習はWindowsをひとまず使っている。(Linuxでもできると思うがあまり調べていないため)
\item ビデオ録画(PT2を使用)はFreeBSDで行っている。自宅webサーバがFreeBSDだったので、それに相乗りする形でそのまま使い続けているから
\end{itemize}

\end{prework}

\begin{prework}{ Kiwamu Okabe }

Haskell!Debianが最も適しています!

\end{prework}

\begin{prework}{ 田原悠西 }

コンピュータを使ってやりたいことであり、やっていることは仕事とかインターネットを使ったりとか。Debianを使ってできています。

\end{prework}

\begin{prework}{ Arnaud Fontaine }

I'm a Debian developer who arrived in Tokyo last year and interested in meeting up some other developers. I'm using Debian only at work and home.

\end{prework}

\begin{prework}{ Hirotaka Kawata }

\begin{enumerate}
\item 趣味で Python で簡単なプログラムの作成や、アセンブラ(binutils)やコンパイラ(gcc)の新しいアーキテクチャへの移植。仕事で PHP, Rails や HTML, CSS をつかった Web プログラムの作成。遊びでゲームとか(フライトシミュレータなど)
\item できる
\end{enumerate}

\end{prework}

\begin{prework}{ mitty }

ファイル等のバックアップや重複管理をもっと楽にしたい。

普段使いのPCとは別に、ファイルサーバ(NAS)としてUbuntuを用いているが、バックアップや世代化、ダウンロードしてきたisoイメージなどのファイル重複チェックなど、ほとんど適宜手動でやっているため、ある程度自動化出来ると嬉しい。
\end{prework}

\begin{prework}{ osamu@debian.org }

Debian文書の整理(DDPのXETEX対応)、内容更新 ー 使用理由: 自明

実際は私物の写真の編集・整理ぐらいがDEBIAN「利用目的」です。 ー 他のプラットフォームはすぐ消えていくが、いつも最新システムでいられるしコンンパティビリティーの心配がない。なんと言ってもお金がかからない。こんな所が理由です。

正直なところ、ブラウザーを使う以外は意外とDEBIANを使っていません。Debianは少々荒削りですが、使いやすいのと、今までの習慣で使っています。最初は日本語サポートできるがベースが英語というのが理由でしたが、いまとなってはUBUNTUをなぜ使わないかとなります。まあ、あえていえば自分たちで作っているという気がするのが理由ですかね。
\end{prework}

\begin{prework}{ pi8027 }

\begin{enumerate}
\item \begin{itemize}
\item プログラム、証明、文書などの作成
\item 講義のノートを取る作業、レポートを作成する作業
\item 情報収集
\item 自分自身に関わる色々な物事の管理(予定管理とか)
\item これらの作業の部分的もしくは完全な自動化
\end{itemize}
\item できます
\end{enumerate}

\end{prework}

\begin{prework}{ taitioooo(長谷川) }

家のKVMやXenserverを管理できるセルフポータル整備しようとしています。Linuxでできますが、debianでできません。Centos使ってます。

\end{prework}

\begin{prework}{ 野島 貴英 }

\begin{enumerate}
\item \begin{enumerate}
\item 自分独自仕様のクールなGUIとクールな入出力デバイスを持つクールなPC/モバイル機を作って人に自慢する
\item とにもかくにもクールな爆安PC/モバイル機を作って社会的/経済的に厳しいところでとにかく流行らせる。コンテンツ作らせる。お金まわす
\end{enumerate}
\item いや、その。Debian(Linux)以外だとかなり難易度高いっすよね?\\
※中年になってから、厨二病患ってます...
\end{enumerate}
\end{prework}


\begin{prework}{ 岩松 信洋 }
  \begin{itemize}
  \item やりたいこと
  \begin{enumerate}
  \item DVCam を使った Debianでの Ustream 配信。
  昔はできたんですけどね。サウンド周りを直していません。
  \item 完全自動クロスコンパイル環境の構築。今は一部のパッケージを
  ネイティブでコンパイルされている必要があるので。
  \end{enumerate}
  \item やっていること
  \begin{enumerate}
  \item いつでも、どこでも、誰とでも Debian を使っています。
  \item インターネットを見るのもDebianだし、開発環境もDebianです。
  \end{enumerate}
  \end{itemize}
\end{prework}



\begin{prework}{ まえだこうへい }

  \begin{enumerate}
  \item やりたいことは、自分や家族の生活を便利にすること
  \item こまめ監視カメラなど、部分的には出来ているけど、まだ完全ではない。今の勤務先もCentOSやUbuntuで、Debianは使ってない。前者は技術的(ハードウェア)やリソース(要は自分の時間と金)の問題、後者は政治的、人的な理由
  \end{enumerate}
  Debianだけな生活を送るために、頑張るのです。
\end{prework}

