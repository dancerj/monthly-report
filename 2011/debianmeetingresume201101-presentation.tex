%; whizzy paragraph -pdf xpdf -latex ./whizzypdfptex.sh
%; whizzy-paragraph "^\\\\begin{frame}"
% latex beamer presentation.
% platex, latex-beamer でコンパイルすることを想定。 

%     Tokyo Debian Meeting resources
%     Copyright (C) 2009 Junichi Uekawa
%     Copyright (C) 2009 Nobuhiro Iwamatsu

%     This program is free software; you can redistribute it and/or modify
%     it under the terms of the GNU General Public License as published by
%     the Free Software Foundation; either version 2 of the License, or
%     (at your option) any later version.

%     This program is distributed in the hope that it will be useful,
%     but WITHOUT ANY WARRANTY; without even the implied warreanty of
%     MERCHANTABILITY or FITNESS FOR A PARTICULAR PURPOSE.  See the
%     GNU General Public License for more details.

%     You should have received a copy of the GNU General Public License
%     along with this program; if not, write to the Free Software
%     Foundation, Inc., 51 Franklin St, Fifth Floor, Boston, MA  02110-1301 USA

\documentclass[cjk,dvipdfmx,12pt]{beamer}
\usetheme{Tokyo}
\usepackage{monthlypresentation}

%  preview (shell-command (concat "evince " (replace-regexp-in-string "tex$" "pdf"(buffer-file-name)) "&")) 
%  presentation (shell-command (concat "xpdf -fullscreen " (replace-regexp-in-string "tex$" "pdf"(buffer-file-name)) "&"))
%  presentation (shell-command (concat "evince " (replace-regexp-in-string "tex$" "pdf"(buffer-file-name)) "&"))

%http://www.naney.org/diki/dk/hyperref.html
%日本語EUC系環境の時
\AtBeginDvi{\special{pdf:tounicode EUC-UCS2}}
%シフトJIS系環境の時
%\AtBeginDvi{\special{pdf:tounicode 90ms-RKSJ-UCS2}}

\title{東京エリアDebian勉強会}
\subtitle{第72回 2011年1月度}
\author{上川純一 dancer@debian.org\\IRC nick: dancerj}
\date{2011年1月15日}
\logo{\includegraphics[width=8cm]{image200607/openlogo-light.eps}}

\begin{document}

\frame{\titlepage{}}

\section{}
\begin{frame}
 \frametitle{Agenda}
\begin{minipage}[t]{0.45\hsize}
  \begin{itemize}
  \item 注意事項
	\begin{itemize}
	 \item 飲食禁止
	 \item 宗教禁止
	 \item 営利活動禁止
	\end{itemize}
  \item 最近あったDebian関連のイベント報告
	\begin{itemize}
	 \item 2010年12月勉強会
	\end{itemize}
 \end{itemize}
\end{minipage} 
\begin{minipage}[t]{0.45\hsize}
 \begin{itemize}
  \item Debian勉強会アンケートシステム
  \item kinect
  \item CACert assurance
 \end{itemize}
\end{minipage}
\end{frame}

\begin{frame}
 \frametitle{前回}
\begin{minipage}[t]{0.45\hsize}
  \begin{itemize}
  \item 注意事項
	\begin{itemize}
	 \item 飲食禁止
	 \item 宗教禁止
	 \item 営利活動禁止
	\end{itemize}
  \item 最近あったDebian関連のイベント報告
	\begin{itemize}
	 \item ??
	\end{itemize}
 \end{itemize}
\end{minipage} 
\begin{minipage}[t]{0.45\hsize}
 \begin{itemize}
  \item 2010年のDebian勉強会
  \item CACertの準備
  \item libsane
  \item Debian miniconf
  \item 2011 を妄想
 \end{itemize}
\end{minipage}
\end{frame}


\emtext{イベント報告}

\section{DWN quiz}
\begin{frame}{Debian 常識クイズ}

Debian の常識、もちろん知ってますよね?
知らないなんて恥ずかしくて、知らないとは言えないあんなことやこんなこと、
みんなで確認してみましょう。

今回の出題範囲は\url{debian-devel-announce@lists.debian.org} に投稿された
内容とDebian Project Newsからです。

\end{frame}

\subsection{問題}
%; whizzy-master ../debianmeetingresume201101.tex
% $B0J>e$N@_Dj$r$7$F$$$k$?$a!"$3$N%U%!%$%k$G(B M-x whizzytex $B$9$k$H!"(Bwhizzytex$B$,MxMQ$G$-$^$9!#(B
%
% $B$A$J$_$K!"%/%$%:$OJL%V%i%s%A$G:n@.$7!"$N$A$K%^!<%8$7$^$9!#5U$K%^!<%8$7(B
% $B$J$$$h$&$K$7$^$7$g$&!#(B
% (shell-command "git checkout quiz-prepare")

\santaku
{RC$B%P%0$N8=>u$O$O$I$3$G3NG'$G$-$k$+(B}
{\url{http://bugs.debian.org/release-critical/}}
{\url{http://localhost/}}
{\url{http://debianmeeting.appspot.com/}}
{A}

\santaku
{Debian$BJY6/2qM=Ls%7%9%F%`$N(BURL$B$O$I$l$+(B}
{\url{http://www.2ch.net/}}
{\url{http://atnd.org/events/}}
{\url{http://debianmeeting.appspot.com/}}
{C}

\santaku
{events@debian.org$B$O$I$3$HE}9g$5$l$?$+(B}
{merchants@debian.org}
{hoge@debian.org}
{fuga@debin.org}
{A}

\santaku
{antiharassment@debian.org $B$N$&$i$K$$$J$$$N$OC/$+(B}
{Amaya Rodrigo Sastre}
{Patty Langasek}
{Kouhei Maeda}
{C}

\santaku
{$B8=:_$$$/$D$+$N(BSprint$B$,3+:E$5$l!"4k2h$5$l$F$$$k!#8=:_4k2h$9$i$5$l$F$$$J(B
$B$$(Bsprint$B$O$I$l$+(B}
{-www sprint}
{security sprint}
{tokyo sprint}
{C}

\santaku
{DACA$B$O$I$3$r8+$l$P$h$$$+(B}
{\url{http://qa.debian.org/daca/}}
{\url{http://daca.debian.org/}}
{\url{file:/tmp}}
{A}

\santaku
{DEP $B$O2?$NN,$+(B}
{Debian Enhancement Proposal}
{Device Enhancement Protocol}
{$B$G$C$W(B}
{A}

\santaku
{DEP5$B$GDs0F$5$l$F$$$k(Bdebian/copyright$B$N5!3#2DFI7A<0$O$I$&$$$&$b$N$+(B}
{S$B<0(B}
{RFC822$BIw(B}
{XML}
{B}



\emtext{prework}

{\footnotesize
 %; whizzy-master ../debianmeetingresume201101.tex
% $B0J>e$N@_Dj$r$7$F$$$k$?$a!"$3$N%U%!%$%k$G(B M-x whizzytex $B$9$k$H!"(Bwhizzytex$B$,MxMQ$G$-$^$9!#(B

\begin{prework}{$B>e@n(B $B=c0l(B}

\preworksection{2011$BG/$NJY6/2q$G$J$7$H$2$?$$$3$H$r@k8@$7$F$/$@$5$$(B }

$B:#G/$O(BLAMP $B%9%?%C%/!"(BKVS$B$J$I!"%/%i%&%I%$%s%U%i$r(BDebian$B$G0lDL$j9=C[$9$k;E(B
$BAH$_$r<B8=$9$kJ}K!$K$D$$$F0lDL$jJY6/$7$F4XO"$7$?%Q%C%1!<%8%s%0$K$^$D$o$k(B
$BLdBj$K$D$$$F@0M}$7$?$$!#(B

\preworksection{2015$BG/$N(BDebian$B$,$I$&$J$C$F$$$k$+BgC@$KLQA[$7$F$/$@$5$$!#(B}
 
Debian$B$O$^$@@8$-;D$C$F$$$F!"(BSqueeze+2$B$N%j%j!<%9=`Hw$GK;$7$$!#(B
$B%G%P%$%9$O7HBSEEOC$H%?%V%l%C%H%G%P%$%9$N%5%]!<%H$,DI2C$5$l$F$*$j!"(B
$B%a%$%s%9%H%j!<%`$N(BCPU$B%"!<%-%F%/%A%c$O(Bx86\_64 $B$H(B arm$B!#(B

$B%G%P%C%0MQ$N>pJs$,%G%U%)%k%H$GA4It$N%Q%C%1!<%8$K$D$$$FDs6!$5$l$k!#(B

$BA4(BVCS$B$r6&DL$GMxMQ$G$-$k$h$&$J%$%s%?%U%'!<%9$,@0Hw$5$l!"(B
Debian$B$N%=!<%9%Q%C%1!<%8$rI8=`E*$J(BVCS$B$H$7$FDs6!$5$l$k$h$&$K$J$k!#(B

Debian$B$N%5!<%P%j%=!<%9$O(BP2P$B%*!<%P%l%$%M%C%H%o!<%/7PM3$GDs6!$5$l!"(B
$B6&M-$N%S%k%I%5!<%P$J$I$,MxMQ$G$-$k$h$&$K$J$C$F$$$k!#(B

\end{prework}

\begin{prework}{$B$^$($@$3$&$X$$(B}

\preworksection{2011$BG/$NJY6/2q$G$J$7$H$2$?$$$3$H$r@k8@$7$F$/$@$5$$!#(B}

 $B%9%^!<%H%U%)%s4XO"$N(BWeb$B%"%W%jMQ%i%$%V%i%j$H%M%$%F%#%V%"%W%jJQ49%D!<%k$J(B
 $B$I$N3+H/4D6-$r(BDebian$B$G@0Hw$9$kJ}K!!"2?$,$I$&M%$l$F$$$k$+$rJY6/$7$?$$!#(B
 Debian$B$G4D6-$r@0$($k0Y$K$O$I$l$,(BDebian$B%Q%C%1!<%8(B
 $B$K$J$C$F$$$k$N$+!"$I$l$OBP1~2DG=$G!"$I$l$,IT2DG=$J$N$+!#(BDebian$B>e$G@8@.(B
 $B$7$?%"%W%j$r(BDebian$B$+$iG[?.$9$k!"$"$k$$$O4{B8$NG[?.$N;EAH$_$HO"F0$9$k$?(B
 $B$a$N;EAH$_$O$G$-$J$$$+!#(B

\preworksection{2015$BG/$N(BDebian$B$,$I$&$J$C$F$$$k$+BgC@$KLQA[$7$F$/$@$5$$!#(B}

 $B0U30$H(BWheezy$B$N<!$N%j%j!<%9$,=P$F$7$^$C$F$$$k!#%9%^!<%H%U%)%s8~$1$N%$%s(B
 $B%9%H!<%i$,EP>l$7!"(Bjail break$B$d(Brooted$B$7$J$/$F$b!"4JC1$K(BDebian$B$r%$%s%9%H!<(B
 $B%k$G$-$F$7$^$&!#(BwebOS$B$,(Bipkg$B$r$d$a$F!"(BDebian$B%Q%C%1!<%8$r;H$&$h$&$K@hADJV(B
 $B$j$7$F$$$k(B(webOS$B$K$O5/;`2s@8%[!<%`%i%s$G@8$-;D$C$F$$$F$[$7$$(B)$B!#$=$b$=$b(B
 2015$BG/$K$O%9%^!<%H%U%)%s$H$+%/%i%&%I$H$+$$$&8@MU$O(B $B>C$($F!"$^$?JL$N?7$7(B
 $B$$%P%:%o!<%I$KJQ$o$C$F$$$k$K0c$$$J$$!#(B

\end{prework}

}

\emtext{Debian勉強会アンケートシステム}

\begin{frame}{アンケートシステムの使い方(管理者)}

\includegraphics[width=0.8\vsize]{image201101/enquete-edit.png}

\end{frame}

\begin{frame}{アンケートシステムの使い方(ユーザ)}
 \includegraphics[width=0.8\vsize]{image201101/enquetemail.jpg}
\end{frame}

\begin{frame}{アンケートシステムの使い方(ユーザ)}
 \includegraphics[width=0.8\vsize]{image201101/enquete-respond.png}
\end{frame}


\begin{frame}{アンケートシステムの設計と実装}
 \begin{itemize}
  \item Debian勉強会予約システムの1モジュール
  \item シンプルな設計を目指す
 \end{itemize}
\end{frame}

\begin{frame}[containsverbatim]{アンケートシステムの設計と実装(データストア)}
\begin{commandline}
class EventEnquete(db.Model):
    """Enquete questions for an event."""
    eventid = db.StringProperty()
    overall_message = db.TextProperty()
    question_text = db.StringListProperty()
    timestamp = db.DateTimeProperty(auto_now_add=True)

class EventEnqueteResponse(db.Model):
    """Enquete respnose for an event by one person."""
    eventid = db.StringProperty()
    # responses for 1-5 questions. 0 is N/A
    question_response = db.ListProperty(long)
    overall_comment = db.TextProperty() # a general comment from user.
    timestamp = db.DateTimeProperty(auto_now_add=True)
    user = db.UserProperty()
\end{commandline}

\end{frame}

\begin{frame}{アンケートシステムの設計と実装(テンプレート)}
 \begin{itemize}
 \item EnqueteAdminEdit.html 管理者用のアンケート編集画面
 \item EnqueteAdminSendMail.txt 管理者が参加者にアンケートを依頼するメー
       ル送信するときのメールのテンプレート
 \item EnqueteAdminShowEnqueteResult.txt アンケートの結果をCSV形式で表示
       する。
 \item EnqueteRespond.html 参加者がアンケートを返答する際に表示される
       HTML。
 \item EnqueteRespondDone.txt 参加者がアンケートを回答したときに送信され
       る確認メール。
\end{itemize}
\end{frame}

\begin{frame}{アンケートシステムの設計と実装(集計)}
 
 集計はシステムで面倒をみないで、CSVで出力するだけ。

\end{frame}

\begin{frame}[containsverbatim]{先月のアンケート結果}

\begin{commandline}
$ R
[中略]
> enquete <- read.csv('201012enquete.csv')
> summary(enquete)
  事前課題紹介   X2010年のDebianを振り返って.2011年を企画する
 Min.   :0.000   Min.   :0.000                               
 1st Qu.:3.000   1st Qu.:3.000                               
 Median :4.000   Median :4.000                               
 Mean   :3.333   Mean   :3.556                               
 3rd Qu.:4.000   3rd Qu.:5.000                               
 Max.   :4.000   Max.   :5.000                               
 CACertの準備に何が必要か 俺のlibsaneが火をふくぜ Debian.Miniconf.企画
 Min.   :4.000            Min.   :0.000           Min.   :1.000       
 1st Qu.:4.000            1st Qu.:4.000           1st Qu.:2.000       
 Median :5.000            Median :5.000           Median :3.000       
 Mean   :4.556            Mean   :3.667           Mean   :2.778       
 3rd Qu.:5.000            3rd Qu.:5.000           3rd Qu.:4.000       
 Max.   :5.000            Max.   :5.000           Max.   :4.000       
\end{commandline}

\end{frame}

\begin{frame}[containsverbatim]{相関を見る}
 \includegraphics[width=0.4\hsize]{image201101/oreno-cacert.png}

 \begin{commandline}
 > cor(enquete$CAC, enquete$俺の)
 [1] 0.186339
 > cor(enquete$俺の, enquete$事前)
 [1] 0.7572402
 > cor(enquete$CAC, enquete$事前)
 [1] -0.2988072
 \end{commandline}

データセットの性質を考えるとこれはちょっと無理があるか・・・

\end{frame}

\emtext{Kinect on Debian GNU/Linux}
\emtext{CACert Assurance}

\begin{frame}{今後のイベント}
 
\begin{itemize}
 \item 2月のDebian勉強会は神保町
 \item 3月 OSC
 \item CACert keysigning
 \item OSC Kobe
\end{itemize}
\end{frame}

\begin{frame}{今日の宴会場所}

「魚こう」

\end{frame}

\end{document}

;;; Local Variables: ***
;;; outline-regexp: "\\([ 	]*\\\\\\(documentstyle\\|documentclass\\|emtext\\|section\\|begin{frame}\\)\\*?[ 	]*[[{]\\|[]+\\)" ***
;;; End: ***
