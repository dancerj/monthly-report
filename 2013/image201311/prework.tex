%; whizzy-master ../debianmeetingresume201311.tex
% 以上の設定をしているため、このファイルで M-x whizzytex すると、whizzytexが利用できます。
%

\begin{prework}{ 野首 }

後発でありながら優位性のみられないMirに意味があるとすれば、それはCanonicalがコントロール可能な点だけなのかなという気がします。

\end{prework}

\begin{prework}{ dictoss(杉本 典充) }

調べてきてWaylandがX.Orgの後継として設計されて、WaylandからforkしたものがMirらしいということがわかった。過去との互換性をもっているWaylandのほうがdebianとしては採用しやすい分勝っているのかなと思う。

\end{prework}

\begin{prework}{  清野陽一 }

普段あまり意識したことがなかったので、これを気に勉強できればと思います。
\end{prework}

\begin{prework}{ mtoshi }

さっぱり分かりません(涙)
\end{prework}

\begin{prework}{ まえだこうへい }

名前しか耳にしたことがないので、さっぱり分からない。
\end{prework}

\begin{prework}{ 野島 貴英 }

waylandも頑張った!mirは自分はよくわからんが、頑張ってる!Xも負けてない!
いやー、こちらの戦いは目がはなせませんネー。何事も競争相手がいるって良いことですネー。
mirとの比較は検討したことないので、違いについて誰かよろしくおねがいします。いずれにしても、自分としては、組み込み含めて簡単に遊べそうだし、中身すっごいわかりやすい実装である、waylandとしばらく戯れようと思ってます。

※mirは調べてないよ?

\end{prework}

\begin{prework}{ 上川純一 }

x11のプロトコルは改良の余地があると思うので頑張って開発が進み競争があるのは好ましいと思います。
\end{prework}

\begin{prework}{ 吉野(yy\_{}y\_{}ja\_{}jp) }

Wayland になってほしいです.
\end{prework}
