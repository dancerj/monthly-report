\documentclass[cjk,dvipdfmx,10pt,compress,%
hyperref={bookmarks=true,bookmarksnumbered=true,bookmarksopen=false,%
colorlinks=false,%
pdftitle={第 71 回 関西 Debian 勉強会},%
pdfauthor={倉敷・のがた・佐々木・かわだ・八津尾},%
%pdfinstitute={関西 Debian 勉強会},%
pdfsubject={資料},%
}]{beamer}

\title{第 71 回 関西 Debian 勉強会}
\subtitle{$\sim$発表資料$\sim$}
\author[かわだ てつたろう]{{\large\bf 倉敷・のがた・佐々木・かわだ・八津尾}}
\institute[Debian JP]{{\normalsize\tt 関西 Debian 勉強会}}
\date{{\small 2013 年 4 月 28 日}}

%\usepackage{amsmath}
%\usepackage{amssymb}
\usepackage{graphicx}
\usepackage{moreverb}
\usepackage[varg]{txfonts}
\AtBeginDvi{\special{pdf:tounicode EUC-UCS2}}
\usetheme{Kyoto}
\def\museincludegraphics{%
  \begingroup
  \catcode`\|=0
  \catcode`\\=12
  \catcode`\#=12
  \includegraphics[width=0.9\textwidth]}
%\renewcommand{\familydefault}{\sfdefault}
%\renewcommand{\kanjifamilydefault}{\sfdefault}
\begin{document}
\settitleslide
\begin{frame}
\titlepage
\end{frame}
\setdefaultslide

\begin{frame}[fragile]
\frametitle{Agenda}

\tableofcontents

\end{frame}

\section{最近の Debian 関係のイベント}

\takahashi[40]{最近の Debian\\関係のイベント}

\begin{frame}[fragile]
  \frametitle{第 70 回関西 Debian 勉強会}
  \begin{itemize}
  \item 日時: 3 月 24 日(日)
  \item 場所: 港区民センター
  \end{itemize}
  \begin{block}{内容}
    \begin{itemize}
    \item 「UbuntuとGNOME Shellと私」
    \item 「管理者視点からのGNOMEの大規模な配置」
    \end{itemize}
  \end{block}
\end{frame}

\begin{frame}[fragile]
  \frametitle{第 99 回 東京エリア Debian 勉強会}
  \begin{itemize}
  \item 日時: 4 月 20 日(土)
  \item 場所: あんさんぶる荻窪 第一教室
  \end{itemize}
  \begin{block}{内容}
    \begin{itemize}
    \item 「debootstrapを有効活用してみよう」
    \item 「Debianの認証をWindowsに統合してみたり」
    \item 「PowerPC の Vector Facility や Vector-Scalar Floating-Point Operation の Opcode
Map を調べてみた。」
    \end{itemize}
  \end{block}
\end{frame}

\begin{frame}[fragile]
  \frametitle{Debian Project}
  \begin{itemize}
  \item 新プロジェクトリーダ、Lucas Nussbaum さん
  \item DebConf14 は、アメリカ、オレゴン州ポートランドで開催
  \item 5 月 4 日、5 日の週末に wheezy をリリース予定
  \end{itemize}
\end{frame}

\takahashi[50]{そんな\\こんなで}
\takahashi[120]{次}

\section{事前課題発表}

\takahashi[50]{事前課題}

\begin{frame}[fragile]
  \frametitle{事前課題}
  \begin{block}{今回の事前課題}
    \begin{description}
    \item[事前課題1] 
      「Debian 7.0 -- リリースノート」を読んできてください。
      
      誤植や誤訳などの気づいたこと、わからないなと思うことなどがあれば教え
      てください。
    \item [事前課題2]%
      Debian をこれから使ってみようとされている方は Debian もしくは wheezy
      に対して気になっていることがあれば教えてください。
    \item [事前課題3]%
      Debian をすでにお使いの方は squeeze から wheezy へのアップグレードで
      はまりそうなところ、気になるところがあれば教えてください。
      
      もしくは、お使いの squeeze 環境を一つ、wheezy にアップグレードしてみ
      て下さい。そしてその結果を教えてください。
    \item [事前課題4]%
      クラウド的なものをさわったことのない方は、(あれば)理由を教えてくださ
      い。ある人は、セッション中にいろいろツッコミしてください。
    \end{description}
  \end{block}
\end{frame}

\takahashi[50]{事前課題\\発表}

\begin{frame}
  \frametitle{ kozo2 }
  \begin{enumerate}
  \item %
    すみません着くまでに読みます
  \item %
    \begin{description}
    \item[Debianに対して] \\
      Firefoxをinstallする方法はppaを追加するしか無いのでしょうか。
    \item[wheezyに対して] \\
      たしかkernelのversionが3.2あたりではなかったでしょうか。
      sidにせずwheezyのままkernelやその他のpackageを新しくすることは可能でしょうか。
      というのはsidではskypeのinstallができなかった記憶があるので。
    \end{description}
  \item %
    いつもsidにしていたので書けません。すみません。
  \item %
    あるのでセッション中に伺います。AWSはすごく興味があるので。研究系の現場でもAWSを使った方がいい状況になっていると思うのですが恐らく日本ではAWSにお金を使わせてもらっているところはまだ(ほとんど)無いのではないかと思うので。
  \end{enumerate}
\end{frame}

\begin{frame}
  \frametitle{ 佐々木洋平 }

  \begin{enumerate}
  \item %
    発表に備えて \texttt{git svn clone ...} と叩いたら, なかなか clone 終わらなくて歴史を感じている次第です.
  \item %
    普段から使っておりますので...
  \item %
    リモートでデスクトップの upgrade をしたら, 起動しなくなる事案が発生しました.
    %
  \item %
    VPS として S@@Ses のVPS を複数台使っているぐらいですが, これはクラウドなんでしょうか?
    Xen 環境なのでカーネルを上げられないのがちょっと辛いですが, それ以外はまあなんとか.
    というか,
    そもそもクラウドって何ですか?

    AWS 自体は使ってみたいとは思ってましたが, 以前の状況では予算の都合で諦めました.
    そのうち触ってみようと思います.
  \end{enumerate}
\end{frame}

\begin{frame}\frametitle{ かわだてつたろう }
  \begin{enumerate}
  \item %
    これから読みます。
    \setcounter{enumi}{2}
  \item %
    Debian 以外のパッケージを使用していると multiarch, /run 周りではまりそうな気がします。
  \item %
    特に理由は無いのですが、さわったことがないです。
  \end{enumerate}
\end{frame}

\begin{frame}
\frametitle{ 山城の国の住人 久保博 (1/2)}
  \begin{enumerate}
  \item %
    はい、これから読みます。
  \item %
    もう使いましたので…
  \item %
    気になることをいくつか挙げます.
    \def\theenumii{\alph{enumii}}
    \def\labelenumii{\theenumii.~}
    \begin{enumerate}
    \item %
      \texttt{/etc/X11/xorg.conf} が存在したら、アップグレード前に名前を変えて取っておく方が良いと思います。おそらく、そのまま存在すると却って邪魔です。
    \item %
      X の反応が却って悪くなりました。カーネルモードのドライバに移行したからでしょうか。
    \item %
      カーネルまわりで多くのバグレポートが残っているので、古めのハードウェアやちょっと変わったデバイスがあったりするとはまるかも。バグレポートは見ておくといいかもしれません。 私は一つ踏んづけました。
    \end{enumerate}
  \end{enumerate}
\end{frame}

\begin{frame}
\frametitle{ 山城の国の住人 久保博 (2/2)}
  \begin{enumerate}
    \setcounter{enumi}{2}
  \item %
    \def\theenumii{\alph{enumii}}
    \def\labelenumii{\theenumii.~}
    \begin{enumerate}
      \setcounter{enumii}{3}
    \item %
      ノートPCで、時々ブート途中にコンソールが消えて画面が真っ暗になるようになりました。dm-crypt のパスワードを入力する所なので、困ってます。
    \item %
      GNOME3 初心者は、dist-upgrade 後に reboot する前に、 gnome-shell-extensions パッケージをインストールすることをお勧めします。
    \item %
      ruby 関係の移行には気をつけたい
    \item %
      dist-upgrade が disk fullで途中で止まっても諦めるな。要らないものを消して、 -f オプションで切り抜けろ。
    \end{enumerate}
  \item %
    クラウド的なものはほとんど使っていません。新しいものに適応する力が弱っているので。
  \end{enumerate}
\end{frame}

\begin{frame}
\frametitle{ 甲斐正三 }
  事前課題にお答えします。
  \begin{enumerate}
  \item %
    kernel upgrade についても触れていただきたい。
  \item %
    特にありません。
  \item %
    Debian Squeeze上のqemu-kvmにDebian Squeezeの
    ベースシステムとデスクトップシステムを新規インストー
    ルし、これをWheezyにアップグレードしてみました。
    気付いた点は以下の3点でした。
    \def\theenumii{\arabic{enumii}}
    \def\labelenumii{(\theenumii)~}
    \begin{enumerate}
    \item %
      upgrade中、パネル上のepiphanyのアイコンが消えて、グレーの四角になる。
    \item %
      「GNOME3の読み込みに失敗しました。」メッセージが出て
      フォールバックする。VM上のwheezy/sidも同様。
    \item %
      kernelが、'3.2.0-2-amd64'にならず、リリースノートに
      その説明もない。'Linux kvm-sqz 2.6.32-5-amd64'のまま。
      'dist-upgrade'したのち、'apt-get install linux-image-3.2.0-2-amd64'で
      インストールした。インストール成功。
    \end{enumerate}
  \item %
    触ったことがありません。
    理由:差し迫った必要性がない。
  \end{enumerate}
\end{frame}

\begin{frame}
  \frametitle{ 川江 }
  \begin{enumerate}
  \item %
   細かく見ていませんが、ないような気がします。
 \item %
   compizは使えるようになりました?
 \item %
   もともと、wheezyなのでどうしましょう?
 \item %
   これから「自宅クラウド」を作ろうと思っていますので。
 \end{enumerate}
\end{frame}

\begin{frame}
  \frametitle{ 大林 }
  \begin{enumerate}
  \item %
   一通り目を通しました。行くまでにもう少しじっくり読みます。
   \setcounter{enumi}{2}
 \item %
   なにかありますかねえ。以前squeezeへアップグレードしたときには特に問題なくできたのでなんとかなる気がしています。
 \item %
   つかったことはありません。
 \end{enumerate}
\end{frame}

\begin{frame}
  \frametitle{ 西山和広 (1/3)}
  \begin{enumerate}
  \item
    \begin{itemize}
    \item %
      ch-upgrading で sources.list だけではなく sources.list.d への言及もあった方が良いのではないかと思いました。
    \item %
      4.5.8. コンソール接続へセッションの変更:

        異なったテキストモードのターミナル間で切り替えを行うには、Alt+左矢印 か Alt+右矢印 も使えます。

      \noindent
      「異なった」は「ターミナル」にかかると思うのですが、「テキストモード」にかかっているように見えてしまいます。
    \item %
      4.5.9.1. Sudo:

      「visudo -f」はおすすめなので、もっと知られると良さそうです。
    \item %
      4.5.9.2 Screen:

      tmux はどうなのか気になりました。
    \item %
      4.8. 廃止予定のコンポーネント:

      まだないのは後で書かれる予定でしょうか?
    \end{itemize}
  \end{enumerate}
\end{frame}

\begin{frame}
  \frametitle{ 西山和広 (2/3)}
  \begin{enumerate}
  \item
    \begin{itemize}
    \item %
      4.9. 時代遅れ (Obsolete) のパッケージ:

        portmap: 後継となるパッケージは rpcbind です。

      \noindent
      /etc/hosts.allow などで tcp wrapper の設定をしている場合は変更が必要なのかどうかが気になりました。
    \item %
      5.2. LDAP サポート:

      「 libnss-ldap パッケージを、すべての LDAP 検索に分割されたデーモン (nslcd) を使っている新しいライブラリである libnss-ldapd に置き換えることをお勧めします。libpam-ldap の代替品は libpam-ldapd です。 」
      のところがわかりにくい気がしました。
    \item %
      5.10. pdksh から mksh への移行:

      PS1 が pre になっているのは意図的なのでしょうか? (原文でも同じ)
    \item %
      6.2. 助けを求めるには:

      日本語訳には訳注で debian.or.jp の情報を付け足しても良いのではないかと思いました。
    \end{itemize}
  \end{enumerate}
\end{frame}

\begin{frame}
  \frametitle{ 西山和広 (3/3)}
  \begin{enumerate}
  \item 
    \begin{itemize}
    \item %
      A.3. 古く不要になった設定ファイルを削除する:

      「*.dpkg-\{new,old\}」の他に「smb.conf.ucf-dist」のような「*.ucf-*」も書いてあった方が良いのではないかと思いました。
    \end{itemize}
    \setcounter{enumi}{2}
  \item %
    Ubuntu 10.04 から Ubuntu 12.04 で nslcd の変更により 1 文字のユーザー名や smbldap-tools で設定された空白の入ったグループ名 (Domain Users など) が見えなくなっていて、 /etc/nslcd.conf で

\# allow name with only 1 char and spaces (e.g. "Domain Users")
validnames  /\^[a-z0-9.\textbackslash\_\{\}@\$]([ a-z0-9.\textbackslash\_\{\}@\$\textbackslash\textbackslash\~-]*[a-z0-9.\textbackslash\_\{\}@\$\~-])?\$/i

    \noindent
    という設定を入れたので、 squeeze から wheezy へのアップグレードでも同じ問題にはまりそうだと思いました。
  \end{enumerate}
\end{frame}

\begin{frame}
  \frametitle{ おくの }
  \begin{enumerate}
  \item %
    読みました。リリースが楽しみです。
  \item %
    Multiarchが気になります。
  \item %
    新規インストールするつもりなので……
  \item %
    四月からクラウドをやるかもしれない会社に転職したので、勉強中ですです。
  \end{enumerate}
\end{frame}

\begin{frame}
  \frametitle{ lurdan }
  \begin{enumerate}
  \item %
   ざっくり読みましたが、特に気になる点はありませんでした。一部未訳部分がありますが、更新中なので仕方ありませんね。
 \item %
   もう使ってます。
 \item %
   特に気になることはありません。
 \item %
   ツッコミお待ちしております
 \end{enumerate}
\end{frame}

\takahashi[50]{そんな\\こんなで}
\takahashi[120]{次}

\section{クラウド初心者が AWS に Debian をのっけて翻訳サービスの試行に挑戦してみた}
\takahashi[30]{クラウド初心者が\\ AWS に\\ Debian をのっけて\\翻訳サービスの試行に\\挑戦してみた\\by\\倉敷 悟}

\takahashi[50]{そんな\\こんなで}
\takahashi[120]{次}

\section{リリースノートを読んでみよう}
\takahashi[30]{リリースノートを\\読んでみよう\\by\\佐々木洋平}

\takahashi[50]{そんな\\こんなで}
\takahashi[120]{次}

\section{今後の予定}
\begin{frame}[fragile]
\frametitle{今後の予定 (1)}

\begin{block}{第 72 回関西 Debian 勉強会}
  \begin{itemize}
  \item 日時: 5 月 26 日(日)
  \item 会場: 福島区民センター 302号室
  \item 内容: 新年度入門ネタ

    Wheezy インストール大会

    月刊 Debian Policy
  \end{itemize}
\end{block}

\begin{block}{第 100 回東京エリア Debian 勉強会}
  \begin{itemize}
  \item 日時: 5 月 18 日(土)
  \item 会場: 荻窪地域区民センター
  \item 内容: 未定
  \end{itemize}
\end{block}

\end{frame}

\begin{frame}[fragile]
\frametitle{今後の予定 (2)}

\begin{block}{大統一 Debian 勉強会}
  \begin{itemize}
  \item 日時: 6 月 29 日(土)
  \item 会場: 東京 日本大学 駿河台キャンパス
  \item 公式サイトオープン \url{http://gum.debian.or.jp/}
  \item Call For Presentation の締め切りは今日!
  \end{itemize}
\end{block}

\end{frame}


\takahashi[50]{  }


\end{document}
%%% Local Variables:
%%% mode: japanese-latex
%%% TeX-master: t
%%% End:
