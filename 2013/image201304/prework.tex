%; whizzy-master ../debianmeetingresume201304.tex
% 以上の設定をしているため、このファイルで M-x whizzytex すると、
% whizzytexが利用できます

\begin{prework}{ べつやく }

\preworksection{最近Debianのアカウント・システム管理で考えていること}

ユーザアカウントの管理について特に意識したことがありません。ログイン時の
 パスワードを共通化できる仕組みがあるらしい・・・ぐらいの認識です。

\end{prework}

\begin{prework}{ koedoyoshida }

\preworksection{最近Debianのアカウント・システム管理で考えていること}

会社のサーバはLinux系で統一されているので、uid,gid,アカウント名をそれに合わせる程度です。
ldap連携も可能ですが、自分の管理しているマシンでは(社内向けのサービスを
 提供していないので)行っていません。

\preworksection{Debianでの開発環境をどう管理しているか}

stable環境にchrootでsid環境を作るorVMware環境上へ構築しています。

\preworksection{SambaをWindowsのドメインに参加させたことはありますか。}

「SambaをWindowsのドメインに参加させたこと」、「Linuxのユーザ認証をWindowsに統合したこと」はありません。
逆にsambaをDCとしてwindowsクライアントを参加させたことはあります。
\end{prework}

\begin{prework}{ ribbon@ns.ribbon.or.jp }

\preworksection{SambaをWindowsのドメインに参加させたことはありますか。}
あります。
\preworksection{参加させたことがある方、ハマったところを教えてください}
バージョンによって挙動が違うところです。
\end{prework}

\begin{prework}{ dictoss(杉本 典充) }

\preworksection{最近Debianのアカウント・システム管理で考えていること}

サーバで使っていても自分しか使用しないので、個人アカウントを作って完
 結してしまっている。

\preworksection{Debianでの開発環境をどう管理しているか}

amd64マシンを使っているので、i386環境はchrootで入れるように設定している。tarballを持ってきたアプリがビルド時にunameするものが一部あり、うまくビルドが通らないな場合を考えてKVM環境も併用している。ただ、ネットワーク関係のテストをするときはiptablesを使う場合が多いので最初からKVM環境で作って試している。
\end{prework}

\begin{prework}{ henrich }

\preworksection{最近Debianのアカウント・システム管理で考えていること}

Fedoraのようにユーザーがアカウントを取得して、ウェブからアクセスできる様々な活動ができるといいな、と思っています。例: https://admin.fedoraproject.org/updates

\preworksection{Debianでの開発環境をどう管理しているか}

どう管理、というのがちょっと意味が掴み取れません。普通にcowbuilderとpiupartsとlintian使ってるぐらいですが、これをどのように変更していけば効率の良い環境になるのかは知りたい所です。
\end{prework}

\begin{prework}{ 野首 }

\preworksection{最近Debianのアカウント・システム管理で考えていること}
Debianのアカウント管理は、自分の周りのマシンではいまだ/etc/passwdベース
 ばかりです。

\preworksection{Debianでの開発環境をどう管理しているか}
開発環境の管理としては、パッケージにsvn-buildpackageとgit-buildpackageを併用しています。特別使い分けたわけではなく、後になってgitを覚えてからgit-buildpackageを使ってみただけなので、これに特別な利点があったりはしません。むしろ設計思想が全然違うので一本化できず困っています。
あとはsid環境にlxcを使ってみたり、armやsparcのテストのためにqemuを使い始めたりしている程度です。

\end{prework}

\begin{prework}{ 石井一夫 }

\preworksection{最近Debianのアカウント・システム管理で考えていること}

セキュリティは結構深刻で、内部の者も悪さをする恐れがあり、かなり、気を使います。サーバでは、一般ユーザにsudo権限を取得できないようにし、管理者用アカウントを作ってwheel のグループに所属させ、それだけが、sudo権限を取得できるようにしています。パスワードは、3ヶ月ごとに変更。Kerberosは気になりますが、使いこなせていません。

\preworksection{Debianでの開発環境をどう管理しているか}

GCCとか、JDKとかインストールできるものは、可能な限りインストールしていますが、環境管理とまでは行っていません。これからの課題です。
\end{prework}

\begin{prework}{ 上川純一 }

\preworksection{最近Debianのアカウント・システム管理で考えていること}

個人の環境は自分と目的用途別に作っているアカウントで、自分のアカウント以
外はインタラクティブに利用する用途ではないので基本的にはパスワードを設
定せず利用しています。

昔はファイル共有でNFSを使っていたのでシステム間でUIDを一致させないといけないなどの面
倒がありましたが最近はsshfs とか rsync/ssh でファイルを共有するので片方向
の鍵認証でファイルのアクセス権が決まるモデルになっています。こっちのほう
が管理者としてはやりやすいですね。

\preworksection{Debianでの開発環境をどう管理しているか}

cowbuilderでamd64 sidの環境のみを維持管理しています。昔はいろいろなCPUアー
 キテクチャーとかOSをとかをためしていたので大量のイメージがありました。
cowbuilder だと常に最新版のバイナリを使うことになります。また、必要なソ
 フトウェアを必要なときにJust In Timeでインストールしてくれることになり
 ます。
OSイメージの構成・履歴管理などを省略してくれるのでいいです。

開発環境の仮想マシンを管理するとどういうメモリを割り当てるかとかネットワー
クの設定をどうするかだとかというのも心配することになるんですが、
cowbuilderではホストOSの環境をそのままつかっているので楽です。開発環境の
メンテナンスが目的ではなく開発環境を使うことが目的なので開発環境はメンテ
ナンスフリーなのが理想だと思います。

各種クラウドサービスを使うとOSイメージのインストールからさせられることが
多いのですがそこではじめていかに世間の人が環境の維持管理を重要なタスクだ
と思っていて、cowbuilderが楽なのかを認識した気がします。

\end{prework}

\begin{prework}{ まえだこうへい }

\preworksection{最近Debianのアカウント・システム管理で考えていること}

Debianの、というかLDAPに代わるもっとシンプルなアカウント管理の仕組みを作りたいなと思っているだけで何もやってません。

\preworksection{Debianでの開発環境をどう管理しているか}

Pythonでのツールの開発はのSidを使っていますが、Sid自体はVirtual Box上にあるので、Shutdown時に都度snapshotを取っています。

\preworksection{SambaをWindowsのドメインに参加させたことはありますか。}

某所でSamba使っていましたが、某所のADには参加できない(その某所とは別の法人組織だったので)が、アクセスは某所のPCでしかできないので、ユーザアカウントを同一にするため、某所のシステムからユーザアカウント引っこ抜いてLDIFに変換してLDAPでアカウント管理してました。(Sambaとは直接連携してない)
\end{prework}

\begin{prework}{ 吉野(yy\_{}y\_{}ja\_{}jp) }

\preworksection{最近Debianのアカウント・システム管理で考えていること}

スタンドアロンで使うとき,VMで使うときがありますがどちらも考える必要を感じてません.

\preworksection{Debianでの開発環境をどう管理しているか}

どちらの場合も特に何もしてないです.

\preworksection{SambaをWindowsのドメインに参加させたことはありますか。}
ありません.
\preworksection{Linuxのユーザ認証をWindowsに統合したことはありますか。}
ありません.

\end{prework}

\begin{prework}{ 岩松 信洋 }

\preworksection{最近Debianのアカウント・システム管理で考えていること}

特にアカウント管理に関して考えてないです。
昔ながらのユーザとグループで管理しています。

\preworksection{Debianでの開発環境をどう管理しているか}

chroot と alias, シェルスクリプトによる切り替えを使っています。

\preworksection{「SambaをWindowsのドメインに参加させたことはありますか。参加させたことがない方、もしお時間があれば実際にやってみて、ハマったところを教えてください。参加させたことがある方、ハマったところを教えてください。」、「Linuxのユーザ認証をWindowsに統合したことはありますか。ある場合はその際の方法やハマったところについて教えてください。ない場合は、もしお時間があれば実際にやってみてハマったところを教えてください」にお答えください。}

誰でもアクセスできるようなsambaサーバしか構築したことないので、特にハマったということはないです。
\end{prework}
