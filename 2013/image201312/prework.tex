\begin{prework}{ 河本拓 }

 インタビュー形式で、参加者のみなさんが「オープンソースに関わったきっかけ」などのお話が聞けたら嬉しいです。2014年最初の勉強会だと思うので、新年の抱負などと合わせて...(?)
\\
追記:初参加です。Linux初心者ですがよろしくお願いします。
\end{prework}

\begin{prework}{ dictoss(杉本 典充) }

基本的にはハックカフェのようにdebianに関する作業をする場として集まり、質問や議論があれば集まった人たちの中で話す、という場を提供することにするとよさそう。
参加の条件として行った作業や成した事、成せなかった事をまとめてブログ記事を必ず1つその場で書くこととする、というのはどうだろうか。そのとき各個人のブログサイトではなく、「勉強会ブログ」のようなサイトで一元的に記事を登録するようにして、対外的にdeibanのトレンドを発信することも兼ねるというのはどうだろうか。
ただ参加者の間でパッケージやdebianの仕組みについて勉強会が必要と判断することもあるので、その場合はセミナー形式で開催すればいいと思う。
\end{prework}

\begin{prework}{ 吉野(yy\_{}y\_{}ja\_{}jp) }

特に変わらないと思っていました
\end{prework}

\begin{prework}{ 野島 貴英 }

原稿/プレゼン集まればセミナー形式、集まらなければ、ハッカソン(作業時間)にするのがよいかなー?とりあえず、来年こそは、debian組み込みネタか、ハードウェアネタをやりたい。
\end{prework}
