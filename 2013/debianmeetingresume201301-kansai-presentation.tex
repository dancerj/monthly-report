\documentclass[cjk,dvipdfmx,10pt,compress,%
hyperref={bookmarks=true,bookmarksnumbered=true,bookmarksopen=false,%
colorlinks=false,%
pdftitle={第 68 回 関西 Debian 勉強会},%
pdfauthor={倉敷・のがた・佐々木・かわだ},%
%pdfinstitute={関西 Debian 勉強会},%
pdfsubject={資料},%
}]{beamer}

\title{第 68 回 関西 Debian 勉強会}
\subtitle{$\sim$発表資料$\sim$}
\author[かわだ てつたろう]{{\large\bf 倉敷・のがた・佐々木・かわだ}}
\institute[Debian JP]{{\normalsize\tt 関西 Debian 勉強会}}
\date{{\small 2013 年 1 月 27 日}}

%\usepackage{amsmath}
%\usepackage{amssymb}
\usepackage{graphicx}
\usepackage{moreverb}
\usepackage[varg]{txfonts}
\AtBeginDvi{\special{pdf:tounicode EUC-UCS2}}
\usetheme{Kyoto}
\def\museincludegraphics{%
  \begingroup
  \catcode`\|=0
  \catcode`\\=12
  \catcode`\#=12
  \includegraphics[width=0.9\textwidth]}
%\renewcommand{\familydefault}{\sfdefault}
%\renewcommand{\kanjifamilydefault}{\sfdefault}
\begin{document}
\settitleslide
\begin{frame}
\titlepage
\end{frame}
\setdefaultslide

\begin{frame}[fragile]
\frametitle{Agenda}

\tableofcontents

\end{frame}

\section{最近の Debian 関係のイベント}

\takahashi[40]{最近の Debian\\関係のイベント}

\begin{frame}[fragile]
  \frametitle{第 67 回関西 Debian 勉強会}
  \begin{itemize}
  \item 日時: 12 月 23 日
  \item 場所: 福島区民センター
  \end{itemize}
  \begin{block}{内容}
    \begin{itemize}
    \item「Android 端末(Asus Transformer TF201)への Debian インストール奮闘記」
    \item 「月刊 Debian Policy パッケージ管理スクリプトとインストール手順」
    \item 「2012年の振り返りと2013年の企画」
    \end{itemize}
  \end{block}
\end{frame}

\begin{frame}[fragile]
  \frametitle{第 96 回 東京エリア Debian 勉強会}
  \begin{itemize}
  \item 日時: 1 月 16 日
  \item 場所: スクウェア・エニックス
  \end{itemize}
  \begin{block}{内容}
    \begin{itemize}
    \item 「Debian予約システムアンケート集計」
    \item 「gdbのpython拡張(その1)」
    \item 「月刊Debhelper」
    \end{itemize}
  \end{block}
\end{frame}

\begin{frame}[fragile]
  \frametitle{福岡 Debian 勉強会 2 回目}
  \begin{itemize}
  \item 日時: 1 月 26 日
  \item 場所: paperboy\&co.(ペパボ)
  \end{itemize}
  \begin{block}{内容}
    \begin{itemize}
    \item 「Debianに片足も両足も突っ込んだ皆さん、こんにちは(仮)」
    \item 「d-i usbメモリの作り方 or wheezyについて何か(仮)」
    \end{itemize}
  \end{block}
\end{frame}



\takahashi[50]{そんな\\こんなで}
\takahashi[120]{次}

\section{事前課題発表}

\takahashi[50]{事前課題}

\begin{frame}[fragile]
  \frametitle{事前課題}
  \begin{block}{今回の事前課題}
    \begin{description}
    \item[事前課題1] 2015年ではDebianはどうなっているかを大胆に予想してください。
    \item[事前課題2] これまでDebianでどんなCMSやWebアプリフレームワークを使ったことがありますか?
    \item[事前課題3] あえてDebianのパッケージを使わなかった事はありますか?
    \item[事前課題4] Drupalの標準ファイル配置は /var/www以下のドキュメントルートにすべてのファイルを一括配置します。
      Debianパッケージの場合はどう変化するか説明してください。\\
      どのコマンドを使えば調べられるでしょうか。
    \item[事前課題5] 可能であれば Drupal 7 - Content Management Framework の VMを動かしてみてください。
    \end{description}
  \end{block}
\end{frame}

\takahashi[50]{事前課題\\発表}

\begin{frame}
  \frametitle{ かわだてつたろう }
  \begin{enumerate}
  \item  -devel は相変わらず init.d の話題で炎上しているが、Jessie が滞りなくリリースされる。

    DebConf15 が日本で開催されようとしている。といいなぁ。
  \item Trac, WordPress と Symfony, CakePHP
  \item Symfony, CakePHP などのWeb アプリフレームワークはプロジェクトのソース管理に含めて管理したいので使わないことが多い。

    デプロイ環境が Debian で無かったり、root 権限が無い場合に都合がよい。
  \item debian 以下をみると次のように変更される。

    (省略)
\item 試してみます。
  \end{enumerate}
\end{frame}

\begin{frame}
\frametitle{ のがたじゅん (1)}
  \begin{enumerate}
  \item[1] Debianなスマートフォンが出て幸せになる(とよいなぁ)
  \item[2] パッケージで使ったのはWordpress, DokuWiki, tDiaryあたりとか。CMS/フレームワークではないけどRedmineをパッケージで入れたら、めちゃ簡単でよかったですね。
  \item[3] 管理で楽をしたいので、なるべくならパッケージを利用したいです。パッケージを使わないときは、最新のものを使いたい時とか?
  \end{enumerate}
\end{frame}

\begin{frame}
\frametitle{ のがたじゅん (2)}
  \begin{enumerate}
  \item[4] apt-file searchとやってもいいけど、packages.debian.orgで検索してファイル一覧を見てみました。

    {\tt /usr/share/drupal7}以下に本体があります。apacheのaliasで使うようになってますね。README.Debianを見ればセットアップ方法もわかるけど、Wordpressのパッケージに比べるとちょっと不親切かな。

    drushがパッケージになってたので、{\tt /usr/share/drupal7}以下で使ってパッケージのファイルを破壊しました(・ω<) てか、どう使ったらいいんだろ?

    モジュールをパッケージ化するdh-make-drupalも使ってみたかったけど時間切れ。
  \item[5] 見ました。今どきっぽくなってて日本語のメッセージカタログをインストールするのに苦労しました。(メニューがタブになってるの気づかなかったりとか…)あと最初からモジュールが入ってるけど、どれがどれかサッパリ。
  \end{enumerate}
\end{frame}

\begin{frame}
  \frametitle{ yyatsuo }
  \begin{enumerate}
  \item
    \begin{itemize}
    \item popcon で arm"hf" が amd64 を抜く
    \item debconf15 日本開催
    \item ローリングリリースにすべきだ!という一派が現われて devel が荒れる
    \item Hurd がまさかの大躍進
    \end{itemize}
  \item ブログは WordPress を使ってます

    Tornado 勉強中です
  \item コンパイルオプションとかの関係でオレオレパッケージ作ったりします
  \item 調べておきます

    "dpkg -L パッケージ名" で調べられたはず
  \item 時間があれば…
  \end{enumerate}
\end{frame}

\begin{frame}
  \frametitle{ kino }

(無回答)

\end{frame}

\begin{frame}
  \frametitle{ 佐々木洋平 (1)}
  \begin{enumerate}
  \item[1] GNU/k$*$BSD が増えて, さらに Universal OS としての地位を確立する。粛々と Jessie がリリースされる。miniconf in Japan が 2 回ぐらい開催されており, スポンサーもつきつつあるので, Debconf in Japan が現実となりつつある。
  \item[2] Trac, Drupal, Xoops. Rails 関連では Redmine, Radiant CMS. 静的 CMS として Octpress。
  \item[3] Rails 関連は新しめの gem に依存することが多いので, あえてパッケージを使わないこともしばしば。当然 unstable にはパッケージとして突っ込むけれども. あと, 業者から貰ったソースが古いバージョンに依存していることも多いので, 必要に応じて古いバージョンをソースからビルドしたり. こういう場合は chroot 切って reverse proxy で動かしたりしている.
  \end{enumerate}
\end{frame}

\begin{frame}[fragile]
  \frametitle{ 佐々木洋平 (2)}
  \begin{enumerate}
  \item[4] {\tt /usr/share/drupal7} 以下に置いてますね. {\tt /usr/share/doc/drupal7/README.Debian.gz} に書いてます. 調べ方は
    \begin{commandline}
      $ apt-get source drupal7
      $ cd drupal7-7.14/debian
      $ lv README.Debian
    \end{commandline}
    % $
  \item[5] 入れて動かしてみました. Drupal6 から結構変わってますねぇ...うーむ。
  \end{enumerate}
\end{frame}


\begin{frame}
  \frametitle{ 山城の国の住人 久保博 }
  \begin{enumerate}
  \item Windows XP からの置き換えで Debian のデスクトップ利用が爆発的に進む。
  \item Pukiwiki, Trac, Redmine, ... CMS ではないですかね。
  \item あえてパッケージを使わなかったことはあります。 {\tt /usr/local} 以下に自分でインストールするのに慣れていたので。
  \item drupal7 パッケージで試しました。 {\tt /usr/share/drupal7} の下です。
    コマンドは

{\tt \% apt-file list drupal7}

    {\tt /usr/share/doc/drupal7/README.Debian.gz} を読んでもそれらしい説明がありました。
  \item 時間切れ。すぐできる環境もなく、できませんでした。残念。
  \end{enumerate}
\end{frame}

\begin{frame}
  \frametitle{ lurdan }
  \begin{enumerate}
  \item 今とそう大差なく、バズトピックに飛びつきすぎず、デスクトップ用途に偏りすぎず、プレーンでカスタマイズしやすい、そんな Debian でいて欲しいッ(願望)
  \item tDiary/Rails なあれこれ/Zope ちょびっと/ とかそんな感じですかね。PHP は (実行環境としてセットアップするのが) ヤダ。
  \item CMS 系のソフトウェアはパッケージングと相性が悪いと思っていて、パッケージを使わないことは多いですね
  \item {\tt /usr/share/drupal} あたり? (apt-file search)
  \item VMを動かしてみて、というよりはとりあえず動く Drupal を触ってみて、ということ?
  \end{enumerate}
\end{frame}

\begin{frame}
  \frametitle{ joe }
  \begin{enumerate}
  \item あまり変わってほしくないですが「ORACLEに買収される」というのは冗談で、NetBSDのサポート?2年毎のリリースを止めるとか
  \item PukiWikiや独自の物を利用していましたが、色々問題があってDrupalにしていきたいと考えています。
  \item PostgreSQL、Drush
  \item 申し訳ありません質問の意図が分かりません。site-enable下の設定によると思うのですが、、
  \item ためさせていただきます
  \end{enumerate}
\end{frame}


\takahashi[50]{そんな\\こんなで}
\takahashi[120]{次}

\section{月刊 Debian Policy 第8回 「オペレーティングシステム」}
\takahashi[25]{月刊 Debian Policy\\第8回 \\「オペレーティングシステム」\\by\\のがた じゅん}

\takahashi[50]{そんな\\こんなで}
\takahashi[120]{次}

\section{Using Drupal on Debian − CMSから入った人のDebian体験}
\takahashi[20]{Using Drupal on Debian\\CMSから入った人のDebian体験\\by\\紀野}

\takahashi[50]{そんな\\こんなで}
\takahashi[120]{次}

\section{今後の予定}
\begin{frame}[fragile]
\frametitle{今後の予定 (1)}

\begin{block}{第 1 回 Debian パッケージング道場}
  \begin{itemize}
  \item 日時: 2 月 9 日(土) 10:00〜17:00
  \item 会場: 京都大学 理学研究科3号館 数学教室 109 号室
  \end{itemize}
\end{block}

\begin{block}{第 69 回関西 Debian 勉強会}
  \begin{itemize}
  \item 日時: 2 月 24 日(日)
  \item 会場: GREE 大阪スタジオ
  \end{itemize}
\end{block}

\end{frame}

\begin{frame}[fragile]
\frametitle{今後の予定 (2)}

\begin{block}{OSC 浜松}
  \begin{itemize}
  \item 日時: 2 月 9 日(土)
  \end{itemize}
\end{block}

\begin{block}{第 97 回東京エリア Debian 勉強会}
  \begin{itemize}
  \item 日時: 2 月 22 日(金)、23 日(土)
  \item 会場: OSC Tokyo/Spring
  \end{itemize}
\end{block}

\end{frame}


\takahashi[50]{  }


\end{document}
%%% Local Variables:
%%% mode: japanese-latex
%%% TeX-master: t
%%% End:
