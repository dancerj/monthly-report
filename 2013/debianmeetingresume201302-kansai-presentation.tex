\documentclass[cjk,dvipdfmx,10pt,compress,%
hyperref={bookmarks=true,bookmarksnumbered=true,bookmarksopen=false,%
colorlinks=false,%
pdftitle={第 69 回 関西 Debian 勉強会},%
pdfauthor={倉敷・のがた・佐々木・かわだ},%
%pdfinstitute={関西 Debian 勉強会},%
pdfsubject={資料},%
}]{beamer}

\title{第 69 回 関西 Debian 勉強会}
\subtitle{$\sim$発表資料$\sim$}
\author[かわだ てつたろう]{{\large\bf 倉敷・のがた・佐々木・かわだ}}
\institute[Debian JP]{{\normalsize\tt 関西 Debian 勉強会}}
\date{{\small 2013 年 2 月 24 日}}

%\usepackage{amsmath}
%\usepackage{amssymb}
\usepackage{graphicx}
\usepackage{moreverb}
\usepackage[varg]{txfonts}
\AtBeginDvi{\special{pdf:tounicode EUC-UCS2}}
\usetheme{Kyoto}
\def\museincludegraphics{%
  \begingroup
  \catcode`\|=0
  \catcode`\\=12
  \catcode`\#=12
  \includegraphics[width=0.9\textwidth]}
%\renewcommand{\familydefault}{\sfdefault}
%\renewcommand{\kanjifamilydefault}{\sfdefault}
\begin{document}
\settitleslide
\begin{frame}
\titlepage
\end{frame}
\setdefaultslide

\begin{frame}[fragile]
\frametitle{Agenda}

\tableofcontents

\end{frame}

\section{最近の Debian 関係のイベント}

\takahashi[40]{最近の Debian\\関係のイベント}

\begin{frame}[fragile]
  \frametitle{第 68 回関西 Debian 勉強会}
  \begin{itemize}
  \item 日時: 1 月 27 日(日)
  \item 場所: 国際奈良学セミナーハウス研修施設
  \end{itemize}
  \begin{block}{内容}
    \begin{itemize}
    \item「Using Drupal on Debian」
    \item「月刊 Debian Policy オペレーティングシステム」
    \end{itemize}
  \end{block}
\end{frame}

\begin{frame}[fragile]
  \frametitle{第 97 回 東京エリア Debian 勉強会}
  \begin{itemize}
  \item 日時: 2 月 23 日(土)
  \item 場所: オープンソースカンファレンス 2013 Tokyo/Spring
  \end{itemize}
  \begin{block}{内容}
    \begin{itemize}
    \item 「Debian update - Debianの最新動向について語ります」
    \end{itemize}
  \end{block}
\end{frame}

\begin{frame}[fragile]
  \frametitle{オープンソースカンファレンス 2013 Hamamatsu}
  \begin{itemize}
  \item 日時: 2 月 9 日(土)
  \item 場所: 浜松市市民協働センター 2F ギャラリー
  \end{itemize}
  \begin{block}{ブース展示物}
    \begin{itemize}
    \item Wheezy PCの展示
    \item あんどきゅめんてっどでびあんの展示
    \item Debian のインフォグラフィック日本語版の展示と配布
    \item Debianステッカーの配布
    \item 東京エリア/関西Debian勉強会の紹介
    \end{itemize}
  \end{block}
\end{frame}

\begin{frame}[fragile]
  \frametitle{第 1 回 Debian パッケージング道場}
  \begin{itemize}
  \item 日時: 2 月 9 日(土)
  \item 場所: 京都大学 理学研究科 3 号館 数学教室 109 号室
  \item まとめ: \url{http://debian-pkg-dojo.titanpad.com/1}
  \end{itemize}
  \begin{block}{内容}
    \begin{itemize}
    \item 基本的なパッケージの作成方法
    \item 最新のパッケージング事情と使い方
    \item パッケージ作成作業
    \end{itemize}
  \end{block}
\end{frame}

\takahashi[50]{そんな\\こんなで}
\takahashi[120]{次}

\section{事前課題発表}

\takahashi[50]{事前課題}

\begin{frame}[fragile]
  \frametitle{事前課題}
  \begin{block}{今回の事前課題}
    \begin{description}
    \item[事前課題1] リリースノートの「付録 B. preseed を利用したインストールの自動化」\footnote{\url{http://www.debian.org/releases/stable/i386/apb.html.ja}}を読んできて下さい。

    また、読んで気になる点などがあれば、教えて下さい。
    \item[事前課題2] Wheezy もしくは sid 環境に apt-get install ruby してきて下さい。
    \item[事前課題3] 普段(業務もしくは趣味等)お使いの Ruby ソフトウェアがあれば、教えて下さい。

    また、それが Debian パッケージになっていないならば、ご指摘下さい。
    \end{description}
  \end{block}
\end{frame}

\takahashi[50]{事前課題\\発表}

\begin{frame}
  \frametitle{ 西山和広 1/3 }
  \begin{enumerate}
  \item[1] ちゃんと読んでいたら長くなってしまったので \url{https://gist.github.com/znz/831f7bd6b0c5eff06530} に置きました。

    \begin{itemize}
    \item[B.1.1.]
      hd-media の hd が何なのか気になりました。
    \item[B.2.1.]
      インストーラが確実に正しい事前設定ファイルを取得するのに、このファイルのチェックサムを指定できます。現在、これには md5sum 値の指定が必要です。指定した値と事前設定ファイルの値は一致し
      なければなりません。一致しない場合は、インストーラは事前設定ファイルを使用しません。
   
      必須かどうかわかりにくいと思いました。
    \item[B.2.2.]
      注意

      現在の Linux カーネル (2.6.9 以降) では、最大 (インストーラがデフォルトで指定するオプションを含め) コマンドラインオプションを 32 個、環境オプションを 32 個受け取れます。この数を超え    ると、カーネルはパニック (クラッシュ) してしまいます (以前のカーネルではこの数字がもっと少ないです)。
   
      長さの制限があるのかどうか気になりました。

    \end{itemize}
  \end{enumerate}
\end{frame}


\begin{frame}
  \frametitle{ 西山和広 2/3 }
  \begin{enumerate}
  \item[1] (続き)
    \begin{itemize}
    \item[B.2.3.]
    auto ブートラベルは、まだどこにも定義されていません。
   
    「auto url=...」というのが使えるのかどうかわかりませんでした。

    \item[B.2.5.]
    この文字列で、ネットワーク上の全マシンに preseed でインストールするのではなく、特定のホストに対して行うようにもできます。
   
    この文字列とは?

    \item[B3]
    型と値の間はよくありません。
   
    型はタイプ (質問タイプ) でしょうか?

    \item[B.4.1.]
    この方法は費用に使うのが容易ですが、言語、国、ロケールの利用可能な組み合わせをすべて preseed できるわけではありません[29]。言語と国は、どちらもブートパラメータで指定できます。
   
    費用に使うのが容易というところの意味が分かりませんでした。

    \item[B.4.5.]
    パスワードを知っている事前設定ファイルが誰でもアクセスできるために、
   
    「事前設定ファイルが」は「事前設定ファイルは」とか「事前設定ファイルを」とかの方が良いのではないでしょうか。

    \end{itemize}
  \end{enumerate}
\end{frame}

\begin{frame}
  \frametitle{ 西山和広 3/3 }
  \begin{enumerate}
  \item[1] (続き)
    \begin{itemize}
    \item[B.4.10.]
    このパラメータの値は、カーネルコマンドラインにそのまま渡されるので、カンマか空白で区切ったパッケージのリストを取れます。
   
    カーネルコマンドラインでは値に空白区切りは使えないはずなので、誤訳のように見えます。

    \item[B.5.2.]
    この場合でも質問は行われます。
   
    ここも誤訳でしょうか? 英語の方は「質問済み状態のまま」(だから質問されない) (ので後続の文で未質問状態に戻す、と続く) という意味に見えます。

    \item[B.5.2.]
    派ケージ
   
    typo?

    \item[B.5.3.]
    他のファイルでより確かな設定を指定する
   
    specific を「確かな」と訳すのはちょっと違うように感じました。

    \end{itemize}
  \item[2] インストールしました。
  \item[3] nadoka という IRC の proxy のようなものを使っているのですが、新しいバージョンが Debian パッケージになっていません。
  \end{enumerate}
\end{frame}

\begin{frame}
  \frametitle{ かわだてつたろう }
  \begin{enumerate}
  \item 読んでみました。

    preseed を使ったことはないのですが、警告があるようにパーティション分割がはまりそうな気がします。
  \item してあります。
  \item redmine, nokogiri, rabbit
  \end{enumerate}
\end{frame}

\begin{frame}
  \frametitle{ 山城の国の住人 久保博 }
  \begin{enumerate}
  \item はい、今から頑張ります。
  \item はい。schroot の中に作った sid 環境にインストールしておきます。
  \item Redmine を使っています。
  \end{enumerate}
\end{frame}

\begin{frame}
  \frametitle{ murase\_{}syuka }
  \begin{enumerate}
  \item 軽く読んでみたが良く分からない
  \item rvmでinstall済み
  \item rvm/irb(pry)/rails
  \end{enumerate}
\end{frame}

\begin{frame}
  \frametitle{ のがたじゅん }
  \begin{enumerate}
  \item はい読みました。が、いまだにHDDのパーティションの切り方がよくわかってなかったり。

    \url{http://www.nofuture.tv/linux/debianautoinstall}
  \item はい。入ってます。
  \item tDiaryを使ってます
  \end{enumerate}
\end{frame}

\begin{frame}
  \frametitle{ よしだともひろ }
  \begin{enumerate}
  \item 読んでおきます。
  \item sid環境で、apt-get install ruby 実行しました。
  \item rabbitくらいかも。
  \end{enumerate}
\end{frame}

\begin{frame}
  \frametitle{ yyatsuo }
  \begin{enumerate}
  \item 一通り読んでみました
  \item Ruby 1.9.1 インストール済みです
  \item 普段使うのは RedMine

    今までに使ったことがあるのは Milkode と Termtter
  \end{enumerate}
\end{frame}

\begin{frame}
  \frametitle{ 佐藤誠 }
  \begin{enumerate}
  \item 今読んでます。

    ちょっと試してみたいけど、時間が...(汗
  \item 仮想環境のsqueezeを、昨晩ようやくwheezyにあげました。

    えーと、ruby も入ってるはず(未確認)
  \item tdiaryでブログを書いています。
  \end{enumerate}
\end{frame}

\begin{frame}
  \frametitle{ 佐々木洋平 }
  \begin{enumerate}
  \item preceed よく使ってます
  \item 普段から使ってます。そろそろ mruby と ruby2.0 を入れたいですね。
  \item redmine と Jekyll がないと途方にくれます。あと、今回のプレゼンは rabbit です。
  \end{enumerate}
\end{frame}

\begin{frame}
  \frametitle{ 大林 }
  \begin{enumerate}
  \item 一通り読みました。
  \item Wheezyの実験用環境にインストールしました。

    ただネットワークの向こう側のVMの上なので、会場では使えないかもしれません。

    会場でネットワークがもし使えればいけるのですが。
  \item yard, rmagick, test-unit, NArray, depq など
  \end{enumerate}
\end{frame}

\begin{frame}
  \frametitle{ kazuhito.sumpic }
  (無回答)
\end{frame}

\takahashi[50]{そんな\\こんなで}
\takahashi[120]{次}

\section{Debian Installer トラブルシューティング}
\takahashi[30]{Debian Installer\\トラブルシューティング\\by\\Yuryu}

\takahashi[50]{そんな\\こんなで}
\takahashi[120]{次}

\section{Ruby In Wheezy}
\takahashi[30]{Ruby In Wheezy\\by\\佐々木洋平}

\takahashi[50]{そんな\\こんなで}
\takahashi[120]{次}

\section{月刊 Debian Policy 第8回 「オペレーティングシステム」}
\takahashi[25]{月刊 Debian Policy\\第8回 \\「オペレーティングシステム」\\その2\\by\\のがた じゅん}

\takahashi[50]{そんな\\こんなで}
\takahashi[120]{次}

\section{今後の予定}
\begin{frame}[fragile]
\frametitle{今後の予定 (1)}

\begin{block}{第 70 回関西 Debian 勉強会}
  \begin{itemize}
  \item 日時: 3 月 24 日(日)
  \item 会場: 港区民センター 梅
  \item 内容: 松澤さんとあわしろいくやさんによる GNOME3 話二本立て
  \end{itemize}
\end{block}

\end{frame}

\begin{frame}[fragile]
\frametitle{今後の予定 (2)}

\begin{block}{オープンソースカンファレンス 2013 Tokushima}
  \begin{itemize}
  \item 日時: 3 月 9 日(土)
  \item 会場: とくぎんトモニプラザ(徳島県青少年センター) 3F・4F
  \end{itemize}
\end{block}

\begin{block}{第 98 回東京エリア Debian 勉強会}
  \begin{itemize}
  \item 日時: 3 月 16 日(土)
  \end{itemize}
\end{block}

\end{frame}


\takahashi[50]{  }


\end{document}
%%% Local Variables:
%%% mode: japanese-latex
%%% TeX-master: t
%%% End:
