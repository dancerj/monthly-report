\documentclass[cjk,dvipdfmx,10pt,compress,%
hyperref={bookmarks=true,bookmarksnumbered=true,bookmarksopen=false,%
colorlinks=false,%
pdftitle={第 77 回 関西 Debian 勉強会},%
pdfauthor={倉敷・のがた・佐々木・かわだ・八津尾},%
%pdfinstitute={関西 Debian 勉強会},%
pdfsubject={資料},%
}]{beamer}

\title{第 77 回 関西 Debian 勉強会}
\subtitle{$\sim$発表資料$\sim$}
\author[かわだ てつたろう]{{\large\bf 倉敷・のがた・佐々木・かわだ・八津尾}}
\institute[Debian JP]{{\normalsize\tt 関西 Debian 勉強会}}
\date{{\small 2013 年 10 月 27 日}}

%\usepackage{amsmath}
%\usepackage{amssymb}
\usepackage{graphicx}
\usepackage{moreverb}
\usepackage[varg]{txfonts}
\AtBeginDvi{\special{pdf:tounicode EUC-UCS2}}
\usetheme{Kyoto}
\def\museincludegraphics{%
  \begingroup
  \catcode`\|=0
  \catcode`\\=12
  \catcode`\#=12
  \includegraphics[width=0.9\textwidth]}
%\renewcommand{\familydefault}{\sfdefault}
%\renewcommand{\kanjifamilydefault}{\sfdefault}
\begin{document}
\settitleslide
\begin{frame}
\titlepage
\end{frame}
\setdefaultslide

\begin{frame}[fragile]
\frametitle{Agenda}

\tableofcontents

\end{frame}

\section{最近の Debian 関係のイベント}

\takahashi[40]{最近の Debian\\関係のイベント}

\begin{frame}[fragile]
  \frametitle{第 76 回関西 Debian 勉強会}
  \begin{itemize}
  \item 日時: 9 月 22 日(日)
  \item 場所: 港区民センター
  \end{itemize}
  \begin{block}{内容}
    \begin{itemize}
    \item 「Linuxとサウンドシステム」
    \item 「dgit でソースパッケージを触ってみる」
    \end{itemize}
  \end{block}
\end{frame}

\begin{frame}[fragile]
  \frametitle{第 105 回東京エリア Debian 勉強会}
  \begin{itemize}
  \item 日時: 10 月 20 日(日)
  \item 場所: OSC 2013 Tokyo/Fall
  \end{itemize}
  \begin{block}{内容}
    \begin{itemize}
    \item ブース展示
    \item セッション「Debian 最近の update」
    \end{itemize}
  \end{block}
\end{frame}

\begin{frame}[fragile]
  \frametitle{Debian Project}
  \begin{itemize}
  \item 「Bits from the Release Team (Jessie freeze info)」
    \begin{itemize}
    \item 2014/11/5 23:59(UTC) Jessie フリーズ
    \end{itemize}
  \item 「Call for Jessie Release Goals」
    \begin{itemize}
    \item Native systemd support in every package with sysv scripts
    \item Hardening of ELF binaries (carry over from Wheezy)
    \item debian/rules to honor CC/CXX flags
    \item clang as secondary compiler
    \item piuparts clean archive
    \item Cross Toolchains in the archive
    \item Make the base system cross-buildable
    \item SELinux
    \item UTF-8
    \end{itemize}
  \item debian-devel@d.o
  \end{itemize}
\end{frame}

\takahashi[50]{そんな\\こんなで}
\takahashi[120]{次}

\section{事前課題発表}

\takahashi[50]{事前課題}

\begin{frame}[fragile]
  \frametitle{事前課題}
  \begin{block}{今回の事前課題}
    \begin{description}
    \item[事前課題1]
      パッケージlibasound2-devをインストールし、
      添付の pcm\_minimal.c をコンパイルして実行ファイルを作成してください。
      実行ファイルを実行した結果を教えてください。

    \item[事前課題2]
      git-buildpackage を install してきて下さい。

    \item[事前課題3]
      Debian Developer/Maintainer の方は, 御自身がメンテされているパッケージをどの VCS で管理されているかお答え下さい。
      パッケージをメンテされていない方は、
      \begin{center}
        {\tt{http://anonscm.debian.org/gitweb}}
      \end{center}
      以下にあるパッケージを眺めて触ってみたいパッケージを決めておいて下さい。
    \end{description}
  \end{block}
\end{frame}

\takahashi[50]{事前課題\\発表}

\begin{frame}
  \frametitle{ 坂本 貴史 }
  \begin{enumerate}
  \item ノイズが聞こえました
  \item インストールできました
  \item debootstrapをやりたいと思います
  \end{enumerate}
\end{frame}

\begin{frame}
  \frametitle{ kozo2 }
  \begin{enumerate}
  \item まだやっています。とりいそぎ参加申し込みしておきたいので結果は会場で口頭で述べさせてください。
  \item いまinstallしようとしています
  \item メンテしていないです。pkg-ruby-extras/ruby-bio.git, pkg-mozext/firebug.git, aptitude/aptitude.git を触ってみたいです
  \end{enumerate}
\end{frame}

\begin{frame}
  \frametitle{ lurdan }
  \begin{enumerate}
  \item あとでやっておきます。
  \item essential ですよ?
  \item git
  \end{enumerate}
\end{frame}

\begin{frame}
  \frametitle{ 西原 }
  \begin{enumerate}
  \item noisyでした。
  \item installしました。
  \item 触ってみたいパッケージを決めておきます。
  \end{enumerate}
\end{frame}

\begin{frame}
  \frametitle{ 西山和広 }
  \begin{enumerate}
  \item ザーというアナログテレビの砂嵐のような音が出ました。
  \item インストールしました。
  \item twitter irc gateway の atig が普段使っているので良さそうだと思いました。
  \end{enumerate}
\end{frame}

\begin{frame}
  \frametitle{ おくの }
  \begin{enumerate}
  \item ノイズ音が鳴りました。
  \item インストールしました。
  \item OpenStackパッケージとか個人的に興味ありますが多いですねー
  \end{enumerate}
\end{frame}

\begin{frame}
  \frametitle{ 川江 }
  \begin{enumerate}
  \item コンパイルに失敗しました。
  \item しました
  \item qemu-kvm
  \end{enumerate}
\end{frame}

\begin{frame}
  \frametitle{ かわだてつたろう }
  \begin{enumerate}
  \item なにもおこりませんでした。

    ソースをみるとノイズ音がするのでしょうか。
  \item はい。
  \item メンテしていないので、何か決めておきます。
  \end{enumerate}
\end{frame}

\begin{frame}[containsverbatim]
  \frametitle{ 佐々木洋平 }
  \begin{enumerate}
  \item はい。
    \begin{commandline}
% gcc -Wall -Wextra -c pcm_minimal.c
pcm_minimal.c: 関数 ‘main’ 内:
pcm_minimal.c:48:4: 警告: ‘if’ 文内の空の本体は中括弧で括ることを推奨します [-Wempty-body]
    ; /* cannot resolve XRUN */
    ^
pcm_minimal.c:50:4: 警告: ‘if’ 文内の空の本体は中括弧で括ることを推奨します [-Wempty-body]
    ; /* cannot write all of samples */
    ^
pcm_minimal.c:19:6: 警告: 使用されない変数 ‘err’ です [-Wunused-variable]
  int err;
% gcc -Wall -Wextra -o pcm_minimal pcm_minimal.o -lasound
    \end{commandline}
  \item はい. 
  \item 自分の関与しているパッケージは pkg-ruby-extras に沢山あったりします。
  \end{enumerate}
\end{frame}

\begin{frame}
  \frametitle{ 大林 }
  \begin{enumerate}
  \item コンパイルしました。実行するとヘッドフォンからノイズ音が出力されます。
  \item しました
  \item メンテナではないので何か考えておきます。

    mesa とか yard とか?
  \end{enumerate}
\end{frame}

\begin{frame}
  \frametitle{ yabuki@netfort.gr.jp }
  \begin{enumerate}
  \item Nothing happened
  \item はい、やっときました。
  \item むかし、yc-el を svn-buildpackage でやっていたが、これぼめんてがなくなったので、もうありません。
  \end{enumerate}
\end{frame}

\takahashi[50]{そんな\\こんなで}
\takahashi[120]{次}

\section{ALSAのユーザーランド解説}
\takahashi[30]{ALSAのユーザーランド解説\\by\\坂本 貴史}

\takahashi[50]{そんな\\こんなで}
\takahashi[120]{次}

\section{git-buildpackage入門again}
\takahashi[30]{git-buildpackage入門again\\by\\佐々木 洋平}

\takahashi[50]{そんな\\こんなで}
\takahashi[120]{次}

\section{今後の予定}
\begin{frame}[fragile]
\frametitle{今後の予定}

\begin{block}{第 78 回関西 Debian 勉強会}
  \begin{itemize}
  \item 日時: 11 月 9 日(土) 11:00 -
  \item 会場: KOF 2013 大阪南港 ATC ITM 棟
  \item 内容: 「Debian 7.0 の実情/今後の開発について」
  \end{itemize}
\end{block}

\begin{block}{第 106 回東京エリア Debian 勉強会}
  \begin{itemize}
  \item 日時: 11 月 16 日(土)
  \item 会場: あんさんぶる荻窪
  \item 内容: 「wayland を動かす」(調整中)
  \end{itemize}
\end{block}

\end{frame}

\takahashi[50]{  }

\end{document}
%%% Local Variables:
%%% mode: japanese-latex
%%% TeX-master: t
%%% End:
