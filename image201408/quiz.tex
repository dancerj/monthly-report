%; whizzy-master ../debianmeetingresume201311.tex
% 以上の設定をしているため、このファイルで M-x whizzytex すると、whizzytexが利用できます。
%

\santaku
{2014/8/16にDebianは誕生日を迎えました。さて今年で何歳?}
{25}
{21}
{20}
{B}
{1993/8/16にIan Ashley Murdockさんにより、Debianは創設されました。というわけで、21歳です。Debian Dayはもう過ぎちゃいましたが、本日(2014/8/23)の勉強会後の宴会で祝いたいと思います!}

\santaku
{2014/7時点の各国のアクティブなDebian Developerの数とそれぞれの国の人口との比率が最も多いのはどこの国?}
{Finland}
{Ireland}
{New Zealand}
{A}
{Finlandは本割合は2009年からずっと毎年1位(今年3.61ppm。)2位 Ireland 3位 New Zealand。日本の本割合は0.28ppmで30位でした。単にアクティブDeveloperの絶対数が多い国は1位USA、2位Germany、3位France。}

\santaku
{OpenAmbitが2014/7にパッケージ化されsidへリリースされました。ところで何するパッケージ?}
{AmazonTV用アプリ開発環境}
{ChoromeCast用アプリ開発環境}
{SUNNTO AMBIT用アプリ開発環境}
{C}
{SUNNTO AMBITという腕時計型のGPS搭載の端末のアプリ開発環境となります。楽天とか、Amazonで買えます。スポーツ好きな方はDebianでアプリ開発やってみてください。}

\santaku
{2014/7/31にtechnical committeeによりDebianのデフォルトのJPEG圧縮伸長ライブラリとして採択されたのは以下のうちのどれ?}
{libjpeg6b}
{libjpeg8/9}
{libjpeg-turbo}
{C}
{libjpeg-turboはlibjpeg6 ABI互換のJPEG圧縮伸長ライブラリで、libjpeg6を遥かに凌ぐ速さで動作します。ここで、JPEG圧縮伸長ライブラリのデフォルトとして、libjpeg8/9か、libjpeg-turboかで議論が分かれていましたが、technical comitteeはlibjpeg-turboをデフォルトのJPEG圧縮伸長ライブラリとしました。}

\santaku
{2014/7/31にtechnical committeeから、「Debianパッケージは複数のinitシステムに対応する事」ということが再アナウンスされました。何のパッケージのドタバタがきっかけでしょう?}
{ftp}
{tftp-hpa}
{ncftp}
{B}
{experimentalにtftp-hpa 5.2-17が入ったが、そのchangelogとして''Removing upstart hacks, they are ugly and upstart is dead now.''という内容が入っていた件が物議を醸しました。bug746715参照。}

\santaku
{2014/7/20にsqueeze(注:squeeze-ltsではない)の最後のアップデートが行われました。何回目のアプデートでしょうか?}
{10}
{9}
{8}
{A}
{最後のsqueezeのアップデートです。実はこのアップデートを行っても、Debianですでに報告されているsqueezeのセキュリティホールの一部は残ったままです。しかしながら、これ以上のアップデートは無いので、wheezyにアップデートするか、squeeze-ltsの採用を検討しましょう。}

\santaku
{2014/7/31にJessieに搭載の可能性のあるlinux kernelのバージョンのアナウンスが行われました。バージョンはいくつでしょう?}
{3.14}
{3.12}
{3.16}
{C}
{2014/8/14にkernel.orgでlinux kernel3.16のリリースがアナウンスされました。Debian Kernelチームのアナウンスでは、「3.16がJessieに採用される可能性があるので、現在のパッケージが3.16と互換のあるKernel APIの元で動作するか確認し、問題があれば然るべき対応してほしい」とのこと。}

\santaku
{VCS-*フィールドのVCS情報を参照し、パッケージのchangelogが合致しているかのチェックが行われるようになりました。以下のどのサイトで確認できる?}
{http://qa.debian.org/cgi-bin/vcswatch}
{http://bugs.debian.org/}
{http://codesearch.debian.net/}
{A}
{vcswatchのページにアクセスし、見たいパッケージ名入れると、状況が得られます。詳しくはvcswatchのサイトの説明をご覧ください。}

\santaku
{パッケージに含まれているドキュメントに関して、lintianが新たな警告をするようになりました。以下のどれ?}
{HTMLファイル中の画像/CSS/JS/videoリンクがローカルファイルを指していない場合、警告}
{HTMLファイル中のDebianの綴りが間違っている場合、警告}
{HTMLファイルが入っていない場合、警告}
{A}
{ブラウザがHTMLファイルを表示する場合、ローカルのファイルであっても、表示時にブラウザが自動でリンク先をアクセスしてしまう作用を持つリンクがあります。ここに外部サイトのURLが入っていた場合、その外部サイトは、HTMLファイルが閲覧された時の情報を収集することが出来てしまいます。こういったことはセキュリティ上良くないので、該当するようなHTMLファイルが含まれている場合、lintianが警告するようになりました。}

\santaku
{2014/7にBTSのWEBサイトにreplyのリンクがつくようになりました。このリンクをブラウザからクリックすると何が起きるようになった?}
{BTS上でreplyしている次のメッセージが表示されるようになった。}
{BTS上のreply用のフォームがブラウザに表示されるようになった。}
{メーラが起動し、正しいSubject/宛先/引用が入るようになった。}
{C}
{やってみれば判りますが、Bugレポートに対する返答をする場合に、非常に便利です。BTSから見て、正しい内容と宛先で、返答出来るように下書きメールがメーラに準備されるようになりました。}

\santaku
{2014/8/14にDebian Installer Jessie Beta 1がアナウンスされました。このバージョンのインストーラで導入されるinitシステムは結局どうなった?}
{systemdになった}
{sysvinitになった}
{upstartになった}
{A}
{このバージョンのDebian Installerから、initシステムとしてsystemdが入るようになりました。ちなみにこの件、Installerプログラム本体の話であって、Jessie本体のBeta版リリースという意味では無い事に注意。}




