\begin{prework}{ dictoss }
  \begin{enumerate}
  \item 知っており翻訳作業またはレビューをしたことがある
  \item PC、サーバ、組み込み機器で動くためOSを統一でき利用者には便利なため
  \end{enumerate}
\end{prework}

\begin{prework}{ NOKUBI Takatsugu (knok) }
  \begin{enumerate}
  \item 知っているが翻訳作業はしたことはない
  \item アップグレードが保証されているから
  \end{enumerate}
\end{prework}

\begin{prework}{ sirtetris }
  \begin{enumerate}
  \item 知らなかった
  \item 仕事で管理しているサーバーはDebianです。
  \end{enumerate}
\end{prework}

\begin{prework}{ hihitani }
  \begin{enumerate}
  \item 知らなかった
  \item 今でも32bitを明確にサポートしている
  \end{enumerate}
\end{prework}

\begin{prework}{ Hiroyuki Yamamoto (yama1066) }
  \begin{enumerate}
  \item 知っているが翻訳作業はしたことはない
  \item (回答なし)
  \end{enumerate}
\end{prework}

\begin{prework}{ yy\_y\_ja\_jp }
  \begin{enumerate}
  \item 知っており翻訳作業またはレビューをしたことがある
  \item (回答なし)
  \end{enumerate}
\end{prework}

\begin{prework}{ Takayoshi Shiigi (tcgi) }
  \begin{enumerate}
  \item 知っており翻訳作業またはレビューをしたことがある
  \item (回答なし)
  \end{enumerate}
\end{prework}

\begin{prework}{ レドレ (redred) }
  \begin{enumerate}
  \item 知らなかった
  \item ・派生が多いということは、それだけ多くの人の目が通っていて、安全なOSなのだろうと思える安心感。・日常のアップデート頻度がUbuntuより控えめな気楽さ。・玄人を装えるドヤ自己悦 隙間あればDebian自体が開発しているソフトウェアの翻訳はどう行われているのかの様子も少しお聞きしたい。
  \end{enumerate}
\end{prework}
