%; whizzy-master ../debianmeetingresume201311.tex
% 以上の設定をしているため、このファイルで M-x whizzytex すると、whizzytexが利用できます。
%

\santaku
{2015/2/6に、とある国にsecurity mirrorが新設されたことがアナウンスされました。どこの国?}
{日本}
{アメリカ}
{ジャマイカ}
{A}
{Debianのsecurity mirrorを新設していただいたのは、SAKURA Internetさん(http://www.sakura.ad.jp/)となります。こちらありがとうございます!OSC 2015 Tokyo/Spring(http://www.ospn.jp/osc2015-spring/)へ行かれる方は、SAKURA Internetさんのブースにてよろしくお伝えくださいませ!また、こちらの新設に尽力された、やまねさんとDebian JP Projectの関係者の皆様、お疲れさまでした!}

\santaku
{2015/1/26に、Debian Installerのリリースのアナウンスがされました。こちらのバージョンはいくつ?}
{Jessie RC 1}
{Jessie RC 2}
{Jessie RC 3}
{A}
{Debian Installerのリリースの話で、Jessie本体のリリースの話では無いです。で、今回1/26にRC1がリリースされたことがアナウンスされました。バグ出し、特にUEFI環境をおもちの方、動作テストしてバグ出しとBTSへのレポートよろしくおねがいしますー。}

\santaku
{過去に諸々の理由によりtestingからRemoveされてしまったJessie向けのパッケージについて、再度復活できる最後のチャンスの締切り日はいつでしょう?}
{2/5中}
{2/14中}
{2月末中}
{A}
{2/5 23:59:59 UTCまでだそうです。もう終わっちゃいましたね。皆様のパッケージは無事でしたでしょうか?}

\santaku
{2015/2/13にてGSoC 2015のDebianに関するプロジェクト募集とGSoCのメンター募集が行われました。2/15現在寄せられていないプロジェクトは?}
{Apport of Debian}
{Coinstallable PHP Versions}
{Debian Gnu/Hurd enhancements}
{C}
{詳しくは、https://wiki.debian.org/SummerOfCode2015/Projects。Apportは、プログラムがクラッシュしたときに割り込み、デバッグ情報や環境の情報をBTSへ送るシステム。Coinstallable PHP VersionはDebianにて、php 5.Xの各バージョンごとのPHPパッケージ/PHP環境を同居して導入できるようにするプロジェクト。}

\santaku
{2015/2/13にReproducible Buildsの報告がありました。現在mainパッケージの何\%が完了しているでしょう?}
{62\%}
{83\%}
{100\%}
{B}
{Reproducible Buildsとは、現在お使いのDebianの各バイナリパッケージが、本当に現在のソースパッケージから生成されたもので、悪意のあるバイナリが混じっていないかを調べる試み。簡単に言うと、ソフト製造者側の製品本体のウィルス感染チェックみたいなもの。Debianの他にも、Fedora/OpenSUSE/NixOSでも行われている。詳しくは https://wiki.debian.org/ReproducibleBuilds。また、本プロジェクトのについてのプレゼンは:http://reproducible.alioth.debian.org/presentations/2014-02-01-FOSDEM14.pdf}
