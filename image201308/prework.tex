
\begin{prework}{ Yoshida Shin }

VPN を構築した事は有りません。
また、プライベートでも使用した事は有りません。

VPN は仕事で以下の 2個の使い方をしています。
\begin{enumerate}
 \item  データセンターのサーバーにログインする
 \item  緊急対応のため、自宅から職場につなぐ
\end{enumerate}

でも、緊急対応は行わないで済むようにするべきだし、
(緊急対応が無いと仮定すれば、)多くの場合はデータセンターへの接続も
職場からの IP 制限だけで十分だと考えています。

VPN は万一の為に用意するものであり、
あまり積極的に使いたいと思わないです。
\end{prework}

\begin{prework}{ 吉野(yy\_{}y\_{}ja\_{}jp) }

個人的には使ってません.
\end{prework}

\begin{prework}{ sakai }

今のところVPNは利用していない。
そのうち、外出した時用にVPNで自宅とつなごうかなー、ということを考える程度。
\end{prework}

\begin{prework}{ 野島 貴英 }

以前、苦肉の策で2つの拠点間をインターネット経由でopenvpn使って一時的にLANを組み、複数のWEBサイトのWEB-DB間のアクセスを遠方のDBに常時流し込みつづけてそのまま半月ぐらいサービス維持した事があります。使った結果ですが、予想外にも、簡単/大変タフ/非常に安定したVPN経路が組めた記憶があります。ただ、ちゃんとしたデータ取ってないので、感想以上の事がいえない状況ではあります。
\end{prework}

\begin{prework}{ dictoss(杉本 典充) }

最近自宅サーバにopenvpnを入れて、出先でテザリングをしながらサーバにアクセスしている。
それまではsshとポートフォワードでがんばっていたが、サーバの台数が増えるとポートフォワードの数が増えるので設定が疲れました。
VPSから自宅サーバにVPNセッションを張ろうかと思っているが自宅サーバはDynamicDNSを利用しているので切れずにつながっていられるか不明。

会社だと拠点間をつなぐためにL2TP/IPsecでVPNを張る場合や、グロバールIPを持たない(=NAT配下)に設置されるが管理上サーバからsshする必要があるPCにはクライアントからVPNを張らせることでsshできるようにする、といった使い方をしている。
\end{prework}

\begin{prework}{ まえだこうへい }

会社環境に繋ぐのに、OpenVPNと、Cisco Anycoonect を使ってますが、後者の環境では、Anyconnectクライアントではなく、OpenConnectを使ってます。RSA OneTimePasswordのモジュールを併用して二要素認証にしていますが、これにも対応してます。

個人環境ではSSHのProxy機能を使っているので、接続先ノード数増えても特に困らずVPNって面倒だよなぁとふと思い出しながら仕事してます。
\end{prework}
