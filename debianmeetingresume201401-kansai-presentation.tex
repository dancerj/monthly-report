\documentclass[cjk,dvipdfmx,10pt,compress,%
hyperref={bookmarks=true,bookmarksnumbered=true,bookmarksopen=false,%
colorlinks=false,%
pdftitle={第 80 回 関西 Debian 勉強会},%
pdfauthor={倉敷・のがた・佐々木・かわだ・八津尾},%
%pdfinstitute={関西 Debian 勉強会},%
pdfsubject={資料},%
}]{beamer}

\title{第 80 回 関西 Debian 勉強会}
\subtitle{$\sim$発表資料$\sim$}
\author[かわだ てつたろう]{{\large\bf 倉敷・のがた・佐々木・かわだ・八津尾}}
\institute[Debian JP]{{\normalsize\tt 関西 Debian 勉強会}}
\date{{\small 2014 年 1 月 26 日}}

%\usepackage{amsmath}
%\usepackage{amssymb}
\usepackage{graphicx}
\usepackage{moreverb}
\usepackage[varg]{txfonts}
\AtBeginDvi{\special{pdf:tounicode EUC-UCS2}}
\usetheme{Kyoto}
\def\museincludegraphics{%
  \begingroup
  \catcode`\|=0
  \catcode`\\=12
  \catcode`\#=12
  \includegraphics[width=0.9\textwidth]}
%\renewcommand{\familydefault}{\sfdefault}
%\renewcommand{\kanjifamilydefault}{\sfdefault}
\begin{document}
\settitleslide
\begin{frame}
\titlepage
\end{frame}
\setdefaultslide

\begin{frame}[fragile]
\frametitle{Agenda}

\tableofcontents

\end{frame}

\section{最近の Debian 関係のイベント}

\takahashi[40]{最近の Debian\\関係のイベント}

\begin{frame}[fragile]
  \frametitle{第79回関西Debian勉強会}
  \begin{itemize}
  \item 日時: 12月22日(日)
  \item 場所: 港区民センター
  \end{itemize}
  \begin{block}{内容}
    \begin{itemize}
    \item 「2013年の振り返りと2014年の企画」
    \end{itemize}
  \end{block}
\end{frame}

\begin{frame}[fragile]
  \frametitle{第108回東京エリアDebian勉強会}
  \begin{itemize}
  \item 日時: 1月18日(土)
  \item 場所: 株式会社スクウェア・エニックス セミナールーム
  \end{itemize}
  \begin{block}{内容}
    \begin{itemize}
    \item 「Debian Pure Blends」
    \item もくもくの会
    \end{itemize}
  \end{block}
\end{frame}

\begin{frame}[fragile]
  \frametitle{Debian Project}
  \begin{itemize}
  \item delegation
  \item 「Valve games for Debian Developers」
  \item 「ci.debian.net」
  \end{itemize}
\end{frame}

\takahashi[50]{そんな\\こんなで}
\takahashi[120]{次}

\section{事前課題発表}

\takahashi[50]{事前課題}

\begin{frame}[fragile]
  \frametitle{事前課題}
  \begin{block}{今回の事前課題}
    \begin{description}
    \item[事前課題1]
      Debian で、やろうとしてできていない上手くいってないこと、もしくは
      やろうとしていることを教えてください。
    \item[事前課題2]
      LT 歓迎です。何かお話したい方はタイトルを下さい。
    \end{description}
  \end{block}
\end{frame}

\takahashi[50]{事前課題\\発表}

\begin{frame}
  \frametitle{ 佐々木洋平 }
  \begin{enumerate}
  \item tDiary のパッケージングが上手くいってません。
  \item 「LT: jenkins + jenkins-debian-glue + freight で野良リポジトリ
    作った」です。
  \end{enumerate}
\end{frame}

\begin{frame}
  \frametitle{ takata }
  \begin{itemize}
  \item 上手くいっていないこと: jessieで Redmineがうまく動作しない
  \item Debianでruby関連のパッケージは皆さんどのように使用されているの
    でしょう? 
  \item Windows8ノート機に Debianをインストールする場合の推奨方法とい
    うのはあるでしょうか?
  \end{itemize}
\end{frame}

\begin{frame}
  \frametitle{ 西山和広 }
  \begin{enumerate}
  \item squeeze のサポート終了が近くなってきたので、そろそろ wheezy に
    上げたいと思っていますが、まだ出来てません。
  \item さくらのVPSでIPv6設定
  \end{enumerate}
\end{frame}

\begin{frame}
  \frametitle{ かわだてつたろう }
  \begin{enumerate}
  \item 
    \begin{itemize}
    \item イベントなどの会計整理。
    \item JP サイトのポリシー更新。
    \end{itemize}
  \end{enumerate}
\end{frame}

\begin{frame}
  \frametitle{ 木下 }
  \begin{enumerate}
  \item 
    \begin{itemize}
    \item PANDABOARDへのGPUデバイスドライバのインストール。\\
       →NONフリーの為か、もしくは利用者が少ない為か情報が少ないように思います。
    \item 業務用システムOSとしての導入\\
       →CentOSを好まれてしまう(社内ですが・・・)。\\
       →クラウドサービス、データセンター等においてもDebianの選択肢が少ないような気がします。

    \end{itemize}
  \item 
     何かあればよいのですが・・・
  \end{enumerate}
\end{frame}

\begin{frame}
  \frametitle{ 川江 }
  \begin{enumerate}
  \item kVMのVMでWWWサーバを立ち上げて、HTML5のサイトを作る予定。
  \item 特になし。
  \end{enumerate}
\end{frame}

\begin{frame}
  \frametitle{ 榎真治 }
  \begin{enumerate}
  \item
    \begin{itemize}
    \item Becky!からsylpheedへのメールデータの移行と、ローカルでのimapサ
      ーバの設定(1つのメールアカウントでは成功したものの複数アカウント
        を扱う方法を調べ中)
    \item また、バージョンごとの動作の違いを調べたいので、LibreOfficeの
      TDFビルドを複数バージョン共存させる方法を探そうとしています。例え
      ば、4.1.4を入れると4.1.3は削除されます。
    \end{itemize}
  \end{enumerate}
\end{frame}

\begin{frame}
  \frametitle{ yyatsuo }
  \begin{enumerate}
  \item 
    \begin{itemize}
    \item fcitx-skk のパッケージング
    \item kernel-handbookとDPNの翻訳
    \item RepRap on Debian 的なやつ
    \end{itemize}
  \item 時間があれば小ネタを用意します
  \end{enumerate}
\end{frame}

\begin{frame}
  \frametitle{ lurdan }
  \begin{enumerate}
  \item Weblate の実験環境が壊れっぱなしなのでなんとかする
  \end{enumerate}
\end{frame}

\begin{frame}
  \frametitle{ あかべ }
  \begin{itemize}
  \item apt-getするだけで統計機械翻訳システム一式を導入できるようにし
    たい。
    \begin{itemize}
    \item 翻訳・言語モデルの作成→フリーソフトウェアのドキュメントとか
      から取り出した対訳文を使って学習(だからフリー!)
    \item 利用例→フリーソフトウェアのドキュメントとかの翻訳作業を補助
    \end{itemize}
  \end{itemize}
\end{frame}

\begin{frame}
  \frametitle{ 大北剛史 }
  \begin{enumerate}
  \item Kdeを使用しているが、resolutionをあげる方法がわかりません。
  \end{enumerate}
\end{frame}

\takahashi[50]{そんな\\こんなで}
\takahashi[120]{次}

\section{ライトニングトーク}
\takahashi[30]{ライトニングトーク}

\takahashi[50]{そんな\\こんなで}
\takahashi[120]{次}

\section{もくもくの会}
\takahashi[30]{もくもくの会}

\takahashi[50]{そんな\\こんなで}
\takahashi[120]{次}

\section{今後の予定}
\begin{frame}[fragile]
\frametitle{今後の予定}

\begin{block}{第81回関西Debian勉強会}
  \begin{itemize}
  \item 日時: 2月23日(日) 13:30 -
  \item 場所: 福島区民センター
  \end{itemize}
\end{block}

\begin{block}{第109回東京エリアDebian勉強会}
  \begin{itemize}
  \item 日時: 2月15日(土)
  \item 場所: 株式会社スクウェア・エニックス セミナールーム
  \item 内容: もくもくの会
  \end{itemize}
\end{block}

\end{frame}

\takahashi[50]{  }

\end{document}
%%% Local Variables:
%%% mode: japanese-latex
%%% TeX-master: t
%%% End:
