\documentclass[cjk,dvipdfmx,10pt,%
hyperref={bookmarks=true,bookmarksnumbered=true,bookmarksopen=false,%
colorlinks=false,%
pdftitle={第 59 回 関西 Debian 勉強会},%
pdfauthor={倉敷・のがた・かわだ・佐々木},%
%pdfinstitute={関西 Debian 勉強会},%
pdfsubject={資料},%
}]{beamer}

\title{第 59 回 関西 Debian 勉強会}
\subtitle{{\scriptsize 資料}}
\author[かわだ てつたろう]{{\large\bf 倉敷・のがた・かわだ・佐々木}}
\institute[Debian JP]{{\normalsize\tt 関西 Debian 勉強会}}
\date{{\small 2012 年 5 月 27 日}}

%\usepackage{amsmath}
%\usepackage{amssymb}
\usepackage{graphicx}
\usepackage{moreverb}
\usepackage[varg]{txfonts}
\AtBeginDvi{\special{pdf:tounicode EUC-UCS2}}
\usetheme{Kyoto}
\def\museincludegraphics{%
  \begingroup
  \catcode`\|=0
  \catcode`\\=12
  \catcode`\#=12
  \includegraphics[width=0.9\textwidth]}
%\renewcommand{\familydefault}{\sfdefault}
%\renewcommand{\kanjifamilydefault}{\sfdefault}
\begin{document}
\settitleslide
\begin{frame}
\titlepage
\end{frame}
\setdefaultslide

\begin{frame}[fragile]
\frametitle{Agenda}

\tableofcontents

\end{frame}

\section{最近の Debian 関係のイベント}

\takahashi[40]{最近の Debian\\関係のイベント}

\begin{frame}[fragile]
\frametitle{第 58 回関西 Debian 勉強会}

\begin{itemize}
\item 日時: 4 月 22 日
\end{itemize}

\begin{block}{内容}
  \begin{itemize}
  \item フリーソフトウェアと戯れるための著作権入門
  \item 月刊 Debian Policy
  \item 月刊(?) Konoha
  \end{itemize}
\end{block}
ネタ出しは随時行なっております! 皆様よろしく!!
\end{frame}

\begin{frame}[fragile]
  \frametitle{第 88 回 東京エリア Debian 勉強会}
  \begin{itemize}
  \item  日時: 5 月 19 日
  \end{itemize}
  \begin{block}{内容}
    \begin{itemize}
    \item Python初心者が「Pythonプロフェッショナルプログラミング」を読んでみた
    \item coffeescriptを使ってみた
    \end{itemize}
  \end{block}
\end{frame}

\takahashi[50]{そんな\\こんなで}
\takahashi[120]{次}

\section{事前課題発表}

\takahashi[50]{事前課題}

\begin{frame}[fragile]
\frametitle{事前課題}

\begin{block}{今回の事前課題}
  \begin{description}
  \item[事前課題1] ITPをしたことがない方は、ITPしたいパッケージを探してみて概要を紹介してください。\\
     ITPをしたことがある方は、最初のITPパッケージの概要とITPをするきっかけなどを語ってください。
  \item[事前課題2] Essential なパッケージをひとつ挙げてパッケージのコントロールフィールドを確認し\\
    「アーカイブエリア」「アーカイブセクション」「パッケージ名」「バージョン」「メンテナ」\\
    をおしえてください。そのメンテナが個人かグループかも判定して下さい。
  \end{description}
\end{block}

\end{frame}

\takahashi[50]{事前課題\\発表}

\begin{frame}
  \frametitle{ 佐々木洋平 }
  \begin{enumerate}
  \item rttool、だった気がする。
  \item 
    \begin{description}
    \item [アーカイブ] main
    \item [アーカイブセクション] libs
    \item [パッケージ名] libc-bin
    \item [バージョン] 2.13-32
    \item [メンテナ] GNU Libc Maintainers $<$debian-glibc@lists.debian.org$>$
      \begin{description}
      \item Team ですね。
      \end{description}
    \end{description}
  \end{enumerate}
\end{frame}

\begin{frame}
    \frametitle{ 川江 }
  \begin{enumerate}
  \item
    \begin{description}
    \item [PPP] ポイント to ポイントでパソコン同士を繋ぐパッケージ
    \end{description}
  \item dpkg main dpkg 1.16.3 Dpkg Developers
  \end{enumerate}
\end{frame}

\begin{frame}
  \frametitle{ のがたじゅん }
  \begin{enumerate}
  \item ITPはしたことないです。してみたいなと思ってるソフトを挙げてみた(てか、その後の言葉が予想出来るだけに無理はしてないはず)
    \begin{description}
    \item [luppp] Abelton Liveに似たオーディオ・ループ・シーケンサー。
    \item [cromfs] ユーザー空間で利用できる圧縮ROMファイルシステム
    \item [AzMiniPainter] ドット線描画用の簡易ペイントソフト
    \end{description}
  \item build-essentialにある/usr/share/build-essential/essential-packages-listを見ました。\\
    その中からbase-filesを見ました。
    \begin{description}
    \item [アーカイブエリア] main
    \item [アーカイブセクション] admin
    \item [パッケージ名] base-files
    \item [バージョン] 3.9.1
    \item [メンテナ] Santiago Vila $<$sanvila@debian.org$>$ さん
    \end{description}
    個人だと思います。
  \end{enumerate}
\end{frame}

\begin{frame}
    \frametitle{ かわだてつたろう }
  \begin{enumerate}
  \item
    \begin{description}
    \item [stl-manual] C++-STL documentation in HTML\\
      SGI の STL プログラマーズガイド\\
      (ITA され RFS も出されていますが止まっているようです。)
    \end{description}
  \item 
    \begin{description}
    \item [アーカイブエリア] main
    \item [アーカイブセクション] utils 
    \item [パッケージ名] grep
    \item [バージョン] 2.12-2
    \item [メンテナ] Anibal Monsalve Salazar $<$anibal@debian.org$>$
    \end{description}
    メンテナは個人です。\\
    アーカイブエリアはコントロールフィールドに記載されていませんでしたが、Essential なパッケージなので main 以外にないでしょう。
  \end{enumerate}
\end{frame}

\begin{frame}
  \frametitle{ 山城の国の住人 久保博 }
  \begin{enumerate}
  \item ITPしたことがありません。するとすれば、次のソフトかな。でも、似たような etherwake や wakeonlan というパッケージがあるので、 ITP する意義が薄い気がしています。
    \begin{description}
    \item [wakeoverlan] と名づけている、自作の C プログラム。 Wake On LAN パケットを指定した IP アドレスへ向けて飛ばす。
    \end{description}
  \item aptitude ?Essential で出した一覧から sed を選びました。\\
    squeeze 環境で、
    \begin{description}
    \item [アーカイブエリア] main
    \item [アーカイブセクション] utils
    \item [パッケージ名] sed
    \item [バージョン] 4.1.5-6
    \item [メンテナ] Clint Adams $<$schizo@debian.org$>$
    \end{description}
    でした。BTS でのやりとりの様子を見るに、メンテナは実在の人だと思います。
  \end{enumerate}
\end{frame}

\begin{frame}
  \frametitle{ lurdan }
  \begin{enumerate}
  \item
    \begin{description}
    \item [yaskkserv] 当時、アーカイブにある skkserv はほとんどメンテされていない昔の残骸ばかりでした。自分で使う上で apt-get で入って欲しかったので、自分でつっこむことにしました。
    \end{description}
  \item 何かと話題の sysvinit
    \begin{description}
    \item [アーカイブエリア] main
    \item [アーカイブセクション] admin
    \item [パッケージ名] sysvinit
    \item [バージョン] 2.88dsf-22.1
    \item [メンテナ] Debian sysvinit maintainers (グループ)
    \end{description}
  \end{enumerate}
\end{frame}

\begin{frame}
  \frametitle{ yyatsuo }
  \begin{enumerate}
  \item
    \begin{description}
    \item [lpc21isp] ARMマイコン書き込み用ISPツール
    \end{description}
  \item 
    \begin{description}
    \item [アーカイブエリア] main
    \item [アーカイブセクション] utils
    \item [パッケージ名] coreutils
    \item [バージョン] 8.13-3.2
    \item [メンテナ]  Michael Stone 
    \end{description}
  \end{enumerate}
\end{frame}

\begin{frame}
  \frametitle{ 西山和広 }
  \begin{enumerate}
  \item ITP はしたことがありません。気になっているものとしては
    \begin{description}
    \item [cocot] script コマンドのように端末とプログラムの間に入って、エンコーディング変換をしてくれるプログラムです。
    \item [ccollect] rsync の --link-dest を使って pdumpfs のような変化のなかったファイルはハードリンクにするというバックアップをとれるプログラムです。
    \end{description}
  \item apt を選びました。
    \begin{description}
    \item [アーカイブエリア] main
    \item [アーカイブセクション] admin
    \item [パッケージ名] apt
    \item [バージョン] 0.8.10.3+squeeze1
    \item [メンテナ] APT Development Team $<$deity@lists.debian.org$>$
    \end{description}
    でグループメンテナでした。
  \end{enumerate}
\end{frame}

\begin{frame}
  \frametitle{ よだ たくや }
  \begin{enumerate}
  \item
    \begin{description}
    \item [ITP経験] なし
    \item [ITPしたいパッケージ] 思いつきませんでした。
    \end{description}
  \item 
    \begin{description}
    \item [アーカイブエリア] main
    \item [アーカイブセクション] utils
    \item [パッケージ名] tar
    \item [バージョン] 1.26-4
    \item [メンテナ] Bdale Garbee
    \end{description}
  \end{enumerate}
\end{frame}

\begin{frame}
\frametitle{ 甲斐正三 }
  \begin{enumerate}
  \item 
    'ITP'はできませんが、'jfbterm'には関心がありますので、調べた結果を報告します。(Squeeze上'aptitude show'による)\\
    フルネームは'J Framebuffer Terminal/Multilingual Extension'と言い、フレームバッファ上でCJKマルチリンガルテキストを表示するソフトウエアである。
    \item essential(必須)なソフトウエアとして'dpkg'を揚げました。
      \begin{description}
      \item [アーカイブエリア] main
      \item [アーカイブセクション] admin
      \item [パッケージ名] dpkg
      \item [バージョン] 1.16.3
      \item [メンテナー] Dpkg Developers(グループのようです)
      \end{description}
  \end{enumerate}
\end{frame}

\begin{frame}
 \frametitle{ kozo2 }
  \begin{enumerate}
  \item
    \begin{description}
    \item [Cytoscape] というネットワーク可視化ソフト

      \#465331を見る限りではITPからRFPにステータスが変わったようだった。Javaアプリなので大変そうではあるが興味がある。
    \end{description}
  \item
    \begin{description}
    \item [アーカイブエリア] パッケージディレクトリにない
    \item [アーカイブセクション] パッケージディレクトリにない
    \item [パッケージ名]: emacs-snapshot
    \item [バージョン] 2:20120521-1
    \item [メンテナ] Julien Danjou(個人)
    \end{description}
  \end{enumerate}
\end{frame}

\begin{frame}
  \frametitle{ よしだともひろ }
  \begin{enumerate}
  \item 3月31日にsdic-inline-elでITPデビューしました。
    \begin{description}
    \item [sdic-inline-el] はポイント下の単語の意味をミニバッファに表示するEmacs Lispです。
    \end{description}
    普段からsdic-inlineを使っていて、まだDebianパッケージになっていなかったので思い切ってITPしてみました。
  \item aptitude search \textasciitilde{}E で essentialなパッケージを探しsedのcontrolファイルを見ました。
    \begin{description}
    \item [アーカイブエリア] main
    \item [アーカイブセクション] utils
    \item [パッケージ名] sed
    \item [バージョン] 3.9.1
    \item [メンテナ] Clint Adams(個人)
    \end{description}
  \end{enumerate}
\end{frame}


\takahashi[50]{そんな\\こんなで}
\takahashi[120]{次}

\section{ITP から始めるパッケージメンテナへの道 by よしだ ともひろ}
\takahashi[25]{ITP から始めるパッケージメンテナへの道 \\by\\ よしだ ともひろ}

\takahashi[50]{そんな\\こんなで}
\takahashi[120]{次}

\section{月刊 Debian Policy 第4回 「バイナリパッケージ」 by 山城国の住人 久保博}
\takahashi[25]{月刊 Debian Policy 第4回\\「バイナリパッケージ」 \\by\\ 山城国の住人 久保博}

\takahashi[50]{そんな\\こんなで}
\takahashi[120]{次}

\section{今後の予定}
\begin{frame}[fragile]
\frametitle{今後の予定}

\begin{block}{大統一Debian勉強会}
\begin{itemize}
  \item 日時: 6 月 23 日
  \item 会場: 京都大学理学部3号館 108, 109, 110
  \item 内容: 
    \url{http://gum.debian.or.jp/}
    \begin{itemize}
    \item 一般参加申し込み/懇親会受付中
    \end{itemize}
\end{itemize}
\end{block}

\begin{block}{Wheezy Freeze}
\begin{itemize}
  \item 6 月中旬
\end{itemize}
\end{block}

\end{frame}



\takahashi[50]{  }


\end{document}
%%% Local Variables:
%%% mode: japanese-latex
%%% TeX-master: t
%%% End:
