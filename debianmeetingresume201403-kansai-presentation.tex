\documentclass[cjk,dvipdfmx,10pt,compress,%
hyperref={bookmarks=true,bookmarksnumbered=true,bookmarksopen=false,%
colorlinks=false,%
pdftitle={第 82 回 関西 Debian 勉強会},%
pdfauthor={倉敷・のがた・佐々木・かわだ・八津尾},%
%pdfinstitute={関西 Debian 勉強会},%
pdfsubject={資料},%
}]{beamer}

\title{第 82 回 関西 Debian 勉強会}
\subtitle{$\sim$発表資料$\sim$}
\author[かわだ てつたろう]{{\large\bf 倉敷・のがた・佐々木・かわだ・八津尾}}
\institute[Debian JP]{{\normalsize\tt 関西 Debian 勉強会}}
\date{{\small 2014 年 3 月 23 日}}

%\usepackage{amsmath}
%\usepackage{amssymb}
\usepackage{graphicx}
\usepackage{moreverb}
\usepackage[varg]{txfonts}
\AtBeginDvi{\special{pdf:tounicode EUC-UCS2}}
\usetheme{Kyoto}
\def\museincludegraphics{%
  \begingroup
  \catcode`\|=0
  \catcode`\\=12
  \catcode`\#=12
  \includegraphics[width=0.9\textwidth]}
%\renewcommand{\familydefault}{\sfdefault}
%\renewcommand{\kanjifamilydefault}{\sfdefault}
\begin{document}
\settitleslide
\begin{frame}
\titlepage
\end{frame}
\setdefaultslide

\begin{frame}[fragile]
  \frametitle{Disclaimer}
  \begin{itemize}
  \item 疑問、質問、ツッコミ、茶々、\alert{大歓迎}
  \item その場でインタラクティブにどうぞ
  \item ハッシュタグ \#kansaidebian
\end{itemize}
\end{frame}

\begin{frame}[fragile]
\frametitle{Agenda}

\tableofcontents

\end{frame}

\section{最近の Debian 関係のイベント}

\takahashi[40]{最近の Debian\\関係のイベント}

\begin{frame}[fragile]
  \frametitle{第81回関西Debian勉強会}
  \begin{itemize}
  \item 日時: 2月23日(日)
  \item 場所: 福島区民センター
  \end{itemize}
  \begin{block}{内容}
    \begin{itemize}
    \item もくもくの会
    \item LT
      \begin{enumerate}
      \item upstart と docker
      \item タイリングマネージャ i3
      \item systemd を入れみてた
      \end{enumerate}
    \end{itemize}
  \end{block}
\end{frame}

\begin{frame}[fragile]
  \frametitle{第110回東京エリアDebian勉強会}
  \begin{itemize}
  \item 日時: 3月1日(土)
  \item 場所: OSC 2014 Tokyo/Spring
  \end{itemize}
  \begin{block}{内容}
    \begin{itemize}
    \item 「Debian Update \& Debian のEFI/UEFI対応について」
    \item ブース展示
    \end{itemize}
  \end{block}
\end{frame}

\begin{frame}[fragile]
  \frametitle{第111回東京エリアDebian勉強会}
  \begin{itemize}
  \item 日時: 3月15日(土)
  \item 場所: 株式会社スクウェア・エニックス 会議室
  \end{itemize}
  \begin{block}{内容}
    \begin{itemize}
    \item 「debianとiphone5」
    \item もくもくの会
    \end{itemize}
  \end{block}
\end{frame}

\begin{frame}[fragile]
  \frametitle{Debian Project}
  \begin{itemize}
  \item Code of Conduct
  \item Debian Installer Jessie Alpha1 release
  \item Bits from the Security Team
  \item Packages
  \item Bits from keyring-maint: Pushing keyring updates. Let us bury your old 1024D key!
  \item 800000th Bug Contest
  \item More than 20,000 source packages in Debian unstable
  \end{itemize}
\end{frame}

\takahashi[50]{そんな\\こんなで}
\takahashi[120]{次}

\section{事前課題発表}

\takahashi[50]{事前課題}

\begin{frame}[fragile]
  \frametitle{事前課題}
  \begin{block}{今回の事前課題}
    \begin{description}
    \item[事前課題1]
      もくもくの会で行なう作業、質問などの課題を用意して教えてください。
    \item[事前課題2]
      前回(第81回)の勉強会に参加された方は、前回の作業や課題がその後どう
      なったか結果を教えてください。
    \item[事前課題3]
      LT 歓迎です。何かお話したい方はタイトルを下さい。
    \end{description}
  \end{block}
\end{frame}

\takahashi[50]{事前課題\\発表}

\begin{frame}
  \frametitle{ のがたじゅん }

とくに考えてないです
\end{frame}

\begin{frame}
  \frametitle{ yyatsuo }
  \begin{enumerate}
  \item fcitx-skk の RFS します
  \item 前回不参加でした
  \item 特になし
  \end{enumerate}
\end{frame}

\begin{frame}
  \frametitle{ 木下 }
  \begin{enumerate}
  \item
    \begin{enumerate}
    \item グリッドコンピューティング関連の調査・研究
      \begin{itemize}
      \item GlobusToolkitで何ができる?
        →AndroidOSのクロスコンパイルで使えたら嬉しいかも。
      \end{itemize}
    \item Eucalyptus(プライベートクラウドとして)の調査・研究
    \item Debian7 on PANDABOARDの調査・研究
      \begin{itemize}
      \item WiFiモジュール(On Board:TI製)の有効化
      \item GPUデバイスドライバの有効化
      \end{itemize}
    \end{enumerate}
  \item
    \begin{enumerate}
    \item Debian7 on PANDABOARDの調査・研究
      \begin{itemize}
      \item WiFiモジュール(USB)の接続

        実績:完了
      \item GPUデバイスドライバの有効化

        実績:保留
      \end{itemize}
    \end{enumerate}
  \end{enumerate}
\end{frame}

\begin{frame}
  \frametitle{ 川江 }
  \begin{enumerate}
  \item 特にないです。ま、状況に応じてVMのdebianをインストールします。
  \item 前回はspiceの件ですが、課題のサーバのUEFIをupdateしたためにGRUB
    が飛んで、Debianを再インストールしている状況です。
  \item できれば4月ぐらいにkvmについてまとめたいと思っています。
  \end{enumerate}
\end{frame}

\begin{frame}
  \frametitle{ 山城の国の住人 久保博 }
  \begin{enumerate}
  \item fpm2 の Segmentation fault が発生する不具合 \#647440 を修正する
    パッチを投稿できるように作業する。
  \item 前回は来ていませんでした。
  \end{enumerate}
\end{frame}

\begin{frame}
  \frametitle{ Hideaki Oose }
  \begin{enumerate}
  \item ApacheSolrを使い情報検索の基礎を学ぶ。
  \item kfreebsd環境での無線LAN使用の可否を調べていますが、現在も未
    解決の状態です。同一環境下でネイティブFreeBSDと比較し可能性を検証
    中です。
  \end{enumerate}
\end{frame}

\begin{frame}
  \frametitle{ murase\_{}syuka }
  \begin{enumerate}
  \item mruby-debianパッケージの更新
    \begin{itemize}
    \item 更新作法がよくわかってないので
    \end{itemize}
  \end{enumerate}
\end{frame}

\begin{frame}
  \frametitle{ かわだてつたろう }
  \begin{enumerate}
  \item VCSで設定ファイルをどう管理しているのか聞いてみたい。
    あと、年度末事務作業を片付けたい。。
  \item すいません、寝込んでました。
  \end{enumerate}
\end{frame}

\begin{frame}
  \frametitle{ 佐々木洋平 }
  \begin{enumerate}
  \item 毎度言ってる気がするが, tDiary と jekyll をなんとかしたい。
  \item 同上
  \item 多分、ネタは無いです。
  \end{enumerate}
\end{frame}

\begin{frame}
  \frametitle{ lurdan }
  \begin{enumerate}
  \item python-social-auth のパッケージを作る予定です
  \item 3D Graphics のハードルが高すぎるので諦め気味です (MMP)
  \end{enumerate}
\end{frame}

\begin{frame}
  \frametitle{ daism666 }
ワイヤレス環境構築にチャレンジしたいです。

Potato版を大学時代に触ってましたが、ここ八年触っておらず。

最近、WinXpを捨てこちらに戻ってきました。

よろしくお願いいたします。
\end{frame}

\begin{frame}
  \frametitle{ 立川 勝宣 }
以前、一度参加させて頂きました。

3Dプリンター挑戦中です。
興味が有り、

参加させてください。

よろしくお願い致します。

実は、LTの意味解りません。
\end{frame}


\takahashi[50]{そんな\\こんなで}
\takahashi[120]{次}

\section{Debianで楽しむ3Dプリンティング}
\takahashi[30]{Debianで楽しむ\\3Dプリンティング\\by\\八津尾}

\takahashi[50]{そんな\\こんなで}
\takahashi[120]{次}

\section{もくもくの会}
\takahashi[30]{もくもくの会}

\takahashi[50]{そんな\\こんなで}
\takahashi[120]{次}

\section{今後の予定}
\begin{frame}[fragile]
\frametitle{今後の予定}

\begin{block}{第83回関西Debian勉強会}
  \begin{itemize}
  \item 日時: 4月27日(日) 13:30 -
  \item 場所: 福島区民センター
  \end{itemize}
\end{block}

\begin{block}{第112回東京エリアDebian勉強会}
  \begin{itemize}
  \item 日時: 4月19日(土)
  \item 場所、内容: 未定
  \end{itemize}
\end{block}

\end{frame}

\takahashi[50]{  }

\end{document}
%%% Local Variables:
%%% mode: japanese-latex
%%% TeX-master: t
%%% End:
