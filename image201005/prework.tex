
\begin{prework}{ 和田健 }

普段使っているLinuxディストリビューション:Debian /GNU Linux
使ってみたいLinuxディストリビューション:特になし
使っている理由:パッケージ管理が楽
魅力:パッケージ管理が楽
不満:特になし


\end{prework}



\begin{prework}{ 藤沢理聡(risou) }

普段使っているディストリビューションはDebianとRedHat。
RedHatは仕事でしか使ってません。Debianを使ってる理由は大学時代に所属していた研究室のサーバがDebianだったから。他のディストリビューションに触れる機会もあったけれど、結局一番使いなれたものをずっと使っている感じです。使ってみたいディストリビューションは、UbuntuやGentooなど。今まで使う機会がなかったので、一度使ってみたい、という簡単な動機です。

\end{prework}



\begin{prework}{ 吉野(yy\_y\_ja\_jp) }

普段 Debian を使っています.色々な意味で自由だからです.不満はあまり感じてません.Ubuntu さんはより global な Debian にもっと成果を還元していただけるとうれしく思います.

\end{prework}



\begin{prework}{ 原口秀康 }

debian
軽い

個人的にはubuntuですが
会社がred hatを使用しているのでred hatにも興味があります。

\end{prework}



\begin{prework}{ hard2259 }

普段使っているLinuxディストリビューション
\begin{itemize}
\item VineLinux
\item Ubuntu
\item CentOS
\end{itemize}

使っている理由

\begin{itemize}
\item Vine\\
      学校の授業で指定されてるから。\\
      いい意味で枯れた技術といわれたものを多く使っているため、Linux系を一から勉強するにはもってこいのモノらしいので。\\

\item Ubuntu \\
      ゼミ室の先輩から勧められた\\
      Windowsに近いモノがあるから

\item CentOS \\
      ゼミ室のサーバのOSがCentOSだから使わざるを得ない。RedHat系の構造と似ているから、会社はいったとき役立つかも。\\
      ※不満を言えるほど使い込んでませんごめんなさい。
\end{itemize}


使ってみたいLinuxディストリビューション\\
・Gentoo

使ってみたい理由\\
・Gentoo\\
 色々難しいといわれているから挑戦してみたい。

\end{prework}



\begin{prework}{ キタハラ }

普段使っているLinuxディストリビューション:debian, Ubuntu
  無料で使える、ライセンス管理で悩まなくて良い、何か使いたいソフトが
あると「apt-get」ですぐ試せる。 でも、環境によって音が出なかったり、
グラフィックやプリンタのドライバで悩んだり、Windows依存のWebページに
アクセスできなかったり、人に勧めるには少々悩ましい所がある。

使ってみたいLinuxディストリビューション:Ubuntu(ARM), Android(?)
  モバイル用途で利用している「Zaurus」の製品寿命がつきているので、
その後継として「NetWalker」か「JN-DK01」とか「IS01」のAndroid機が
欲しいなぁーと、思ってまして・・・。


\end{prework}



\begin{prework}{ KIM\_TPDN }

デスクトップ用途でopenSUSE、VPSでCentOSを使っています。ちなみに自宅鯖はFreeBSDを使用しています。

使っている理由
openSUSE
・YaSTがかなり便利。
・KDEが好き。

CentOS
・初めて使ったディストロだから。一通りのことをこれで学習した。
・VPSや専鯖のOS選択肢にはたいてい入ってる。

使ってみたいディストリビューション
・Arudius
・Ubuntu Studio



\end{prework}



\begin{prework}{ 上川純一 }

いまふと振り返ると普段もっとも使っている Linux ディストリビューションは Android です。
携帯電話にプリインストールされているので使い倒しています。
不満点はカーネルをいじれないことと、emacsが動かないことです。


\end{prework}



\begin{prework}{ compozz }

・普段使っているディストリビューション
 Ubuntu
・使ってみたいディストリビューション
 Fedora

・(Ubuntuを)使っている理由
 ユーザーの多さや安定性がある(と思っている)からです。

・魅力
 KNOOPIXのLiveCDを使おうとした時、起動時からディスプレイドライバではまったけれど、Ubuntuは設定変更しなくてもインストール画面からGUIで感動しました。
 とにかくLinuxの中では導入や運用が楽だと思うので、普段使いにとても役立っています。

・不満点
 ディストリビューションの不満はほとんど無いですが、詰まったときにUbuntuのバグだったりしてがっかりしたことがあります。

・(Fedoraを)使ってみたい理由
 使ってみたことがないからです。 Gentooは時間があれば。
Debian勉強会なのにすみません。。。

\end{prework}



\begin{prework}{ mnakao }

普段使っているLinuxディストリビューション:
それはDebianでしょう!

使ってみたいLinuxディストリビューション:
openSUSEを仕事で少し使ってみたけど、結構便利そうなので気になってます。

(Debianを)使っている理由、魅力:
aptが便利なのと、一度インストールしたら、再インストールする必要がないこと(私はヘタレなので、何度もクリーンインストールしてますが:p)

(Debianの)不満点:
入門者はその文化に慣れるのが大変な所。

(Suseを)使ってみたい理由:
YaSTがあるから。LDAPの設定もYaSTで出来るので便利そう。


\end{prework}



\begin{prework}{ koedoyoshida }

普段使っているLinuxディストリビューション
Debian,Ubuntu,Asianux,RHEL
使ってみたいLinuxディストリビューションは何ですか?
Gentoo
使っている理由
Debian:パッケージが多い。stableのバージョンアップが遅いので、サーバ向けに安定している。インストール後の手間が掛からない。
Ubuntu:パッケージが多い。カーネルが新しく、LiveCDが標準なので新型機で各種デバイスの動作を確認するのに便利。インストールの手間が掛からない。
Asianux:飯の種
RHEL:良くも悪くも標準で参考になる
魅力、不満点、使ってみたい理由
Gentoo:パッケージの多さ、手間がかかりそうなところ、マニア向け


\end{prework}



\begin{prework}{ moli3 }

普段使っているディストリビューション
CentOS
両方初心者にはなかなか難しいよく躓くことがあります。
サーバに向いていると聞いてサーバとして使っていました。
Debian
最近サーバとディスクトップをサーバにしました。CentOSと違う点で
戸惑うときがあります。
使ってみたいディストリビューション
gentoo
よくわからないけど話題なので入れてみたいです。Debianからchrootで
やろうかと思ってます。

\end{prework}



\begin{prework}{ 村田信人 }

使っている: Ubuntu
+ 半年ごとに完成度の高いバージョンがリリースされるというサイクル
+ Humanityアイコンテーマ
+ Mark Shuttleworthの行動力
- Mark Shuttleworthの行動力

使ってみたい: Debian
stable, testing, unstableを並列で進めているところ。

\end{prework}



\begin{prework}{ akedon }

普段使っているLinuxディストリビューションはDebian,CentOS,MIKO GNYO/Linux (Ubuntu Desktop) です。
使ってみたいLinuxディストリビューションはSuSEですね。
前記のLinuxを使っている理由はDebianは使い始めた当時は圧倒的にaptツールが他の管理ツールrpmより使いやすかったので、そこから使い続けて慣れているからです。CentOSはXenを使い始めた時に一番手軽で確実に動いたからです。MIKO GNYO/Linuxは壁紙が素敵ですよね、落ち着いて作業するには最適です。 
魅力はLinux全体に言えることですが、不要な機能を止めたり削除したり、必要最低限に絞り込みやすく、使いやすい様にカスタマイズしやすいという点にあります。
不満点は、デバイスドライバ等、ベンダに依存しているものがWindowsやMacOSに見劣りするものがあったりする点になります。
SuSEを使ってみたい理由は単純で、未だ試していないからです。

\end{prework}



\begin{prework}{ yama1066 }

普段使っているLinuxディストリビューション:
Debian GNU/Linux (sid, squeeze on lenny)
使っている理由:なんとなく。
魅力:なにかあったかな?(お
不満点:特に無し。

普段使ってはいないが、apt-get upgrade だけしているLinuxディストリビューション:
Ubuntu Linux (maverick)
使っている理由:使ってない。
魅力:Debian上でも chroot で共存可能。
不満点:特に無し。

使ってみたいLinuxディストリビューション:Fedora (Rawhide)
使ってみたい理由:人柱は重要だよね。

\end{prework}



\begin{prework}{ 岩松 信洋 }

普段使っているLinuxディストリビューション
Debian, Gentoo, buildroot, openembedded, オレオレbusyboxベースディストリ

使ってみたいLinuxディストリビューション
Arch Linux とか。なんか人気があるようです。
使っている理由、魅力、不満点、使ってみたい理由
Debian: 仕事と開発用。
Gentoo: 開発用で主にGCCのHEADを追っかけるのに使っています。
buildroot: 仕事で。クロスコンパイラとか一式をとりあえず作る場合に利用しています。ディストリというのかは不明。
openembedded: 最近いじりはじめました。shサポートとか。
オレオレbusyboxベースディストリ: 


\end{prework}



\begin{prework}{ monoqlo }

普段使っているLnuxディストリビューション:Ubuntu
使ってみたいLinuxディストリビューション:しばらくUbuntuでいいです。
使っている理由:日本語の情報がたくさんあることと、初めてLinuxを使おうと思った時にWindowsからの移行が楽だと感じたため。
魅力:初心者にも勧めやすい
不満点:とくになし

\end{prework}



\begin{prework}{ なかおけいすけ }

使っているディストリビューションはDebianで、長期的に使えそうなのが魅力です。
使ってみたいディストリビューションは、Scientific Linuxです。

\end{prework}



\begin{prework}{ henrich }

普段は debian、その上の KVM で fedora と ubuntu がたまに動きます。

使ってみたいのは…CentOSかなぁ。何が良いのか魅力がよく分からんので。後大元の RHEL。2.1 ぐらいの時にインストールした記憶しかないので。

\end{prework}



\begin{prework}{ まえだこうへい }

普段使っているディストロはDebianです。他に使ってみたいのは無いです。複数種類のアーキテクチャで同じシステムを使える(サポートしているアーキテクチャが多い)のが一番の理由です。(いつも言っている理由と同じで代わり映えしませんが)

\end{prework}



\begin{prework}{ 倉世古 恭平 }

普段: Fedora 11。適当に動くので。情報が多くて詰まって死ぬことが少ない。変にこだわりをもって使ってる人が多いのでたまに議論がかみ合わなくて困る
使ってみたい: OpenSUSE。結構敷居が高いと思っていたら、最近周囲で使っている人が増えたので。

\end{prework}



\begin{prework}{ 堀本 貴幸 (opentaka) }

普段使っているLinuxディストリビューションと使ってみたいLinuxディストリビューションは何ですか?
---
Gentoo Linuxをメインに、Debianをeee pcで使っている。
LFS(Linux From Scratch)を使いこなしてみたい。
---

使っている理由、魅力、不満点、使ってみたい理由を教えてください。
---
Gentoo Linux: 
Portageに多くのパッケージがある。

不満点: bleeding edgeすぎて、本当にbleedingする時がある。(壊れる)
---

---
Debian(debian eee project): 
eeepc向けのディストリビューションで一番簡単にインストールできそうだったから。
---

\end{prework}


