\documentclass[cjk,dvipdfmx,10pt,%
hyperref={bookmarks=true,bookmarksnumbered=true,bookmarksopen=false,%
colorlinks=false,%
pdftitle={第 63 回 関西 Debian 勉強会},%
pdfauthor={倉敷・のがた・かわだ・佐々木},%
%pdfinstitute={関西 Debian 勉強会},%
pdfsubject={資料},%
}]{beamer}

\title{第 63 回 関西 Debian 勉強会}
\subtitle{{\scriptsize 資料}}
\author[かわだ てつたろう]{{\large\bf 倉敷・のがた・かわだ・佐々木}}
\institute[Debian JP]{{\normalsize\tt 関西 Debian 勉強会}}
\date{{\small 2012 年 8 月 26 日}}

%\usepackage{amsmath}
%\usepackage{amssymb}
\usepackage{graphicx}
\usepackage{moreverb}
\usepackage[varg]{txfonts}
\AtBeginDvi{\special{pdf:tounicode EUC-UCS2}}
\usetheme{Kyoto}
\def\museincludegraphics{%
  \begingroup
  \catcode`\|=0
  \catcode`\\=12
  \catcode`\#=12
  \includegraphics[width=0.9\textwidth]}
%\renewcommand{\familydefault}{\sfdefault}
%\renewcommand{\kanjifamilydefault}{\sfdefault}
\begin{document}
\settitleslide
\begin{frame}
\titlepage
\end{frame}
\setdefaultslide

\begin{frame}[fragile]
\frametitle{Agenda}

\tableofcontents

\end{frame}

\section{最近の Debian 関係のイベント}

\takahashi[40]{最近の Debian\\関係のイベント}

\begin{frame}[fragile]
\frametitle{第 62 回関西 Debian 勉強会@OSC2012Kansai@Kyoto}

\begin{itemize}
\item 日時: 8 月 3 日、4 日
\item 場所: 京都リサーチパーク
\end{itemize}
\begin{block}{内容}
  \begin{itemize}
  \item 「Debian 7.0 ``wheezy'' frozen」
  \end{itemize}
\end{block}
\end{frame}

\begin{frame}[fragile]
  \frametitle{第 91 回 東京エリア Debian 勉強会}
  \begin{itemize}
  \item  日時: 8 月 21 日
  \end{itemize}
  \begin{block}{内容}
    \begin{itemize}
    \item Debconf12 参加報告
    \item 月刊 Debhelper
    \item ソフト開発以外の簡単 Debian contribution(ドラフト版!)
    \item Debian での C++11 開発環境
    \end{itemize}
  \end{block}
\end{frame}

\takahashi[50]{そんな\\こんなで}
\takahashi[120]{次}

\section{事前課題発表}

\takahashi[50]{事前課題}

\begin{frame}[fragile]
\frametitle{事前課題}

\begin{block}{今回の事前課題}
  \begin{description}
  \item[事前課題1] MIT版Kerberosでサービスに必要となるプロセス名と、そのプロセスが使用するポートを 2 つ以上挙げてください。

  \item[事前課題2] DebConf12での資料を予習しておいてください。\\
   \url{http://penta.debconf.org/dc12_schedule/events/894.en.html}
  \end{description}
\end{block}

\end{frame}

\takahashi[50]{事前課題\\発表}

\begin{frame}
  \frametitle{ kazken3 }
  (無回答)
\end{frame}

\begin{frame}
  \frametitle{ 岡野孝悌 }
  (無回答)
\end{frame}

\begin{frame}
  \frametitle{ 川江 }
  \begin{enumerate}
  \item 認証プロセス 749 750
  \end{enumerate}
\end{frame}

\begin{frame}
  \frametitle{ とみー }
  \begin{enumerate}
  \item まったくわかりませんが、参加して勉強したいとおもいます。
  \end{enumerate}
\end{frame}

\begin{frame}
  \frametitle{ かわだてつたろう }
  \begin{enumerate}
  \item sid 環境で krb5-\{admin-server,kdc\} をインストールして確認。
    \begin{itemize}
    \item krb5kdc 88/udp 750/udp
    \item kadmind 464/udp 464/tcp 749/tcp
    \end{itemize}
  \item 見ておきます。
  \end{enumerate}
\end{frame}

\begin{frame}{ okumura.d }
  \begin{enumerate}
  \item kadmind プロセス。tcp及びudpポート番号は88,749.\\
    \#750(Kerberos RFile)はWindows限定..?
  \end{enumerate}
\end{frame}

\begin{frame}
  \frametitle{ 佐々木洋平 }
  \begin{enumerate}
  \item 
    \begin{itemize}
    \item process
      \begin{itemize}
      \item krb5-admin-server
      \item krb5-kdc
      \end{itemize}
    \item port
      \begin{itemize}
      \item 88/udp, 754/tcp, 760/tcp
      \end{itemize}
    \end{itemize}
  \item 見ておきまする。 
  \end{enumerate}
\end{frame}

\begin{frame}
  \frametitle{ 木下 }
  \begin{enumerate}
  \item ケルベロス認証では、「KDC(Key Distribution Center)」と呼ばれるサーバーを用意し、そこに認証情報を一元管理。\\
    →ユーザーが複数サーバーを利用する場合、一度認証を受けるだけで、ほかのサーバーへもアクセスできるようになり、何度もログインする必要がなくなる。
    \begin{enumerate}
    \item サービスに必要となるプロセス名
      \begin{itemize}
      \item krb5kdc(Kerberos サーバー)
      \item kadmin(kadmin ユーティリティ)
      \item krb524(?不明)
      \end{itemize}
    \item サービスに必要となるプロセス名
      \begin{itemize}
      \item すみません。時間がなく調べられていません。
      \end{itemize}
    \end{enumerate}
  \item 時間の許す限り見ておきます。
  \end{enumerate}
\end{frame}

\begin{frame}
  \frametitle{ yyatsuo }
  \begin{enumerate}
  \item 勉強しておきます。
  \item 勉強しておきます。
  \end{enumerate}
\end{frame}

\begin{frame}
  \frametitle{ 0xBCD1BC92 }
  \begin{enumerate}
  \item インストールしたらわかるのかもしれないが、/etc/init.dのプロセスをキックする名前でお茶を濁しておく。したの/etc/serivcesと連動しているかもしれず。
    \begin{itemize}
    \item krb5-admin-server
    \item krb5-kdc 
    \end{itemize}
    /etc/services から拾ったから正解だと思う。
    \begin{itemize}
    \item 88/udp
    \item 754/tcp         krb5\_prop hprop
    \item 760/tcp         kreg
    \end{itemize}
  \item 資料は、仕事の間ににらんでおきます。
  \end{enumerate}
\end{frame}

\begin{frame}
  \frametitle{ 西山和広 }
  \begin{enumerate}
  \item kadmind で TCP の 749 と 464
  \end{enumerate}
\end{frame}

\begin{frame}
  \frametitle{ 安部武志 }
  \begin{enumerate}
  \item プロセス名: krb5kdc (KDC)\\
    使用するポート番号: 88 および 750
  \end{enumerate}
\end{frame}

\begin{frame}
  \frametitle{ 岡 大輔 }
  \begin{enumerate}
  \item どこを調べればいいのかわからない。Kerberos認証がどういうものかはググることでわかった。もう少しクローリングしてみる。
  \end{enumerate}
\end{frame}

\begin{frame}
  \frametitle{ 山城の国の住人 久保博 }
  \begin{enumerate}
  \item squeeze 環境で MIT Kerberos を動かして、サービスが使うポート番号を次のように調べました。
    \begin{quote}
lsof -p `pgrep krb5kdc`\\
lsof -p `pgrep kadmind`
    \end{quote}
    \begin{itemize}
    \item プロセス名: krb5kdc 
    \item ポート番号: 88/udp 750/udp
    \item プロセス名: kadmind
    \item ポート番号: 749/tcp 464/tcp 464/udp
    \end{itemize}
  \item はい。当日までには何とか。
  \end{enumerate}
\end{frame}

\begin{frame}
  \frametitle{ lurdan }
  \begin{enumerate}
  \item 続きは We\^{}h\^{}h セッションで!
  \item 一応ビデオ見ました。難しいです。
  \end{enumerate}
\end{frame}

\takahashi[50]{そんな\\こんなで}
\takahashi[120]{次}

\section{Debian ではじめる Kerberos 認証}
\takahashi[25]{Debian ではじめる\\ Kerberos 認証 \\by\\ 倉敷 悟}

\takahashi[50]{そんな\\こんなで}
\takahashi[120]{次}

\section{News from EDOS: finding outdated packages}
\takahashi[25]{News from EDOS:\\ finding outdated packages \\by\\ Ralf Treinen}

\takahashi[50]{そんな\\こんなで}
\takahashi[120]{次}

\section{今後の予定}
\begin{frame}[fragile]
\frametitle{今後の予定}

\begin{block}{第 64 回関西 Debian 勉強会}
\begin{itemize}
  \item 日時: 9 月 23 日(日)
  \item 会場: 福島区民センター
\end{itemize}
\end{block}

\begin{block}{東京エリア Debian 勉強会}
  \begin{itemize}
  \item OSC Tokyo Fall
  \item 日時: 9 月 15 日(土)
  \end{itemize}
\end{block}

\end{frame}


\takahashi[50]{  }


\end{document}
%%% Local Variables:
%%% mode: japanese-latex
%%% TeX-master: t
%%% End:
