%; whizzy-master ../debianmeetingresume201211.tex
% 以上の設定をしているため、このファイルで M-x whizzytex すると、whizzytexが利用できます。
%

\santaku
{DebConf13 の開催地と開催日は?}
{日本 東京都 6月29日}
{ニカラグア マナグア 7月8-14日}
{スイス ヴォーマルキュ 8月11-18日}
{C}
{ニカラグアはDebConf12の開催地です。
DebConf13はスイスのキャンプ地で開催です。
6/29は皆さん予定を空けておきましょう。}

\santaku
{世界のWebサーバで最も人気のあるLinux ディストリビューション(W3Techs調べ)は?}
{CentOS}
{Debian}
{Ubuntu}
{B}
{\url{http://w3techs.com/technologies/history_details/os-linux}に結果のグラフがあります。
現在 Linux を使用している web サーバの 32.9\% が Debian を利用しており、その割合は現在も増加を続けているそうです。}

\santaku
{現在Debian プロジェクトリーダ選挙 2013(Debian Project Leader Elections 2013)が開催されています。現在(3/16)のステータスは?}
{指名期間\\Nomination period}
{選挙運動期間\\Campaigning period}
{投票期間\\Voting period}
{B}
{指名期間:2013 年 3 月 3 日(日) 00:00:00 UTC 〜\\
3 月 9 日(土) 23:59:59 UTC
\\
選挙運動期間:2013 年 3 月 10 日(日) 00:00:00 UTC 〜\\
3 月 30 日(土) 23:59:59 UTC
\\
投票期間:2013 年 3 月 31 日(日) 00:00:00 UTC 〜\\
4 月 13 日(土) 23:59:59 UTC
\\
\url{http://www.debian.org/vote/2013/vote_001}}


\santaku
{Ben Hutchings さんが次期 Debian 安定版と一緒に出荷される Linux カーネルに (3.2 系列の mainline には無い) 追加機能が搭載される予定であると述べています。
多くの追加点の中に含まれないものは何?}
{リアルタイム性を強化するPREEMPT\_RT}
{Hyper-V guest drivers\\強化}
{TCP接続高速化技術\\TCP Fast Open}
{C}
{Ben Hutchings さんはDebian カーネルチームのメンバーであり、kernel.org の 3.2.y 安定版系列のメンテナ。
PREEMPT\_RTはI/Oの遅延と揺らぎを低減するaudio/videoアプリケーション利用等向、
linux-image-rt-amd64 , linux-image-rt-686-paeで使用可能。
Hyper-V guest driversは3.2にも含まれていますが、より改善された3.4からの修正を導入。
TCP Fast Open(TFO)はTCPの3way-handshakeを改善し、2回目以降の通信開始時にSYNとdataを含んだパケットを送ることで高速化する技術、3.6時点ではクライアント側のみ実装、サーバ側は3.7から。}

\santaku
{Wookeyさんがアナウンスしたalpha版のDebian port ARM64 imageは?}
{Debian/Ubuntu \\port image}
{Debian/KFreeBSD \\port image}
{Debian/GnuHurd \\port image}
{A}
{x86環境を使用しないself-bootstrap imageです。\url{http://wiki.debian.org/Arm64Port}でステータスが確認できます。mainline kernelのARM64/AArch64(ARMv8アーキテクチャ:Cortex A57やCortex A53と呼ばれる予定のCPU向け)対応は 3.7から。}

\santaku
{700,000番目のバグが報告された日を当てる700000thBugContestの結果が出ました。その予想日と対象バグの報告日は?}
{予想日:2013/02/04、\\報告日:2013/02/14}
{予想日:2013/02/07、\\報告日:2013/02/14}
{予想日:2013/02/14、\\報告日:2013/02/07}
{C}
{最も近い2013/02/14を予想したChristian Perrierさんが当てました。結果は\url{http://wiki.debian.org/700000thBugContest}で公開されています。
また、800,000/1,000,000番目のバグが報告される日を当てるコンテスト\url{http://wiki.debian.org/800000thBugContest}も開催されています。}

\santaku
{master.debian.orgが新しい機械に移行されました。これは何のサーバでしょうか ?}
{@debian.orgのメールサーバ}
{パッケージのマスターサーバ}
{パッケージのスポンサー(mentor)を探すサーバ}
{A}
{古いサーバはディスク障害等があったので、寿命と判断され、データが損失する前に新しいサーバに移行されました。ftp-master.debian.orgはDebianの official package リポジトリです。パッケージのスポンサー(mentor)を探すのはmentors.debian.net。 }

\santaku
{pbuilder でビルドする際に gcc の代替として Clang を使いやすくするパッチが追加されました。誰が書いたパッチでしょうか?}
{Sylvestre Ledru}
{Junichi Uekawa}
{Hideki Yamane}
{C}
{Sylvestre Ledru さんはClangのメンテナーの一人です。彼はClang 3.2 を使用してアーカイブをリビルドした結果について報告しました(12.1\% が失敗、つまり18264packages中、2204が失敗)。失敗したパッケージの情報\url{http://clang.debian.net/pts.php}。DebianのClangサポートは着々と進んでいます。}

\santaku
{DPN - 2013年3月4日号に取り上げられた日本のイベントは}
{Open Source Conference 2013 Tokyo/Spring}
{Open Source Conference 2013 Hamamatsu}
{Open Source Conference 2013 Tokushima}
{A}
{\url{http://henrich-on-debian.blogspot.jp/2013/02/open-source-conference-2013-tokyospring.html} 詳細は後ほど。}

