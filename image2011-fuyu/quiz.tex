%; whizzy-master ../debianmeetingresume201101.tex
% 以上の設定をしているため、このファイルで M-x whizzytex すると、whizzytexが利用できます。
%
% ちなみに、クイズは別ブランチで作成し、のちにマージします。逆にマージし
% ないようにしましょう。
% (shell-command "git checkout quiz-prepare")

\santaku
{alioth.debian.orgが2台に分かれました。そのサーバ名は?}
{vasks.debian.org と wagner.debian.org}
{volks.debian.org と don.debian.org}
{dennys.debian.org と gusto.debian.org}
{A}
{ほかはファミレスの名前}

\santaku
{現在行われているPerl transition のPerlバージョンは?}
{5.12}
{5.13}
{5.14}
{A}
{5.14はまだexperimentalです。}

\santaku
{プライマリミラーサーバが新しく追加された国は?}
{チュニジア}
{中国}
{マダガスカル}
{B}
{チュニジアとマダカスカルはミラー。プライマリではない。}

\santaku
{mentors.debian.net を構築しているwebアプリケーションが変更されました。何に変わったでしょう?}
{Debmemtors}
{Debcomike}
{Debexpo}
{C}
{Python とTurbogears で
書かれたWebアプリケーション。
パッケージレビューやテストスィートを提供するらしい。}

\santaku
{debian-ports に追加された新しいアーキテクチャは?}
{s390x}
{ppc64}
{blackfin}
{A}
{s390x。aurel32 によって開始。blachfinはまだサポートされていない。}

\santaku
{新しくサポートされた圧縮形式は?}
{rar}
{cab}
{xz}
{C}
{可逆圧縮アルゴリズム LZMA
(Lempel-Ziv-Markov chain-Algorithm)を使った圧縮形式。
GNU zip に比べ、
約40\%圧縮率が向上している。
圧縮には時間がかかるが、伸長には時間がかからない。
}

\santaku
{Samuel Thibault がアナウンスした Debian GNU/Hurd の内容は?}
{Wheezy で Debian GNU/Hurd をリリースします!}
{なんつーか、飽きた。}
{DVDが読めないのでDVDイメージは配布しません。}
{A}
{Whezzy のリリースゴール対象に入れるようです。PorterBox も用意されました。}

\santaku
{Emdebian Grip はなぜ Debianのリポジトリに入れる事が可能なのか?}
{Debianだから。}
{Freeだから。}
{パッケージの互換性があるから。}
{C}
{Emdebian Grip はパッケージからドキュメントファイルなどのa 
組み込みには必要のないファイルを削除したパッケージ
を提供するディストリビューション。}

\santaku
{mentors.debian.net を構築しているwebアプリケーションが変更されました。何に変わったでしょう?}
{Debmemtors}
{Debcomike}
{Debexpo}
{C}
{Python とTurbogears で
書かれたWebアプリケーション。
パッケージレビューやテストスィートを提供するらしい。}

\santaku
{debian-ports に追加された新しいアーキテクチャは?}
{s390x}
{ppc64}
{blackfin}
{A}
{s390x。aurel32 によって開始。blachfinはまだサポートされていない。}

\santaku
{新しくサポートされた圧縮形式は?}
{rar}
{cab}
{xz}
{C}
{可逆圧縮アルゴリズム LZMA
(Lempel-Ziv-Markov chain-Algorithm)を使った圧縮形式。
GNU zip に比べ、
約40\%圧縮率が向上している。
圧縮には時間がかかるが、伸長には時間がかからない。
}

\santaku
{Samuel Thibault がアナウンスした Debian GNU/Hurd の内容は?}
{Wheezy で Debian GNU/Hurd をリリースします!}
{なんつーか、飽きた。}
{DVDが読めないのでDVDイメージは配布しません。}
{A}
{Whezzy のリリースゴール対象に入れるようです。PorterBox も用意されました。}

\santaku
{Emdebian Grip はなぜ Debianのリポジトリに入れる事が可能なのか?}
{Debianだから。}
{Freeだから。}
{パッケージの互換性があるから。}
{C}
{Emdebian Grip はパッケージからドキュメントファイルなどのa 
組み込みには必要のないファイルを削除したパッケージ
を提供するディストリビューション。}

\santaku
{Debian温泉2011の1日目はいつでしょうか?}
{9/17}
{9/18}
{9/19}
{A}
{さっきの話を聞いて(読んで)いればわかって当然ですね}

\santaku
{8月にDebianは誕生日を迎えました。何周年でしたでしょうか?}
{17}
{18}
{19}
{B}
{今年もお祝いしましたよね}

\santaku
{最新のDebian Newsはいつ発行されたでしょうか?}
{9/17}
{9/18}
{9/19}
{C}
{購読していれば知っていて当然ですね}

\santaku
{10/17の''delegation for the DSA team''で代表団に任命されなかったのは誰でしょう?}
{Faidon Liambotis}
{Luca Filipozzi}
{Nobuhiro Iwamatsu}
{C}
{他に任命されたのは、の全部で合計名です。}

\santaku
{Wheezyフリーズの予定はいつでしょう?}
{2012年4月}
{2012年6月}
{2012年8月}
{B}
{あと6ヶ月ですよ!}

\santaku
{1.16.1がリリースされたdpkgに該当するのはどれ?}
{\texttt{dpkg-buildpackage}コマンドでは \texttt{CFLAGS, CXXFLAGS, LDFLAGS, CPPFLAGS, FFLAGS}の\texttt{export}が必須になった}
{\texttt{dpkg-deb}コマンドに\texttt{--verbose}オプションが追加された}
{Multi-Archフィールドがサポートされた}
{B}
{\texttt{dpkg-buildpackage}ではこれらのオプションが不要になりました。Multi-Archは1.16.2からサポートされる予定です。\texttt{dpkg-deb -x/--extract -v/--verbose}で\texttt{dpkg-deb -X/--xextract}と同じ動きをするようになりました。}
