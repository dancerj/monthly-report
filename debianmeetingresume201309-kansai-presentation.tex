\documentclass[cjk,dvipdfmx,10pt,compress,%
hyperref={bookmarks=true,bookmarksnumbered=true,bookmarksopen=false,%
colorlinks=false,%
pdftitle={第 76 回 関西 Debian 勉強会},%
pdfauthor={倉敷・のがた・佐々木・かわだ・八津尾},%
%pdfinstitute={関西 Debian 勉強会},%
pdfsubject={資料},%
}]{beamer}

\title{第 76 回 関西 Debian 勉強会}
\subtitle{$\sim$発表資料$\sim$}
\author[かわだ てつたろう]{{\large\bf 倉敷・のがた・佐々木・かわだ・八津尾}}
\institute[Debian JP]{{\normalsize\tt 関西 Debian 勉強会}}
\date{{\small 2013 年 9 月 22 日}}

%\usepackage{amsmath}
%\usepackage{amssymb}
\usepackage{graphicx}
\usepackage{moreverb}
\usepackage[varg]{txfonts}
\AtBeginDvi{\special{pdf:tounicode EUC-UCS2}}
\usetheme{Kyoto}
\def\museincludegraphics{%
  \begingroup
  \catcode`\|=0
  \catcode`\\=12
  \catcode`\#=12
  \includegraphics[width=0.9\textwidth]}
%\renewcommand{\familydefault}{\sfdefault}
%\renewcommand{\kanjifamilydefault}{\sfdefault}
\begin{document}
\settitleslide
\begin{frame}
\titlepage
\end{frame}
\setdefaultslide

\begin{frame}[fragile]
\frametitle{Agenda}

\tableofcontents

\end{frame}

\section{最近の Debian 関係のイベント}

\takahashi[40]{最近の Debian\\関係のイベント}

\begin{frame}[fragile]
  \frametitle{第 75 回関西 Debian 勉強会}
  \begin{itemize}
  \item 日時: 8 月 25 日(日)
  \item 場所: 福島区民センター
  \end{itemize}
  \begin{block}{内容}
    \begin{itemize}
    \item 「puppet による構成管理の実践」
    \end{itemize}
  \end{block}
\end{frame}

\begin{frame}[fragile]
  \frametitle{第 104 回東京エリア Debian 勉強会}
  \begin{itemize}
  \item 日時: 9 月 21 日(土)
  \item 場所: あんさんぶる荻窪
  \end{itemize}
  \begin{block}{内容}
    \begin{itemize}
    \item 10 月の OSC 2013 Tokyo/Fall に延期
    \end{itemize}
  \end{block}
\end{frame}

\begin{frame}[fragile]
  \frametitle{Debian Project}
  \begin{itemize}
  \item 「Bits from the Release」
    \begin{itemize}
    \item Changing testing migration (NEW-TEST)
    \item Key-packages and automated removals (AUTO-RM)
    \item Automating Architecture (re-)Qualification (ARCH)
    \item Roll call for porters (ROLL-CALL)
    \end{itemize}
  \item 「Call for Jessie Release Goals」
  \item 「Debian Project News」
  \end{itemize}
\end{frame}

\takahashi[50]{そんな\\こんなで}
\takahashi[120]{次}

\section{事前課題発表}

\takahashi[50]{事前課題}

\begin{frame}[fragile]
  \frametitle{事前課題}
  \begin{block}{今回の事前課題}
    \begin{description}
    \item[事前課題1]
      サウンドデバイスを持つPCあるいはARMボードにDebian GNU/Linux Wheezy をインストールし、パッケージ「alsa-base」、「libasound2」、
      \sout{「libasound2-data」、}「libasound2-plugins」、「alsa-utils」をインストールしておいてください。

    \item[事前課題2]
      「dgit」を動かせる環境を用意しておいてください。

     (パッケージを持ってくるだけで wheezy 環境にもインストールできます。)
    \end{description}
  \end{block}
\end{frame}

\takahashi[50]{事前課題\\発表}

\begin{frame}
  \frametitle{ かわだてつたろう }
  \begin{enumerate}
  \item  sid 環境にインストールしました。
  \item インストールしましたが、clone に失敗しました。
  \end{enumerate}
\end{frame}

\begin{frame}
  \frametitle{ 西山和広 }
  \begin{enumerate}
  \item 仮想マシンで用意しました。
  \item 1で用意した環境を使える予定です。
  \end{enumerate}
\end{frame}

\begin{frame}
  \frametitle{ 木下 }
  \begin{enumerate}
  \item 間に合えばですが、PANDABOARDで試してみます。
  \item こちらも間に合えばやっておきます。
  \end{enumerate}
\end{frame}

\begin{frame}
  \frametitle{ おくの }
  勉強会までに間に合わせます$><$
\end{frame}

\begin{frame}
  \frametitle{ 川江 }
  了解です。
\end{frame}

\begin{frame}
  \frametitle{ yyatsuo }
  用意しておきます
\end{frame}

\begin{frame}
  \frametitle{ 山城の国の住人 久保博 }
  \begin{enumerate}
  \item はい、インストールしました
  \item はい、ソースパッケージを wheezy 環境でビルドしてインストールしました。
  \end{enumerate}
\end{frame}

\begin{frame}
  \frametitle{ 佐々木洋平 }
  インストールしました。dgit 便利そうですね。
\end{frame}

\begin{frame}
  \frametitle{ 坂本 貴史 }
  (無回答)
\end{frame}

\begin{frame}
  \frametitle{ 西原 }
  各パッケージインストール済みPCをを持参します。
\end{frame}

\begin{frame}{ lurdan }
  \begin{enumerate}
  \item すでにインストールされていました
  \item すみません動きません
  \end{enumerate}
\end{frame}
\takahashi[50]{そんな\\こんなで}
\takahashi[120]{次}

\section{Linuxとサウンドシステム}
\takahashi[30]{Linuxとサウンドシステム\\by\\坂本 貴史}

\takahashi[50]{そんな\\こんなで}
\takahashi[120]{次}

\section{dgit でソースパッケージを触ってみる}
\takahashi[30]{dgit でソースパッケージを触ってみる\\by\\倉敷 悟}

\takahashi[50]{そんな\\こんなで}
\takahashi[120]{次}

\section{今後の予定}
\begin{frame}[fragile]
\frametitle{今後の予定}

\begin{block}{第 77 回関西 Debian 勉強会}
  \begin{itemize}
  \item 日時: 10 月 27 日(日)
  \item 会場: 港区民センター
  \item 内容: 未定
  \end{itemize}
\end{block}

\begin{block}{第 104 回東京エリア Debian 勉強会}
  \begin{itemize}
  \item 日時: 10 月 19 日、20 日
  \item OSC 2013 Tokyo/Fall
  \end{itemize}
\end{block}

\end{frame}

\takahashi[50]{  }

\end{document}
%%% Local Variables:
%%% mode: japanese-latex
%%% TeX-master: t
%%% End:
