%; whizzy paragraph -pdf xpdf -latex ./whizzypdfptex.sh
%; whizzy-paragraph "^\\\\begin{frame}\\|\\\\emtext"
% latex beamer presentation.
% platex, latex-beamer でコンパイルすることを想定。 

%     Tokyo Debian Meeting resources
%     Copyright (C) 2012 Junichi Uekawa

%     This program is free software; you can redistribute it and/or modify
%     it under the terms of the GNU General Public License as published by
%     the Free Software Foundation; either version 2 of the License, or
%     (at your option) any later version.

%     This program is distributed in the hope that it will be useful,
%     but WITHOUT ANY WARRANTY; without even the implied warreanty of
%     MERCHANTABILITY or FITNESS FOR A PARTICULAR PURPOSE.  See the
%     GNU General Public License for more details.

%     You should have received a copy of the GNU General Public License
%     along with this program; if not, write to the Free Software
%     Foundation, Inc., 51 Franklin St, Fifth Floor, Boston, MA  02110-1301 USA

\documentclass[cjk,dvipdfmx,12pt]{beamer}
\usetheme{Tokyo}
\usepackage{monthlypresentation}

%  preview (shell-command (concat "evince " (replace-regexp-in-string "tex$" "pdf"(buffer-file-name)) "&")) 
%  presentation (shell-command (concat "xpdf -fullscreen " (replace-regexp-in-string "tex$" "pdf"(buffer-file-name)) "&"))
%  presentation (shell-command (concat "evince " (replace-regexp-in-string "tex$" "pdf"(buffer-file-name)) "&"))

%http://www.naney.org/diki/dk/hyperref.html
%日本語EUC系環境の時
\AtBeginDvi{\special{pdf:tounicode EUC-UCS2}}
%シフトJIS系環境の時
%\AtBeginDvi{\special{pdf:tounicode 90ms-RKSJ-UCS2}}

\newenvironment{commandlinesmall}%
{\VerbatimEnvironment
  \begin{Sbox}\begin{minipage}{1.0\hsize}\begin{fontsize}{8}{8} \begin{BVerbatim}}%
{\end{BVerbatim}\end{fontsize}\end{minipage}\end{Sbox}
  \setlength{\fboxsep}{8pt}
% start on a new paragraph

\vspace{6pt}% skip before
\fcolorbox{dancerdarkblue}{dancerlightblue}{\TheSbox}

\vspace{6pt}% skip after
}
%end of commandlinesmall

\title{東京エリアDebian勉強会}
\subtitle{第126回 2015年5月度}
\author{野島貴英}
\date{2015年5月23日}
\logo{\includegraphics[width=8cm]{image200607/openlogo-light.eps}}

\begin{document}

\begin{frame}
\titlepage{}
\end{frame}

\begin{frame}{設営準備にご協力ください。}
会場設営よろしくおねがいします。
\end{frame}

\begin{frame}{Agenda}
 \begin{minipage}[t]{0.45\hsize}
  \begin{itemize}
   \item 注意事項
	 \begin{itemize}
	  \item 写真はセミナールーム内のみ可です。
          \item 出入りは自由でないので、もし外出したい方は、野島まで一声くださいませ。
	 \end{itemize}
   \item 事前課題発表
  \end{itemize}
 \end{minipage} 
 \begin{minipage}[t]{0.45\hsize}
  \begin{itemize}
   \item 最近あったDebian関連のイベント報告
	 \begin{itemize}
	 \item 第125回 東京エリアDebian勉強会
	 \end{itemize}
   \item Debian Trivia Quiz
   \item 自然言語処理チームとパッケージ
   \item 今後のイベント
   \item 今日の宴会場所
  \end{itemize}
 \end{minipage}
\end{frame}

\section{事前課題}
\emtext{事前課題}
{\footnotesize
\begin{prework}{ $BLnEg(B }
  \begin{enumerate}
  \item Q.hack time$B$K2?$r$7$^$9$+!)(B\\
    A. Nook HD+$B$r$=$m$=$m(Bdebian$B$K!#(B
  \item ($B%*%W%7%g%s(B)Q.$B2?$K$D$$$FJ9$-$?$$!?;22C<T$HOC$r$7$?$$$G$9$+!)(B\\
    A. $B%=%U%H%&%'%"<+M3$r<i$jDL$9;v$KBP$7$F!"2?$,=PMh$k$+$K$D$$$F!#(B
  \end{enumerate}
\end{prework}

\begin{prework}{ wskoka }
  \begin{enumerate}
  \item Q.hack time$B$K2?$r$7$^$9$+!)(B\\
    A. MIPS Debian $B$N0\?"(B
  \item ($B%*%W%7%g%s(B)Q.$B$I$3$G:#2s$NJY6/2q$N3+:E$rCN$j$^$7$?$+!)(B\\
    A. $B$=$NB>(B
  \end{enumerate}
\end{prework}

\begin{prework}{ koedoyoshida  }
  \begin{enumerate}
  \item Q.hack time$B$K2?$r$7$^$9$+!)(B\\
    A. DDTSS,PyconJP$B4XO"(B
  \item ($B%*%W%7%g%s(B)Q.$B$I$3$G:#2s$NJY6/2q$N3+:E$rCN$j$^$7$?$+!)(B\\
    A. $BM'C#$dCN$j9g$$$+$iD>@\(B
  \end{enumerate}
  DDTSS: http://ddtp.debian.net/ddtss/index.cgi/ja
\end{prework}

\begin{prework}{ dictoss }
  \begin{enumerate}
  \item Q.hack time$B$K2?$r$7$^$9$+!)(B\\
    A. kFreeBSD$B$N(BIPsec$B4XO"$rD4$Y$k(B
  \item ($B%*%W%7%g%s(B)Q.$B$I$3$G:#2s$NJY6/2q$N3+:E$rCN$j$^$7$?$+!)(B\\
    A. Debian JP$B$N%a!<%j%s%0%j%9%H(B
  \end{enumerate}
\end{prework}

\begin{prework}{ yy\_y\_ja\_jp }
  \begin{enumerate}
  \item Q.hack time$B$K2?$r$7$^$9$+!)(B\\
    A. DDTSS 
  \item ($B%*%W%7%g%s(B)Q.$B$I$3$G:#2s$NJY6/2q$N3+:E$rCN$j$^$7$?$+!)(B\\
    A. $B$=$NB>(B
  \item ($B%*%W%7%g%s(B)Q.$B2?$K$D$$$FJ9$-$?$$!?;22C<T$HOC$r$7$?$$$G$9$+!)(B\\
    A. DDTSS
  \end{enumerate}
\end{prework}

}

\section{イベント報告}
\emtext{イベント報告}

\begin{frame}{第125回東京エリアDebian勉強会}

\begin{itemize}
\item 場所はスクウェア・エニックスさんのセミナルームをお借りしての開催でした。
\item 参加者は13名でした。
\item セミナ内容はまえだこうへいさんによる「.debからPythonパッケージへの変遷」でした。
\item 残りの時間でhack timeを行い、成果発表をしました。
\item 宴会の代わりに、「祥龍房 新宿イーストサイドスクエア店」で夕食会をやりました。
\end{itemize} 
  
\end{frame}

\begin{frame}{第125回東京エリアDebian勉強会(つづき)}

 セミナはまえださんにより、PythonパッケージのローカルDebian CIに関する発表と、Pythonに関するDebian公式パッケージの考察が行われました。大統一Debian勉強会から、毎年発表のはや3年越しのシリーズものとなります。昨今の言語関係の流行りや事情から、言語のパッケージをDebianのパッケージ化を検討した方が良い場合と、あまり向かない場合がある状況とのことです。会場にHaskelの詳しい人、Perlに詳しい人がおり、様々に議論がなされました。
  
\end{frame}

  
\section{Debian Trivia Quiz}
\emtext{Debian Trivia Quiz}
\begin{frame}{Debian Trivia Quiz}

  Debian の常識、もちろん知ってますよね?
知らないなんて恥ずかしくて、知らないとは言えないあんなことやこんなこと、
みんなで確認してみましょう。

今回の出題範囲は\url{debian-devel-announce@lists.debian.org},
\url{debian-news@lists.debian.org} に投稿された
内容などからです。

\end{frame}

\subsection{問題}

%; whizzy-master ../debianmeetingresume201311.tex
% $B0J>e$N@_Dj$r$7$F$$$k$?$a!"$3$N%U%!%$%k$G(B M-x whizzytex $B$9$k$H!"(Bwhizzytex$B$,MxMQ$G$-$^$9!#(B
%

\santaku
{Jessie$B$,L5;v%j%j!<%9$5$l$^$7$?!#$$$D$@$C$?$G$7$g$&$+!)(B}
{2015/4/11}
{2015/4/18}
{2015/4/25}
{C}
{$BL5;v(B4/25$B$K(BJessie(Debian 8)$B$,%j%j!<%9$5$l$^$7$?!#(BLinux kernel$B$O(B3.16.7$B!"(BGNU Compiler Collection 4.9.2$B!"(BGNOME 3.14$B!"(BApache 2.4.10$B!"(BPHP 5.6.7$B$J$I$N%P!<%8%g%s$N$b$N$,Ek:\$5$l$^$7$?!#Ek:\$5$l$F$$$k%Q%C%1!<%8?t$OA4It$G(B43,000$B%Q%C%1!<%80J>e$b$"$j$^$9!#(B}

\santaku
{Jessie$B$G=i$a$FDI2C$5$l$?$b$N$G$O$J$$$b$N$,:.$6$C$F$$$^$9!#$I$l!)(B}
{Debian Games Blend}
{OpenJDK}
{androidsdk-tools}
{B}
{Jessie$B$G$NJQ99E@$O%j%j!<%9%N!<%H(B(https://www.debian.org/releases/ jessie/amd64/release-notes/index.ja.html)$B$K$"$j$^$9!#(BDebian Games$B%A!<%`$h$j%2!<%`$N5M$a9g$o$;$N(BDebian Blend$B$,DI2C$5$l$?$j!"(Bandroid$B$N3+H/%D!<%k$N0lIt$bDI2C$5$l$?$j$7$F$$$^$9!#(B}

\santaku
{Debian GNU/Hurd 2015$B$b(B2015/4/30$B$K%j%j!<%9$5$l$^$7$?!#?4B!It$N(BGNU Mach$B$N%P!<%8%g%s$O$$$/$D!)(B}
{1.6}
{1.5}
{1.4}
{B}
{2015/4/10$B!A(B2015/4/15$B<~JU$G(BGNU/Hurd$B!J(Bupstream$BB&(B)$B$N%"%C%W%G!<%H$,9T$o$l!"(BGNU Hurd$B$O(Bversion 0.6$B$K!"(BGNU Mach$B$O(B1.5$B$K$J$j$^$7$?!#L$$@$K(BKVM/QEMU$B$J$I$N2>A[4D6-$G$NF0:n$,?d>)$N(B32bit$B%7%9%F%`$G$9$,!"B?$/$N(BOS$B$,%/%i%&%I4D6-$GF0:n$7$F$$$k@$$NCf$G$9$N$G!"%[%S!<MQES$NMxMQ$G$=$m$=$mCmL\$5$l$F$bNI$$$+$b!)$H;W$&<!Bh$G$9!#(B}

\santaku
{2015$BG/$N(BGSoC$B$K:NBr$5$l$?!"(BDebian MIPS ports$B$K$D$$$F$N3+H/FbMF$O<!$N$I$l!)(B}
{$BB??t$N%S%k%I=PMh$J$$%Q%C%1!<%8$r!"$A$c$s$H%S%k%I$G$-$k$h$&$K$9$k!#(B}
{$B?7$7$$(BMIPS CPU$B$X$NBP1~(B}
{$B?7$7$$(BMIPS CPU$BEk:\@=IJ$X$NBP1~(B}
{A}
{Debian MIPS ports$B$NB??t$N%Q%C%1!<%8$O:F%S%k%I$,=PMh$J$$!"%$%s%9%H!<%k$K<:GT$9$k$b$N$,B??t$"$k>u67$G$9!#$3$A$i$r:F%S%k%I$G$-$k$h$&$K=$@5$7!"%$%s%9%H!<%k=PMh$k$h$&$K$9$k;v$,:NBr$5$l$^$7$?!#(BGSoC$B$K8B$i$:!"$$$D$G$bK\7o$N6(NO<TJg=8Cf$G$9$N$G!"(BMIPS ports$B$K6=L#$"$kJ}$O%Q%C%1!<%8$N=$@5$K@'Hs$46(NO$/$@$5$$$^$;!<!#(B}

\santaku
{$B?k$K(Bhttp.debian.net$B$,(Bdebian.org$B$N%$%s%U%i$K0\F0$H$J$j$^$7$?!#?7$7$$(BURL$B$O$I$l!)(B}
{http://http.debian.org/debian}
{http://httpredir.debian.org/debian}
{http://www.debian.org/}
{B}
{http.debian.net$B$O(BHTTP$B$N(BRedirect$B$r3hMQ$7!"%f!<%6$K:G$bE,@Z$J>l=j$K$"$k!"%Q%C%1!<%8$N%_%i!<@h$r65$($F$/$l$k%5!<%S%9$G$9!#?k$K!"<B83E*%5!<%S%9$N07$$$G$"$k(Bdebian.net$B$+$i(BDebian$B8x<0%5!<%S%9$G$"$k(Bdebian.org$B$X0\F0$H$J$j$^$7$?!#(Bhttp.debian.net$B$r(Bsource.list$B$K@_Dj$7$F$$$k?M$O!"(Bhttpredir.debian.org$B$X=$@5$r$*$M$,$$$7$^$9!#$^$?!"IT6q9g$O(B"mirrors" pseudo-package$B$G(BBTS$B$9$k$h$&$G$9!#;2>H!'(Bhttp://bugs.debian.org/mirrors$B!!(B}

\santaku
{debian$B$X$N(Blibdvdcss/ZFS$B%Q%C%1!<%8Ek:\$N7o$K$D$$$F!"(B2015/5$B$N(BBit From DPL$B$GJs9p$5$l$?>u67$O0J2<$N$I$l(B?}
{DPL$B$,$$$m$$$m5DO@$7$F2s$C$F$$$k>u67(B}
{$BEk:\F|3NDj$7$?(B}
{$BEk:\$rD|$a$?(B}
{A}
{2015/5/19$B8=:_!"(BFTPMasters, SFLC, FSF$B$H5DO@Cf$H$N$3$H!#$5$i$K6a!9(BLinux Foundation$B$H$b5DO@M=Dj!#8=:_!"Ev=iA[Dj$7$?$h$j$b!"<B8=$K4X$9$k>u67$,J#;($H$N$3$H$G!"K\7o$$$D7hCe$9$k$+8+$($J$/$J$C$?$H$N;v$G$9!#(B}



\section{自然言語処理チーム(pkg-nlp-ja)とパッケージ}
\emtext{自然言語処理チーム(pkg-nlp-ja)とパッケージ}

\section{今後のイベント}
\emtext{今後のイベント}
\begin{frame}{今後のイベント}
\begin{itemize}
 \item 2015 OSC Hokkaido
  \url{}
 \item 関西エリアDebian勉強会
 \item 東京エリアDebian勉強会(6/20)
\end{itemize}
\end{frame}

\section{今日の宴会場所}
\emtext{今日の宴会場所}
\begin{frame}{今日の宴会場所}
未定
\end{frame}

\end{document}

;;; Local Variables: ***
;;; outline-regexp: "\\([ 	]*\\\\\\(documentstyle\\|documentclass\\|emtext\\|section\\|begin{frame}\\)\\*?[ 	]*[[{]\\|[]+\\)" ***
;;; End: ***
