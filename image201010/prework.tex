今回の事前課題は以下です:

\begin{prework}{ 石川勇 }
 本業ではフロントのサービスばかりを作っている自分ですが自宅にはdebian
 etchサーバが二台あります。以前zopeと相性がいいとかセキュアだとかいうの
 を小耳に挟んでセットアップしたものです。

 こいつらあまりメンテしておらず自宅で運営してるWEBサービスが吐くapachelog
 を圧縮してログサーバに転送したりdnsサービスを使って定期更新しているといっ
 た最低限の利用でした。


 ところがある日、SSDをサイズ違いで買い間違えるわけです。返品してもいいの
 ですが、せっかくなのでSSDなサーバでも作ってみようかなと思いたちました。
 ちょこちょこ調べてみると、まだコストパフォーマンス的には使えないという
 意見が多数。それでもいいから遊ぶなら何の用途に使うと楽しいのかなと探し
 ているとmysqlの高速化や巨大memcachedとして使う意見、などちらほらあった
 のですが、ioがボトルネックになり、かつ書き換え回数がそんなに多くない、
 などの観点で流行のキーバリュー型データストアが面白そうです。

 というわけでdebianに最小構成から肉付けしてキーバリューデータストアの
 cassandraか何かをセットアップしようと思い立ちsshをはじめちょこちょこ
 aptitudeしていくと失敗するわ失敗するわ。どうやらlibedit2がないらしいの
 ですが、etchのものはunavailableのよう。解決できなく、etchのCDが古いから
 と逃げてlennyを焼きなおしてやっとスタート地点についた次第です。このまま
 なぁなぁでcassandraを使い始めるのが今までの自分でしたが常々debianで万能
 感を持ったサーバ管理ができればと思っていたので思い切って勉強させていた
 だければと思っています。

 右も左もわからず、ゼロからのスタートとなりますがよろしくお願いいたします。
\end{prework}

\begin{prework}{ あけど }
\subsubsection{朝}
Debianなサーバのメンテナンス、\texttt{cron-apt}からのメールが来てな
 いかチェック、来てたら\texttt{aptitude update; aptitude -sy
 safe-upgrade}として更新。内容を確認して\texttt{aptitude -y
 safe-upgrade}を実行

\subsubsection{昼}
Debian系の各種MLを読んでみる。

\subsubsection{夜}
Debianなデスクトップ環境のメンテナンスとか、朝と同じく
 \texttt{aptitude update ; aptitude -sy safe-upgrade}として更新内容を確
 認して、更新があったら\texttt{aptitude -y safe-upgrade}を実行、他にも
 GPG のキーの更新とか
\end{prework}

\begin{prework}{ やまねひでき }
 ある初夏の日の朝の話。

 寒い。夏の初めに似合わぬ事を思いながら目を覚ます。まだ目覚ましもなってい
 ない。どうやら掛け毛布が剥ぎ取られた上に扇風機がいつの間にか回されてい
 るせいだと気づく。やれやれ。

 喉が多少痛むので昨日のうちに淹れておいた水出しコーヒーで湿らせる。とにか
 く温かいものを、と思い余ったカレーパンをレンジで温める。ホットコーヒー
 にした方が良かったかな、と寝ぼけた頭でぼんやり考える。レンジから音がし
 たので取り出して頬張る。駅構内のチェーン店にしてはいける味。だが同じ駅
 構内でも赤羽のアレはダメだ。

 頭が動かないのでシャワーを浴び、ようやく人心地に。まだエアコンが取り付
 けされていない自室へ行き、デスクトップPCが轟音を立ているなか、ノートPC
 をスリープから立ち上げる。もちろん立ち上がってくるのは Debian だ。gdm
 のパスワード入力に答え、雑然としたデスクトップの状態を眺める。さて、メー
 ル処理からやってみるか。sylpheed の画面を開く。昨晩でびあん傘の注文を受
 けて返事を書いたが、まだ送信してないのに気づいたので一気に送信。今回は
 何名か GPG encrypt してくれたが、sylpheed からサクッと処理が出来ずにロー
 カルの適当なファイルにコピペしてターミナルから \texttt{gpg --decrypt}なんぞをやっ
 ている、とほほ。相変わらず「手間を減らす」事が下手なのに嫌気が。ついで
 に Google Docs に注文状況のメモ書きをまとめる。ふと inbox を見るとスイ
 スのその傘の首謀者からメール。郵送に EMS を使ってほしかったが、「それ聞
 いたことが無い」とのこと。彼が示した URL から EMS の国別状況のページが
 見つかったので、そちらをメール。うまくいくといいが。

 事務処理はまだ続く。Debconf10 にいく事は決めたのだが、チケットの都合上、
 早く行くことになってしまった。宿を取らないと…と思って HIS で適当なのを
 予約したが、よくよく聞くとDebconf 会場でも支払いさえすれば前乗りできる
 様子。ということでNYの締切りに間に合うように penta.debconf.org から登
 録情報を修正し、SPIへクレジットで振込処理。ふー。HISの方はDebconf側が取
 れたのを見計らってキャンセルだな。

 まだ事務処理。昨年寄稿した Software Design 誌の Debian JP サイトへの再
 利用許可が gihyo 方面から許可を出していただけたので、理事会に共有してお
 く。実際にウェブページへの反映はいつやろうかね…。Debconf記事の売り込
 みも返信ついでにやっておく。旅費が、ね。それから、DebconfといえばGPG
 キーサインパーティ。今年の手順はこれだぞ!と岩松先生から送られてきてい
 たので、登録作業を実施。
 \url{http://people.debian.org/~anibal/ksp-dc10/ksp-dc10.html} を見なが
 ら、「あれー俺のキーっていくつ?」などとフザケたことを思い、\texttt{gpg
 --list-keys}して見つけるなど。pub key が分かればあとはページどおりに進
 めるだけ。一応 port25 ブロックの影響で届かないとか嫌なので自分宛にもメー
 ル、届いた。


 まだメール処理。identi.ca で Gregkh 先生に「あなたの twitter クライア
 ント (bti)、OAuth 対応してる?」とダイレクトで質問してたのに返答がきて
 た。「まだなんだよねー、わかってんだけど」とのこと。squeeze のフリーズ
 と twitter の OAuth 以外拒否が近いので、対応してないクライアントは一旦
 ドロップしないといけない。リストを簡単にアップデートしておく。とりあえ
 ず、キリがなさそうなので一旦メール処理終了。


 バグ潰しに入る。といっても自分のパッケージのバグではなく、FTBFS なバグ。
 Lucas Nussbaum さんが大量に登録してくれているので、彼の登録したバグをみ
 ていくことにしている。昨日のうちにめぼしいものは pbuilder を使ってロー
 カルでビルドするなどしてリストアップしてあるので、後は処理メールだけ。
 17個ほど同じ原因の RC bug に対して \texttt{<bugnumber>-done@bugs.debian.org}宛
 にメール。1個は手元の\texttt{pbuilder}で再現しないので unreproducible タグをつ
 けて\texttt{control@bugs.debian.org}へ。もう2個はエラーが指し示すとおりに
 Build-Depends を微調整するだけでなおるので、ビルドできる事を確認してか
 ら patch タグをつけて送る。これで20個ほどバグが減る方向へ進んだわけだ。
 残りがこれくらい簡単なのばかりだと楽なんだが。
\end{prework}

\begin{prework}{ キタハラ }
 寝床のWeb巡回マシンがDebian(lenny)です。 少々怪しいリンクを踏んでも、
 Windowsより耐性があるかな?というあまり根拠のない理由からです。 最近は
 flashも動きますし、Videoも見れますし、不自由な事はあまりありませんね。
\end{prework}

\begin{prework}{ 山本浩之 }
 ある晴れた日曜の一日

 ぐったりとして、朝、目覚める。 まずモニタをつけて昨日の debuild の確認。
 む、なんか gcc-4.4 がビルドエラーしてるな。どうやら symbols の不整合で、
 最後の最後に、lib32gomp1 パッケージを組めずにエラー吐いているらしい。そ
 ういえば gcc-4.3 から ppc64 のパッチは投げられてなかったな。さて、どう
 しよう。よし、てきとーに、lib32gomp1.symbols.ppc64 にコピペしてみるか。
 おk。debuild 開始。

 (んで、MythTV で録画しておいたONEPIECEを見て、モニタを消して、寝る)

 夕方、再度目覚める。ああ、よく寝た。む、またlib32gomp1 パッケージを組め
 ずにエラー吐いたな。ええい、くそ、こうなったら libgomp1.symbols.common
 にコピペだ。よし、おk。debuild 開始。

 (んで、MythTV で録画しておいた龍馬伝を見て、モニタを消して、寝る)

 以上、ある晴れた日曜の一日でした。
\end{prework}

\begin{prework}{ 鈴木崇文 }
 仕事のデスクトップ環境と、自宅サーバ兼ルータとして使用しています。デス
 クトップ環境については、Debianの様々なパッケージが利用でき、さらにrpm系
 のコマンドも利用できるため、rpm系の作業もdeb系の作業も手元でできるのが
 利点です。自宅サーバ兼ルータについては、potateの時代にdebianをインストー
 ルして以来アップグレートをして日々使用し続けていますが、ほとんど問題な
 く稼働しています。
\end{prework}

\begin{prework}{ まえだこうへい }

 \subsubsection{起き抜けの一発}
 ドスン! 「グヘッ」

 鳩尾にいつもの衝撃が目が覚める。こまめが朝の餌をヨコセと、今朝もやってき
 た。無視して寝直す。ヨメの方の掛け布団の上の方に移動したようだ。タイマーでめざましテレビがついた。さて、起きるか。こまめも同時に''チャチャン''という鈴の音を立てて先回り。こまちゃんの餌をやる前にMacBookの電源を入れる。起動させている間にこまめの餌と水をやる。

 顔を洗い、着替えを手にしてサーバルームへ。起動したMacBookのSidにログイ
 ンし、\texttt{apt-get update; apt-get upgrade}を実行し、その間に執筆中
 の本の原稿を\texttt{git pull}する。終わったら、バナナを食べて6:33の始発
 のシャトルバスで出かける。夏に電車のダイヤ改正に合わせて、シャトルバス
 のダイヤも変わってしまったので、通勤中に座れなくなってしまった。無論、
 通勤中にはMacBookを使えないので、milestoneでRSSフィードの購読をする。

\subsubsection{始業前の一時間}
8時過ぎに会社のビルに到着後、朝食を調達し、朝一の小便を済ませ、会社のリ
フレッシュルームで原稿を書く。この一時間弱が一日で唯一のプライベートタイ
ムだ。Debian勉強会で発表をするときも、ここで資料を作成する。

\subsubsection{始業後の日課}
9時になりオフィスの自席で、Sidの入ったデスクトップPCを立ち上げる。仕事で
の検証は専らこの貧弱なマシンで行う。SidのCouchDBは未だ0.11なのでsvnのリ
ポジトリから\texttt{git-svn}で最新のリビジョンを持ってきてビルドする。おっ
と、今日はCouchAppのリポジトリも更新されているようだ。\texttt{git pull}
後に、debパッケージ化してアップデートしておこう。

\subsubsection{昼休みのメンテ}
昼休み、今日は組合サーバにログインする。表のサーバは未だSargeなの
だが、SargeのAPTはProxyサーバのNTLM認証に対応していないので仕方ない。
Sargeの背後で動かしているSqeuuzeのパッケージアップデートだけを行った。

\subsubsection{帰宅後のこまめ、家事}
帰宅後、着替えて、こまめの餌と水をやる。今日はワシが食事当番なので、いつ
もの納豆料理をする。作り終わる直前にちょうどヨメが帰宅。今夜も美味くでき
たな。

食後にヨメが食器洗いをしている間にOOoで管理している家計簿をつけながら、
メールのチェック。Debianパッケージのセキュリティ通知があるのに気づき、
OpenBlockSのLennyのパッケージをアップデートし、TripwireのDBを更新してお
いた。そろそろOpenBlockSもSqueezeにアップグレードしないとあかんなぁ。ぷらっ
とホーム提供のファームウェアはLinux Kernel 2.6.16ベースなのでSqueezeには
そのままアップグレードできない。クロスコンパイル環境を作って、ファームウェ
アのビルドを計画しなければ。そんなことを思うが、やはり夜は頭回らない。さっ
さと寝ることにしよう。
\end{prework}

\begin{prework}{ 荒木靖宏 }
\subsubsection{朝}
まずtwitter clientを確認。ここでcdn.debian.netに何かあれば通知されていることが多い。
次にar@debian.orgあてのメールを確認。

\subsubsection{昼}
Debian、Ubuntu、Fedora、CentOSなどがいりまじった環境をいじっている時は、問題にあたれば検索、さらに確認をくりかえす。
実はBTSをみない日のほうが多いかも。

\subsubsection{夜}
IRC会議がなければ特になにもやる元気もなくたおれる。
\end{prework}

\begin{prework}{ 小室文 }
 朝起きたら、Debianのdesktopを立ち上げ、\texttt{aptitude
 update/safe-upgrade}をして、ほげほげして、会社についたらDebianのノート
 ブックを立ち上げ、リモートのDebianサーバをほげほげして、ローカルな
 Debianマシンで、新しいパッケージを試してはニヤニヤして、満足して帰ると
 いう一日を毎日しています。
\end{prework}

\begin{prework}{ 吉野(yy\_y\_ja\_jp) }
いろいろ考えてみましたが,一日のうちDebianをPC以外では利用していない気がします...
\end{prework}

\begin{prework}{ 日比野啓 }
朝の通勤中に座れたときには Debian sid の入ったノートで OCaml のプログラムを書く。
会社に着いたら、やはり Debian sid の端末で、emacs の twittering-mode を使って twitter をチェック。
すっかり関数型クラスタになっている friend timeline から情報収集。
おもしろそうな論文を発見したら取ってきておく。
OCamlやHaskellのおもしろそうなライブラリを発見したら、
Debian package になっているかチェックしつつ試しにインストールして遊んでみる。

関数型脳があたたまってきたところで、Haskellで書いている業務プログラムの拡充。
とかやっている間にアルバイトの学生さんが質問しにきたりする。
彼らの半数はやはり Debian を使って開発している。
頭の中を関数型からPerlに切り換えたりJavaに切り換えたり。

気がつくと夜になっていてへろへろなので帰る。
帰りの電車の中でも OCaml 書き。

\end{prework}

\begin{prework} {tai@rakugaki.org}
 \subsection{生活環境として…}
 実はXは見切りをつけてGUI環境はWindowsなのだが、これはDebian上のVM。そし
 て、ウェブサーバーやデータベースなどの各種サーバも同様にVMにしてkvmや
 openvz/lxcで稼動。なので、

 \begin{enumerate}
  \item 手元PCでおもむろにリモートデスクトップを起動(mstsc.exeやrdesktop)
  \item VMなWindowsに接続
  \item その中でホストにログイン
  \item emacsやw3mを開いて生活(GUIものはWindows側)
  \item 特定のサーバをいじるときだけvzctl enter/lxc-consoleで接続
 \end{enumerate}
 というのが毎日の生活手順。

 Debianらしい?ユースケースとしては、kvm/openvz/lxcが使うrootfsは実体は
 aufsでクローニングされた環境で、それをrootdir/nfsroot対象にして稼動して
 いたり。このあたりは非標準なaufsまでビルドしており、さらに各種の仮想化
 機能を組み込んだカーネルパッケージを提供してくれるDebianは大変ありがた
 い。

 \subsection{端末・ルータとして…}
 上の生活環境にアクセスするための端末や各所のGWも多くがDebian。

 端末は持ち歩き用はWindowsだが、その他の各所に置きっぱなしの固定端末は全
 てディスクをDebian入りCF/USBに差し替えて、rdesktop端末になっている。ロー
 カルで動かすのはratpoisonとrdesktopだけ。ここでもaufs大役立ちで、真の/
 を/roに移動させて/全体をaufsでラップすると、chroot /ro+upgradeで更新で
 きて非常に便利。昔のオンメモリで動かす方式や一部フォルダだけtmpfsを重ね
 る方式ではこうはいかない。

 ルータとしては、Debianには標準でaiccu/miredo/radvdやらがあるのでIPv6ルー
 タにもすぐなって楽。もっともTokyo6to4の登場で不要になった…(そもそも
 IPv6は管理用とipv6.2ch.netとかつてMSが主催したInstallManiaxでしか使って
 ない…)。最近Vyattaにも手を出してますが、いずれも必要なら素Debianに抜
 けてopenvpnでもvtunでもhttptunnelでも何でも使って穴を通せるのはとても安
 心。

 そういうわけで「困ったらDebian。とりあえずDebian」な一日です。
 え?RedHat/CentOS?そういう時はfebootstrap/rinseですよ。(ホント何でも
 あるな…)
\end{prework}

\begin{prework}{hattorin}
出社すぐDebian数十台マシンのMuninチェック
Cronバッチで動くOracleからDebianのPostgresへのDB連携動作チェック
Quaggaが動作するDebianマシンのパケットフォワーディング状況チェック

Debian上で動くRailsコントローラーの改造しつつ、日々のネットワーク運用に必要なツール群が動作するPerlスクリプトの高速化処理を研究しつつ適用。
おもむろに全Debianへの\texttt{apt-get update}。\texttt{apt-get upgrade}。

帰宅後、家のマシンで複数ホストへ仕掛けているSmokepingの動作確認。
kernel.orgにて新しいパッチがでていないか確認。
\end{prework}
