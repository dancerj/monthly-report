%; whizzy document
% latex beamer presentation.
% platex, latex-beamer でコンパイルすることを想定。 

%     Tokyo Debian Meeting resources
%     Copyright (C) 2006 Junichi Uekawa

%     This program is free software; you can redistribute it and/or modify
%     it under the terms of the GNU General Public License as published by
%     the Free Software Foundation; either version 2 of the License, or
%     (at your option) any later version.

%     This program is distributed in the hope that it will be useful,
%     but WITHOUT ANY WARRANTY; without even the implied warranty of
%     MERCHANTABILITY or FITNESS FOR A PARTICULAR PURPOSE.  See the
%     GNU General Public License for more details.

%     You should have received a copy of the GNU General Public License
%     along with this program; if not, write to the Free Software
%     Foundation, Inc., 51 Franklin St, Fifth Floor, Boston, MA  02110-1301 USA


\documentclass[cjk,dvipdfmx]{beamer}
\usetheme{Warsaw}
%  preview (shell-command (concat "xpdf " (replace-regexp-in-string "tex$" "pdf"(buffer-file-name)) "&"))
%  presentation (shell-command (concat "xpdf -fullscreen " (replace-regexp-in-string "tex$" "pdf"(buffer-file-name)) "&"))

%http://www.naney.org/diki/dk/hyperref.html
%日本語EUC系環境の時
\AtBeginDvi{\special{pdf:tounicode EUC-UCS2}}
%シフトJIS系環境の時
%\AtBeginDvi{\special{pdf:tounicode 90ms-RKSJ-UCS2}}


\title[Debian 勉強会クイズ問題]{Debian勉強会クイズ}
\subtitle{2006年6月17日版}
\author{上川}
\date{2006年6月17日}

% 三択問題用
\newcounter{santakucounter}
\newcommand{\santaku}[5]{%
\addtocounter{santakucounter}{1}
\frame{\frametitle{問題\arabic{santakucounter}. #1}
%問題\arabic{santakucounter}. #1
\begin{minipage}[t]{0.7\hsize}
 \begin{itemize}
 \item □ A #2\\
 \item □ B #3\\
 \item □ C #4\\
 \end{itemize}
\end{minipage}
}
\frame{\frametitle{問題\arabic{santakucounter}. #1}
%問題\arabic{santakucounter}. #1
\begin{minipage}[t]{0.7\hsize}
\begin{itemize}
\item □ A #2\\
\item □ B #3\\
\item □ C #4\\
\end{itemize}
\end{minipage}
\begin{minipage}[t]{0.2\hsize}
答えは:


\vspace{1cm}

{\huge \hspace{1cm}#5}
\end{minipage}}
}


\begin{document}
\frame{\titlepage{}}

\section{DWNQuiz}
%% debianmeetingresume200606.texから以下コピー


\subsection{2006年16号}
\url{http://www.debian.org/News/weekly/2006/16/}
にある4月18日版です。

\santaku
{DPL選挙の結果リーダーとして選出されたのは}
{Branden Robinson}
{Ted Walther}
{Anthony Towns}
{C}


\santaku
{X11R7のリリースで何がおきたか}
{パッケージはまだアップロードされていないのでわからない}
{今までうごいていたビデオカードは原則として全部動かないように改変された}
{X独自のディレクトリツリー構造を廃棄し、/usr/bin 以下などに直接バイナリ
がインストールされるようになった}
{C}

\subsection{2006年17号}
\url{http://www.debian.org/News/weekly/2006/17/}
にある4月25日版です。

\santaku
{単独のパッケージをあたらしく共同でメンテナンスするためにはAliothのどの機能を使うのが有効か?}
{新規プロジェクトの申請}
{collab-maint パッケージ}
{IRCチャンネル}
{B}

\santaku
{mozilla はどうなるか}
{サポートされなくなるので削除され、xulrunnerに移行が必要}
{mozillaは永遠です}
{使いにくいのでIEに置き換える}
{A}

\subsection{2006年18号}
\url{http://www.debian.org/News/weekly/2006/18/}
にある5月2日版です。

\santaku
{debian-wwwでwww.debian.orgのライセンスが議論された理由は}
{現状のライセンスがDFSGフリーではないのだが、DFSGフリーであるライセンス
に合意がとれなかった}
{www.debian.orgのライセンスはnon-freeでそんなものはDebianのウェブページ
として存在して良い分けが無いから}
{www.debian.orgをホスティングしているサーバが障害で停止したから}
{A}


\santaku
{buildd.net で何がおきたか}
{創始者が引退した}
{Debian以外に拡張された}
{ソースが公開された}
{C}

\subsection{2006年19号}
\url{http://www.debian.org/News/weekly/2006/19/}
にある5月9日版です。

\santaku
{Christian Perrier によると stable/unstable/testing は何か}
{suite/branch}
{distribution}
{release}% potato, woody, sarge
{A}

\santaku
{bts-linkは何をしてくれるものか?}
{リンクに失敗したらBTSに報告してくれるリンカー}
{BTSを自分の現在作業している内容とリンク}
{Debian BTS と upstream の BTS の連係}
{C}

\subsection{2006年20号}
\url{http://www.debian.org/News/weekly/2006/20/}
にある5月16日版です。

\santaku
{Canonical が HP のためにまとめた multi-arch についての調査報告書が提案したのは}
{必要なあらゆる機能を dpkg で実現するため、dpkg 2.0 を実装}
{対象アーキテクチャのために chroot を複数メンテナンスする}
{biarchを実装する}
{A}

\santaku
{apt 0.6.44 で実装された機能は何か}
{コマンドラインで実行するとコンソール画面にAAで牛があらわれて去って行くだけの apt-moo 機能}
{最近の用途パターンから今後必要なパッケージを分析して勝手にインストール
してくれるプロビジョニング機能}
{apt-get update の際に差分ファイルを利用してダウンロード量節約する機能}
{C}

\subsection{2006年21号}
\url{http://www.debian.org/News/weekly/2006/21/}
にある5月23日版です。

\santaku
{debian-installer のグラフィカル版が最初に追加されたアーキテクチャは}
{i386}
{amd64}
{hppa}
{A}

\santaku
{Debconf6 は何回目のDebconfか。}
{4}
{6}
{7}
{C}

\subsection{2006年22号}
\url{http://www.debian.org/News/weekly/2006/22/}
にある5月30日版です。

\santaku
{irc.debian.org に接続すると今後どこのIRCネットワークに接続するようになるのか}
{freenode}
{OFTC}
{WIDE}
{B}

\santaku
{solaris/i386 への 移植版について問題になったのは}
{思想的に十分フリーでないOSへの嫌悪}
{あまりにもLinuxと互換性がなさすぎること}
{GPL互換ではないライブラリとリンクする必要があること}
{C}

\subsection{2006年23号}
\url{http://www.debian.org/News/weekly/2006/23/}
にある6月6日版です。

\santaku
{Martin KrafftがDebconfで実施した実験とは何か}
{実は同時にアメリカで Benjamin Mako Hill の結婚式にも出席していた}
{別人のふりをして Keysigning party に参加した}
{偽の身分証明書をもって keysigning party に参加した}
{C}

\santaku
{Debian の広報を改善するために創られたメーリングリストは何か?}
{debian-propaganda}
{debian-publicity}
{debian-daihonneihappyou}
{B}

\subsection{2006年24号}
\url{http://www.debian.org/News/weekly/2006/24/}
にある6月13日版です。

\santaku
{HPのスポンサーでDebianは何の試験に通過したか}
{CGL}
{あれげ検定}
{BSD}
{A}

\santaku
{Joey Hess は Debian のインテグレーションができていないパッケージをなん
と読んだか}
{Superman}
{Supermarket}
{Warmart}
{B}


\subsection{終了}
\begin{frame}
 \frametitle{点数は?}

 全問正解した方、いますか?
\end{frame}

\section{Debconf}

\begin{frame}
\frametitle{Debconfの過去の沿革}
 {\footnotesize
 \begin{tabular}{|c|c|c|r|}
 \hline
 年 & 名前 & 場所 & 参加人数 \\
 \hline
 2000 & debconf 0 &フランス ボルドー & \\
 2001 & debconf 1 &フランス ボルドー & \\
 2002 & debconf 2 &カナダ トロント & 90名 \\
 2003 & debconf 3 &ノルウェー オスロ & 140名 \\
 2004 & debconf 4 &ブラジル ポルトアレグレ &  150名 \\
 2005 & debconf 5 &フィンランド ヘルシンキ & 200名 \\
 2006 & debconf 6 &メキシコ オアスタペック & 300名 \\
 \hline
 \end{tabular}
}
\end{frame}

\begin{frame}
 \frametitle{参加者分布}
 {\tiny
\begin{center}
 \begin{tabular}[t]{@{\vrule width 1pt}c|r@{\ \vrule width 1pt}}
\hline
国 & 人数 \\
\hline
 MEXICO & 144 \\
 UNITED STATES & 48 \\
 VENEZUELA & 31 \\
 GERMANY & 29 \\
 UNITED KINGDOM & 17 \\
 ITALY & 17 \\
 SPAIN & 16 \\
 EL SALVADOR & 16 \\
 BRAZIL & 16 \\
 FINLAND & 15 \\
 FRANCE & 9 \\
 COLOMBIA & 8 \\
 ARGENTINA & 8 \\
 NORWAY & 6 \\
 JAPAN & 5 \\
 CANADA & 5 \\
 BELGIUM & 5 \\
 PERU & 4 \\
 BELIZE & 4 \\
 SWITZERLAND & 3 \\
 SWEDEN & 3 \\
 NETHERLANDS & 3 \\
 INDIA & 3 \\
 GREECE & 3 \\
 CAMEROON & 3 \\
 AUSTRIA & 3 \\
 AUSTRALIA & 3 \\
 RUSSIAN FEDERATION & 2 \\
 ROMANIA & 2 \\
 NIGERIA & 2 \\
 BOSNIA AND HERZEGOVINA & 2 \\
 BOLIVIA & 2 \\
 UKRAINE & 1 \\
 NEW ZEALAND & 1 \\
 LATVIA & 1 \\
 KENYA & 1 \\
 ISRAEL & 1 \\
 IRELAND & 1 \\
 INDONESIA & 1 \\
 GUINEA & 1 \\
 GUATEMALA & 1 \\
 GAMBIA & 1 \\
 EGYPT & 1 \\
 CZECH REPUBLIC & 1 \\
 CUBA & 1 \\
 CROATIA & 1 \\
 CHINA & 1 \\
 CHILE & 1 \\
 CAMBODIA & 1 \\
 BANGLADESH & 1 \\
\hline
 \end{tabular}
\end{center} 
}
\end{frame}


\begin{frame}
 \frametitle{セッション}
 \begin{center}
 {\tiny
 \begin{tabular}{|l|l|p{20em}|r|}
\hline
日 & 時間 & タイトル & 参加人数 \\
\hline
 2006-05-14 Sunday & 11:00-11:45 & Welcome by DebConf Organizers &  50 \\
 2006-05-14 Sunday & 12:50-13:35 & wiki.debian.org BoF by Joey Hess &  29 \\
 2006-05-14 Sunday & 12:50-13:35 & OpenSolaris, Java dn Debian:  can we be friends? Simon Phipps, Alvaro Lopez Ortega &  92 \\
 2006-05-14 Sunday & 15:20-16:05 & Advanced tools for wasting time by Enrico Zini &  90 \\
 2006-05-14 Sunday & 18:00-19:00 & Multithreading:  Why and how we should use it by Ben Huthcings &  30 \\
 2006-05-15 Monday & 10:05-11:45 & Python BoF by Andreas Barth et al &  28 \\
 2006-05-15 Monday & 10:05-10:50 & Embedding Debian by Wookey &  31 \\
 2006-05-15 Monday & 11:00-11:45 & Topper: An Open Source Driver Framework by Maxim Alt and Dario Rapisardi &  59 \\
 2006-05-15 Monday & 11:55-12:40 & Ubuntu annual report by Mark Shuttleworth &  122 \\
 2006-05-15 Monday & 12:50-13:35 & i18n Infrastructure AdHoc Session I by Christian Perrier &  ~35 \\
 2006-05-15 Monday & 15:20-16:05 & Representing Debian - Doing the best for the best? by Alexander Schmehl &  ~35 \\
 2006-05-15 Monday & 16:15-17:55 & Security Enhanced Linux UML instances - an Introcution and recipe by Manoj Srivastava &  ~100 \\
 2006-05-15 Monday & 18:00-19:00 & Resurecting Computers with Free Software by Vagrant Cascadian and Hector Colina &  30 \\
 2006-05-15 Monday & 19:00-20:00 & debian-installer and SELinux by Manoj Srivastava &  ~25 \\
 2006-05-15 Monday & 21:30-22:30 & debian-installer BoF by Joey Hess &  38 \\
 2006-05-16 Tuesday & 10:05-10:50 & stable release BoF by Andreas Barth &  21 \\
 2006-05-16 Tuesday & 10:05-10:50 & ideas for repository of meta-information (watchfiles et al) by Filippo Giunchedi &  35 \\
 2006-05-16 Tuesday & 11:00-11:45 & Common Lisp development in Debian by Peter van Eynde &  8 \\
 2006-05-16 Tuesday & 11:00-11:45 & Optimizing boot time by Margarita Manterola &  100 \\
 2006-05-16 Tuesday & 11:55-13:35 & GPLv3 by Don Armstrong &  120 \\
 2006-05-16 Tuesday & 15:20-16:05 & Debian Community Guidelines by Enrico Zini &  120 \\
 2006-05-16 Tuesday & 16:15-17:55 & Let's port together. Debian fun for everyone by Peter de Schrijver and Steve Langasek &  110 \\
 2006-05-16 Tuesday & 18:00-19:00 & BoF Debian en Latinoamerica by Anibal Monsalve Salazar and David Moreno Garza &  37 \\
 2006-05-16 Tuesday & 19:00-20:00 & Scratchbox 2, bringing crosscompiling to Debian by Riku Voipio &  12 \\
 2006-05-16 Tuesday & 21:30-22:30 & Webapps Common: Tthe central point in developing a next-generation web server and web application policy by Neil McGovern &  21 \\
 2006-05-18 Thursday & 10:05-10:50 & Debian and the \$ 100 Laptop by Jim Gettys &  28 \\
 2006-05-18 Thursday & 10:05-10:50 & Governance of the Debian Project by Bdale Garbee &  57 \\
 2006-05-18 Thursday & 11:00-11:45 & X.org status and plans by Keith packard &  95 \\
 2006-05-18 Thursday & 11:55-13:35 & releasing in time - etch in December 06 by Andreas Barth and Steve Langasek &  98 \\
 2006-05-18 Thursday & 15:20-17:00 & Debian installer internals by Frans Pop &  ~~60?? \\
 2006-05-18 Thursday & 17:10-18:50 & Weeding out security bugs by Javier Fernandez-Sanguino &  47 \\
 2006-05-19 Friday & 10:05-10:50 & i18n Infrastructure AdHoc Session II by Christian Perrier &  ~30 \\
 2006-05-19 Friday & 10:05-10:50 & AM BoF &  30 \\
 2006-05-19 Friday & 11:00-11:45 & The X Community - History and Directions by Keith Packard &  70 \\
 2006-05-19 Friday & 11:55-12:40 & Experiences with large CDD-installations by Knut Yrvin &  ~100 \\
 2006-05-19 Friday & 12:50-13:35 & LTSP Muekow Next Generation by Vagrant Cascadian and Octavio H. Ruiz Cervera &  ? \\
 2006-05-19 Friday & 12:50-13:35 & the future of the NM process by Christoph Berg &  ~75 \\
 2006-05-19 Friday & 15:20-16:05 & Packaging shared libraries by Josselin Mouette &  ~50 \\
 2006-05-19 Friday & 16:15-17:55 & Cheap Thrills - Instant inspiration for the masses by Meike Reichle &  55 \\
 2006-05-19 Friday & 21:30-22:30 & What's new and cool with MySQL by Jorge del Conde &  ? \\
 2006-05-20 Saturday & 10:05-10:50 & Ubuntu Question and Answer Bof by Mark Shuttleworth &  ? \\
 2006-05-20 Saturday & 10:05-10:50 & Alternative developer's interface to APT: libapt-front by Petr Rockai &  ? \\
 2006-05-20 Saturday & 11:00-11:45 & Codes of Value: An Anthropological Analysis of Hacker Values by Gabriella Coleman &  ? \\
 2006-05-20 Saturday & 11:55-12:40 & Lightning Talks by Joey Hess et al &  ? \\
 2006-05-20 Saturday & 12:50-13:35 & www.debian.org redesign by Agnieszka Czajkowska &  ? \\
 2006-05-20 Saturday & 12:50-13:35 & Debian's Debugging Debacle: the Debrief by Erinn Clark and Anthony Towns &  ? \\
 2006-05-20 Saturday & 15:20-16:05 & debconf[67] by Andreas Schuldei &  80 \\
 2006-05-20 Saturday & 16:15-17:55 & state of the art for Debian i18n/l10n by Christian Perrier and Javier Fernandez-Sanguino &  50 \\
 2006-05-20 Saturday & 19:00-20:00 & Devotee and the temple of Doom by Manoj Srivastava &  ? \\
 2006-05-20 Saturday & 21:30-22:30 & zeroconf BoF by Joey Hess &  ? \\
\hline
 \end{tabular}
 }
 \end{center}
\end{frame}


\begin{frame}
\frametitle{lightning talk}
\begin{itemize}
	\item Actively Discovering bugs/issues with packages
	\item Walkthrough : Make your Country love Debian
	\item Debian in the greater Linux ecosystem
	\item WNPP: Automatizing the unautomatizable
	\item How far can we go with a collaborative maintenance infrastructure
	\item How to get debian-admin to help you
	\item Learning from Gentoo
	\item Datamining on Debian packages metadata
	\item Tracking MIA developers
	\item How to pronounce Jeroen van Wolffelaar, and other names
\end{itemize}
\end{frame}

\subsection{.}
\begin{frame}
 \frametitle{}

以上
\end{frame}

\section{pbuilder/cowdancer/cowbuilder}
\subsection{pbuilder}

\begin{frame}
\frametitle{pbuilder}
\begin{itemize}
 \item pbuilder create
 \item pbuilder update
 \item pbuilder build
 \item chroot 内にパッケージビルド用の環境準備
 \item chroot 内部で testsuite を実行
 \item sid の最新版にアップデートした状態で常にテストできる!
 \item stable 向けにバックポートする環境の維持が楽!
\end{itemize}
\end{frame}

\subsection{cow-shell}
\begin{frame}
\frametitle{cowdancer}
\begin{itemize}
 \item cow-shell
 \item ハードリンク
 \item cp -al 大活躍
\end{itemize}
\end{frame}

\subsection{cowbuilder}
\begin{frame}
\frametitle{cowbuilder}
\begin{itemize}
 \item cowbuilder --create
 \item cowbuilder --update
 \item cowbuilder --build
 \item cowdancer を活用、毎回 200MB の tarball の展開をしなくて済む!
\end{itemize}
\end{frame}

\subsection{パフォーマンス}
\begin{frame}
 \frametitle{パフォーマンス}
\begin{center}
 \begin{tabular}[t]{@{\vrule width 1pt}l|r|r|r@{\ \vrule width 1pt}}
 \hline
 \hline
 オペレーション & pbuilder & cowbuilder & speed \\
 \hline
 update & 150  & 16 & 10x \\
 build (N/W down) & 80 & 18 & 5x \\
 build (pbuilder) & 177 & 86 & 2x\\
 login & 80 & 4 & 20x\\
 \hline
 \hline
 \end{tabular}
\end{center}

\end{frame}

\end{document}