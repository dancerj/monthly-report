%; whizzy document
% latex beamer presentation.
% platex, latex-beamer でコンパイルすることを想定。 


%     Tokyo Debian Meeting resources
%     Copyright (C) 2006 Junichi Uekawa

%     This program is free software; you can redistribute it and/or modify
%     it under the terms of the GNU General Public License as published by
%     the Free Software Foundation; either version 2 of the License, or
%     (at your option) any later version.

%     This program is distributed in the hope that it will be useful,
%     but WITHOUT ANY WARRANTY; without even the implied warranty of
%     MERCHANTABILITY or FITNESS FOR A PARTICULAR PURPOSE.  See the
%     GNU General Public License for more details.

%     You should have received a copy of the GNU General Public License
%     along with this program; if not, write to the Free Software
%     Foundation, Inc., 51 Franklin St, Fifth Floor, Boston, MA  02110-1301 USA


\documentclass[cjk,dvipdfmx]{beamer}
\usetheme{Warsaw}
%  preview (shell-command (concat "xpdf " (replace-regexp-in-string "tex$" "pdf"(buffer-file-name)) "&"))
%  presentation (shell-command (concat "xpdf -fullscreen " (replace-regexp-in-string "tex$" "pdf"(buffer-file-name)) "&"))


\title[Debian 勉強会]{sidの勧め}
\subtitle{2006年3月18日}
\author{やまね@Debian勉強会}
\date{2006年3月18日}

\begin{document}

\frame{\titlepage{}}

 \section{debianの特徴}
 
 \begin{frame}
  \frametitle{debianの特徴}
  「安定していて」「パッケージ沢山」
  
 \end{frame}
 
 
 
 \begin{frame}
  \frametitle{debianの特徴}
  「でもFed○raより古いよね」\\
  「この雑誌には、Fed○raならパッケージがあるって書いてあるよ。\\
   Debian はパッケージが古いからダメなんでしょ?」
 \end{frame}
 
 \begin{frame}
  \frametitle{debianの特徴}
  本当?\\
  そんなあなたに
  
  Sid
 \end{frame}
 

 \section{What's Sid?}
 \begin{frame}
  \frametitle{ What's Sid?}
  
  unstable のコードネーム
  
  Toystory にでてくる\\
  おもちゃを壊す悪ガキ
 \end{frame}
 
 \begin{frame}
  \frametitle{ What's Sid?}
  
  パッケージ間の依存関係が壊れていることがある
  
  ソフトウェア自体も壊れていることがある\\
  (たまに)
  
  安定していない?\\
  実はそれなりに使える
  
  なぜ?
 \end{frame}
 
 \begin{frame}
  \frametitle{ What's Sid?}
  unstable = developers version
 \end{frame}

 \begin{frame}
  \frametitle{ What's Sid?}
 
 開発者が生活するのは基本的に sid の上
  
  「壊れっぱなしだと作業に支障がでる」=自然と目が向く/直そうとする

 でも、「アップロードする前のせめて自分の所で確認してからにしろよ!!」
  と思うときもある
 \end{frame}
 
 \section{sid って何がいいの?}
 \begin{frame}
  \frametitle{sid って何がいいの?}

  新しい
  日々変化がある
 \end{frame}

 \begin{frame}
  \frametitle{「士別三日即更刮目相待」}
 「男子三日会わざれば刮目して見るべし」\\
  数日前のものとは別物になる事が暫し
 \end{frame}

 \begin{frame}
  
  \frametitle{大量のパッケージアップデートの様子}
  
 kernel も変わる\\
 glibc も変わる\\
 もう何でも変わる\\
 昨日できなかったことが出来るように!\\
  (昨日までできていたことが出来なくなるようにも)
 \end{frame}
 
 \section{利用方法}
 
 \begin{frame}
  \frametitle{利用方法}
  
  /etc/apt/sources.list\\
s/stable/unstable/g\\
  $\sharp$\texttt{aptitude update; aptitude dist-upgrade}\\
  大量にパッケージがある状態で upgrade すると依存関係がややこしいことになりがちなので、最小限のインストールから upgrade したほうが楽\\
  注意\\
  でかいパッケージ群だと即座に最新にならなかったりします\\
  (例:gnome)
 \end{frame}
 

 \begin{frame}
  \frametitle{利用方法}
  安定版で一部のパッケージだけを新しくして使いたい\\
  \url{http://www.backports.org/}\\
  Norbert Tretkowski さんの個人プロジェクト、最近は参加者が増えている\\
  まだ Debian の公式プロジェクトではないから注意
 \end{frame}
 
 \section{参考}
 \begin{frame}
  \frametitle{参考}
  \begin{itemize}
   \item  \texttt{apt-listbugs} によるバグ報告の確認
   \item  \texttt{apt-listchanges} によるChangeLogの確認
   \item  \url{debian-devel@irc.debian.org} のトピック
   \item  \url{debian-devel@irc.oftc.net} のトピック
   \item  \url{http://wiki.debian.org/StatusOfUnstable}
  \end{itemize}
 \end{frame}
 
\end{document}
