%; whizzy document
% latex beamer presentation.
% platex, latex-beamer でコンパイルすることを想定。 


%     Tokyo Debian Meeting resources
%     Copyright (C) 2006 Junichi Uekawa

%     This program is free software; you can redistribute it and/or modify
%     it under the terms of the GNU General Public License as published by
%     the Free Software Foundation; either version 2 of the License, or
%     (at your option) any later version.

%     This program is distributed in the hope that it will be useful,
%     but WITHOUT ANY WARRANTY; without even the implied warranty of
%     MERCHANTABILITY or FITNESS FOR A PARTICULAR PURPOSE.  See the
%     GNU General Public License for more details.

%     You should have received a copy of the GNU General Public License
%     along with this program; if not, write to the Free Software
%     Foundation, Inc., 51 Franklin St, Fifth Floor, Boston, MA  02110-1301 USA


\documentclass[cjk,dvipdfmx]{beamer}
\usetheme{Warsaw}
%  preview (shell-command (concat "xpdf " (replace-regexp-in-string "tex$" "pdf"(buffer-file-name)) "&"))
%  presentation (shell-command (concat "xpdf -fullscreen " (replace-regexp-in-string "tex$" "pdf"(buffer-file-name)) "&"))

%http://www.naney.org/diki/dk/hyperref.html
%日本語EUC系環境の時
\AtBeginDvi{\special{pdf:tounicode EUC-UCS2}}
%シフトJIS系環境の時
%\AtBeginDvi{\special{pdf:tounicode 90ms-RKSJ-UCS2}}


\title[Debian 勉強会クイズ問題]{Debian勉強会クイズ}
\subtitle{2006年4月15日版}
\author{上川}
\date{2006年4月15日}

% 三択問題用
\newcounter{santakucounter}
\newcommand{\santaku}[5]{%
\addtocounter{santakucounter}{1}
\frame{\frametitle{問題\arabic{santakucounter}. #1}
%問題\arabic{santakucounter}. #1
\begin{itemize}
\item □ A #2\\
\item □ B #3\\
\item □ C #4\\
\end{itemize}
}
\frame{\frametitle{問題\arabic{santakucounter}. #1}
%問題\arabic{santakucounter}. #1
\begin{itemize}
\item □ A #2\\
\item □ B #3\\
\item □ C #4\\
\end{itemize}
\vfill{}
#5
}
}


\begin{document}
\frame{\titlepage{}}

\section{DWNQuiz}
%% debianmeetingresume200604.texから以下コピー
\subsection{2006年8号}
\url{http://www.debian.org/News/weekly/2006/08/}
にある2月22日版です。

\santaku
{Debian etch beta1 インストール用メディアにどういう問題があったか}
{最新じゃないのでつかってられない}
{メディアが水に濡れて使えなくなった}
{Debianアーカイブの変更の影響で動かなくなった}
{C}

\santaku
{Debian Live Initiativeは何をしようとするものか}
{新しい開発をがんがんする}
{DebianのLive CDを統合する}
{リアルタイムハック実況中継のための環境を提供する}
{B}

\subsection{2006年9号}
\url{http://www.debian.org/News/weekly/2006/09/}
にある2月28日版です。

\santaku
{ミラーシステムについてAnthony Townsが発表したのは何か}
{i386とamd64だけに限定して今後は運用する}
{全アーキテクチャを含めた巨大なミラーを継続}
{アーキテクチャ毎にミラーを分割する}
{C}

\santaku
{NMUを実施する際に、注意するべきことは何か}
{BTSを通してメンテナに通知すること}
{NMUなんてしてる暇があったら自分のバグを直す}
{できるだけメンテナにばれないように実施する}
{A}

\subsection{2006年10号}
\url{http://www.debian.org/News/weekly/2006/10/}
にある3月7日版です。

\santaku
{AMD64/kFreeBSDについて何がおきたか}
{はじめてパッケージが動いた}
{glibc/gcc/binutilsがポーティングできた}
{chroot内部で動作するようになり、builddが動いている}
{C}

\santaku
{バックポートのサポートが公式になるのか、という質問についての回答は}
{Utunubu 広報担当によるとDebianはもう時代遅れだ}
{Joseph Smidt によると、Debianはバックポートを主体として今後は活動を続け
る}
{Norbert Tretkowski によると、公式なサポートつきのバックポートを提供することは考えにく
い}
{C}

\subsection{2006年11号}
\url{http://www.debian.org/News/weekly/2006/11/}
にある3月14日版です。

\santaku
{Bastian Blank が発表した、Debian カーネルチームの作業内容は}
{kernel-image-という名前からlinux-image-という名前に変更しました}
{カーネルはFreeBSDのものに入れ換えました}
{今後はSMP版とUniprocessor版というだけでなく、何CPUのSMPかということで
flavorを分けます}
{A}

\santaku
{Martin の後の安定版リリースマネージャは誰にならなかったか}
{Martin Zobel-Helas}
{Andreas Barth}
{Nobuhiro Iwamatsu}
{C}

\subsection{2006年12号}
\url{http://www.debian.org/News/weekly/2006/12/}
にある3月21日版です。

\santaku
{JBoss4のDebianパッケージは存在するか}
{Guido Guentherが作成したものが存在する}
{non-freeなのでそんなものは存在しない}
{ボスって何?}
{A}

\santaku
{パッケージに含めるドキュメントの形式はPDFだけにしたい、というメールに対しての反応は}
{HTMLのほうがgrepしやすいので、HTMLをいれてほしい}
{DVIのほうがファイル構造が安定しているのでDVIにしてほしい}
{プレインテキスト形式のほうが小さいのでプレインテキスト形式にしてほしい}
{A}

\subsection{2006年13号}
\url{http://www.debian.org/News/weekly/2006/13/}
にある3月28日版です。

\santaku
{David Moreno Garzaが作成したのは}
{DWNのmixi風インタフェース}
{DWNのRSSフィード}
{DWNの2ch風インタフェース}
{B}

\santaku
{google groupsを利用してDebianバグを検索するにはどのニュースグループをみ
ればよいか}
{there.is.no.bugs ニュースグループ}
{bugs.debian.org ニュースグループ}
{linux.debian.bugs.dist ニュースグループ}
{C}

\subsection{2006年14号}
\url{http://www.debian.org/News/weekly/2006/14/}
にある4月4日版です。

\santaku
{ndiswrapperがmainに入っているのがただしいのかという議論の原因は何か}
{ndiswrapperのソースコードは実は全部暗号文で構成されているから}
{ndiswrapperは思想的におかしいから}
{ndiswrapperはエミュレーションする対象のドライバが無いと実用できないから}
{C}

\santaku
{Debian Project Leader の投票についてClytieが苦情をいったのは何故か}
{Debian Developer でないと投票権がないことに気づかずに投票した}
{立候補者がどれもえらびようがないような人達ばっかりだった}
{Branden Robinson が立候補していなかった}
{A}


\subsection{2006年15号}
\url{http://www.debian.org/News/weekly/2006/15/}
にある4月11日版です。

\santaku
{XenのDebianパッケージはもともとJulien DanjouたちとBlastian Blankが別個
に作業していた。それが統合されたのはいつか}
{4/5}
{3/1}
{1/1}
{A}

\santaku
{sudoのセキュリティーフィックスでどういう変更がなされたか}
{アプリケーションを実際に実行するのではなく、実行しているっぽくみせかけ
るだけになった}
{ルート権限でプログラムを実行しなくなる}
{実行プログラムに引き渡される環境変数を制限}
{C}


\end{document}