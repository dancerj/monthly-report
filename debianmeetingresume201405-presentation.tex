%; whizzy paragraph -pdf xpdf -latex ./whizzypdfptex.sh
%; whizzy-paragraph "^\\\\begin{frame}\\|\\\\emtext"
% latex beamer presentation.
% platex, latex-beamer でコンパイルすることを想定。 

%     Tokyo Debian Meeting resources
%     Copyright (C) 2012 Junichi Uekawa

%     This program is free software; you can redistribute it and/or modify
%     it under the terms of the GNU General Public License as published by
%     the Free Software Foundation; either version 2 of the License, or
%     (at your option) any later version.

%     This program is distributed in the hope that it will be useful,
%     but WITHOUT ANY WARRANTY; without even the implied warreanty of
%     MERCHANTABILITY or FITNESS FOR A PARTICULAR PURPOSE.  See the
%     GNU General Public License for more details.

%     You should have received a copy of the GNU General Public License
%     along with this program; if not, write to the Free Software
%     Foundation, Inc., 51 Franklin St, Fifth Floor, Boston, MA  02110-1301 USA

\documentclass[cjk,dvipdfmx,12pt]{beamer}
\usetheme{Tokyo}
\usepackage{monthlypresentation}

%  preview (shell-command (concat "evince " (replace-regexp-in-string "tex$" "pdf"(buffer-file-name)) "&")) 
%  presentation (shell-command (concat "xpdf -fullscreen " (replace-regexp-in-string "tex$" "pdf"(buffer-file-name)) "&"))
%  presentation (shell-command (concat "evince " (replace-regexp-in-string "tex$" "pdf"(buffer-file-name)) "&"))

%http://www.naney.org/diki/dk/hyperref.html
%日本語EUC系環境の時
\AtBeginDvi{\special{pdf:tounicode EUC-UCS2}}
%シフトJIS系環境の時
%\AtBeginDvi{\special{pdf:tounicode 90ms-RKSJ-UCS2}}

\newenvironment{commandlinesmall}%
{\VerbatimEnvironment
  \begin{Sbox}\begin{minipage}{1.0\hsize}\begin{fontsize}{8}{8} \begin{BVerbatim}}%
{\end{BVerbatim}\end{fontsize}\end{minipage}\end{Sbox}
  \setlength{\fboxsep}{8pt}
% start on a new paragraph

\vspace{6pt}% skip before
\fcolorbox{dancerdarkblue}{dancerlightblue}{\TheSbox}

\vspace{6pt}% skip after
}
%end of commandlinesmall

\title{東京エリアDebian勉強会}
\subtitle{第113回 2014年5月度}
\author{野島貴英}
\date{2014年5月17日}
\logo{\includegraphics[width=8cm]{image200607/openlogo-light.eps}}

\begin{document}

\begin{frame}
\titlepage{}
\end{frame}

\begin{frame}{設営準備にご協力ください。}
会場設営よろしくおねがいします。
\end{frame}

\begin{frame}{Agenda}
 \begin{minipage}[t]{0.45\hsize}
  \begin{itemize}
   \item 注意事項
	 \begin{itemize}
	  \item 写真はセミナールーム内のみ可です。
          \item 出入りは自由でないので、もし外出したい方は、野島まで一声くださいませ。
	 \end{itemize}
   \item 最近あったDebian関連のイベント報告
	 \begin{itemize}
	  \item 第111回 東京エリアDebian勉強会
	 \end{itemize}
  \end{itemize}
 \end{minipage} 
 \begin{minipage}[t]{0.45\hsize}
  \begin{itemize}
   \item Debian Trivia Quiz
  \end{itemize}
 \end{minipage}
\end{frame}

\section{Debian Trivia Quiz}
\emtext{Debian Trivia Quiz}
\begin{frame}{Debian Trivia Quiz}

  Debian の常識、もちろん知ってますよね?
知らないなんて恥ずかしくて、知らないとは言えないあんなことやこんなこと、
みんなで確認してみましょう。

今回の出題範囲は\url{debian-devel-announce@lists.debian.org},
\url{debian-devel@lists.debian.org} に投稿された
内容などからです。

\end{frame}

\subsection{問題}

%; whizzy-master ../debianmeetingresume201311.tex
% 以上の設定をしているため、このファイルで M-x whizzytex すると、whizzytexが利用できます。
%

\santaku
{2014/4/26に、とあるアーキテクチャがtestingから外されました。次のうちのどれでしょう?}
{i386}
{armel}
{sparc}
{C}
{リリースチームの見解によれば、ツールチェインの問題、安定性の問題、また今後Jessieリリースに向けての開発について明確な見解が開発チームらから得られなかったとのことです。}

\santaku
{2014/4/26にdebianの安定版がリリースされました。バージョンはいくつでしょう?}
{7.4}
{7.5}
{7.6}
{B}
{5回目の更新リリースとなります。安定版を使っている人でアップデートしていない人は、早速apt-get upgradeをお勧めしておきます。主にバグフィックスとセキュリテイ対策となります。}





\section{事前課題}
\emtext{事前課題}
{\footnotesize
\begin{prework}{ Koji Hasebe }
MAC上でDebian作業環境を構築します。\\
その環境を業務で使えるところまで行けたらと思っています。
\end{prework}

\begin{prework}{ dictoss(杉本 典充) }
Debian GNU/kFreeBSDをいろいろ試す。
\end{prework}

\begin{prework}{ 吉野(yy\_{}y\_{}ja\_{}jp) }
\begin{itemize}
\item DDTSS
\item manpages-ja
\end{itemize}
\end{prework}

\begin{prework}{ wbcchsyn }
kpatch を読む\\
\url{https://github.com/dynup/kpatch}
kpatch は、OS を停止せずに linux kernel にパッチを当てる為のパッチとの事。(開発中)\\

 Linux kernel のパッチなので Gnu Linux でも使用可能なはずだが、RedHat 中心に開発されているとの事なので、Debian 向けのドキュメントや環境が整うには時間がかかりそう。なので、自分で開発中の GitHub を読んでみる。
\end{prework}

\begin{prework}{ regonn(Kenta Tanoue) }
 Debian初心者なのでDebianリファレンスを読んでいきます。(目標は半分の6章まで。知らなかったことをどんどんメモしていく)
\end{prework}

\begin{prework}{ shinyorke }

\begin{description}
\item [課題] Pythonistaとして、Debianに慣れる。
\item [内容] 
\begin{itemize}
\item 自分で作ったPythonアプリ(Django)をDebian上で動かす(Python + Django + MySQL)
\item nginxを入れて、Pythonアプリをリバースプロキシする
\end{itemize}
\item [環境] 
\begin{itemize}
\item ホストOS: Mac OS X(Mavericks)
\item ゲストOS: Debian ※virtualbox上で動作
\end{itemize}
\end{description}
 今回、外向けの公開はしない。

\end{prework}

\begin{prework}{ 野島 貴英 }

 Debianによるimmutable infrastractureな環境の実験と試行錯誤。
(んでもって、何か見つけたらバグレポ)

\end{prework}

\begin{prework}{ まえだこうへい }

先月の続き。

\begin{itemize}
\item Golang関係のパッケージ化の続き
\item \url{http://qa.debian.org/developer.php?login=mkouhei@palmtb.net}のバグ潰し&パッケージアップデート
\end{itemize}
\end{prework}

}

\section{docker.io}
\emtext{docker.io}

\begin{frame}{dockerとは}

 dockerは使うとわかるのですが、単にlinux上にコンテナ環境を作成するという機能の他
に、aufsを利用してベースのOSのシステムに迅速に変更差分を適用するという動作により
、非常に素早くカスタム化されたコンテナ環境の作成・変更・管理が出来ます。

 また、変更した内容をdockerリポジトリ(\url{https://index.docker.io})に登録する
ことにより、dockerが動作する環境さえあれば、こちらのリポジトリから全く同じコンテ
ナ環境を作成・動作させることができます。

 今回はdebianをdockerホストにしてdockerを使ってみた事について発表します。

\end{frame}

\begin{frame}{今回利用のdebian}

今回発表で評価したdebianはunstable(jessie/sid)となります。

 なお、残念ながら安定版のdebian wheezyでは未だdockerはパッケージ化されていない
状況です。本誌を読まれているdebian使いの方は、ぜひこの機会にdebian unstable(jess
ie/sid)にアップグレードしてパッケージからdockerをお試しください。

\end{frame}

\begin{frame}{docker仕組み}

勉強会資料の図1参照。

\end{frame}


\section{今後のイベント}
\emtext{今後のイベント}
\begin{frame}{今後のイベント}
\begin{itemize}
 \item 2014年5月17日(土) 東京エリアDebian勉強会
\end{itemize}
\end{frame}

\section{今日の宴会場所}
\emtext{今日の宴会場所}
\begin{frame}{今日の宴会場所}
未定
\end{frame}

\end{document}

;;; Local Variables: ***
;;; outline-regexp: "\\([ 	]*\\\\\\(documentstyle\\|documentclass\\|emtext\\|section\\|begin{frame}\\)\\*?[ 	]*[[{]\\|[]+\\)" ***
;;; End: ***
