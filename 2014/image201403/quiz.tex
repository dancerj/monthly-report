%; whizzy-master ../debianmeetingresume201311.tex
% 以上の設定をしているため、このファイルで M-x whizzytex すると、whizzytexが利用できます。
%

\santaku
{2014年GSoCのメンター募集が行われています。2014年のGSoCにて採択されていないものはどれ}
{hurd-i386の開発}
{clangでDebianのパッケージをコンパイルできるようにする}
{Android上でDebian環境を作れる件の改良を行う}
{A}
{他にもいろいろなProjectがDebian Projectから採択されています。Elektra\url{http://www.libelektra.org}で設定ファイルのアップグレードを改良するとか、libstdc++からlibc++を使うようにDebianを変更する件や、パッケージ管理にMuonを使う件など。参考:\url{https://wiki.debian.org/SummerOfCode2014/Projects}}

\santaku
{2014/2/14にバグレポートのIDが\#740000を向かえました。\#730000からどのぐらいの期間がたったでしょう?}
{1ヶ月と3日}
{3ヶ月と4日}
{10ヶ月と10日}
{B}
{毎年、Christian Perrierさんにより、バグレポートのIDについて、将来いつ何万番台を迎えるかについて当てるコンテストが行われています。}

\santaku
{Debianのコミュニティにより提供されているWebサービスについて調査が行われています。この調査の名前は?}
{Debian Services Servey}
{Outreach Program For Women}
{Debian Services Census}
{C}
{2014/2/13に呼びかけが行われました。現在のサービスの名前とURLのリストは、\url{https://wiki.debian.org/Services}にまとめられています。}

\santaku
{毎年恒例のDPL選挙が始まりました。2014年のDPL立候補者は誰?}
{Takahide Nojima}
{Lucas Nussbaum}
{Stefano Zacchiroli}
{B}
{lucusは2013年DPLですが、2年連続立候補となります。他の2名の方は、Gergely Nagyさん、Neil McGovernさんとなります。
 選挙期間は2014/3/31〜4/13となります。各候補者の声明は、\url{http://www.debian.org/vote/2014/platforms/}に掲載予定です。}
