\begin{prework}{ wbcchsyn }
発表予定のツールをブラッシュアップする。\\
\url{https://github.com/wbcchsyn/dVault\_prototype}
フィードバックがあれば、取り込み。\\
無ければ、ドキュメント作成とか自分で気がついている部分の改修とか。
\end{prework}

\begin{prework}{ 野島 貴英 }
\begin{itemize}
\item Debian ARM64まわりを調べる。
\item OSS化したsoftetherをdebianパッケージとしてどうするか考える
\item Debian Gnu/Hurdネタを調べる
\end{itemize}

 まあ、将来の東京エリアDebian勉強会ネタの仕込み作業となりますなぁ。
\end{prework}

\begin{prework}{ yyuu }

 先々月に続いて、pyenv というスクリプトの deb 化のための作業を行なおうと考えています。

 \url{https://github.com/yyuu/pyenv}

 可能であれば pyenv-virtualenv プラグインも deb 化したいです。

 \url{https://github.com/yyuu/pyenv-virtualenv}

\end{prework}

\begin{prework}{ zinrai }

direnvのdebパッケージ化

\url{https://github.com/zimbatm/direnv}

\end{prework}

\begin{prework}{ koedoyoshida }

あんどきゅめんてっどでびあんの編集作業

\end{prework}

\begin{prework}{ regonn(Kenta Tanoue) }

AWS debian環境で、
Rails+publifyを使って自分のブログを作ります。(作業工程を記事にしたい)
インストールの隙間時間でDDTSSの作業もできたらいいな。
\end{prework}

\begin{prework}{ なかおけいすけ }

7月の勉強会で発表せよとご指名いただいたので、その準備をします。

\end{prework}

\begin{prework}{ 野首(@knok) }

\begin{itemize}
\item KAKASIのバグフィックス
\item gnu.org翻訳
\item ECS Livaでのdebian kernelでのmmc関連修正
\end{itemize}

\end{prework}
