%; whizzy-master ../debianmeetingresume201311.tex
% 以上の設定をしているため、このファイルで M-x whizzytex すると、whizzytexが利用できます。
%

\santaku
{MATE desktop環境が先日Debianのパッケージに入りました。対象のMATEのバージョンは?}
{1.7}
{1.8}
{1.9}
{B}
{Gnome 2.28ベース以降のGnomeからフォークしたデスクトップ環境とのことです。読み方は「マテ」(マテ茶のマテ)と読むとのこと。}

\santaku
{ARM64アーキテクチャへの対応の協力者募集が行われました。残念ながらDebianは動きませんが、民生品で手に入るARM64搭載の機材は次のどれでしょう?}
{iphone 5s}
{NEXUS 7}
{OpenBlocks A7}
{A}
{iphone 5sから搭載されているA7 SoCはARMv8搭載でARM64動作も出来るそうです。ARM64を搭載した民生品はこれが最初かも。
一方、DebianのARM64ポーティングでは全く別の開発用向けに提供されている機材が利用されています。
我々がDebianの動くARM64機材を民生品で手に出来る日は何時ー(涙)?}

\santaku
{edos.debian.netがqa.debian.org/doseで復活しました。ところで、このサイト何するサイト?}
{各パッケージの依存関係がお互いに満たされている状態かをチェック}
{各パッケージがどれだけupstreamから乖離しているかチェック}
{各パッケージがどれだけ人気があるかをチェック}
{A}
{dose-*パッケージに含まれるチェックコマンドを使い、各アーキテクチャのdebianリポジトリを毎日チェックして、依存関係が満たされていないパッケージをチェックした結果を教えるサイトとなります。膨大な量のパッケージの依存関係を迅速にチェック出来るところが凄い点です。仕組みについてIEEEの論文まで出ているようです。参考:http://hal.archives-ouvertes.fr/ docs/00/14/95/66/PDF/ase.pdf}

\santaku
{先日Debian GNU/Hurdの中間報告がなされました。initが変更になったそうですが、どれになったでしょうか?}
{Hurd独自実装のinit}
{systemd}
{Sysvinit}
{C}
{去年のGSoCの成果がそのまま開発継続し、遂にSysvinitにinitが変更されたそうです。このおかげで、Network、halt/shutdown時の動作、ブート時のfilesystemのマウント方式がやっとDebian流になったようです。}

\santaku
{DEP-12がPTSによりサポートされたそうです。ところで、DEP-12の内容って何?}
{CIツールによる自動テストの為の定義ファイルの提案}
{upstreamに関する様々な情報を記載したメタデータの提案}
{copyrightファイルを機械処理がしやすいフォーマットにする提案}
{B}
{DEP-12はソースパッケージ中の''debian/upstream/metadata''等のファイルをに用意し、upstreamに関する様々な情報を記載しようという提案です。
 一方、「CIツールによる自動テストの為の...」はDEP-8であり、こちらは''debian/tests/''以下のファイルを使い、ci.debian.netでパッケージ開発について継続的インテグレーションを実施するものとなります。先日紹介のアナウンスありましたね。}

\santaku
{dputする前にはduckで必ずテストしてほしいとのこと。ところでduckって何?}
{debian/以下のファイルに含まれるURL/メールアドレスの存在をチェックするツール}
{状況・様子から名前を断定するツール}
{依存関係をチェックするツール}
{A}
{debian/controlファイル、debian/upstream関係のファイルに記載されているURL,メールアドレスのドメインパート(例:hoge@fuga.comのfuga.com部分)が実際に存在するか/アクセス可能かを調べてくれるツールがduckとなります。}

\santaku
{mentors.debian.netでレビュー待ちのパッケージはいくつある?}
{40〜50}
{50〜60}
{70以上}
{C}
{東京エリアDebian勉強会へいらっしゃるような皆様は、ぜひdebian-devel@debian.or.jpなどでDDにmentor役をお願いする、あるいはdebianの開発チームでパッケージがメンテナンスをされている場合はチームの誰かにmentor役をお願いするのがパッケージアップロードの道として早いです。不定期に開かれるDebian Hack Cafeで面前で直接頼むという手もあります。}
