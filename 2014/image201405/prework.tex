\begin{prework}{ Koji Hasebe }
MAC上でDebian作業環境を構築します。\\
その環境を業務で使えるところまで行けたらと思っています。
\end{prework}

\begin{prework}{ dictoss(杉本 典充) }
Debian GNU/kFreeBSDをいろいろ試す。
\end{prework}

\begin{prework}{ 吉野(yy\_{}y\_{}ja\_{}jp) }
\begin{itemize}
\item DDTSS
\item manpages-ja
\end{itemize}
\end{prework}

\begin{prework}{ wbcchsyn }
kpatch を読む\\
\url{https://github.com/dynup/kpatch}
kpatch は、OS を停止せずに linux kernel にパッチを当てる為のパッチとの事。(開発中)\\

 Linux kernel のパッチなので Gnu Linux でも使用可能なはずだが、RedHat 中心に開発されているとの事なので、Debian 向けのドキュメントや環境が整うには時間がかかりそう。なので、自分で開発中の GitHub を読んでみる。
\end{prework}

\begin{prework}{ regonn(Kenta Tanoue) }
 Debian初心者なのでDebianリファレンスを読んでいきます。(目標は半分の6章まで。知らなかったことをどんどんメモしていく)
\end{prework}

\begin{prework}{ shinyorke }

\begin{description}
\item [課題] Pythonistaとして、Debianに慣れる。
\item [内容] 
\begin{itemize}
\item 自分で作ったPythonアプリ(Django)をDebian上で動かす(Python + Django + MySQL)
\item nginxを入れて、Pythonアプリをリバースプロキシする
\end{itemize}
\item [環境] 
\begin{itemize}
\item ホストOS: Mac OS X(Mavericks)
\item ゲストOS: Debian ※virtualbox上で動作
\end{itemize}
\end{description}
 今回、外向けの公開はしない。

\end{prework}

\begin{prework}{ 野島 貴英 }

 Debianによるimmutable infrastractureな環境の実験と試行錯誤。
(んでもって、何か見つけたらバグレポ)

\end{prework}

\begin{prework}{ まえだこうへい }

先月の続き。

\begin{itemize}
\item Golang関係のパッケージ化の続き
\item \url{http://qa.debian.org/developer.php?login=mkouhei@palmtb.net}のバグ潰し&パッケージアップデート
\end{itemize}
\end{prework}
