%; whizzy-master ../debianmeetingresume201311.tex
% 以上の設定をしているため、このファイルで M-x whizzytex すると、whizzytexが利用できます。
%

\santaku
{2014/2/1、次期DebianのバージョンであるJessieにて、とあるアーキテクチャが取り除かれた事がアナウンスされました。それは、どのアーキテクチャ?}
{hurd-i386}
{s390x}
{ia64}
{C}
{長い間お疲れ!>ia64。AMDの戦略に負けた、世間に負けた。ところで、hurd-i386アーキテクチャとsparcアーキテクチャも、2014/1時点でRelease Teamによればわりと崖っぷちな状況のようです。参考:\url{https://lists.debian.org/debian-devel-announce/2014/01/msg00008.html}}

\santaku
{2014/2あたまにリリースされたstable版のDebianのバージョンはいくつでしょう?}
{7.3}
{7.4}
{7.5}
{B}
{wheezy使いの人は早速アップデートだ!今回もセキュリティに関するBugfixが主です。}

\santaku
{Debianの資産(Assetの事です)を任せることのできる「信頼に足る組織(Trusted Organization)」の定義が先日レビューされていました。信頼に足る組織の条件に当てはまらない組織はどれ}
{公式Debian開発者が居ない組織}
{素早い応答/対応ができる組織}
{Debian社会憲章と対立しない組織}
{A}
{最新版は、\url{https://wiki.debian.org/Teams/DPL/TrustedOrganizationCriteria}
に掲載されています。ちなみに、信頼に足る組織が何をするのかの定義については、Debianプロジェクト憲章の9.4章にあります。今まで明確な基準がなかったのか?というのがちょっと驚き。}

\santaku
{2013/1/23にPCゲームをネットで売るサービス(Steam)運営で有名なValve社が、いくつかのLinux対応のゲームを無料で提供しますと決めました。どんな人が対象でしょうか?}
{全Debianユーザ}
{全Debian公式開発者}
{全IT企業のDebianサーバー戦士}
{B}
{これもコミュニティへの企業の寄付の方法として面白いと思いました。いわゆる「現物支給」という奴ですな。}

\santaku
{Debianの公報チームが、ソーシャルメディアの公式アカウントでの発言する内容の募集をしています。最初に投稿されるのはどのアカウントでしょうか?}
{twitterのdebian}
{google+のdebian}
{identi.caのdebian}
{C}
{debianのソーシャルメディアの公式アカウントからDebianの活動をアピりたい人は応募してみると良いとおもいます。\url{https://wiki.debian.org/Teams/Publicity/Identica}}

\santaku
{Debian MemberへSIPサービスが提供されました。Debian Memberじゃない人がDebian MemberとSIPを使ったコミュニケーションをするときに便利なサイトは?}
{rtc.debian.org}
{freephonebox.net}
{www.nttdocomo.co.jp}
{B}
{DSA頑張った!Debian Memberの人は\url{rtc.debian.org}にxxxx@debian.orgをSIPアカウントにしてログインしておくと、Debian Member以外の人はfreephonebox.netからxxxx@debian.org宛に連絡を取ることができるとの事。}

\santaku
{Debconf14の開催日は?}
{8月23日〜31日}
{8月10日〜24日}
{7月21日〜31日}
{A}
{2014年はアメリカ オレゴン州 ポートランドで開かれます。先日スポンサー募集の案内が流れました。Debconf14については\url{http://debconf14.debconf.org/}。Debconf13の様子は\url{http://www.irill.org/videos/debconf13}で見れます。}

\santaku
{Jessieのデフォルトのinitシステムが投票により決定しました。さて何になったでしょう?}
{sysvinit}
{upstart}
{systemd}
{C}
{長い間の論争にケリがついたようです。早速systemdの使い方を覚えないと。
 参考:ctteの投票アナウンス\url{https://lists.debian.org/debian-ctte/2014/02/msg00281.html}、結論\url{https://lists.debian.org/debian-ctte/2014/02/msg00405.html}
}
