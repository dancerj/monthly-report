\documentclass[cjk,dvipdfmx,10pt,compress,%
hyperref={bookmarks=true,bookmarksnumbered=true,bookmarksopen=false,%
colorlinks=false,%
pdftitle={第 91 回 関西 Debian 勉強会},%
pdfauthor={倉敷・のがた・佐々木・かわだ},%
%pdfinstitute={関西 Debian 勉強会},%
pdfsubject={資料},%
}]{beamer}

\title{第 91 回 関西 Debian 勉強会}
\subtitle{$\sim$発表資料$\sim$}
\author[かわだ てつたろう]{{\large\bf 倉敷・のがた・佐々木・かわだ}}
\institute[Debian JP]{{\normalsize\tt 関西 Debian 勉強会}}
\date{{\small 2014 年 11 月 23 日}}

%\usepackage{amsmath}
%\usepackage{amssymb}
\usepackage{graphicx}
\usepackage{moreverb}
\usepackage[varg]{txfonts}
\AtBeginDvi{\special{pdf:tounicode EUC-UCS2}}
\usetheme{Kyoto}
\def\museincludegraphics{%
  \begingroup
  \catcode`\|=0
  \catcode`\\=12
  \catcode`\#=12
  \includegraphics[width=0.9\textwidth]}
%\renewcommand{\familydefault}{\sfdefault}
%\renewcommand{\kanjifamilydefault}{\sfdefault}
\begin{document}
\settitleslide
\begin{frame}
\titlepage
\end{frame}
\setdefaultslide

\begin{frame}[fragile]
  \frametitle{Disclaimer}
  \begin{itemize}
  \item 疑問、質問、ツッコミ、茶々、\alert{大歓迎}
  \item その場でインタラクティブにどうぞ
  \item ハッシュタグ \#kansaidebian
\end{itemize}
\end{frame}

\begin{frame}[fragile]
\frametitle{Agenda}

\tableofcontents

\end{frame}

\section{最近の Debian 関係のイベント}

\takahashi[40]{最近の Debian\\関係のイベント}

\begin{frame}[fragile]
  \frametitle{第89回関西Debian勉強会}
  \begin{itemize}
  \item 日時: 10月26日(日)
  \item 場所: 福島区民センター
  \end{itemize}
  \begin{block}{内容}
    \begin{itemize}
    \item もくもくの会
    \end{itemize}
  \end{block}
\end{frame}

\begin{frame}[fragile]
  \frametitle{第90回関西Debian勉強会@関西オープンソース2014}
  \begin{itemize}
  \item 日時: 11月8日(土)
  \item 場所: 大阪南港ATC ITM棟 10F
  \item 内容: 「Debian 8 "jessie" frozen」10F ショーケース2 13:00〜
  \end{itemize}
\end{frame}

\begin{frame}[fragile]
  \frametitle{Debian Project}
  \begin{itemize}
  \item Bits from the release team: Jessie Freeze
  \item so long and thanks for all the fish
  \item Results for GR - Init system coupling
  \end{itemize}
\end{frame}

\takahashi[50]{そんな\\こんなで}
\takahashi[120]{次}

\section{事前課題発表}

\takahashi[50]{事前課題}

\begin{frame}[fragile]
  \frametitle{事前課題}
  \begin{block}{今回の事前課題}
    \begin{description}
    \item[事前課題1]
      Jessieのインストールを試みることが出来る機材をご持参ください
      (仮想環境へのインストールテストでもOKです)。
    \item[事前課題2]
      もしくは作業、質問などの課題を用意して教えてください。
    \end{description}
  \end{block}
\end{frame}

\begin{frame}
  \frametitle{ 木下 }

  当日となり申し訳ございません。

  \begin{enumerate}
  \item VM環境を利用予定

  \item
    \begin{enumerate}
    \item プライベートクラウドの調査・研究
      \begin{itemize}
      \item Openstackの研究
      \item Eucalyptusの研究/実験
      \end{itemize}

    \item グリッドコンピューティング関連の調査・研究
      \begin{itemize}
      \item GlobusToolkitで何ができる?

        →AndroidOS等のJavaVMのコンパイルで使えたら嬉しいかも。
      \end{itemize}

    \item Debian7 on PANDABOARDの調査・研究
      \begin{itemize}
      \item WiFiモジュール(On Board:TI製)の有効化
      \item GPUデバイスドライバの有効化
      \end{itemize}
    \end{enumerate}
  \end{enumerate}
\end{frame}

\begin{frame}
  \frametitle{ かわだてつたろう }
  \begin{enumerate}
  \item VirtualBox環境を用意しておきます。
  \end{enumerate}
\end{frame}

\begin{frame}
  \frametitle{ 川江 }
  \begin{enumerate}
  \item 用意します 
  \item HTML5のサイトの作成。
  \end{enumerate}
\end{frame}

\takahashi[50]{事前課題\\発表}


\takahashi[50]{そんな\\こんなで}
\takahashi[120]{次}

\section{インストーラテスト}
\takahashi[30]{インストーラテスト}

\section{もくもくの会}
\takahashi[30]{もくもくの会}

\takahashi[50]{そんな\\こんなで}
\takahashi[120]{次}

\section{今後の予定}
\begin{frame}[fragile]
\frametitle{今後の予定}

\begin{block}{第92回関西Debian勉強会}
  \begin{itemize}
  \item 日時: 12月28日(日)
  \item 場所: 福島区民センター 304
  \end{itemize}
\end{block}

\begin{block}{第120回東京エリアDebian勉強会}
  \begin{itemize}
  \item 日時: 11月29日(土)
  \item 場所: 株式会社スクウェア・エニックス セミナールーム
  \item 内容: 未定
  \end{itemize}
\end{block}

\end{frame}

\takahashi[50]{  }

\end{document}
%%% Local Variables:
%%% mode: japanese-latex
%%% TeX-master: t
%%% End:
