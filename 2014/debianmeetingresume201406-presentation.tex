%; whizzy paragraph -pdf xpdf -latex ./whizzypdfptex.sh
%; whizzy-paragraph "^\\\\begin{frame}\\|\\\\emtext"
% latex beamer presentation.
% platex, latex-beamer でコンパイルすることを想定。 

%     Tokyo Debian Meeting resources
%     Copyright (C) 2012 Junichi Uekawa

%     This program is free software; you can redistribute it and/or modify
%     it under the terms of the GNU General Public License as published by
%     the Free Software Foundation; either version 2 of the License, or
%     (at your option) any later version.

%     This program is distributed in the hope that it will be useful,
%     but WITHOUT ANY WARRANTY; without even the implied warreanty of
%     MERCHANTABILITY or FITNESS FOR A PARTICULAR PURPOSE.  See the
%     GNU General Public License for more details.

%     You should have received a copy of the GNU General Public License
%     along with this program; if not, write to the Free Software
%     Foundation, Inc., 51 Franklin St, Fifth Floor, Boston, MA  02110-1301 USA

\documentclass[cjk,dvipdfmx,12pt]{beamer}
\usetheme{Tokyo}
\usepackage{monthlypresentation}

%  preview (shell-command (concat "evince " (replace-regexp-in-string "tex$" "pdf"(buffer-file-name)) "&")) 
%  presentation (shell-command (concat "xpdf -fullscreen " (replace-regexp-in-string "tex$" "pdf"(buffer-file-name)) "&"))
%  presentation (shell-command (concat "evince " (replace-regexp-in-string "tex$" "pdf"(buffer-file-name)) "&"))

%http://www.naney.org/diki/dk/hyperref.html
%日本語EUC系環境の時
\AtBeginDvi{\special{pdf:tounicode EUC-UCS2}}
%シフトJIS系環境の時
%\AtBeginDvi{\special{pdf:tounicode 90ms-RKSJ-UCS2}}

\newenvironment{commandlinesmall}%
{\VerbatimEnvironment
  \begin{Sbox}\begin{minipage}{1.0\hsize}\begin{fontsize}{8}{8} \begin{BVerbatim}}%
{\end{BVerbatim}\end{fontsize}\end{minipage}\end{Sbox}
  \setlength{\fboxsep}{8pt}
% start on a new paragraph

\vspace{6pt}% skip before
\fcolorbox{dancerdarkblue}{dancerlightblue}{\TheSbox}

\vspace{6pt}% skip after
}
%end of commandlinesmall

\title{東京エリアDebian勉強会}
\subtitle{第114回 2014年6月度}
\author{野島貴英}
\date{2014年6月14日}
\logo{\includegraphics[width=8cm]{image200607/openlogo-light.eps}}

\begin{document}

\begin{frame}
\titlepage{}
\end{frame}

\begin{frame}{設営準備にご協力ください。}
会場設営よろしくおねがいします。
\end{frame}

\begin{frame}{Agenda}
 \begin{minipage}[t]{0.45\hsize}
  \begin{itemize}
   \item 注意事項
	 \begin{itemize}
	  \item 写真はセミナールーム内のみ可です。
          \item 出入りは自由でないので、もし外出したい方は、野島まで一声くださいませ。
	 \end{itemize}
   \item 最近あったDebian関連のイベント報告
	 \begin{itemize}
	  \item 第113回 東京エリアDebian勉強会
	 \end{itemize}
  \end{itemize}
 \end{minipage} 
 \begin{minipage}[t]{0.45\hsize}
  \begin{itemize}
   \item Debian Trivia Quiz
   \item GPG 秘密鍵取り扱い方法の提案
   \item 今後のイベント
   \item 今日の宴会場所
  \end{itemize}
 \end{minipage}
\end{frame}

\section{イベント報告}
\emtext{イベント報告}

\begin{frame}{第113回 東京エリアDebian勉強会}

 東京エリアDebian勉強会113回目は(株)スクウェア・エニックスさんで開催されました。
8名の参加者がありました。

\begin{itemize}
\item debianでdocker.ioを使う件について
  \begin{itemize}
    \item debianパッケージを利用した使い方について
    \item 実演
    \item Google Compute Engine(こちらもdebian!)でのdocker.io利用について。
 \end{itemize}
について発表がありました。
\item 参加者全員で、各自の作業を行い、最後に成果発表をしました。
\end{itemize}

\end{frame}

\section{Debian Trivia Quiz}
\emtext{Debian Trivia Quiz}
\begin{frame}{Debian Trivia Quiz}

  Debian の常識、もちろん知ってますよね?
知らないなんて恥ずかしくて、知らないとは言えないあんなことやこんなこと、
みんなで確認してみましょう。

今回の出題範囲は\url{debian-devel-announce@lists.debian.org},
\url{debian-devel@lists.debian.org} に投稿された
内容などからです。

\end{frame}

\subsection{問題}

%; whizzy-master ../debianmeetingresume201311.tex
% 以上の設定をしているため、このファイルで M-x whizzytex すると、whizzytexが利用できます。
%

\santaku
{MATE desktop環境が先日Debianのパッケージに入りました。対象のMATEのバージョンは?}
{1.7}
{1.8}
{1.9}
{B}
{Gnome 2.28ベース以降のGnomeからフォークしたデスクトップ環境とのことです。読み方は「マテ」(マテ茶のマテ)と読むとのこと。}

\santaku
{ARM64アーキテクチャへの対応の協力者募集が行われました。残念ながらDebianは動きませんが、民生品で手に入るARM64搭載の機材は次のどれでしょう?}
{iphone 5s}
{NEXUS 7}
{OpenBlocks A7}
{A}
{iphone 5sから搭載されているA7 SoCはARMv8搭載でARM64動作も出来るそうです。ARM64を搭載した民生品はこれが最初かも。
一方、DebianのARM64ポーティングでは全く別の開発用向けに提供されている機材が利用されています。
我々がDebianの動くARM64機材を民生品で手に出来る日は何時ー(涙)?}

\santaku
{edos.debian.netがqa.debian.org/doseで復活しました。ところで、このサイト何するサイト?}
{各パッケージの依存関係がお互いに満たされている状態かをチェック}
{各パッケージがどれだけupstreamから乖離しているかチェック}
{各パッケージがどれだけ人気があるかをチェック}
{A}
{dose-*パッケージに含まれるチェックコマンドを使い、各アーキテクチャのdebianリポジトリを毎日チェックして、依存関係が満たされていないパッケージをチェックした結果を教えるサイトとなります。膨大な量のパッケージの依存関係を迅速にチェック出来るところが凄い点です。仕組みについてIEEEの論文まで出ているようです。参考:http://hal.archives-ouvertes.fr/ docs/00/14/95/66/PDF/ase.pdf}

\santaku
{先日Debian GNU/Hurdの中間報告がなされました。initが変更になったそうですが、どれになったでしょうか?}
{Hurd独自実装のinit}
{systemd}
{Sysvinit}
{C}
{去年のGSoCの成果がそのまま開発継続し、遂にSysvinitにinitが変更されたそうです。このおかげで、Network、halt/shutdown時の動作、ブート時のfilesystemのマウント方式がやっとDebian流になったようです。}

\santaku
{DEP-12がPTSによりサポートされたそうです。ところで、DEP-12の内容って何?}
{CIツールによる自動テストの為の定義ファイルの提案}
{upstreamに関する様々な情報を記載したメタデータの提案}
{copyrightファイルを機械処理がしやすいフォーマットにする提案}
{B}
{DEP-12はソースパッケージ中の''debian/upstream/metadata''等のファイルをに用意し、upstreamに関する様々な情報を記載しようという提案です。
 一方、「CIツールによる自動テストの為の...」はDEP-8であり、こちらは''debian/tests/''以下のファイルを使い、ci.debian.netでパッケージ開発について継続的インテグレーションを実施するものとなります。先日紹介のアナウンスありましたね。}

\santaku
{dputする前にはduckで必ずテストしてほしいとのこと。ところでduckって何?}
{debian/以下のファイルに含まれるURL/メールアドレスの存在をチェックするツール}
{状況・様子から名前を断定するツール}
{依存関係をチェックするツール}
{A}
{debian/controlファイル、debian/upstream関係のファイルに記載されているURL,メールアドレスのドメインパート(例:hoge@fuga.comのfuga.com部分)が実際に存在するか/アクセス可能かを調べてくれるツールがduckとなります。}

\santaku
{mentors.debian.netでレビュー待ちのパッケージはいくつある?}
{40〜50}
{50〜60}
{70以上}
{C}
{東京エリアDebian勉強会へいらっしゃるような皆様は、ぜひdebian-devel@debian.or.jpなどでDDにmentor役をお願いする、あるいはdebianの開発チームでパッケージがメンテナンスをされている場合はチームの誰かにmentor役をお願いするのがパッケージアップロードの道として早いです。不定期に開かれるDebian Hack Cafeで面前で直接頼むという手もあります。}


\section{事前課題}
\emtext{事前課題}
{\footnotesize
 \begin{prework}{ wbcchsyn }
発表予定のツールをブラッシュアップする。\\
\url{https://github.com/wbcchsyn/dVault\_prototype}
フィードバックがあれば、取り込み。\\
無ければ、ドキュメント作成とか自分で気がついている部分の改修とか。
\end{prework}

\begin{prework}{ 野島 貴英 }
\begin{itemize}
\item Debian ARM64まわりを調べる。
\item OSS化したsoftetherをdebianパッケージとしてどうするか考える
\item Debian Gnu/Hurdネタを調べる
\end{itemize}

 まあ、将来の東京エリアDebian勉強会ネタの仕込み作業となりますなぁ。
\end{prework}

\begin{prework}{ yyuu }

 先々月に続いて、pyenv というスクリプトの deb 化のための作業を行なおうと考えています。

 \url{https://github.com/yyuu/pyenv}

 可能であれば pyenv-virtualenv プラグインも deb 化したいです。

 \url{https://github.com/yyuu/pyenv-virtualenv}

\end{prework}

\begin{prework}{ zinrai }

direnvのdebパッケージ化

\url{https://github.com/zimbatm/direnv}

\end{prework}

\begin{prework}{ koedoyoshida }

あんどきゅめんてっどでびあんの編集作業

\end{prework}

\begin{prework}{ regonn(Kenta Tanoue) }

AWS debian環境で、
Rails+publifyを使って自分のブログを作ります。(作業工程を記事にしたい)
インストールの隙間時間でDDTSSの作業もできたらいいな。
\end{prework}

\begin{prework}{ なかおけいすけ }

7月の勉強会で発表せよとご指名いただいたので、その準備をします。

\end{prework}

\begin{prework}{ 野首(@knok) }

\begin{itemize}
\item KAKASIのバグフィックス
\item gnu.org翻訳
\item ECS Livaでのdebian kernelでのmmc関連修正
\end{itemize}

\end{prework}

}

\section{GPG 秘密鍵取り扱い方法の提案}
\emtext{GPG 秘密鍵取り扱い方法の提案}

\section{今後のイベント}
\emtext{今後のイベント}
\begin{frame}{今後のイベント}
\begin{itemize}
 \item 2014年6月15日 Debian meetup Hokkaido 14.06 
 \item 2014年7月 東京エリアDebian勉強会
\end{itemize}
\end{frame}

\section{今日の宴会場所}
\emtext{今日の宴会場所}
\begin{frame}{今日の宴会場所}
未定
\end{frame}

\end{document}

;;; Local Variables: ***
;;; outline-regexp: "\\([ 	]*\\\\\\(documentstyle\\|documentclass\\|emtext\\|section\\|begin{frame}\\)\\*?[ 	]*[[{]\\|[]+\\)" ***
;;; End: ***
