
\begin{prework}{ dictoss(杉本 典充) }

Build Debian with another compiler.

gcc以外のコンパイラーでdebianをビルドできると面白い。freebsd 10はclangへコンパイラ変更を考えているため、debianにおけるkfreebsdのビルドシステムにも関わるため興味がある。
\end{prework}

\begin{prework}{ 野島 貴英 }

1. Games Team BOF

 現在の活動状況サマリと質疑応答/今後のアイデア出し。ゲームのデータのサイズが大きい場合についての議論(参考:ryzomではデータ8GB、vegastrikeはデータが大きすぎてパッケージがRejectされている)、ゲーム用途のサーバー設置についての意見が出た。

2. Machine learning threats and oppotunities for Debian and Free
 Software

  機械学習の技術とこれを応用した技術を主な例に、データそのものが心臓部となりやすいアプリケーション/技術が昨今増えてきている関係で、このデータについてもユーザの改変の自由/配布の自由に関する考え方を守り、ライセンスを保護するには我々はどうしていくべきかについてのセッション。クラウド環境のサービスを利用して動作するソフトウェアの場合、クラウド環境で動作するプログラムについても同様の事が生じる事についても触れられている。

以上1、2のように自分には聞こえました。チャンチャン。自分の英語力が無いので間違っていたら教えてーっ。
\end{prework}

\begin{prework}{ 本庄 }

見てませんすいません。
\end{prework}

\begin{prework}{ henrich }

パッケージアーカイブの縮小のセッション :)
\end{prework}

\begin{prework}{ yamamoto }

AArch64 planning のセッションに興味ありましたが、AArch64 を Debian で移植する具体的な計画や、取りまとめている人物の情報等が無いようで、残念。しかし Linaro では Ubuntu をベースとして toolchain のクロスコンパイルを開始していることは分かった。ちなみに Debian でのアーキテクチャ名は arm64 として dpkg 1.16.4 で取り込まれている。
\end{prework}

\begin{prework}{ 野首 }

新型OpenBlocksについて興味を持った人が多かったとのことなので、そのあたりを聞きたいです。
あと、SecureBootに関する話題も出たという話なのでそちらも気になります。

\end{prework}

\begin{prework}{ まえだこうへい }

まだビデオ自体を見る時間を取れてません…。
タイトルで気になっているのは次の5つです。なんか仕事がらみのが多いですね…。

\begin{itemize}
 \item EFI in Debian
 \item ARM port(s) update
 \item AArch64 planning 
 \item XCP, Openstack, Debian and the Cloud
 \item Advanced linux kernel networking features for virtualization
\end{itemize}
\end{prework}

\begin{prework}{ 日比野 啓 }

Multiarch の lib*-dev パッケージのヘッダまわりがどうなってるのが気になっていたので Multiarch packaging workshop を見てみました。
Multiarch にしたときの依存関係まわりがどうなっているのか。ライブラリのようにdynamic linkしているものは同じアーキテクチャが必要。
そうではなく、コマンドに依存している場合は動作が問題なければアーキテクチャは関係ないはず。
しかしコマンドについては異なるアーキテクチャはインストールできるようにはなっていない。

Multiarch crossbuilding あたりも気になっていますがまだ見ていないです。

\end{prework}
