\documentclass[cjk,dvipdfmx,10pt,compress,%
hyperref={bookmarks=true,bookmarksnumbered=true,bookmarksopen=false,%
colorlinks=false,%
pdftitle={第 97 回 関西 Debian 勉強会},%
pdfauthor={倉敷・のがた・佐々木・かわだ},%
%pdfinstitute={関西 Debian 勉強会},%
pdfsubject={資料},%
}]{beamer}

\title{第 97 回 関西 Debian 勉強会}
\subtitle{$\sim$発表資料$\sim$}
\author[かわだ てつたろう]{{\large\bf 倉敷・のがた・佐々木・かわだ}}
\institute[Debian JP]{{\normalsize\tt 関西 Debian 勉強会}}
\date{{\small 2015 年 4 月 26 日}}

%\usepackage{amsmath}
%\usepackage{amssymb}
\usepackage{graphicx}
\usepackage{moreverb}
\usepackage[varg]{txfonts}
\AtBeginDvi{\special{pdf:tounicode EUC-UCS2}}
\usetheme{Kyoto}
\def\museincludegraphics{%
  \begingroup
  \catcode`\|=0
  \catcode`\\=12
  \catcode`\#=12
  \includegraphics[width=0.9\textwidth]}
%\renewcommand{\familydefault}{\sfdefault}
%\renewcommand{\kanjifamilydefault}{\sfdefault}
\begin{document}
\settitleslide
\begin{frame}
\titlepage
\end{frame}
\setdefaultslide

\begin{frame}[fragile]
  \frametitle{Disclaimer}
  \begin{itemize}
  \item 疑問、質問、ツッコミ、茶々、\alert{大歓迎}
  \item その場でインタラクティブにどうぞ
  \item ハッシュタグ \#kansaidebian
  \end{itemize}
\end{frame}

\begin{frame}[fragile]
\frametitle{Agenda}

\tableofcontents

\end{frame}

\section{最近の Debian 関係のイベント}

\takahashi[40]{最近の Debian\\関係のイベント}

\begin{frame}[fragile]
  \frametitle{第96回関西Debian勉強会}
  \begin{itemize}
  \item 日時: 3月29日(日)
  \item 場所: 福島区民センター
  \end{itemize}
  \begin{block}{内容}
    \begin{itemize}
    \item「某所 VPS を先走って Jessie に上げてみた」(佐々木洋平)
    \end{itemize}
  \end{block}
\end{frame}

\begin{frame}[fragile]
  \frametitle{第125回東京エリアDebian勉強会}
  \begin{itemize}
  \item 日時: 4月18日(土)
  \item 場所: 株式会社スクウェア・エニックス セミナールーム
  \end{itemize}
  \begin{block}{内容}
    \begin{itemize}
    \item 「DebianパッケージからPythonパッケージへの変遷、そしてPythonパッケージの公式Debianパッケージ化について考える」
    \end{itemize}
  \end{block}
\end{frame}

\begin{frame}[fragile]
  \frametitle{Debian Project}
  \begin{itemize}
  \item 
  \end{itemize}
\end{frame}

\takahashi[50]{そんな\\こんなで}
\takahashi[120]{次}

\section{事前課題発表}

\takahashi[50]{事前課題}

\begin{frame}[fragile]
  \frametitle{事前課題}
  \begin{block}{今回の事前課題}
    \begin{description}
    \item[事前課題1]
      Debian8 "Jessie" リリースについてなにかひとことどうぞ。
    \end{description}
  \end{block}
\end{frame}

\takahashi[50]{事前課題\\発表}

\begin{frame}
  \frametitle{ tosihisa }
  \begin{enumerate}
  \item よろしくお願いします。
  \end{enumerate}
\end{frame}

\begin{frame}
  \frametitle{ 木下聖士 }
\end{frame}

\begin{frame}
  \frametitle{ Say-no }
  \begin{enumerate}
  \item おめでとうございます! 

    申し込み遅くなってごめんなさい!
  \end{enumerate}
\end{frame}

\begin{frame}
  \frametitle{ t3rkwd }
  \begin{enumerate}
  \item いろいろありましたが、おつかれさまでした、おめでとうございます。
  \end{enumerate}
\end{frame}

\begin{frame}
  \frametitle{ 奥野 由紀 }
  \begin{enumerate}
  \item ARM64対応に興味があります
  \end{enumerate}
\end{frame}

\begin{frame}
  \frametitle{ ItSANgo }
  \begin{enumerate}
  \item おめでとうございます。

    私は何も貢献できませんでしたが、 今後FreeWnnをお守りするつもりでいますのでよろしくお願いいたします。
  \end{enumerate}
\end{frame}

\begin{frame}
  \frametitle{ Daizoh Nakashima }
  \begin{enumerate}
  \item Wheezy からの移行にあたり、クリティカルな変更点が少なくないと聞いており、
    身構えています。
  \end{enumerate}
\end{frame}

\begin{frame}
  \frametitle{ 川江 浩 }
  \begin{enumerate}
  \item ヒャホー!!
  \end{enumerate}
\end{frame}

\begin{frame}
  \frametitle{ むんくさん }
  \begin{enumerate}
  \item 皆様お疲れ様ですおめでとうございます。
  \end{enumerate}
\end{frame}

\begin{frame}
  \frametitle{ lurdan }
  \begin{enumerate}
  \item 思うところはいろいろあるけど、ひとまず皆さんおつかれさまー
  \end{enumerate}
\end{frame}

\begin{frame}
  \frametitle{ 寺崎織人 }
  \begin{enumerate}
  \item おめでとうございます。
  \end{enumerate}
\end{frame}

\begin{frame}
  \frametitle{ 寺崎彰洋 }
  \begin{enumerate}
  \item フリーズから約半年でのリリース。前回より速くてよかったです。
  \end{enumerate}
\end{frame}

\begin{frame}
  \frametitle{ 佐々木洋平 }
  \begin{enumerate}
  \item キタ━(゜∀゜)━!
  \end{enumerate}
\end{frame}

\begin{frame}
  \frametitle{ Yukiharu YABUKI }
  \begin{enumerate}
  \item 参加予定
  \end{enumerate}
\end{frame}

\takahashi[50]{そんな\\こんなで}
\takahashi[120]{次}

\section{Wheezy から Jessie まで}
\takahashi[30]{Wheezy から Jessie まで\\by\\かわだてつたろう}

\begin{frame}
  \frametitle{ Wheezy }
  \begin{block}{おさらい}
    \begin{description}
    \item[2012年6月30日]
      フリーズ
    \item[2013年5月4日]
      リリース
    \end{description}
  \end{block}
\end{frame}

\takahashi[50]{Jessieへ}

\begin{frame}
  \frametitle{ Jessie }
  \begin{block}{2013年}
    \begin{description}
    \item[5月6日]
      Wheezy is out! Jessie is created and receives updates!
    \item[8月11日]
      DebConf13
    \item[8月25日]
      Bits from the Release
    \item[9月11日]
      Call for Jessie Release Goals
    \item[10月13日]
      Bits from the Release Team (Jessie freeze info)
    \item[11月28日]
      Release sprint results - team changes, auto-rm and arch status
    \end{description}
  \end{block}
\end{frame}

\begin{frame}
  \frametitle{ Jessie }
  \begin{block}{ 2013年(systemd in debian-devel@) }
    \begin{itemize}
    \item systemd effectively mandatory now due to GNOME
    \item Proposal: switch default desktop to xfce
    \item Proposal: switch init system to systemd or upstart
    \end{itemize}
  \end{block}
\end{frame}

\begin{frame}
  \frametitle{ Jessie }
  \begin{block}{2014年 (1/2)}
    \begin{description}
    \item[2月11日]
      [CTTE \#727708] Default init system for Debian
    \item[3月19日]
      Debian Installer Jessie Alpha1 release
    \item[4月26日]
      SPARC removed from Jessie
    \item[5月1日]
      Bits from the Release Team - Freeze, removals and archs
    \item[5月22日]
      Bits from the Debian GNU/Hurd porters
    \item[6月3日]
      MATE 1.8 has now fully arrived in Debian
    \end{description}
  \end{block}
\end{frame}

\begin{frame}
  \frametitle{ Jessie }
  \begin{block}{2014年 (2/2)}
    \begin{description}
    \item[6月18日]
      Accepted glibc 2.19-3experimental0 (source all)
    \item[7月30日]
      Linux kernel version for jessie
    \item[8月13日]
      Debian Installer Jessie Beta 1 release
    \item[8月22日]
      DebConf14
    \item[9月4日]
      Cinnamon environment now available in testing
    \item[9月19日]
      Accepted tasksel 3.25 (source all) into unstable
    \item[11月5日]
      Bits from the release team: Jessie Freeze
    \end{description}
  \end{block}
\end{frame}

\begin{frame}
  \frametitle{ Jessie }
  \begin{block}{ 2014年(systemd in debian-devel@) }
    \begin{itemize}
    \item systemd-fsck?
    \item How to avoid stealth installation of systemd?
    \item systemd now appears to be only possible init system in testing
    \end{itemize}
  \end{block}

  \begin{block}{ }
    \begin{itemize}
    \item Code of Conduct
    \item General Resolution: init system coupling
    \item so long and thanks for all the fish
    \end{itemize}
  \end{block}
\end{frame}

\begin{frame}
  \frametitle{ Jessie }
  \begin{block}{ Squeeze LTS }
    \begin{itemize}
    \item Bits from the Security Team
    \item Long term support for Debian 6.0 Announced
    \item Debian 6 debuts its long term support period
    \end{itemize}
  \end{block}
\end{frame}

\begin{frame}
  \frametitle{ Jessie }
  \begin{block}{2015年}
    \begin{description}
    \item[1月26日]
      Debiain Installer Jessie RC1
    \item[2月4日]
      Permanent removals from testing for Jessie
    \item[3月8日]
      Status on the Jessie release
    \item[3月28日]
      Debian Installer Jessie RC 2 release
    \item[3月31日]
      Jessie Release Date: 2015-04-25
    \item[4月19日]
      Debian Installer Jessie RC 3 release
    \item[4月23日]
      Ready for Jessie! (aka bits from the debian-cd team)
    \item[4月25日]
      Debian 8 "Jessie" released
    \end{description}
  \end{block}
\end{frame}

\begin{frame}
  \frametitle{ Stretch, Buster... }
  \begin{block}{}
    \begin{itemize}
    \item Debian 9 ``Stretch''
    \item Debian 10 ``Buster''
    \end{itemize}
  \end{block}
\end{frame}

\takahashi[50]{そんな\\こんなで}
\takahashi[120]{次}

\section{Debian 8 Jessie リリースパーティ}
\takahashi[30]{Debian 8 Jessie\\リリースパーティ}

\takahashi[120]{次}

\section{今後の予定}
\begin{frame}[fragile]
\frametitle{今後の予定}

\begin{block}{第98回関西Debian勉強会}
  \begin{itemize}
  \item 日時: 5月24日(日)
  \item 場所: 福島区民センター 304号
  \end{itemize}
\end{block}

\begin{block}{第126回東京エリアDebian勉強会}
  \begin{itemize}
  \item 日時: 5月16日(土)
  \item 場所: 株式会社スクウェア・エニックス セミナールーム
  \item 内容: 未定
  \end{itemize}
\end{block}

\end{frame}

\takahashi[50]{  }

\end{document}
%%% Local Variables:
%%% mode: japanese-latex
%%% TeX-master: t
%%% End:
