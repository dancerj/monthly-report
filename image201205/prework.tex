%; whizzy-master ../debianmeetingresume201205.tex
% 以上の設定をしているため、このファイルで M-x whizzytex すると、whizzytexが利用できます。
%

\begin{prework}{ BeatenAvenue }

1.おすすめ本:

「windowsプロフェッショナルゲームプログラミング(全2巻?)」
DirectX関連の内容もあった気がしますがwin32APIについての内容が多かった気がします。昔の本ですがタスク処理の考え方など現在でも通用する部分は多いと思います。この本を買ってからC++の勉強を始めたこともあって・・・個人的に思い出がたくさんあります。

「DirectX逆引き大全500の極意」
入門的な優しい解説から一歩踏み込んだプラスアルファまで揃っています。残念ながら絶版で図書館から借りて読みました。DirectX9の解説書では一番よいものかと思います。

「GameProgrammingGems(シリーズ)」
海外のゲームプログラマの方々が書いた記事をまとめた本。3巻だったかと思い
 ますがNaughty Dogの方が書いた"Jak and Daxter: The Precursor legacy"でのマップ移動処理についての内容が好きです。お値段以上。

2.初心者におすすめするスクリプト言語:
Debianとは全く関係ないですがDOSバッチファイルとExcelVBAのちょっとした使い方は覚えないと事務仕事が進みません。解説サイトも多いので付きっきりで教える必要もあまりないです。
Linuxだとbashのスクリプトなんでしょうか。私が初心者なのでそれしか触っていません・・・。
\end{prework}

\begin{prework}{ amotoki }

1. 昨日友人から勧められて気になっているのが「情熱プラグラマー」です。自分の人生を自分で切り開いていくために必要なことが分かりやすく整理されているけど、自分を前に進めてくれる情熱を感じたとのこと。さっそく注文した。

2. 今おすすめするとしたらPythonをお薦めします。オブジェクト指向も書きやすいし、ライブラリも充実していて、マニュアルも実例がそろっているので、プログラミングを学んでいく上でよいと思います。大きめのOSSプロジェクトでも使われているので、知っておいて損することもありません。最初の一歩としては「Pythonチュートリアル」がよいと思います。今でもときどき見ることがあります。
\end{prework}

\begin{prework}{ 吉野(yy\_y\_ja\_jp) }

1. Gitによるバージョン管理 実際のプロジェクトでのGitの利用法が書かれているようです.

2. シェルスクリプト
気軽に使えるからです.
bash(1), dash(1)
\end{prework}

\begin{prework}{ dictoss(杉本 典充) }

1.「インテル スレッディング・ビルディング・ブロック -- マルチコア時代のC++並列プログラミング」オライリー・ジャパン、James Reinders著
Intelが開発し現在はGPLv2で公開しているC++の並列計算用ライブラリ「Threading Building Blocks」の解説を行っている本。マルチコア時代の中で複数のCPUコアを効率的に使用して計算性能を上げるための知識が詰まっている。あくまで単一ノードで計算性能を上げるための手法であり、複数ノードで計算性能を向上させるクラスタリング技術の話ではないので注意。

2.お勧めはpython。理由はプログラミング初心者ということで開発者によって書き方に差異が出にくい分webで紹介されているコードに癖がなくとっつきやすいため。
おすすめサイトは、うーん、なんでしょ?自分は別の言語が書けるようになってからpythonを始めたので"\url{http://www.python.jp/doc/release/}"を確認します。
\end{prework}

\begin{prework}{ kamonshohei }

1. シェルスクリプトシェルスクリプト基本リファレンス

2. シェルスクリプトでしょうか。リナックスのコマンドだけで、ちゃちゃっと実装できる手軽さがいいです。最初の一歩で案内する書籍は 1であげたシェルスクリプト基本レファレンスです。
\end{prework}

\begin{prework}{ emasaka }

1. ケン・スミス「誰も教えてくれない聖書の読み方」。聖書に書かれているそのままの文面を真面目に読んでユーモラスに紹介(?)している本(Debian関係ない)

2. Bashと書こうかと思ったけどRuby。理由は「オブジェクト指向を使っても使わなくてもいい」ではなくて「オブジェクト指向を強制される」から。書籍は「たのしいRuby」

\end{prework}

\begin{prework}{ 本庄 }

1. Debian 勉強会参加者に紹介したい書籍を1冊以上挙げて、内容を簡単に紹介してください。
定番ですが『ハッカーと画家』とかどうでしょう。オタクの怨念が込められています。親として子供の将来に不安を感じます。

2. あなたが何かスクリプト言語をプログラミング初心者にお勧めするとして「その言語を選んだ理由」と「最初の一歩として案内する書籍/サイト」を教えてください。
PHPがおすすめです。ほかに比べて仕事が多そうという理由です。多いかどうかは実際のところわかりませんが、

\url{http://www.google.co.jp/trends/?q=PHP,+Perl,+Python,+Ruby,+Javascript,+Haskell&ctab=0&geo=jp&geor=all&date=ytd&sort=0}

ここ見ると多そうです。案内する書籍として定番はマンモス本だと思いますが、読んだことはありません。そして古い情報かもしれません。
以前、とあるPHP方面の方とお話しする機会があり、オライリーの本はちょっと…的なことを話したら、「ああいう本もいいかなと思ってます」といわれました。翻訳者でした。

\end{prework}

\begin{prework}{ henrich }

1. 「入門Debianパッケージ」。書名から内容が分かるかとは思いますが、Debianパッケージの作り方の書籍です。続刊が期待されます

2. Python を選びました。Perlは人によってとても書き方が変わるところがあまり嬉しくなく、Rubyはバージョン間の移行が若干乱暴に感じられたので。何度もプログラミングに挫折している私ですが、今回Pythonを学ぶのに選んだ「初めてのコンピュータサイエンス」がとても良い書籍でした。
\end{prework}

\begin{prework}{ 野島 貴英 }

1.「イノベーションのジレンマ」(ISBN10:4798100234)と、「のうだま」(ISBN10:4344015959)。「イノベーションのジレンマ」は、技術革新が既存のものをぶち壊していく過程において、既存技術において優秀な組織であればあるほど技術革新についていけなくなってしまう現象を理詰めで説明した本。「のうだま」は人の行動においては、実は習慣が先でやる気は後からついてくるものであるという事を説明した本。

2. プログラミング初心者には今時の状況からjavascript/HTML5を勧めたいのですが、肝心の自分が未評価。最初の一歩はゲーム遊びたさにプログラム覚えた経験を元に、\url{http://wise9.jp/} と、 \url{http://enchantjs.com/} がおすすめなのかな?
\end{prework}

\begin{prework}{ yamamoto }

1. Debian 勉強会参加者に紹介したい書籍を1冊以上挙げて、内容を簡単に紹介してください(特に技術書には限りません)。

有名だと思うので、紹介するほどのことはないかもしれませんが、「入門UNIXシェルプログラミング-シェルの基礎から学ぶUNIXの世界 Bruce Blinn 著・山下哲典 訳」を愛読しています。まあ、シェルスクリプトプログラミングをシェルプログラミングと言っているタイトル (勿論シェル自体の解説もありますけどね) はアレですが、Bシェルのスクリプトを書くときには良く開いています。

2. あなたが何かスクリプト言語をプログラミング初心者にお勧めするとして「その言語を選んだ理由」と「最初の一歩として案内する書籍/サイト」を教えてください。

\#!/bin/sh 万歳!
どこでも大概動くからおすすめ。
Bシェルの作法に従っていれば、今のdashなら大体動く?みたい (未確認)。
だめなら \#!/bin/bashで。
とかいいながらも、おいらのシェルスクリプトには外部コマンドバリバリ入れてますけど。(うひ

上記の書籍でもいいですが、ぐぐるさんにお伺いしたら、参考になるサイトは山ほど出てくるでしょう。
\end{prework}
