\documentclass[cjk,dvipdfmx,10pt,compress,%
hyperref={bookmarks=true,bookmarksnumbered=true,bookmarksopen=false,%
colorlinks=false,%
pdftitle={第 132 回 関西 Debian 勉強会},%
pdfauthor={かわだ},%
%pdfinstitute={関西 Debian 勉強会},%
pdfsubject={資料},%
}]{beamer}

\title{第 133 回 関西 Debian 勉強会}
\subtitle{$\sim$発表資料$\sim$}
\author[Yosuke OTSUKI]{{\large\bf Yosuke OTSUKI}}
\institute[Debian JP]{{\normalsize\tt 関西 Debian 勉強会}}
\date{{\small 2018 年 3 月 25 日}}

%\usepackage{amsmath}
%\usepackage{amssymb}
\usepackage{graphicx}
\usepackage{moreverb}
\usepackage[varg]{txfonts}
\AtBeginDvi{\special{pdf:tounicode EUC-UCS2}}
\usetheme{Kyoto}
\def\museincludegraphics{%
  \begingroup
  \catcode`\|=0
  \catcode`\\=12
  \catcode`\#=12
  \includegraphics[width=0.9\textwidth]}
%\renewcommand{\familydefault}{\sfdefault}
%\renewcommand{\kanjifamilydefault}{\sfdefault}
\begin{document}
\settitleslide
\begin{frame}
\titlepage
\end{frame}
\setdefaultslide

\begin{frame}[fragile]
  \frametitle{Disclaimer}
  \begin{itemize}
  \item 疑問、質問、ツッコミ、茶々、\alert{大歓迎}
  \item その場でインタラクティブにどうぞ
  \item ハッシュタグ \#kansaidebian
  \end{itemize}
\end{frame}

\begin{frame}[fragile]
\frametitle{Agenda}

\tableofcontents

\end{frame}

\section{最近の Debian 関係のイベント}
\takahashi[40]{最近の Debian\\関係のイベント}

\begin{frame}[fragile]
  \frametitle{第132回関西Debian勉強会}
  \begin{itemize}
  \item 日時: 2月25日(日)
  \item 場所: 福島区民センター
  \begin{block}{内容}
    \begin{itemize}
    \item 「ufw 再入門」 by 西山和広さん
    \end{itemize}
  \end{block}
\end{itemize}
\end{frame}

\begin{frame}[fragile]
  \frametitle{第161回東京エリアDebian勉強会}
  \begin{itemize}
  \item 日時: 3月24日(土)
  \item 場所: 溝口クリニック
  \end{itemize}
  \begin{block}{内容}
    \begin{itemize}
    \item 「go/debian での機械学習環境構築について」 by ysaito さん
    \end{itemize}
  \end{block}
\end{frame}

\begin{frame}[fragile]
  \frametitle{Debian Project}
  \begin{itemize}
  \item 2018/Mar/10th Stretch 9.4 がリリース
  \item 2018/Mar/16th Debian Conference 2018 registration 開始 
  	\begin{itemize}
 		\item call for parapers は 2018/June/18th https://debconf18.debconf.org/cfp/  
  	\end{itemize}
  \end{itemize}
\end{frame}

\takahashi[50]{そんな\\こんなで}
\takahashi[120]{次}

\section{事前課題}
\takahashi[50]{事前課題}

\begin{frame}[fragile]
  \frametitle{事前課題}
  \begin{block}{今回の事前課題}
    \begin{enumerate}
    \item なし
    \end{enumerate}
  \end{block}
\end{frame}

\takahashi[30]{- 家のネットワーク図 \\ - 今後どうしたいか? \\ を教えてください}

\begin{frame}
  \frametitle{ ipv6waterstar }
  \begin{enumerate}
  \item 家の中を 10G ネットワークにしたい。
  \item Nuro 10 G があることをみんなで確認 3980 円なり。
  \end{enumerate}
\end{frame}

\begin{frame}
  \frametitle{YukiharuYABUKI}
  \begin{enumerate}
  \item 
  \end{enumerate}
\end{frame}

\begin{frame}
  \frametitle{ yosuke\_san }
  \begin{enumerate}
  \item 借りていいる VPS で CI 環境を構築している
  \item AMQP server + luige でパイプライン処理したい
  \item 開発に必要なツールを持つコンテナ OS を meta インストールしてくれるようにしたい。
  \end{enumerate}
\end{frame}

\begin{frame}
  \frametitle{gdevmjc}
  \begin{enumerate}
  \item  一戸建て
  \item Yamaha の router + Wifi 無線
  \item 物理マシンのみ
  \end{enumerate}
\end{frame}

\begin{frame}
  \frametitle{sato\_makoto}
  \begin{enumerate}
  \item モバイル回線のみ、ipv6 で使用でしている
  \end{enumerate}
\end{frame}

\begin{frame}
  \frametitle{nogajun}
  \begin{enumerate}
  \item Flets 光 
  \item Flets の光のルータ
  \item 実機サーバー廃止して、VPS に移行するため Docker を触っている
  \end{enumerate}
\end{frame}

\begin{frame}
  \frametitle{murase\_syuka}
  \begin{enumerate}
  \item 光 
  \item 家は有線 + 無線 
  \item 家のファイルサーバーに外から繋いでいる。ポートはその都度開く。
  \item 随時開きたいので、セキュリティのよいルーターがほしい。
  \end{enumerate}
\end{frame}

\begin{frame}
  \frametitle{kawada}
  \begin{enumerate}
  \item マンションに回線が来たので、乗り換えた
  \item アクセスポイント VyOS から Debian
  \item アクセスポイント, ファイルサーバ、マクロサーバー
  \item 5 GHz を飛ばしたいが、更新をかけるたびにカーネルリビルド。めんどくさい。
  \item ファイルサーバーの raid 管理がめんどくさい。なんとかしたい。
  \end{enumerate}
\end{frame}

\begin{frame}
  \frametitle{yamada}
  \begin{enumerate}
  \item 無線が多重しており、大変なことになっている
  \item 中継器を買って、ある部分の電波を強くした
  \end{enumerate}
\end{frame}

\takahashi[50]{そんな\\こんなで}
\takahashi[120]{次}

\section{我が家の仮想ネットワーク}
\takahashi[30]{我が家の仮想ネットワーク \\ by \\川江 浩}

\takahashi[50]{そんな\\こんなで}
\takahashi[120]{次}

\section{今後の予定}
\begin{frame}[fragile]
  \frametitle{今後の予定}

  \begin{block}{第134回関西Debian勉強会}
    \begin{itemize}
    \item 日時: 4月22日(日)
    \item 場所: 福島区民センター
    \end{itemize}
  \end{block}

  \begin{block}{第162回東京エリアDebian勉強会}
    \begin{itemize}
    \item 日時: 4月21日(土)
    \item 場所: 朝日ネット様
    \item 内容: 未定 
    \end{itemize}
  \end{block}

\end{frame}

\takahashi[50]{  }

\end{document}
%%% Local Variables:
%%% mode: japanese-latex
%%% TeX-master: t
%%% End:
