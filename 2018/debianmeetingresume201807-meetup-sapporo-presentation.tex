%; whizzy paragraph -pdf xpdf -latex ./whizzypdfptex.sh
%; whizzy-paragraph "^\\\\begin{frame}\\|\\\\emtext"
% latex beamer presentation.
% platex, latex-beamer でコンパイルすることを想定。 

%     Tokyo Debian Meeting resources
%     Copyright (C) 2012 Junichi Uekawa

%     This program is free software; you can redistribute it and/or modify
%     it under the terms of the GNU General Public License as published by
%     the Free Software Foundation; either version 2 of the License, or
%     (at your option) any later version.

%     This program is distributed in the hope that it will be useful,
%     but WITHOUT ANY WARRANTY; without even the implied warreanty of
%     MERCHANTABILITY or FITNESS FOR A PARTICULAR PURPOSE.  See the
%     GNU General Public License for more details.
%     You should have received a copy of the GNU General Public License
%     along with this program; if not, write to the Free Software
%     Foundation, Inc., 51 Franklin St, Fifth Floor, Boston, MA  02110-1301 USA

\documentclass[cjk,dvipdfmx,12pt]{beamer}
\usetheme{Tokyo}
\usepackage{monthlypresentation}

%  preview (shell-command (concat "evince " (replace-regexp-in-string "tex$" "pdf"(buffer-file-name)) "&")) 
%  presentation (shell-command (concat "xpdf -fullscreen " (replace-regexp-in-string "tex$" "pdf"(buffer-file-name)) "&"))
%  presentation (shell-command (concat "evince " (replace-regexp-in-string "tex$" "pdf"(buffer-file-name)) "&"))

%http://www.naney.org/diki/dk/hyperref.html
%日本語EUC系環境の時
\AtBeginDvi{\special{pdf:tounicode EUC-UCS2}}
%シフトJIS系環境の時
%\AtBeginDvi{\special{pdf:tounicode 90ms-RKSJ-UCS2}}

\newenvironment{commandlinesmall}%
{\VerbatimEnvironment
  \begin{Sbox}\begin{minipage}{1.0\hsize}\begin{fontsize}{8}{8} \begin{BVerbatim}}%
{\end{BVerbatim}\end{fontsize}\end{minipage}\end{Sbox}
  \setlength{\fboxsep}{8pt}
% start on a new paragraph

\vspace{6pt}% skip before
\fcolorbox{dancerdarkblue}{dancerlightblue}{\TheSbox}

\vspace{6pt}% skip after
}
%end of commandlinesmall

\title{Debian / Ubuntu ユーザーミートアップ}
\subtitle{in 札幌 2018.07}
\author{杉本典充 / dictoss@live.jp}
\date{2018年7月6日}
\logo{\includegraphics[width=8cm]{image200607/openlogo-light.eps}}

\begin{document}

\begin{frame}
\titlepage{}
\end{frame}

\begin{frame}{Agenda}
  \begin{itemize}
  \item 自己紹介
  \item セミナー
    \begin{itemize}
    \item Debian で Atom、Debian で Visual Studio Code
    \item Debianで自宅にリモート接続 / OpenVPN編
    \end{itemize}
  \item ミートアップタイム
  \item 片づけ
  \end{itemize}
\end{frame}


\section{自己紹介}
\emtext{自己紹介}

{\footnotesize
  \begin{prework}{ niku\_name}
  %\begin{enumerate}
  %\item xxx
  %\end{enumerate}
\end{prework}

\begin{prework}{ yy\_y\_ja\_jp }
  %\begin{enumerate}
  %\item xxx
  %\end{enumerate}
\end{prework}

\begin{prework}{ serra }
  %\begin{enumerate}
  %\item xxx
  %\end{enumerate}
\end{prework}

\begin{prework}{ dictoss }
  \begin{enumerate}
  \item 北海道の道北在住のDebianユーザまたは開発者の方がおられるか知りたいです。
  \end{enumerate}
\end{prework}

\begin{prework}{ 北野大地@JI8GRX (Hatsunejima\_Japan) }
  %\begin{enumerate}
  %\item xxx
  %\end{enumerate}
\end{prework}

}

\section{セミナー1}
\emtext{Debian で Atom、Debian で Visual Studio Code}

\section{セミナー2}
\emtext{Debianで自宅にリモート接続 / OpenVPN編}

\section{近日の予定}
\emtext{近日の予定}

\begin{frame}{近日の予定}
  \begin{itemize}
  \item 2018/7/ 7(土) OSC 2018 Hokkaido
    \begin{itemize}
    \item Debian勉強会でブースを出展しています
    \item セミナー:15:15-16:00 Debian Updates \& Random Topics
    \end{itemize}
  \item 2018/7/29(日)- DebConf 2018 First Day (Hsinchu, Taiwan)
  \end{itemize}
\end{frame}

\end{document}

;;; Local Variables: ***
;;; outline-regexp: "\\([ 	]*\\\\\\(documentstyle\\|documentclass\\|emtext\\|section\\|begin{frame}\\)\\*?[ 	]*[[{]\\|[]+\\)" ***
;;; End: ***
