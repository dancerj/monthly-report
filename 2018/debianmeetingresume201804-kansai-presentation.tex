\documentclass[cjk,dvipdfmx,10pt,compress,%
hyperref={bookmarks=true,bookmarksnumbered=true,bookmarksopen=false,%
colorlinks=false,%
pdftitle={第 132 回 関西 Debian 勉強会},%
pdfauthor={かわだ},%
%pdfinstitute={関西 Debian 勉強会},%
pdfsubject={資料},%
}]{beamer}

\title{第 134 回 関西 Debian 勉強会}
\subtitle{$\sim$発表資料$\sim$}
\author[Yosuke OTSUKI]{{\large\bf Yosuke OTSUKI}}
\institute[Debian JP]{{\normalsize\tt 関西 Debian 勉強会}}
\date{{\small 2018 年 4 月 22 日}}

%\usepackage{amsmath}
%\usepackage{amssymb}
\usepackage{graphicx}
\usepackage{moreverb}
\usepackage[varg]{txfonts}
\AtBeginDvi{\special{pdf:tounicode EUC-UCS2}}
\usetheme{Kyoto}
\def\museincludegraphics{%
  \begingroup
  \catcode`\|=0
  \catcode`\\=12
  \catcode`\#=12
  \includegraphics[width=0.9\textwidth]}
%\renewcommand{\familydefault}{\sfdefault}
%\renewcommand{\kanjifamilydefault}{\sfdefault}
\begin{document}
\settitleslide
\begin{frame}
\titlepage
\end{frame}
\setdefaultslide

\begin{frame}[fragile]
  \frametitle{Disclaimer}
  \begin{itemize}
  \item 疑問、質問、ツッコミ、茶々、\alert{大歓迎}
  \item その場でインタラクティブにどうぞ
  \item ハッシュタグ \#kansaidebian
  \end{itemize}
\end{frame}

\begin{frame}[fragile]
\frametitle{Agenda}

\tableofcontents

\end{frame}

\section{最近の Debian 関係のイベント}
\takahashi[40]{最近の Debian\\関係のイベント}

\begin{frame}[fragile]
  \frametitle{第133回関西Debian勉強会}
  \begin{itemize}
  \item 日時: 3月22日(日)
  \item 場所: 福島区民センター
  \begin{block}{内容}
    \begin{itemize}
    \item 我が家の仮想ネットワーク by  川江 浩
    \end{itemize}
  \end{block}
\end{itemize}
\end{frame}

\begin{frame}[fragile]
  \frametitle{第161回東京エリアDebian勉強会}
  \begin{itemize}
  \item 日時: 4月21日(土)
  \item 場所: 株式会社朝日ネット
  \end{itemize}
  \begin{block}{内容}
    \begin{itemize}
    \item もくもくの会
    \end{itemize}
  \end{block}
\end{frame}

\begin{frame}[fragile]
  \frametitle{Debian Project}
  \begin{block}{内容}
  \begin{itemize}
	\item Debian Conference 2018 registration 中, CFP 2018/June/18th まで
	\item Debian Policy v4.1.4.1 がリリース
  		\begin{itemize}
		\item 差分を確認する get-orig-source が廃止に、かわりに uscan と debian/watch を使うようにとのこと
		\item \url{https://lists.debian.org/debian-devel/2018/04/msg00081.html}
  		\end{itemize}
	\item kFreeBSD と (Hurd) の buildd がパッケージ化
		\begin{itemize}
			\item sid でパッケージ化されたとのこと
			\item \url{https://lists.debian.org/debian-devel/2018/04/msg00492.html}
  		\end{itemize}
	\item m68k 科学計算系のパッケージは直さなくてもいいのでは? 
		\begin{itemize}
			\item R 言語のビルドが m68k qemu のバグが原因で失敗した。	実際に使っているユーザーもいないだろうし、	科学計算系のパッケージがビルドに失敗しても直さなくていいのでは? との提案でした。
			\item m68k ユーザーら反発。曰く「科学計算系のパッケージの品質が悪いかからだ。」とのこと
			\item \url{https://lists.debian.org/debian-devel/2018/03/msg00448.html}
		\end{itemize}
	\end{itemize}
  \end{block}
\end{frame}

\takahashi[50]{そんな\\こんなで}
\takahashi[120]{次}

\section{事前課題}
\takahashi[50]{事前課題}

\begin{frame}[fragile]
  \frametitle{事前課題}
  \begin{block}{今回の事前課題}
    \begin{enumerate}
	\item 以下の資料に目を通しておいて下さい。
	\begin{itemize}
		\item https://wiki.debian.org/sbuild
		\item https://wiki.debian.org/LXC
	\end{itemize}
	\item (推奨)会場で利用できるネットワーク環境をご用意ください。
    \end{enumerate}
  \end{block}
\end{frame}

\takahashi[50]{事前課題\\発表}

\begin{frame}
  \frametitle{ Katsuki Kobayashi }
  \begin{enumerate}
	\item DevConf 行きたいが,予定が
	\item 事前課題はやった
  \end{enumerate}
\end{frame}

\begin{frame}
  \frametitle{uwabami}
  \begin{enumerate}
		  \item 福島に行ってしまった。
		  \item 台湾に行けない。本業の予定が...  
		\item OSC Kyoto 2018
		\item 本日は、バグ修正を実演
  \end{enumerate}
\end{frame}

\begin{frame}
  \frametitle{tomabu }
  \begin{enumerate}
		  \item wiki はみた。
		\item ライブ実演が楽しみ
		\item 台湾は、難しそう。
  \end{enumerate}
\end{frame}

\begin{frame}
  \frametitle{ znz}
  \begin{enumerate}
		  \item 事前課題は
		\item nadoka を引き取る話したが進んでおらず。参考になればと
		\item debconf はわからないが、ruby conf taiwan に行くよよてい。
  \end{enumerate}
\end{frame}
 
\begin{frame}
  \frametitle{ipv6waterstar}
  \begin{enumerate}
		  \item
  \end{enumerate}
\end{frame}
 
\begin{frame}
  \frametitle{t3rkwd}
  \begin{enumerate}
	  \item 行けるのでは? 1 週間全部は無理では。
   		\item sbuild をセットアップするパッケージを試してみた。
		\item DD と keysign をした。
        \item master key の管理に悩む
		\item yubikey の利便性にやなむ ed25519 に対応してたっけ?
		\item 2 factor 認証方式と連動できるようならば、新しいの買っても良いかも
  \end{enumerate}
\end{frame}

\begin{frame}
  \frametitle{yosuke\_san}
  \begin{enumerate}
		\item Debconf 行けるかわからない。
		\item 事前仕様は見たけれど、手を動かしてはいない。
		\item DataWareHouse を作れと言われた
  \end{enumerate}
\end{frame}
 
\begin{frame}
  \frametitle{ItSANgo}
  \begin{enumerate}
		 \item C\# の仕事ばっかり
		 \item Debconf は行けるかわからない。
		 \item 事前課題は win 上の仮想マシンで立ってきた
  \end{enumerate}
\end{frame}

\begin{frame}
  \frametitle{榎真治}
  \begin{enumerate}
		  \item
  \end{enumerate}
\end{frame}

\begin{frame}
  \frametitle{kubo}
  \begin{enumerate}
	  \item drupal を入れてみた
	  \item figaro という pass manager を使っているが、debian からなくなる
      \item 自分で xslt で書いたがその後どうしようか悩み中
	  \item Active Directory と格闘している
  \end{enumerate}
\end{frame}

\takahashi[50]{そんな\\こんなで}
\takahashi[120]{次}

%\section{最近のDebianパッケージ作成環境 - git-buildpackage, sbuild, autopkgtest を例に -}{佐々木 洋平}
\takahashi[50]{最近のDebianパッケージ作成環境 \\ by 佐々木 洋平}

\takahashi[50]{そんな\\こんなで}
\takahashi[120]{次}

\section{今後の予定}
\begin{frame}[fragile]
  \frametitle{今後の予定}

  \begin{block}{関西Debian勉強会}
    \begin{itemize}
	\item {135 回: 5 月 27 日 福島区民センター 304 号室} 
	\item {136 回: 6 月 24 日 福島区民センター 304 号室} 
	\item {137 回: 8 月 26 日 福島区民センター 304 号室} 
	\item {138 回: 9 月 23 日 福島区民センター 304 号室} 
    \end{itemize}
  \end{block}

\end{frame}

\takahashi[50]{  }

\end{document}
%%% Local Variables:
%%% mode: japanese-latex
%%% TeX-master: t
%%% End:
