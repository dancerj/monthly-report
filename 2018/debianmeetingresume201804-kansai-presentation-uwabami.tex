\documentclass[cjk,dvipdfmx,14pt,compress,%
hyperref={bookmarks=true,bookmarksnumbered=true,bookmarksopen=false,%
colorlinks=false,%
pdftitle={第 134 回 関西 Debian 勉強会},%
pdfauthor={佐々木洋平},%
%pdfinstitute={関西 Debian 勉強会},%
pdfsubject={資料},%
}]{beamer}

\title{最近のDebianパッケージ作成環境}
\subtitle{$\sim$第 134 回 関西 Debian 勉強会$\sim$}
\author[佐々木洋平]{{\large\textbf{佐々木洋平}}\newline\texttt{uwabami@debian.or.jp}}
%\email{uwabami@gfd-dennou.org}
\institute[Debian JP]{{\normalsize\texttt{関西 Debian 勉強会}}}
\date{{\small{2018 年 4 月 22 日}}}

\usepackage{amsmath}
\usepackage{amssymb}
\usepackage{graphicx}
\usepackage{moreverb}
\usepackage{ulem}
\renewcommand{\familydefault}{\gtdefault}
\renewcommand{\kanjifamilydefault}{\gtdefault}

\AtBeginDvi{\special{pdf:tounicode EUC-UCS2}}
\usetheme{KansaiDebian}
\begin{document}
\begin{frame}{事前準備}
  以下をインストールしておいて下さい。
  \begin{enumerate}
  \item \texttt{git-buildpackage}
  \item \texttt{sbuild}
  \item \texttt{lintian}, \texttt{piuparts}
  \item \texttt{autopkgtest}, \texttt{debci}
  \end{enumerate}
\end{frame}

\begin{frame}
  \titlepage
\end{frame}

\begin{frame}[fragile]
  \frametitle{おことわり}
  \begin{itemize}
  \item 疑問、質問、ツッコミ、茶々、\alert{大歓迎}
  \item その場でインタラクティブにどうぞ
  \item ハッシュタグ \texttt{\#kansaidebian}
  \end{itemize}
\end{frame}

\takahashi[50]{今日のお題}

\begin{frame}[fragile]
  \frametitle{今日のお題}
  \begin{itemize}
  \item モダンなパッケージ作成環境構築について
    \begin{enumerate}
    \item \texttt{git-buildpackage}
    \item \texttt{sbuild}, \texttt{cowbuilder}
    \item \texttt{lintian}, \texttt{piuparts}
    \item \texttt{autopkgtest}, \texttt{debci}
    \end{enumerate}
  \item \sout{この間自分で仕込んでしまった}バグをその場で修正しアップロード。
  \item その際の作業をライブで見せつつ、皆からツッコミを受ける(受けたい)。
  \end{itemize}
\end{frame}

\begin{frame}
  \frametitle{\texttt{git-buildpackage}}
  \begin{itemize}
  \item %
    Debian パッケージを Git で管理する/
    管理されてるリポジトリを良い具合に扱うツール
  \item %
    インストールしていない人はインストールしましょう。
  \item %
    Debian パッケージ概観
    \begin{itemize}
    \item gbp の時のブランチ構成
    \item patch 管理: quilt, gbp pq
    \item \texttt{packaging-tutorial}: \\チュートリアルドキュメント(PDF)が入手できます。
    \end{itemize}
  \end{itemize}
\end{frame}

\begin{frame}[fragile]
  \frametitle{\texttt{git-buildpackage}: おすすめ}
  わたしはこんな alias, zsh function を仕込んでいます:
  \footnotesize
\begin{verbatim}
git-b:
gbp buildpackage \
 --git-ignore-new \
 --git-builder='debuild -rfakeroot -i.git -I.git -sa -k$GPG_KEY_ID'

git-bs () {
  [ -z $ARCH ] && ARCH=amd64
  [ -z $DIST ] && DIST=unstable
  [ x"$DIST" = x"stretch" ] && DIST=stable
  [ x"$DIST" = x"buster" ] && DIST=testing
  [ x"$DIST" = x"sid" ] && DIST=unstable
  gbp buildpackage \
     --git-ignore-new \
     --git-builder="sbuild --arch=${ARCH} -d ${DIST}"
}
\end{verbatim}
\end{frame}

\begin{frame}
  \frametitle{\texttt{sbuild}}
  \begin{itemize}
  \item パッケージのクリーンルームビルド環境
    \begin{itemize}
    \item %
      最近 \texttt{sbuild-debian-developer-setup} という
      お手軽セットアップパッケージもある。
    \end{itemize}
  \item 同様のソフトウェア:\\
    \texttt{pbuilder, cowbuilder, qemubuilder}
  \end{itemize}
\end{frame}

\begin{frame}
  \frametitle{\texttt{sbuild}の設定: \texttt{\$HOME/.sbuildrc}}
  長いので、以下をご覧下さい。
  \begin{center}
    {\small{\texttt{https://uwabami.github.io/cc-env/DebianPackaging.html}}}
  \end{center}
  \begin{itemize}
  \item \texttt{lintian}
    \begin{itemize}
    \item %
      パッケージ情報の lint
    \end{itemize}
  \item \texttt{piuparts}
    \begin{itemize}
    \item %
      パッケージのインストール、削除のチェッカー
    \item %
      コマンドラインオプションが色々アレなので、注意
    \end{itemize}
  \end{itemize}
\end{frame}

\begin{frame}
  \frametitle{rdepends の autopkgtest: CI}
  \begin{itemize}
  \item \texttt{debci}: Debian CI
    \begin{itemize}
    \item 詳細は \texttt{ci.debian.net} 参照
    \item 個人で使うには、ちょっと大仰. \\
      ⇒ \texttt{sbuild} の post-hook で呼び出すようにしてみた\\
      {\footnotesize{\texttt{https://uwabami.github.io/cc-env/DebianPackaging.html}}}
    \item \texttt{salsa.debian.org} の CI で呼び出すようにしても良いかも。
    \end{itemize}
  \end{itemize}
\end{frame}

\takahashi[50]{何かご質問などありますか?}

\takahashi[50]{ありがとう\\ございました!}

\end{document}
%%% Local Variables:
%%% mode: japanese-latex
%%% TeX-master: t
%%% End:
