%; whizzy paragraph -pdf xpdf -latex ./whizzypdfptex.sh
%; whizzy-paragraph "^\\\\begin{frame}"
% latex beamer presentation.
% platex, latex-beamer でコンパイルすることを想定。 

%     Tokyo Debian Meeting resources
%     Copyright (C) 2009 Junichi Uekawa
%     Copyright (C) 2009 Nobuhiro Iwamatsu

%     This program is free software; you can redistribute it and/or modify
%     it under the terms of the GNU General Public License as published by
%     the Free Software Foundation; either version 2 of the License, or
%     (at your option) any later version.

%     This program is distributed in the hope that it will be useful,
%     but WITHOUT ANY WARRANTY; without even the implied warreanty of
%     MERCHANTABILITY or FITNESS FOR A PARTICULAR PURPOSE.  See the
%     GNU General Public License for more details.

%     You should have received a copy of the GNU General Public License
%     along with this program; if not, write to the Free Software
%     Foundation, Inc., 51 Franklin St, Fifth Floor, Boston, MA  02110-1301 USA

\documentclass[cjk,dvipdfmx,12pt]{beamer}
\usetheme{Tokyo}
\usepackage{monthlypresentation}

%  preview (shell-command (concat "evince " (replace-regexp-in-string
%  "tex$" "pdf"(buffer-file-name)) "&")) 
%  presentation (shell-command (concat "xpdf -fullscreen " (replace-regexp-in-string "tex$" "pdf"(buffer-file-name)) "&"))
%  presentation (shell-command (concat "evince " (replace-regexp-in-string "tex$" "pdf"(buffer-file-name)) "&"))

%http://www.naney.org/diki/dk/hyperref.html
%日本語EUC系環境の時
\AtBeginDvi{\special{pdf:tounicode EUC-UCS2}}
%シフトJIS系環境の時
%\AtBeginDvi{\special{pdf:tounicode 90ms-RKSJ-UCS2}}

\title{Debian で Atom、\\Debian で Visual Studio Code}
\subtitle{Debian / Ubuntu ユーザーミートアップ in 札幌 2018.07}
\author{吉野 与志仁}
\date{2018年7月6日}
\logo{\includegraphics[width=8cm]{image200607/openlogo-light.eps}}

\begin{document}

\frame{\titlepage{}}

%\section{}

\begin{frame}{自己紹介}

\begin{itemize}
 \item 吉野 与志仁(よしの よしひと)
 \item 東京のほうから来ました
 \item @yy\_y\_ja\_jp
 \item Debian 公式開発者ではないです
 \item manpages-ja パッケージのメンテナ
 \item Debian JP Project メンバー
\end{itemize} 
\end{frame}


\begin{frame}{Atom}
\url{https://atom.io/}
\end{frame}

\begin{frame}{Visual Studio Code}
\url{https://code.visualstudio.com/}
\end{frame}

\begin{frame}{Snappy}
Canonical が元々は Ubuntu 向けに開発したツール

Debianでも
\\
{\texttt{
sudo apt install snapd
}}
\\
で使うことはできるようです
\pause

\begin{itemize}[<+->]
 \item {\texttt{
sudo snap install --classic atom
}}
\\
{\scriptsize{\url{https://blog.ubuntu.com/2017/05/11/atom-is-now-available-as-a-snap-for-ubuntu}}}
\\
ただ、このSnapパッケージだと現状日本語入力できないようです
{\scriptsize{\url{http://gihyo.jp/admin/serial/01/ubuntu-recipe/0515?page=2}}}
 \item {\texttt{
sudo snap install --classic vscode
}}
\\
{\scriptsize{\url{https://blog.ubuntu.com/2017/05/19/visual-studio-code-is-now-available-as-a-snap-on-ubuntu}}}

\end{itemize}
\end{frame}

\begin{frame}{APT}
\begin{itemize}[<+->]
 \item {\texttt{
sudo apt install atom
}}
\\
\pause
今のところできません

 \item {\texttt{
sudo apt install vscode
}}
\\
\pause
今のところできません
\end{itemize}
\pause
⇒ できるようになってほしい!
\end{frame}

\begin{frame}{APT}
 Debianはすべてボランティア

 \begin{itemize}
  \item できるようにしたい(ITP)
  \item できるようになってほしい(RFP)
 \end{itemize}
\pause 
\begin{itemize}
 \item \url{https://bugs.debian.org/747824} \\
       \texttt{ITP: atom -- hackable editor}
 \item \url{https://bugs.debian.org/898259} \\
       \texttt{RFP: vscode -- Microsoft Visual Studio Code}
\end{itemize}
\end{frame}

\begin{frame}{APT line}
\texttt{/etc/apt/sources.list} ファイル

Debian 公式パッケージのレポジトリ
\texttt{deb http://ftp.jp.debian.org/debian/ stretch main contrib non-free}

\pause
\begin{itemize}[<+->]
 \item main -- Debian
 \item contrib, non-free -- 正確には Debian ではない
\end{itemize}

\pause
⇒ main に入ってほしい
\end{frame}

\begin{frame}{main}
Debian フリーソフトウェアガイドライン (DFSG) に適合するパッケージのみが
 入っている

\url{https://www.debian.org/social_contract.ja.html\#guidelines}

\pause
\begin{itemize}[<+->]
 \item だれでも使える
 \item パッケージの「すべての」ソースコードを見られる{\footnotesize{(改造もできる)}}
       \begin{itemize}[<+->]
	\item そのアプリ本体のソースコード
	\item そのアプリが使っているライブラリのソースコード
	\item そのアプリをビルドするためのツール(コンパイラ、ビルド
	      支援ツール)のソースコード
       \end{itemize}
\end{itemize}

\pause
⇒ main に入れるには「すべての」ソースコードを手に入れて、それを使ってビ
 ルドする必要がある。Debian公式レポジトリ以外からダウンロードしながらビルドしてはいけ
 ない
\end{frame}

\begin{frame}{Atom}
\begin{itemize}
 \item GitHub - atom/atom: The hackable text editor \url{https://github.com/atom/atom} \\
 ソースコード(MITライセンス) -- DFSGに適合
 \item \url{https://atom.io/} \\
       GitHub レポジトリの releases にある自動ビルド deb と同じものをダ
       ウンロードできる模様
\end{itemize}

 \pause
 特にライブラリとしてNode.jsモジュールElectronを使っている
 \url{https://electronjs.org/}

 \pause
 DebianにElectronパッケージがあるなら(Debian内にソースコードがあるとい
 うことなので)それを使えばよい -- まだありません

\pause
 ⇒ Electronのソースコードも手に入れて、それを使ってビルドする必要がある
\end{frame}

\begin{frame}{Visual Studio Code}
\begin{itemize}[<+->]
 \item Visual Studio Code - Open Source
       \url{https://github.com/microsoft/vscode}\\
       ソースコード(MITライセンス) -- これ自体はDFSGに適合
 \item Microsoft Visual Studio Code \url{https://code.visualstudio.com/}\\
       ソースコードを Microsoft がカスタマイズしてビルドした deb ファ
       イルをダウンロードできる。Microsoft がカスタマイズした部分のソー
       スコードがない -- 「Microsoft による利用状況データ収集」機能など
\end{itemize}

 \pause
 Atomと同様Electronを使っているので、Electronのソースコードも手に入れて、
 それを使ってビルドする必要がある
\end{frame}

  \begin{frame}{パッケージ化}
Debianには、Debianパッケージの形で入れることができる
\begin{itemize}
 \item    ふつうはライブラリごとに別パッケージにする。
   例: electronパッケージを作ってからそれを使ってatomパッケージやvscode
   パッケージを作る
 \item Node.js関連のライブラリ・アプリのDebianパッケージ化をしたい人が集まっ
   ている Debian JavaScript Maintainers というチームがある
   \url{https://wiki.debian.org/Javascript}
 \item    Electronなどのパッケージ化作業の状況なども見られる
   \url{https://wiki.debian.org/Javascript/Nodejs/Tasks}
\end{itemize}

  \end{frame}


\section{まとめ}
\begin{frame}[containsverbatim]{まとめ}
\begin{itemize}
 \item Debian で Atom や Visual Studio Code をアプリのサイトからダウンロー
       ドして動かすことはできる
 \item Snappyでインストールする方法もあるが現時点ではうまく動かないとこ
       ろがあるらしい
 \item Debian 公式レポジトリに atom パッケージや vscode パッケージはまだない。作るに
       は、ビルドツールや使っているライブラリすべてが Debian 公式パッケージになってい
       る必要がある
 \item Debian はボランティア
 \item Debian main からAPTでインストールできる == そのアプリのすべてのソー
       スコードを見られる
 \item 必要なパッケージを少しずつ Debian に入れている人々がいる
\end{itemize}
\end{frame}


\end{document}

;;; Local Variables: ***
;;; outline-regexp: "\\([ 	]*\\\\\\(documentstyle\\|documentclass\\|emtext\\|section\\|begin{frame}\\)\\*?[ 	]*[[{]\\|[]+\\)" ***
;;; End: ***
