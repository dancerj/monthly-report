%; whizzy document
% latex beamer presentation.
% platex, latex-beamer でコンパイルすることを想定。 

%     Tokyo Debian Meeting resources
%     Copyright (C) 2006 Junichi Uekawa

%     This program is free software; you can redistribute it and/or modify
%     it under the terms of the GNU General Public License as published by
%     the Free Software Foundation; either version 2 of the License, or
%     (at your option) any later version.

%     This program is distributed in the hope that it will be useful,
%     but WITHOUT ANY WARRANTY; without even the implied warranty of
%     MERCHANTABILITY or FITNESS FOR A PARTICULAR PURPOSE.  See the
%     GNU General Public License for more details.

%     You should have received a copy of the GNU General Public License
%     along with this program; if not, write to the Free Software
%     Foundation, Inc., 51 Franklin St, Fifth Floor, Boston, MA  02110-1301 USA

\documentclass[cjk,dvipdfmx]{beamer}
\usetheme{Warsaw}
%http://www.naney.org/diki/dk/hyperref.html
%日本語EUC系環境の時
\AtBeginDvi{\special{pdf:tounicode EUC-UCS2}}
%シフトJIS系環境の時
%\AtBeginDvi{\special{pdf:tounicode 90ms-RKSJ-UCS2}}

\title{Debian勉強会プレゼン}
\author{上川}
\date{2005年10月29日}

% 三択問題用
\newcounter{santakucounter}
\newcommand{\santaku}[5]{%
\addtocounter{santakucounter}{1}
\frame{\frametitle{問題\arabic{santakucounter}. #1}
%問題\arabic{santakucounter}. #1
\begin{itemize}
\item □ A #2\\
\item □ B #3\\
\item □ C #4\\
\end{itemize}
}
\frame{\frametitle{問題\arabic{santakucounter}. #1}
%問題\arabic{santakucounter}. #1
\begin{itemize}
\item □ A #2\\
\item □ B #3\\
\item □ C #4\\
\end{itemize}
\vfill{}
#5
}
}


\begin{document}
\frame{\titlepage{}}
\section{DWNQ}
%% debianmeetingresume200510-kansai.texから以下コピー


\subsection{2005年37号}
2005年9月13日です。
%http://www.debian.org/News/weekly/2005/37/
 \santaku{バグトラッキングシステムの見栄えで最近かわったのは何か}
 {CSSを利用するようになった}{DHTMLになった}{XHTMLになった}{A}
 \santaku{Debian UKで問題になったのは何か}{メンバーが活動的でないこと}
 {UKの経済状況がよろしくないこと}{商用利用をしようとした場
 合のDebianという名前の商標の利用の許可をする基準が不明確だったこと}{C}
 \santaku{ソフトウェアを計測する、という論文で発表されたのは何か}{Debian
 sarge には2億3000万行のソースコードが含まれている}{Debian sargeの品質を
 計測した}{Debian sargeの利用しやすさを計測した}{A}
 \santaku{Joey Hess は testing に対して security 対応をすることを発表した。
 それに利用しているサーバはどれか}
 {secure-testing.debian.net}{security.debian.org}{security.debuan.org}{A}
 \santaku{/usr/docをいまだにつかっているパッケージ数はどれくらいか}{100}{200}{500}{C}%C
 \santaku{planet.debian.orgをメーリングリスト経由で配布しようという意見
 に対して出た反論は}{blogの内容は機密事項なので、メーリングリストで配布
 してほしくない}{メーリングリストとして配布するとサーバの負荷が高くなる}{blogの内容を永続的にメーリングリストのアーカイ
 ブとして保存されたくない}{C}%C
 \santaku{/usr/share/doc/パッケージ名/examples/ にあるファイルに実行権限
 をつけることについてはどうするべきか}{サンプルは実行できるものは
 実行権限をつけるべき}{サンプルなんてかざりなので実行しなくてよい}
 {/usr/share以下について実行権限をつけるのはこのましくなく、実行ファイルはbinにおくべきだ。}{C}
 \santaku{sponsors.debian.netが提供するサービスは何か}{金銭的寄付をつの
 るフィッシングサイト}{広告を配信し、広告収入をDebianプロジェクトの発展
 のために利用するサイト}{まだメンテナ
 になっていない人が管理しているパッケージについてスポンサーが必要な状況
 をトラッキングするシステム}{C}
 \santaku{1.0beta3のようなベータ版のバージョン番号が1.0のような最終版の
 バージョン番号より低い、とdpkgが判定してしまう。この状況に関してメンテ
 ナはどう対応するべきか}
 {優先度の低い チルダ記号 \~{ }を利用して、 1.0\~{
 }beta3のような名前にする。
ただまだアーカイブシステムが対応していないので、今後の改善が必要。}{あき
 らめる}{ベータ版はパッケージ化しない}{A}
 \santaku{ソースのみのパッケージのアップロードを可能にするという提案についての反論は
 何か}{バイナリが必要でなくなると、メンテナがテストをしなくなるのではな
 いだろうか、という懸念がある}{ソースのみだとパッケージインフラが破綻す
 る}{katieを改変するのが面倒}{A}
 \santaku{BTSに任意のタグを追加できる機能が追加された、なんという機能か}
 {tagtag}{たぐるんです}{usertag}{C}

\subsection{2005年38号}
2005年9月20日です。
%http://www.debian.org/News/weekly/2005/38
 \santaku{David Moreno Garzaがwnppにてcloseしたバグレポートの数は}{729の
 バグレポート}{100のバグレポート}{123のバグレポート}{A}%A
 \santaku{International Conference on Open Source Systemsに投稿された論
 文の中で説明されていた結果は}{開発者は短期間でどんどん入れ替わる}{メン
 テナは実は幻想で、そんな人は存在しない}{長いあいだアクティブに活動し、パッケー
 ジの数も多くメンテナンスする}{C} % c
 \santaku{Frank Lichtenheldが発表したのは、non-freeなドキュメントを削除
 する処理を開始するということだった。状況をトラッキングするために彼が利
 用したインフラは。}{BTSののusertags機能で
 debian-release@lists.debian.orgユーザのタグとして管理}{
 Wikiページ}{CVS管理のテキストファイル}{A}% A
 \santaku{Software freedom day 05でDebian-womenが行って、
 結果として良かったので今後も継続することになったのは}{debian-women-new
 IRCチャンネルがよい結果をもたらしたので、今後はdebian-womenチャンネルに
 新人を歓迎する時間帯というのをもうける}{CDをたくさん焼いたら人気だった}
 {KatieやBTSなどをインストールしてユーザがいじれるように提供したら人気だったので、今後もやる}{A}
 \santaku{init.dスクリプトは現在直列に実行されているが、今後
 、並列実行を実装する際に便利だろうと思われる
 LSB規格の仕様は}{なんとなく並列に実行しても壊れないようにする仕様}{
 気持の中だけでは並列な年頃}{initスクリプトの中で依存関係を記述できる仕様}{C}
 \santaku{新しいバージョンのパッケージにて問題が解決した場合の、バグレポー
 トをクローズする方法でないのは何か}{changelogでバグ番号を記述し
 アップロードする}
 {バージョンヘッダを付けて、リクエストを -done アドレスに投げる}{btsclose
 コマンドを利用する}{C} %c
 \santaku{Marc Brockschmidtが説明した、新規メンテナプロセスのFront Desk
 の変更とは}{
今後はより厳しい思想チェックを行う}{Debianに
 コントリビュートしていることが要件になり、何もしていない場合は、応募が
 取り消される}{年齢制限を設けます}{B}
 \santaku{security.debian.orgで問題になったのは何か}{セキュリティーアッ
 プデートが遅い}{セキュリティーアップデートが嘘だった}{xfree86のセキュリ
 ティーアップデートがあまりにも高いネットワーク負荷を発生させてしまい、
 security.debian.orgがサーバとして機能しなくなってしまった。}{C}
\subsection{2005年39号}
2005年9月27日です。
%http://www.debian.org/News/weekly/2005/39/
 \santaku{Ben HutchingがDebconfについて報告したのは}{もう終ってしまった
 事は忘れる}{忘れ物がありました}{DVDが入手可能に
 なった}{C}
 \santaku{wiki.debian.orgへの移行で特に手動の労力が必要だったのはどこか}
 {すでにwiki.debian.netからwiki.debian.orgに移行してしまっているページが
 いくつかあったのでそれに対しての手動の対処}{kwikiからmoinmoinへデータ形
 式の変更}{ドメイン名の登録}{A}%Aのつもり
 \santaku{initの時点では/がread-onlyでマウントされているが、その時点でデー
 タを保存するのにはどうしたらよいか。}{メモリファイルシステムを/runにマ
 ウントする}{/mnt以下にメモリファイルシステムをマウントする}{/をrwにマウ
 ントしなおす}{A}
 \santaku{グラフィックライブラリ GLU の実装がDebian内で複数ある理由はな
 ぜか}
 {一部のコードが一部のハー
 ドウェアでしか動かないという状況が続いているから}
 {複数のパッケージをメンテナンスしているほうがかっこいいから}{メンテナの
 仲が悪いから}{A}
 \santaku{Jeroen van Wolffelaarが提案したのは}{libc5を消す}{libc6を消す}
 {libc6.1を消す}{A}%A
 \santaku{piupartsであきらかになる問題は}{purgeする際に、essentialでは
 ないパッケージに依存して、動作しないパッケージ}{インストールしても動かないパッ
 ケージ}{使ってみて使いにくいパッケージ}{A}

\subsection{2005年40号}
2005年10月4日です。
%http://www.debian.org/News/weekly/2005/40/
 \santaku{DPLチームが今後検討する予定の問題について記録する媒体として選
 択したのは}{BTS}{IRC bot}{Wiki}{C}
 \santaku{tetex 3.0はどういう状況になっているか}{今後も入る見通しがない}
 {うごかなくて困っている}{experimentalにアッ
 プロードされ、ライブラリのフリーズが完了したらunstableに入る}{C}
 \santaku{Debian で配布するIA64アーキテクチャ向けのカーネルについて
 Dann FrazierがSMPじゃないカーネルのサポートを削除しようとした、何故か}
 {SMPじゃないといやだから}{時代はSMPです}{IA64で、SMPでないシステムがほ
 とんどなく、あまりテストされていない}{C}
% Chris LameterはSMPじゃないシス
% テムも仮想化環境用などで必要だろうと説明しているが、Debianがそれをすべ
% きかというと違うだろう。
 \santaku{Wolfgang Borgert によると
planet.debian.orgと、メーリングリストの利用方法の違いは}
{メーリングリストは古い技術なので今後はつかわなくなる}{blogはフレームされないで意見を述べることのできるメディアだが、議論す
るのはメーリングリストでして欲しい}{planet.debian.orgは安定していないの
で使わないで欲しい}{B}
 \santaku{pbuttonsdは/dev/input/eventXXを利用しているが、どういう問題が
 あったか}{makedevが、最大32あるうちの4個しかデバイスファイルをつくっていなかっ
 たので、/devを静的に管理しているユーザは一部の機能を利用できていなかっ
 た。}
 {USB接続ではうまく認識できなかった}{電源ボタンがおされたらアプリケーショ
 ンがハングした。}{A}

\subsection{2005年41号}
2005年10月11日です。
%http://www.debian.org/News/weekly/2005/41/
 \santaku{Debian securityで改善したのは}{バックエンドとフロントエンドの
 サーバを分割し、負荷に強い構成に変更した}{特定のユーザが負荷をかけられ
 ないようにスロットリングした}{セキュリティーパッチをリリースしないこと
 でサーバに負荷がかからないようにした}{A}
 \santaku{Carlos Parra Camargoが報告したのは何か}{Wikiが悪意をもったユー
 ザにより書き換えられていたので前のバージョンを復活させた}{Wikiがおもし
 ろくないので改善しよう}{Wikiサーバがダウンしている}{A}
 \santaku{mozilla 1.7.8に対してのセキュリティーアップデートはどういう形
 でリリースされたか}{1.7.10にバージョン1.7.8という名前をつけてリリースし
 た}{セキュリティーパッチをバックポートした}{セキュリティーホールのある
 機能を全てdisableにした}{A}
 \santaku{複数のchrootで同じユーザ情報を利用するのに利用できる方法でない
 のは}{FUSEのshadow etc}{LDAP}{rm /etc/passwd}{C}
 \santaku{ソースコードにローカルに適用したパッチをパッケージの
 アップグレード後も維持するためにはどうしたら一番楽か}
{自分でがんばる}{apt-srcを利用する}{パッチはあてない}{B}
 \santaku{Jurij Smakovがリリースした文書は何か}{Debian Users Handlebook:
 Debianユーザをどうあつかえばよいのか、が書いてある}{Debian Developers
 Handlebook: Debian Developerをどう扱えば良いか、が書いてある。
}{Debian Linux Kernel
 Handbook: Debianでカーネルがどうビルドされているのか、が書いてある}{C}
%http://kernel-handbook.alioth.debian.org/

\subsection{2005年42号}

% http://www.debian.org/News/weekly/2005/42/
% 10月18日

\santaku{Eliveって何?}
{電子的に生きること}
{enlightenmentベースのLiveCD}
{Eliseの新しいバージョン}
{B}

\santaku{m68kについてSteve Langasek が発表したのは}
{m68kを自分もつかいたい}
{m68kがDebianの移植版の中で一番素晴らしい}
{testingに入る条件として、m68kは無視することにした}
{C}

\santaku{etch向けのdebian-installerの状況はどうか}
{もう全アーキテクチャについてインストールできることは確認した}
{もうすでに完全に動いている}
{まだ一部のアーキテクチャではビルドできない}
{C \url{http://people.debian.org/~igloo/status.php?packages=debian-installer}}

\santaku{gnome1のパッケージがビルドできなくなったのはなぜか}
{古いから}
{libpng10が削除されたから}
{gnome2の時代がやっときたから}
{B}

\santaku{ Edd Dumbillが、sargeをインストールする際に、debian-installerを利用しているときに
ハードウェアの問題にあたったときに利用するように提案したのは}
{knoppixでハードウェア認識}
{ハードウェアを買い替える}
{Debianを使う事をあきらめる}
{A}

\santaku{Oldenburg のミーティングの結果として、
Debian security updateでBranden Robinsonが報告したのは}
{security.debian.orgのバックエンドサーバが冗長構成になった}
{security.debian.org サーバがDNSのラウンドロビンで3台存在している構成がとれるようになった}
{security.debian.orgがフィッシングサイトになった}
{B}

\santaku{ソフトウェアに含まれている画像のライセンスにCreative Commons
BY-SAライセンスを利用したものはGPLのパッケージに含める事ができるか}
{やめたほうがよい}
{可能}
{MJ Rayによると、不可能なため、そのような場合はMITライセンスを利用したほうがよい}
{C}



\santaku{Camm McGuireがlibbfdにリンクするのにはどうしたらよいのだ、と質
問したときのDaniel Jacobwitzの回答は}
{libbfdは安定しているのでいくらでもリンクしてくれ}
{よくバイナリレベルの互換性は破壊されるので、binutils-devにあるlibbfd.a
をつかってくれ}
{一般人はlibbfdは使うな}
{B}

\subsection{2005年43号}

% http://www.debian.org/News/weekly/2005/43/
% 10月25日
\santaku{Joerg JaspertがNEWのパッケージをREJECTする理由で多いと指摘したのは}
{読めないドキュメントが多い}
{おもしろくないパッケージが多い}
{debian/copyrightが不正確なものが多い}
{C}

\santaku{Steve Langasek が宣言したetchのリリーススケジュールによると、
etchがリリースされるのは}
{2005年12月}
{2006年6月}
{2006年12月}
{C}

\santaku{今月、Christian Perrierが宣言したかなり完成している、とコメント
していたdebian-installerの機能は何か}
{sidをインストールするインストーラ}
{etch向けのテキストモードのインストーラ}
{GTKを利用したグラフィカルインストーラ}
{C}

\santaku{ypbindなどが動的にポートを確保する場合、その後に起動する
サーバとポート番号がかぶる場合があるそれを回避する方法は}
{portreserve}
{祈る}
{nisなんて使わない}
{A}

\santaku{/etc/hostsに書いてある127.0.0.1のホスト名は現在のsidでは何にな
るか}
{ホスト名}
{localhost.localdomain}
{localhost}
{B}

\santaku{slang用のモジュールパッケージ名は今後'slang-モジュール名'という
形になりそうだが、従来はどういう名前だったか}
{slモジュール名}
{モジュール名-slang}
{モジュール名}
{A}

\santaku{pbuilderの開発体制にどういう変化があったか}
{名前が変わりました}
{チームメンテナンス制をとるために、aliothに移動した}
{おもしろくなくなってきたのでもうやめます}
{B}

\santaku{Daniel Ruosoが提案したDebianの移植版は}
{uclibc移植版}
{minix 3.0 移植版}
{z80移植版}
{A}

\santaku{curlについてopenssl版とgnutls版の両方を提供するようになった、その
理由は}
{GPLのプログラムがopensslとリンクしなくてすむように}
{二種類あったほうが楽しいから}
{GNUのほうがopensslより凄いから}
{A}

\end{document}
