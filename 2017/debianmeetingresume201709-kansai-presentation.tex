\documentclass[cjk,dvipdfmx,12pt,compress,%
hyperref={bookmarks=true,bookmarksnumbered=true,bookmarksopen=false,%
colorlinks=false,%
pdftitle={第 120 回 関西 Debian 勉強会},%
pdfauthor={倉敷・のがた・佐々木・かわだ・おおつき},%
%pdfinstitute={関西 Debian 勉強会},%
pdfsubject={資料},%
}]{beamer}

\title{第 126 回 関西 Debian 勉強会}
\subtitle{$\sim$発表資料$\sim$}
\author[おおつき]{{\large\bf 倉敷・のがた・佐々木・かわだ・おおつき}}
\institute[Debian JP]{{\normalsize\tt 関西 Debian 勉強会}}
\date{{\small 2017 年 08 月 27 日}}

%\usepackage{amsmath}
%\usepackage{amssymb}
\usepackage{graphicx}
\usepackage{moreverb}
\usepackage[varg]{txfonts}
\AtBeginDvi{\special{pdf:tounicode EUC-UCS2}}
\usetheme{KansaiDebian}
\def\museincludegraphics{%
  \begingroup
  \catcode`\|=0
  \catcode`\\=12
  \catcode`\#=12
  \includegraphics[width=0.9\textwidth]}
\renewcommand{\familydefault}{\sfdefault}
\renewcommand{\kanjifamilydefault}{\gtdefault}
\begin{document}
\begin{frame}
\titlepage
\end{frame}

\begin{frame}[fragile]
  \frametitle{Disclaimer}
  \begin{itemize}
  \item 疑問、質問、ツッコミ、茶々、\alert{大歓迎}
  \item その場でインタラクティブにどうぞ
  \item ハッシュタグ \#kansaidebian
  \end{itemize}
\end{frame}

\frametitle{Agenda}

\tableofcontents

\section{最近の Debian 関係のイベント}
\begin{frame}[fragile]
  \frametitle{第125回関西Debian勉強会@OSC2017Kyoto}
  \begin{itemize}
  \item 日時: 08月05日(日)
  \item 場所: 京都リサーチパーク
  \begin{block}{内容}
    \begin{itemize}
        \item{オープンソースカンファレンス 2017@Kyoto}
        \item{Debian updates by 佐々木 洋平}
    \end{itemize}
  \end{block}
\end{itemize}
\end{frame}

\section{最近の Debian Updates}

\begin{frame}[fragile]
  \frametitle{個人的に気になった devel mail}
  \begin{itemize}
    \item Aloith の将来
    \begin{itemize}
       \item Aloith が wheezy で動いている
       \item wheezy の EOL が 2018 年の 5 月
       \item Alioth の管理に結構手間がかかっている
       \item また Forge が管理されていない
       \item alioth が止まるとと、過去のメーリングリストが読めなくなる
    \end{itemize}
  \end{itemize}
\end{frame}

\takahashi[50]{そんな\\こんなで}
\takahashi[120]{次}

\section{事前課題}
\takahashi[50]{事前課題}

\begin{frame}[fragile]
  \frametitle{事前課題}
  \begin{block}{今回の事前課題}
   \begin{itemize}
    \item Lisp処理系をダウンロードし、プログラムの基本的な文法に触れておいてください (括弧や関数定義など)
    \item Debian をお使いの方は、common Lisp の開発環境 sbcl パッケージをお使いください。 
   \end{itemize}
  \end{block}
\end{frame}

\begin{frame}[fragile]
参加者は以下の皆様です。 (申し込み順)
\begin{itemize}
\item znz
\item t3rkwd
\item tomabu
\item YukiharuYABUKI
\item uwabami
\item ipv6waterstar
\item Katsuki Kobayashi
\item yosuke\_san
\item fu7mu4
\item gdevmjc 
\end{itemize}
\end{frame}

\takahashi[50]{突然ですが}
\takahashi[30]{アンケートを取りたいとおもいます。}
\takahashi[30]{Lisp の使用経験と \\ 最も使用頻度の高い言語を教えてください}
\takahashi[30]{注意: 自然言語は除く \\ 例: 日本語、英語、大阪弁、京ことば}

\begin{frame}[fragile]
  \frametitle{znz}
  \begin{block}{Lisp の使用経験 \& よく使う言語}
    \begin{itemize}
      \item 
      \item 
    \end{itemize}
  \end{block}
\end{frame}

\begin{frame}[fragile]
  \frametitle{t3rkwd}
  \begin{block}{Lisp の使用経験 \& よく使う言語}
    \begin{itemize}
      \item 
      \item 
    \end{itemize}
  \end{block}
\end{frame}

\begin{frame}[fragile]
  \frametitle{tomabu}
  \begin{block}{Lisp の使用経験 \& よく使う言語}
    \begin{itemize}
      \item 
      \item 
    \end{itemize}
  \end{block}
\end{frame}

\begin{frame}[fragile]
  \frametitle{YukiharuYABUKI}
  \begin{block}{Lisp の使用経験 \& よく使う言語}
    \begin{itemize}
      \item 
      \item 
    \end{itemize}
  \end{block}
\end{frame}

\begin{frame}[fragile]
  \frametitle{uwabami}
  \begin{block}{Lisp の使用経験 \& よく使う言語}
    \begin{itemize}
      \item 
      \item 
    \end{itemize}
  \end{block}
\end{frame}

\begin{frame}[fragile]
  \frametitle{ipv6waterstar}
  \begin{block}{Lisp の使用経験 \& よく使う言語}
    \begin{itemize}
      \item 
      \item 
    \end{itemize}
  \end{block}
\end{frame}

\begin{frame}[fragile]
  \frametitle{Katsuki Kobayashi}
  \begin{block}{Lisp の使用経験 \& よく使う言語}
    \begin{itemize}
      \item 
      \item 
    \end{itemize}
  \end{block}
\end{frame}

\begin{frame}[fragile]
  \frametitle{yosuke\_san}
  \begin{block}{Lisp の使用経験 \& よく使う言語}
    \begin{itemize}
      \item 経験は皆無 
      \item vim ユーザーなので、使う機会がない. 麻薬 (なれるとやめられない) とか, 変態 Lisp と聞いている. 
      \item ここ半年ぐらいは、Fortran, Python, Batch Script が 4:4:2 ぐらい
    \end{itemize}
  \end{block}
\end{frame}

\begin{frame}[fragile]
  \frametitle{fu7mu4}
  \begin{block}{Lisp の使用経験 \& よく使う言語}
    \begin{itemize}
      \item 
      \item 
    \end{itemize}
  \end{block}
\end{frame}

\begin{frame}[fragile]
  \frametitle{gdevmjc}
  \begin{block}{Lisp の使用経験 \& よく使う言語}
    \begin{itemize}
      \item 
      \item 
    \end{itemize}
  \end{block}
\end{frame}

\takahashi[50]{そんな\\こんなで}
\takahashi[120]{次}

\takahashi[30]{ Debian で Lisp を動かす \\ by 油谷さん}

\takahashi[50]{そんな\\こんなで}
\takahashi[120]{次}

\section{今後の予定}
\begin{frame}[fragile]
  \frametitle{今後の予定}
  \begin{block}{第128回関西Debian勉強会 @ KOF 2017}
    \begin{itemize}
    \item 日時 2017 年 11 月 12 日 (土) 11:00-18:00
    \item 会場: 大阪南港 ATC ITM棟 10F
    \end{itemize}
  \end{block}
\end{frame}

\end{document}
%%% Local Variables:
%%% mode: japanese-latex
%%% TeX-master: t
%%% End:
