\documentclass[cjk,dvipdfmx,12pt,compress,%
hyperref={bookmarks=true,bookmarksnumbered=true,bookmarksopen=false,%
colorlinks=false,%
pdftitle={第 120 回 関西 Debian 勉強会},%
pdfauthor={倉敷・のがた・佐々木・かわだ・おおつき},%
%pdfinstitute={関西 Debian 勉強会},%
pdfsubject={資料},%
}]{beamer}

\title{第 127 回 関西 Debian 勉強会}
\subtitle{$\sim$発表資料$\sim$}
\author[おおつき]{{\large\bf 倉敷・のがた・佐々木・かわだ・おおつき}}
\institute[Debian JP]{{\normalsize\tt 関西 Debian 勉強会}}
\date{{\small 2017 年 08 月 27 日}}

%\usepackage{amsmath}
%\usepackage{amssymb}
\usepackage{graphicx}
\usepackage{moreverb}
\usepackage[varg]{txfonts}
\AtBeginDvi{\special{pdf:tounicode EUC-UCS2}}
\usetheme{KansaiDebian}
\def\museincludegraphics{%
  \begingroup
  \catcode`\|=0
  \catcode`\\=12
  \catcode`\#=12
  \includegraphics[width=0.9\textwidth]}
\renewcommand{\familydefault}{\sfdefault}
\renewcommand{\kanjifamilydefault}{\gtdefault}
\begin{document}
\begin{frame}
\titlepage
\end{frame}

\begin{frame}[fragile]
  \frametitle{Disclaimer}
  \begin{itemize}
  \item 疑問、質問、ツッコミ、茶々、\alert{大歓迎}
  \item その場でインタラクティブにどうぞ
  \item ハッシュタグ \#kansaidebian
  \end{itemize}
\end{frame}

\frametitle{Agenda}

\tableofcontents

\section{最近の Debian 関係のイベント}
\begin{frame}[fragile]
  \frametitle{第129回関西Debian勉強会@KOF}
  \begin{itemize}
  \item 日時: 11月11日(土)
  \item 場所: 大阪南港 ITM 棟 10 階
  \begin{block}{内容}
    \begin{itemize}
     \item ブース展示
     \item「Debian Updates」 by 佐々木洋平
    \end{itemize}
  \end{block}
\end{itemize}
\end{frame}

\section{最近の Debian Updates}

\begin{frame}[fragile]
  \frametitle{個人的に気になった devel mail}
  \begin{itemize}
    \item 9.2
  \end{itemize}
\end{frame}

\takahashi[50]{そんな\\こんなで}
\takahashi[120]{次}

\section{事前課題}
\takahashi[50]{事前課題}

\begin{frame}[fragile]
    \begin{itemize}
	\item 以下の Debian Wiki に目を通しておいてください \\
		\url{https://wiki.debian.org/LXC} \\
		\url{https://wiki.debian.org/LXC/SimpleBridge}
    \item 以下のパッケージをインストールしておいてください。 \\
		lxc libvirt0 libpam-cgroup libpan-cgroup libpam-cgfs bridge-utils
	\item (推奨) 会場で利用できるネットワーク環境をご用意ください。
  \end{itemize}
\end{frame}

\section{参加者・アンケート}
\takahashi[50]{参加者・アンケート}

\begin{frame}[fragile]
  \frametitle{使用している仮想環境がありましたら教えてください}
  \begin{block}{回答}
	\begin{itemize}
	\item znz さん
	\item VirtualBox
	\end{itemize}
  \end{block}
\end{frame}

\begin{frame}[fragile]
  \frametitle{使用している仮想環境がありましたら教えてください}
  \begin{block}{回答}
	\begin{itemize}
	\item uwabami さん
	\item LXC, KVM, VirtualBox
	\end{itemize}
  \end{block}
\end{frame}

\begin{frame}[fragile]
  \frametitle{使用している仮想環境がありましたら教えてください}
  \begin{block}{回答}
	\begin{itemize}
	\item yosuke\_san
	\item LXC
	\end{itemize}
  \end{block}
\end{frame}

\begin{frame}[fragile]
  \frametitle{使用している仮想環境がありましたら教えてください}
  \begin{block}{回答}
	\begin{itemize}
	\item YukiharuYABUKI さん
	\item LXC, KVM
	\end{itemize}
  \end{block}
\end{frame}

\begin{frame}[fragile]
  \frametitle{使用している仮想環境がありましたら教えてください}
  \begin{block}{回答}
	\begin{itemize}
	\item oguraysu さん
	\item 
	\end{itemize}
  \end{block}
\end{frame}

\begin{frame}[fragile]
  \frametitle{使用している仮想環境がありましたら教えてください}
  \begin{block}{回答}
	\begin{itemize}
	\item jsynth21 さん
	\item VMWare、Eucalyptus(認証不具合で現在ほぼ休眠状態ですが・・・)
	\end{itemize}
  \end{block}
\end{frame}

\begin{frame}[fragile]
  \frametitle{使用している仮想環境がありましたら教えてください}
  \begin{block}{回答}
	\begin{itemize}
	\item murase\_syuka さん
	\item virtualbox qemu vagrant docker
	\end{itemize}
  \end{block}
\end{frame}

\begin{frame}[fragile]
  \frametitle{使用している仮想環境がありましたら教えてください}
  \begin{block}{回答}
	\begin{itemize}
	\item ipv6waterstar さん
	\item qemu-kvm, qemu
	\end{itemize}
  \end{block}
\end{frame}

\begin{frame}[fragile]
  \frametitle{使用している仮想環境がありましたら教えてください}
  \begin{block}{回答}
	\begin{itemize}
    \item t3rkwd さん
    \item VirtualBox, Hyper-V
	\end{itemize}
  \end{block}
\end{frame}

\takahashi[50]{そんな\\こんなで}
\takahashi[120]{次}

\takahashi[30]{ Debian Stretch で LXC を動かす \\ by Yosuke OTSUKI}

\begin{frame}[fragile]
  \frametitle{動機}
  \begin{block}{動機}
	\begin{itemize}
    \item 会社: Debian と同じようにクリーンなコンテナの中で、ビルドとテストがしたい
    \item 個人: sid の環境がほしい
	\end{itemize}
  \end{block}

  \begin{block}{コンテナの候補}
	\begin{itemize}
    \item Docker
    \item Linux
	\end{itemize}
  \end{block}
\end{frame}

\begin{frame}[fragile]
  \frametitle{ Docker の問題}
  \begin{block}{問題}
	\begin{itemize}
	\item  Docker の yum DB を調べたところ、本家の CentOS と異なる revision だった	
	\end{itemize}
  \end{block}

  \begin{block}{yum のレポジトリ}
	\begin{itemize}
	\item Docker : \url{https://yum.dockerproject.org/repo/main/centos/7/repodata/repomd.xml} 
	\item CentOS : \url{http://ftp.jaist.ac.jp/pub/Linux/CentOS/7/os/x86_64/repodata/repomd.xml}
	\end{itemize}
  \end{block}
\end{frame}

\begin{frame}[fragile]
  \frametitle{そこで LXC}
   \begin{block}{Linux Container (LXC)}
	\begin{itemize}
    \item Linux Kernel に組み込まれている軽量なコンテナ
    \item chroot みたいに使える  
    \end{itemize} 
  \end{block}

   \begin{block}{Stretch では}
	\begin{itemize}
    \item LXC 2.0  
	\item root 権限がなくてもコンテナを作成できるようになりました 
    \end{itemize} 
  \end{block}
\end{frame}

\begin{frame}[fragile]
  \frametitle{非特権 LXC コンテナを作る}
  \begin{block}{ホスト側: 必要なパッケージをインストールする}
   \begin{verbatim}
   # apt-get lxc libvirt0 libpam-cgroup 
             libpan-cgroup libpam-cgfs bridge-utils
   \end{verbatim}
  \end{block}
\end{frame}

\begin{frame}[fragile]
  \frametitle{ホスト側: ブリッジを作ります}
  \begin{block}{/etc/network/interfaces に bridge を追加}
   \begin{verbatim}
    iface lxcbr0 inet static
    bridge_ports eth0
    address 192.168.100.50/24
   \end{verbatim}
  \end{block}
\end{frame}

\begin{frame}[fragile]
  \frametitle{ホストが LXC に対応しているか確認する}
  \begin{block}{チェックするコマンド}
   \begin{verbatim}
   # lxc-checkconfig
   \end{verbatim}
   すべて enable と表示されれば OK
  \end{block}
\end{frame}

\begin{frame}[fragile]
  \frametitle{カーネルパラメタの変更}
  \begin{block}{非特権コンテナを許可するようにします}
  \begin{verbatim}
  sudo sh -c 'echo "kernel.unprivileged_userns_clone=1" >
                 /etc/sysctl.d/80-lxc-userns.conf'
  \end{verbatim}
  \end{block}
\end{frame}

\begin{frame}[fragile]
  \frametitle{sub*idsを設定する}
  \begin{block}{}
  \begin{verbatim}
# usermod --add-subuids 100000-65536 yosuke
# usermod --add-subgids 100000-65536 yosuke
  \end{verbatim}
  \end{block}
  \begin{block}{}
/etc/subgid と /etc/subuid を確認する
  \end{block}
\end{frame}

\begin{frame}[fragile]
  \frametitle{一般ユーザーが作成できるブリッジを制限する}
  \begin{block}{}
  \begin{verbatim}
echo "$USER veth lxcbr0 10"| sudo tee -i /etc/lxc/lxc-usernet
  \end{verbatim}
  \end{block}
\end{frame}

\begin{frame}[fragile]
  \frametitle{一般ユーザー用のコンテナ設定ファイルを作る}
  \begin{block}{}
  \begin{verbatim}
   /home/yosuke/.config/lxc/default.conf
  \end{verbatim}
  \end{block}

  \begin{block}{}
  \begin{verbatim}
   lxc.include = /etc/lxc/default.conf
   lxc.network.type = veth
   lxc.network.link = lxcbr0
   lxc.network.flags = up
   lxc.network.hwaddr = 00:16:3e:fe:3d:13
  \end{verbatim}
  \end{block}
\end{frame}
   
\begin{frame}[fragile]
  \frametitle{続き: 一般ユーザー用のコンテナ設定ファイルを作る}
  \begin{block}{}
  \begin{verbatim}
   lxc.id_map = u 0 100000 65536
   lxc.id_map = g 0 100000 65536
   
   lxc.mount.auto = proc:mixed sys:ro cgroup:mixed
   
   lxc.network.type = veth
   lxc.network.link = lxcbr0
   lxc.network.flags = up
   lxc.network.hwaddr = f1:53:7f:00:00:01
  \end{verbatim}
  \end{block}
\end{frame}

\begin{frame}[fragile]
  \frametitle{一般ユーザーでコンテナを作る}
  \begin{block}{}
  \begin{verbatim}
lxc-create -n stretch -t download -P ./
  \end{verbatim}
  \end{block}
\end{frame}

\begin{frame}[fragile]
  \frametitle{起動したか確認する}
  \begin{block}{}
  \begin{verbatim}
yosuke@asusx200c:~/work/lxc$ lxc-ls --fancy -P /home/yosuke/work/lxc/
NAME    STATE   AUTOSTART GROUPS IPV4                       IPV6 
stretch RUNNING 0         -      10.0.3.169, 192.168.100.51 -    
  \end{verbatim}
  \end{block}
  \begin{block}{}
  NIC が 2 つ立ち上がっているのは、ゲスト側が DHCP のままのため
  \end{block}
\end{frame}

\begin{frame}[fragile]
  \frametitle{コンテナにログインする}
  \begin{block}{コンテナに attach する}
  \begin{verbatim}
lxc-attach -n strech
  \end{verbatim}
  \end{block}
  \begin{block}{ブリッジ経由でログイン}
  \begin{verbatim}
ssh 192.168.100.50 
  \end{verbatim}
  \end{block}
\end{frame}

\begin{frame}[fragile]
  \frametitle{パッケージをアップグレードしようとしましたが}
  \begin{block}{失敗}
  \begin{verbatim}
root@stretch:/# apt-get update
Reading package lists... Done
W: chown to _apt:root of directory /var/lib/apt/lists/partial failed - SetupAPTPartialDirectory (1: Operation not permitted)
W: chmod 0700 of directory /var/lib/apt/lists/partial failed - SetupAPTPartialDirectory (1: Operation not permitted)
E: Could not open lock file /var/lib/apt/lists/lock - open (13: Permission denied)
E: Unable to lock directory /var/lib/apt/lists/
  \end{verbatim}
  \end{block}
 %  \begin{block}{Ubuntu で似たような報告がある}
 %  \url{https://github.com/lxc/lxd/issues/3310}
 % \end{block}
\end{frame}

\begin{frame}[fragile]
  \frametitle{まとめ}
  \begin{block}{}
  \begin{itemize}
   \item 非特権コンテナの作成
   \item 初見には辛い仕様がある
   \item 非特権コンテナの中でパッケージ更新できなかった問題は追跡中
  \end{itemize}
  \end{block}
\end{frame}

\takahashi[50]{そんな\\こんなで}
\takahashi[120]{次}

\takahashi[30]{ LXC \\ by 佐々木洋平}

\section{今後の予定}
\begin{frame}[fragile]
  \frametitle{今後の予定}
  \begin{block}{第128回関西Debian勉強会 @ KOF 2017}
    \begin{itemize}
    \item 日時 2017 年 12 月 24 日 (日) 13:30-17:00
    \item 会場: 福島区民センター
    \end{itemize}
  \end{block}
\end{frame}

\end{document}
%%% Local Variables:
%%% mode: japanese-latex
%%% TeX-master: t
%%% End:
