%; whizzy paragraph -pdf xpdf -latex ./whizzypdfptex.sh
%; whizzy-paragraph "^\\\\begin{frame}"
% latex beamer presentation.
% platex, latex-beamer でコンパイルすることを想定。 

%     Tokyo Debian Meeting resources
%     Copyright (C) 2009 Junichi Uekawa
%     Copyright (C) 2009 Nobuhiro Iwamatsu

%     This program is free software; you can redistribute it and/or modify
%     it under the terms of the GNU General Public License as published by
%     the Free Software Foundation; either version 2 of the License, or
%     (at your option) any later version.

%     This program is distributed in the hope that it will be useful,
%     but WITHOUT ANY WARRANTY; without even the implied warreanty of
%     MERCHANTABILITY or FITNESS FOR A PARTICULAR PURPOSE.  See the
%     GNU General Public License for more details.

%     You should have received a copy of the GNU General Public License
%     along with this program; if not, write to the Free Software
%     Foundation, Inc., 51 Franklin St, Fifth Floor, Boston, MA  02110-1301 USA

\documentclass[cjk,dvipdfmx,12pt]{beamer}
\usetheme{Tokyo}
\usepackage{monthlypresentation}

%  preview (shell-command (concat "evince " (replace-regexp-in-string
%  "tex$" "pdf"(buffer-file-name)) "&")) 
%  presentation (shell-command (concat "xpdf -fullscreen " (replace-regexp-in-string "tex$" "pdf"(buffer-file-name)) "&"))
%  presentation (shell-command (concat "evince " (replace-regexp-in-string "tex$" "pdf"(buffer-file-name)) "&"))

%http://www.naney.org/diki/dk/hyperref.html
%日本語EUC系環境の時
\AtBeginDvi{\special{pdf:tounicode EUC-UCS2}}
%シフトJIS系環境の時
%\AtBeginDvi{\special{pdf:tounicode 90ms-RKSJ-UCS2}}

\title{第147回東京エリアDebian勉強会 \\ \\debhepler 10の新機能を確認してみよう}
\subtitle{}
\author{Norimitsu Sugimoto (杉本 典充) \\dictoss@live.jp}
\date{2017-01-21}
\logo{\includegraphics[width=8cm]{image200607/openlogo-light.eps}}

\begin{document}

\frame{\titlepage{}}

%\section{}

\begin{frame}{アジェンダ}
  \begin{itemize}
  \item 自己紹介
  \item debhelperとは
  \item debhelper 10の変更点
    \begin{itemize}
    \item autoreconfの自動実行
    \item dh-systemdが不要
    \item parallelオプションの自動付与
    \item dh\_installdebの変更
    \end{itemize}
  \item まとめ
  \item 参考資料
  \end{itemize}
\end{frame}


\begin{frame}{自己紹介}
  \begin{itemize}
  \item Norimitsu Sugimoto (杉本 典充)
  \item dictoss@live.jp
  \item Twitter: @dictoss
  \item Debian使って10年以上、FreeBSD使って5年以上
  \item Debian GNU/kFreeBSDが気になっておりウォッチ中
  \item 仕事はソフトウェア開発者をやってます
  \item pythonとDjangoの組み合わせで使うことが多いです
  \end{itemize}
\end{frame}


\emtext{debhelper}

\begin{frame}[containsverbatim]{debhelperとは}
  \begin{itemize}
  \item debianパッケージを作成するための便利ツール
  \item dhで始まるコマンドが多い
  \item dh\_make、dchなどのdebianディレクトリ配下のファイルの生成、編集
  \item dh、dh\_clean、dh\_installなどパッケージのビルド処理を行う
  \item debian/rules中で、override\_dh\_cleanのようにオーバーライドすることができる
  \end{itemize}
\end{frame}

\emtext{debhelper 10の使い方例}

\begin{frame}[containsverbatim]{GNU helloのパッケージ化の例}
  \begin{commandline}
    $ vi ~/.bashrc
    export DEBEMAIL=''dictoss@live.jp''
    export DEBFULLNAME=''Norimitsu Sugimoto''
    
    $ sudo apt-get install dh-make build-essential
  \end{commandline}
\end{frame}

\begin{frame}[containsverbatim]{GNU helloのパッケージ化の例}
  \begin{commandline}
    $ sudo apt-get install autotools-dev
    $ wget https://ftp.gnu.org/gnu/hello/hello-2.10.tar.gz
    $ tar xf hello-2.10.tar.gz
    $ cd hello-2.10
    $ dh_make -f ../hello-2.10.tar.gz
    $ vi debian/changelog
    $ vi debian/control
    $ vi debian/rules
    $ dpkg-buildpackage -uc -us
  \end{commandline}
\end{frame}


\emtext{debhelper 10の変更点}

\begin{frame}[containsverbatim]{debian/compat}
  \begin{itemize}
  \item debhelperの互換性レベルを記述するファイル
  \item debhelper 10の場合は''10''と書きます
    \begin{commandline}
      $ cat debian/compat
      10
    \end{commandline}
  \end{itemize}
\end{frame}

\begin{frame}[containsverbatim]{autoreconfの自動実行}
  \begin{itemize}
  \item dhコマンドによるビルド処理の一連の流れで、autoreconfが実行されるようになった
    \begin{commandline}
      (省略)
      +   dh_autoreconf_clean
      dh_clean
      dpkg-source -b hello-2.10
      dpkg-source: info: using source format '3.0 (quilt)'
      @@ -64,6 +23,82 @@
      dh build
      dh_testdir
      dh_update_autotools_config
      +   dh_autoreconf
      (省略)
    \end{commandline}
  \end{itemize}
\end{frame}

\begin{frame}[containsverbatim]{Build-Dependsにdh-systemdが不要になった}
  \begin{itemize}
  \item dh-systemdパッケージというのがあります
  \item dh\_systemd\_enable、dh\_systemd\_start、systemd2initコマンドが入っている
  \item systemdによるデーモンのブート時起動設定や開始する処理を行うパッケージのビルド時に必要だった
  \item debian/compatが9以下の場合、dh-systemdが必要
    \begin{itemize}
      \item Build-Depends: debhelper, dh-systemd
    \end{itemize}
  \item debian/compat=10の場合、dh-systemdが不要
    \begin{itemize}
      \item Build-Depends: debhelper (>= 9.20160709)
    \end{itemize}    
  \end{itemize}
\end{frame}

\begin{frame}[containsverbatim]{parallelオプションの自動付与}
  \begin{itemize}
  \item "--parallel"オプションが自動的に付与されるようになった
  \item make -j4のようにビルドのジョブ数が調整される
  \item "--max-parallel"オプションで利用するコア数を指定可能
    \begin{itemize}
    \item --parallelなし時は1
    \item --parallelあり、--max-parallelなし時は論理CPUコア数
    \end{itemize}
  \item --no-parallelオプションもあります
  \end{itemize}
\end{frame}

\begin{frame}[containsverbatim]{parallelオプションの自動付与}
  \begin{itemize}
  \item 2コアHTありのCPUの場合、ジョブ数は4になります
  \end{itemize}
  \begin{commandline}
    $ diff -u hello_2.10-1_amd64.build.compat9 hello_2.10-1_amd64.build.compat10 | grep ``make -j''
    -make -j1
    +make -j4
    -make -j1 check VERBOSE=1
    +make -j4 check VERBOSE=1
    -make -j1 install DESTDIR=/home/norimitu/mkdeb/hello-2.10/debian/hello AM_UPDATE_INFO_DIR=no
    +make -j4 install DESTDIR=/home/norimitu/mkdeb/hello-2.10/debian/hello AM_UPDATE_INFO_DIR=no
  \end{commandline}
\end{frame}

\begin{frame}[containsverbatim]{dh\_installdebの変更}
  \begin{itemize}
  \item dh\_installdebはパッケージのインストール時の特別処理を記述する
  \item debian/package.maintscriptが変更
    \begin{itemize}
    \item パラメータがエスケープされるようになりました
    \end{itemize}
  \item debian/package.shlibsが変更
    \begin{itemize}
    \item debian/compatが9以下はdh\_installdebで実行
    \item debian/compatが10以降はdh\_makeshlibsで実行
    \end{itemize}
  \end{itemize}
\end{frame}

\begin{frame}[containsverbatim]{まとめ}
  \begin{itemize}
  \item debhelper 10の新機能と変更点を紹介しました
  \item stretchはdebhelper 10が含まれておりこれからの標準になるため対応しましょう
  \end{itemize}
\end{frame}

\begin{frame}{参考情報}
  \begin{itemize}
  \item 「debhelper 10 is now available」
    \begin{itemize}
      \item \url{https://nthykier.wordpress.com/2016/09/11/debhelper-10-is-now-available/}
    \end{itemize}
  \item man debhelper(7), dh(1), dh\_installdeb(1)
  \item Debian 新メンテナーガイド
    \begin{itemize}
      \item \url{https://www.debian.org/doc/manuals/maint-guide/}
    \end{itemize}
  \item 「dh-systemdはdebhelper 9.20160709で統合された」
    \begin{itemize}
      \item \url{http://qiita.com/henrich/items/e1651e3284c6b3d0d39e}
    \end{itemize}
  \end{itemize}
\end{frame}

\emtext{質疑応答}

\end{document}


;;; Local Variables: ***
;;; outline-regexp: "\\([ 	]*\\\\\\(documentstyle\\|documentclass\\|emtext\\|section\\|begin{frame}\\)\\*?[ 	]*[[{]\\|[]+\\)" ***
;;; End: ***
