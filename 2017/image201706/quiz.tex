%; whizzy-master ../debianmeetingresume201311.tex
% 以上の設定をしているため、このファイルで M-x whizzytex すると、whizzytexが利用できます。
%

\santaku
{ついにリリースされたDebian 9 (Stretch)。リリースノートに書かれているリリースされたパッケージ数はおよそどのくらいでしょうか。}
{40,000 パッケージくらい}
{50,000 パッケージくらい}
{60,000 パッケージくらい}
{B}
{リリースノートの「2.2. ディストリビューションの最新情報」(\url{https://www.debian.org/releases/stretch/amd64/release-notes/ch-whats-new.html})の記載によると、51687パッケージを超えるとのことです(補足:mainのバイナリパッケージの数です)。DD、DM、PMのみなさま、作業ありがとうございました。}
% $ grep '^Package: ' /var/lib/apt/lists/*\_stretch\_main\_binary-amd64* | wc -l
% 50840

\santaku
{まだ話は早いですが、これからは次のDebian 10に向かって作業を進めることになります。Debian 10のコードネームはなんでしょうか。}
{Potato}
{Bullseye}
{Buster}
{C}
{Debian 10のコードネームはBusterになります。Debianのリリース情報は「\url{https://wiki.debian.org/DebianReleases}」にまとまっていますのでご参照ください。}
