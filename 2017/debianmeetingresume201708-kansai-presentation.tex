\documentclass[cjk,dvipdfmx,12pt,compress,%
hyperref={bookmarks=true,bookmarksnumbered=true,bookmarksopen=false,%
colorlinks=false,%
pdftitle={第 120 回 関西 Debian 勉強会},%
pdfauthor={倉敷・のがた・佐々木・かわだ・おおつき},%
%pdfinstitute={関西 Debian 勉強会},%
pdfsubject={資料},%
}]{beamer}

\title{第 126 回 関西 Debian 勉強会}
\subtitle{$\sim$発表資料$\sim$}
\author[おおつき]{{\large\bf 倉敷・のがた・佐々木・かわだ・おおつき}}
\institute[Debian JP]{{\normalsize\tt 関西 Debian 勉強会}}
\date{{\small 2017 年 08 月 27 日}}

%\usepackage{amsmath}
%\usepackage{amssymb}
\usepackage{graphicx}
\usepackage{moreverb}
\usepackage[varg]{txfonts}
\AtBeginDvi{\special{pdf:tounicode EUC-UCS2}}
\usetheme{KansaiDebian}
\def\museincludegraphics{%
  \begingroup
  \catcode`\|=0
  \catcode`\\=12
  \catcode`\#=12
  \includegraphics[width=0.9\textwidth]}
\renewcommand{\familydefault}{\sfdefault}
\renewcommand{\kanjifamilydefault}{\gtdefault}
\begin{document}
\begin{frame}
\titlepage
\end{frame}

\begin{frame}[fragile]
  \frametitle{Disclaimer}
  \begin{itemize}
  \item 疑問、質問、ツッコミ、茶々、\alert{大歓迎}
  \item その場でインタラクティブにどうぞ
  \item ハッシュタグ \#kansaidebian
  \end{itemize}
\end{frame}

\frametitle{Agenda}

\tableofcontents

\section{最近の Debian 関係のイベント}

\begin{frame}[fragile]
  \frametitle{東京エリアDebian勉強会 出張}
  \begin{itemize}
    \item Debian / Ubuntu ユーザーミートアップ in 札幌 2017.07
    \begin{itemize}
        \item{LXCについて by 杉本さん}
        \item{Debian 9 Strech のネットワークインターフェイス名について by 吉野さん}
    \end{itemize}
    \item 2017年7月OSC 2017 Hokkaido 出張勉強会
    \begin{itemize}
    	\item{Debian Updates by 杉本典充}
    \end{itemize}
  \end{itemize}
\end{frame}

\section{最近の Debian Updates}
\begin{frame}[fragile]
  \frametitle{Debian 9.1 \& Debian 8.9  がリリースされました。}
  \begin{itemize}
    \item 7月22日 にリリース 
  \end{itemize}
\end{frame}

\begin{frame}[fragile]
  \frametitle{第125回関西Debian勉強会@OSC2017Kyoto}
  \begin{itemize}
  \item 日時: 08月05日(日)
  \item 場所: 京都リサーチパーク
  \begin{block}{内容}
    \begin{itemize}
        \item{オープンソースカンファレンス 2017@Kyoto}
        \item{Debian updates by 佐々木 洋平}
    \end{itemize}
  \end{block}
\end{itemize}
\end{frame}

%\begin{frame}[fragile]
%  \frametitle{関西 Debian 勉強会 上半期はイベント}
%  関西 Debian 勉強会 上半期はイベントが多め。 
%  \begin{itemize}
%	\item 1月 第118回関西Debian勉強会 + openSUSE Meetup + LILO\&東海道らぐLT大会
%	\item 2月 第119回関西Debian勉強会 Bug Squashing Party
%	\item 3月 第122回関西Debian勉強会
%	\item 4月 第120回関西Debian勉強会
%	\item 5月 第121回関西Debian勉強会 (10周年記念会)
%	\item 6月 第123回関西Debian勉強会
%	\item 7月 第124回関西Debian勉強会 Debian 9 "Stretch" リリースパーティ
%	\item 8月 第125回関西Debian勉強会@OSC2017Kyoto
%	\item 8月 第126回関西Debian勉強会
%    \end{itemize}
%\end{frame}

\begin{frame}[fragile]
  \frametitle{Debian Conference 2017}
  \begin{itemize}
    \item モントリオール、カナダ
    \item 8月6日 - 8月12日 
  \end{itemize}

録画されたセッションは、以下からご覧に慣れます。\\
\url{http://meetings-archive.debian.net/pub/debian-meetings/2017/debconf17/}

また、 Debconf 2018 の日程が発表されました。7月29日 から 8月5日です。
会場は台湾の新竹市(Hsinchu)です。

\end{frame}

\takahashi[50]{そんな\\こんなで}
\takahashi[120]{次}

\section{事前課題}
\takahashi[50]{事前課題}

\begin{frame}[fragile]
  \frametitle{事前課題}
  \begin{block}{今回の事前課題}
   \begin{itemize}
     \item 「タスク」と「代替コマンド(alternatives)」について、『Debian管理者ハンドブック』の該当の項目に目を通しておいてください。
     \item Debian 管理者ハンドブックは\href{https://debian-handbook.info/browse/ja-JP/stable/}{https://debian-handbook.info/browse/ja-JP/stable/} です。
   \end{itemize}
  \end{block}
\end{frame}

\begin{frame}[fragile]
参加者は以下の皆様です。
\begin{itemize}
  \item ikunya
  \item ipv6waterstar
  \item YukiharuYABUKI
  \item yosuke\_san
  \item lurdan
  \item Say\_no
  \item matsuzawa
  \item t3rkwd
  \item murase\_syuka 
\end{itemize}
\end{frame}

\begin{frame}[fragile]
  \frametitle{よくわかっていないので、調べてきました}
  \begin{block}{alternativesとは}
   \begin{itemize}
     \item 同一システム上で、同じ名前や似たような機能を切り替えるための機能" [1] 
     \item editor とコマンドラインに打ち込んだときに起動するテキストエディタを設定できる 
     \item gcc や ruby のバージョン管理ができる
   \end{itemize}
  \end{block}
\end{frame}

\begin{frame}[fragile]
  \frametitle{editor を叩くとどうなるか?}
\begin{footnotesize}
    \begin{verbatim}
$ which editor
 /usr/bin/editor
$ ls -l /usr/bin/editor
 ... Jun 19 07:49 /usr/bin/editor -> /etc/alternatives/editor
    \end{verbatim}
\end{footnotesize}
  alternatives をさしていることがわかります。
\end{frame}

\begin{frame}[fragile]
  \frametitle{editorのalternatives設定を確認する}
\begin{tiny}
\begin{verbatim}
$ update-alternatives --config editor
There are 3 choices for the alternative editor (providing /usr/bin/editor)

  Selection    Path                Priority   Status
------------------------------------------------------------
* 0            /bin/nano            40        auto mode
  1            /bin/nano            40        manual mode
  2            /usr/bin/vim.basic   30        manual mode
  3            /usr/bin/vim.tiny    15        manual mode
\end{verbatim}
\end{tiny}
どうやら、nano が設定されているようです。
しかし、私は vim ユーザーなので vim が立ち上がって欲しい
\end{frame}

\begin{frame}[fragile]
  \frametitle{editorの設定を変更する}
\begin{tiny}
\begin{verbatim}
# update-alternatives --set editor /usr/bin/vim
update-alternatives: error: alternative /usr/bin/vim for editor not registered; not setting
\end{verbatim}
\end{tiny}
vim を手動選択しようとしましたが、怒られました。
登録されていないと言われていますね
\end{frame}

\begin{frame}[fragile]
  \frametitle{alternatives のリストを確認する}
\begin{footnotesize}
\begin{verbatim}
$ update-alternatives --list editor
/bin/nano
/usr/bin/vim.basic
/usr/bin/vim.tiny
\end{verbatim}
\end{footnotesize}
リストに /usr/bin/vim がないので、登録できなかったようです。
\end{frame}

\begin{frame}[fragile]
  \frametitle{alternatives のリストに追加する}
\begin{tiny}
\begin{verbatim}
# update-alternatives --install /usr/bin/editor editor /usr/bin/vim 50
update-alternatives: using /usr/bin/vim to provide /usr/bin/editor (editor) in auto mode
\end{verbatim}
\end{tiny}
自動選択できるように、優先度 50 で登録します。
\end{frame}

\begin{frame}[fragile]
  \frametitle{リストの再確認}
\begin{tiny}
\begin{verbatim}
  Selection    Path                Priority   Status
------------------------------------------------------------
* 0            /usr/bin/vim         50        auto mode
  1            /bin/nano            40        manual mode
  2            /usr/bin/vim         50        manual mode
  3            /usr/bin/vim.basic   30        manual mode
  4            /usr/bin/vim.tiny    15        manual mode
\end{verbatim}
\end{tiny}
\begin{itemize}
 \item 先程はなかった /usr/bin/vim が選択されていますね。
 \item 最も大きな優先度で設定したので、auto mode で選択されています。
\end{itemize}
\end{frame}

\begin{frame}[fragile]
  \frametitle{リストから除外する}
\begin{tiny}
\begin{verbatim}
# update-alternatives --remove editor /usr/bin/vim
update-alternatives: using /bin/nano to provide /usr/bin/editor (editor) in auto mode
\end{verbatim}
\end{tiny}
すでに /usr/bin/vim.basic が登録されているので、 /usr/bin/vim はいらないですね
それなので、alternatives のリストから削除します。
\end{frame}

\begin{frame}[fragile]
  \frametitle{どういう仕組みで動くのか 1}
   \begin{itemize}
     \item /etc/alternatives/config ファイルにシンボリックリンクが格納されている [2]
     \item alternatives の設定を参照し、対応するリンクをたどる 
   \end{itemize}
\begin{tiny}
\begin{verbatim}
$ ls -l /usr/bin/editor
 ... Jun 19 07:49 /usr/bin/editor -> /etc/alternatives/editor
\end{verbatim}
\end{tiny}
つまり
\begin{tiny}
\begin{verbatim}
 /usr/bin/editor -> /etc/alternatives/editor -> /bin/nano
                                             -> /usr/bin/vim
\end{verbatim}
\end{tiny}
\end{frame}

\begin{frame}[fragile]
  \frametitle{どういう仕組みで動くのか 2}
   \begin{itemize}
     \item alternatives は link group というもので管理されている [2]
     \item link group は master と slave で構成されます
     \begin{itemize}
       \item 例えば、/bin/nano が master で /usr/share/man/*/nano が slave
     \end{itemize}
   \end{itemize}
\begin{tiny}
\begin{verbatim}
$ update-alternatives --display editor 
editor - auto mode
  link best version is /bin/nano
  link currently points to /bin/nano
  link editor is /usr/bin/editor
           :
  slave editor.ja.1.gz is /usr/share/man/ja/man1/editor.1.gz
           :
\end{verbatim}
\end{tiny}
\end{frame}


\begin{frame}[fragile]
  \frametitle{まとめ}
   \begin{itemize}
     \item 同一システム上で、同じ名前や似たような機能を切り替えるための機能
     \item /etc/alternatives/config のシンボリックリンクを参照し、バイナリを探して実行する
     \item link group で依存関係を管理している
     \item update-alternatives コマンドで、alternatives の各種設定ができる
   \end{itemize}
\end{frame}

\begin{frame}[fragile]
  \frametitle{Reference}
   \begin{itemize}
     \item [1] \url{https://wiki.debian.org/DebianAlternatives} (accessed 2017/08/26) \\
     \item [2] man update-alternatives \\
     \item [3] \url{https://askubuntu.com/questions/492029/update-alternatives-install} (accessed 2017/08/26) \\
     \item [4] \url{http://d.hatena.ne.jp/ksmemo/20100525/p1} (accessed 2017/08/26)
   \end{itemize}
\end{frame}

\takahashi[50]{そんな\\こんなで}
\takahashi[120]{次}

\takahashi[50]{Debian Stretchのインプットメソッドの現状\\あわしろいくや}

\section{今後の予定}
\begin{frame}[fragile]
  \frametitle{今後の予定}
  \begin{block}{第127回関西Debian勉強会}
    \begin{itemize}
    \item 日時: 09月24日(日)
    \item 会場: 福島区民センター  305号室
    \end{itemize}
  \end{block}
\end{frame}

\takahashi[50]{  }

\end{document}
%%% Local Variables:
%%% mode: japanese-latex
%%% TeX-master: t
%%% End:
