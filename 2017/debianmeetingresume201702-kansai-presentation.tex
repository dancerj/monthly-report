\documentclass[cjk,dvipdfmx,12pt,compress,%
hyperref={bookmarks=true,bookmarksnumbered=true,bookmarksopen=false,%
colorlinks=false,%
pdftitle={第 120 回 関西 Debian 勉強会},%
pdfauthor={倉敷・のがた・佐々木・かわだ・おおつき},%
%pdfinstitute={関西 Debian 勉強会},%
pdfsubject={資料},%
}]{beamer}

\title{第 120 回 関西 Debian 勉強会}
\subtitle{$\sim$発表資料$\sim$}
\author[かわだ てつたろう]{{\large\bf 倉敷・のがた・佐々木・かわだ・おおつき}}
\institute[Debian JP]{{\normalsize\tt 関西 Debian 勉強会}}
\date{{\small 2017 年 02 月 26 日}}

%\usepackage{amsmath}
%\usepackage{amssymb}
\usepackage{graphicx}
\usepackage{moreverb}
\usepackage[varg]{txfonts}
\AtBeginDvi{\special{pdf:tounicode EUC-UCS2}}
\usetheme{KansaiDebian}
\def\museincludegraphics{%
  \begingroup
  \catcode`\|=0
  \catcode`\\=12
  \catcode`\#=12
  \includegraphics[width=0.9\textwidth]}
\renewcommand{\familydefault}{\sfdefault}
\renewcommand{\kanjifamilydefault}{\gtdefault}
\begin{document}
\begin{frame}
\titlepage
\end{frame}

\begin{frame}[fragile]
  \frametitle{Disclaimer}
  \begin{itemize}
  \item 疑問、質問、ツッコミ、茶々、\alert{大歓迎}
  \item その場でインタラクティブにどうぞ
  \item ハッシュタグ \#kansaidebian
  \end{itemize}
\end{frame}

\takahashi[50]{謝罪\\インフルでダウンしてすいませんでした。}
\takahashi[50]{インフル対策はしっかり}

\begin{frame}[fragile]
\frametitle{Agenda}

\tableofcontents

\end{frame}

\section{最近の Debian 関係のイベント}
\takahashi[40]{最近の Debian\\関係のイベント}

\begin{frame}[fragile]
  \frametitle{関西 Debian 勉強会 + openSUSE Meetup + LILO \& 東海道らぐLT大会}
  \begin{itemize}
  \item 日時: 01月29日(日)
  \item 場所: 大阪市、港区区民センター
  \begin{block}{内容}
    \begin{itemize}
        \item{Leap 42.2, 42.3とAsia Summitについて語る(仮題) / @ftake}
        \item{openSUSE Build Serviceへの愛を語る(仮題) / @ItSANgo}
        \item{LT大会 / LILO \& 東海道らぐ}
        \item{ライトニングトーク (tablet にDebian 入れてみた、ed25519 鍵を使ってみた) }
    \end{itemize}
  \end{block}
\end{itemize}
\end{frame}

%\subsection{第148回東京エリアDebian勉強会、2017年2月勉強会(Bug Squash Party for Debian 9 Stretch}
%
%日時: 2017/02/11
%場所: 朝日ネット
%
%\subsection{関西 Debian 勉強会 Bug Squashing Party}
%
%日時: 2017/02/15
%場所: 京都大学理学部三号館 108 号室

\begin{frame}[fragile]
  \frametitle{第144回東京エリアDebian勉強会}
  \begin{itemize}
  \item 日時: 02月11日(土)
  \item 場所: 株式会社朝日ネット
  \end{itemize}
\end{frame}

\begin{frame}[fragile]
  \frametitle{関西 Debian 勉強会 Bug Squashing Party}
  \begin{itemize}
  \item 日時: 02月12日(日)
  \item 場所: 京都大学理学部三号館
  \end{itemize}
\end{frame}

\begin{frame}[fragile]
  \frametitle{Debian Project}
  \begin{itemize}
  \item Debconf 18 の開催地が、台湾の新竹市になりました。
  \end{itemize}
\end{frame}

\takahashi[50]{そんな\\こんなで}
\takahashi[120]{次}

\section{事前課題}
\takahashi[50]{事前課題}

\begin{frame}[fragile]
  \frametitle{事前課題}
  \begin{block}{今回の事前課題}
   \begin{itemize}
   \item{生産性を高めるために、何か工夫をしていることがあれば教えてください}
  \end{itemize}
    ※ 主催者が connpass の設定で設定するのを忘れてました
  \end{block}
\end{frame}

\takahashi[50]{事前課題\\発表}

\begin{frame}
  \frametitle{ t3rkwd }
  \begin{itemize}
  \item
  \end{itemize}
\end{frame}

\begin{frame}
  \frametitle{ murase\_syuka }
  \begin{itemize}
  \item
  \end{itemize}
\end{frame}

\begin{frame}
  \frametitle{ uwabami }
  \begin{itemize}
  \item
  \end{itemize}
\end{frame}

\begin{frame}
  \frametitle{ shin }
  \begin{itemize}
  \item
  \end{itemize}
\end{frame}

\begin{frame}
  \frametitle{ ipv6waterstar }
  \begin{itemize}
  \item
  \end{itemize}
\end{frame}

\begin{frame}
  \frametitle{ oguraysu }
  \begin{itemize}
  \item
  \end{itemize}
\end{frame}

\begin{frame}
  \frametitle{ gdevmjc }
  \begin{itemize}
  \item
  \end{itemize}
\end{frame}

\begin{frame}
  \frametitle{ Katsuki Kobayashi }
  \begin{itemize}
  \item
  \end{itemize}
\end{frame}

\begin{frame}
  \frametitle{ yosuke\_san }
  \begin{itemize}
  \item 仕事で、Windows と Linux で CI (Buildと結合テスト) はじめました。Jenkins + Python で構築
  \item Windows でも Linux でもできる作業ならば、Linux でやる (早く片付くことが多い)
  \item 0 時には寝る
  \end{itemize}
\end{frame}

\takahashi[50]{そんな\\こんなで}
\takahashi[120]{次}

\section{ qmakeなQtアプリのdebを作ろうとして試行錯誤した話}
\takahashi[35]{ qmakeなQtアプリのdebを作ろうとして試行錯誤した話 \\ by 小林 克希}

\takahashi[50]{そんな\\こんなで}
\takahashi[120]{次}

\section{今後の予定}
\begin{frame}[fragile]
  \frametitle{今後の予定}

  \begin{block}{第121回関西Debian勉強会 10 周年記念回}
    \begin{itemize}
    \item 日時: 03月19日(日)
    \item 場所: 福島区民センター
    \end{itemize}
  \end{block}

  \begin{block}{第149回東京エリアDebian勉強会@OSC Tokyo/Spring}
    \begin{itemize}
    \item 日時: 03月10日(金), 11日(土)
    \item 場所: 明星大学
    \end{itemize}
  \end{block}

\end{frame}

\takahashi[50]{  }

\end{document}
%%% Local Variables:
%%% mode: japanese-latex
%%% TeX-master: t
%%% End:
