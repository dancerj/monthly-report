%; whizzy chapter
% -initex iniptex -latex platex -format platex -bibtex jbibtex -fmt fmt
% 以上 whizzytex を使用する場合の設定。

%     Kansai Debian Meeting resources
%     Copyright (C) 2007 Takaya Yamashita
%     Thank you for Tokyo Debian Meeting resources

%     This program is free software; you can redistribute it and/or modify
%     it under the terms of the GNU General Public License as published by
%     the Free Software Foundation; either version 2 of the License, or
%     (at your option) any later version.

%     This program is distributed in the hope that it will be useful,
%     but WITHOUT ANY WARRANTY; without even the implied warranty of
%     MERCHANTABILITY or FITNESS FOR A PARTICULAR PURPOSE.  See the
%     GNU General Public License for more details.

%     You should have received a copy of the GNU General Public License
%     along with this program; if not, write to the Free Software
%     Foundation, Inc., 51 Franklin St, Fifth Floor, Boston, MA  02110-1301 USA

%  preview (shell-command (concat "evince " (replace-regexp-in-string "tex$" "pdf"(buffer-file-name)) "&"))
% 画像ファイルを処理するためにはebbを利用してboundingboxを作成。
%(shell-command "cd image200708; ebb *.png")

%%ここからヘッダ開始。

\documentclass[mingoth,a4paper]{jsarticle}
\usepackage{kansaimonthlyreport}
\usepackage[dvipdfmx]{xy}
\usepackage{etex}
\usepackage{ulem}

% 日付を定義する、毎月変わります。
\newcommand{\debmtgyear}{2017}
\newcommand{\debmtgdate}{26}
\newcommand{\debmtgmonth}{02}
\newcommand{\debmtgnumber}{120}

\def\fixme#1{{\color{red}{#1}}}

\begin{document}

\begin{titlepage}

% 毎月変更する部分、本文の末尾も修正することをわすれずに

  第\debmtgnumber{}回 関西 Debian 勉強会資料

  \vspace{2cm}

  \begin{center}
    \includegraphics{image200802/kansaidebianlogo.png}
  \end{center}

  \begin{flushright}
    \hfill{}関西Debian勉強会担当者 佐々木・倉敷・のがた・かわだ・おおつき \\
    \hfill{}\debmtgyear{}年\debmtgmonth{}月\debmtgdate{}日
  \end{flushright}

  \thispagestyle{empty}
\end{titlepage}

\dancersection{Introduction}{Debian JP}

\vspace{1em}

関西Debian勉強会はDebian GNU/Linuxのさまざまなトピック
(新しいパッケージ、Debian特有の機能の仕組、Debian界隈で起こった出来事、
などなど)について話し合う会です。

目的として次の三つを考えています。
\begin{itemize}
\item MLや掲示板ではなく、直接顔を合わせる事での情報交換の促進
\item 定期的に集まれる場所
\item 資料の作成
\end{itemize}

それでは、楽しい一時をお過ごしください。

\newpage

\begin{minipage}[b]{0.2\hsize}
  {\rotatebox{90}{\fontsize{80}{80}
{\gt{関西 Debian 勉強会}}}}
\end{minipage}
\begin{minipage}[b]{0.8\hsize}
\hrule
\vspace{2mm}
\hrule
\setcounter{tocdepth}{1}
\tableofcontents
\vspace{2mm}
\hrule
\end{minipage}

\dancersection{最近のDebian関係のイベント報告}{Debian JP}

\subsection{関西 Debian 勉強会 + openSUSE Meetup + LILO \& 東海道らぐLT大会}

日時: 2017/01/29(日)
場所: 大阪市、港区区民センター

30 人近くの方が集まり盛大に開催されました。

\begin{itemize}
\item{Leap 42.2, 42.3とAsia Summitについて語る(仮題) / @ftake}
\item{openSUSE Build Serviceへの愛を語る(仮題) / @ItSANgo}
\item{LT大会 / LILO \& 東海道らぐ}
\item{ライトニングトーク \(tablet にDebian 入れてみた、ed25519 鍵を使ってみた\) }
\end{itemize}

\subsection{第148回東京エリアDebian勉強会、2017年2月勉強会(Bug Squash Party for Debian 9 Stretch)}

日時: 2017/02/11(土)
場所: 朝日ネット

\subsection{関西 Debian 勉強会 Bug Squashing Party}

日時: 2017/02/15(日)
場所: 京都大学理学部三号館 108 号室

\subsection{Debian Project}

\dancersection{事前課題}{関西Debian勉強会}

今回は以下の課題を出題しました。
(connpassに設定し忘れていました)

\begin{itemize}
\item{生産性を高めるために、何か工夫をしていることがあれば教えてください}
\end{itemize}

参加者の皆さんは以下の通りです:

\begin{prework}{ t3rkwd }
\end{prework}

\begin{prework}{murase\_syuka}
\end{prework}

\begin{prework}{ uwabami }
\end{prework}

\begin{prework}{ shin }
\end{prework}

\begin{prework}{ ipv6waterstar }
\end{prework}

\begin{prework}{ oguraysu }
\end{prework}

\begin{prework}{ gdevmjc }
\end{prework}

\begin{prework}{ Katsuki Kobayashi }
\end{prework}

\begin{prework}{ yosuke\_san }
\end{prework}

\dancersection{ qmakeなQtアプリのdebを作ろうとして試行錯誤した話」}{小林 克希}

\subsection{はじめに}

昨年の6月に開催された関西Debian勉強会
\footnote{\url{https://wiki.debian.org/KansaiDebianMeeting/20160626}}
に参加して、生産性向上のための手法の一つであるポモドーロ・テクニック
\footnote{\url{http://cirillocompany.de/pages/pomodoro-technique/}}
用の、Qtで書かれたtomighty
\footnote{\url{http://tomighty.org/}}
という名前のツールのパッケージングに挑戦してみました。

私の作業前の状態は、かなーり昔にAutotoolsで書かれたソフトの
debパッケージをつくったことはある、といった感じでした。
なので、実際のところ、このツールをひとまずdebにするところまでは
\textbf{そこまで}苦労しなかったのですが、
\begin{itemize}
 \item lintianの警告がいっぱい出る!!
 \item qmakeのプロジェクトではdebian/rulesはどう書くのがお作法?
 \item GitHub+TravisCIとかどう使うの……?
\end{itemize}
などの課題が発生したので、
ここでは、それらを解決するためにやってみたり調べた内容を書かせていただこ
うと思います。
なお、全ては解決に至っておりませんので、
ヒントとか色々頂ければうれしいなぁ、と思っております。
どうぞよろしくお願いします。

\subsection{tomighty}

\subsubsection{どんなツール?}

まずは今回パッケージングしたツール、その名も ``tomighty'' の紹介をします。
が、前述した通り、こちらはポモドーロ・テクニック用のツールですので、
まずはポモドーロ・テクニックについて説明しておきます。
このポモドーロ・テクニックは、生産性向上のための手法で、
ざっくりと以下の流れで作業を行なう事が推奨されています。
\begin{enumerate}
 \item 完了させたいタスクを選択して紙に書き出す
 \item ポモドーロ(タイマー)を25分にセットする
 \item タイマが鳴るまで集中する (やる事変えるのはOK)
 \item タイマが鳴ったら紙にチェックを入れる (ここまでの$25$分が$1$ポモドーロという単位)
 \item short-break($5$分程度?)を取る、
   もしくは$4$回ポモドーロを完了していたらlong-break($20\sim 30$分?)を取る
\end{enumerate}
この、''$1$ポモドーロ=$25$分'' というのがミソでして、
集中してやるのに(個人的な主観で)長すぎず、短かすぎず、ちょうど良いくらいの時間で、
後々そのタスクにかけて時間の単位としても使用できる、という感じの手法になっております。
とか書いてますが、私も触りしか知らないので、ボロが出ないこの辺りで止めておきますので、
興味が沸きましたら公式のドキュメント等を読んでいただければと思います。

で、tomightyはなにかというと、
ツールバーに表示されるタイプのポモドーロ専用タイマアプリで、
\begin{itemize}
 \item $1$ポモドーロの$25$分
 \item 休憩時間の$5$分(short)もしくは15分(long)
\end{itemize}
を計測できます。
ちなみに、ポモドーロ(pomodoro)というのはイタリア語でトマトの意味で、
手法の提唱者のFrancesco Cirillo氏が使っていたキッチンタイマーが
トマトの形だったからこの名前になったようです。
ちなみに、tomightyのアイコンもトマトですし、
名前もtomatoから来てるものと思われます。

\subsubsection{まずはビルドしてみる}

では、ひとまず普通にビルドしてみたいと思います。
tomightyのコードはGitHubにて公開されています
\footnote{\url{https://github.com/tomighty/tomighty}}
ので、適当なディレクトリで以下のようにコマンドを叩いて取得します。
\begin{commandline}
 % git clone -b develop https://github.com/tomighty/tomighty
 Cloning into 'tomighty'...
 remote: Counting objects: 4621, done.
 remote: Total 4621 (delta 0), reused 0 (delta 0), pack-reused 4621
 Receiving objects: 100% (4621/4621), 1.83 MiB | 378.00 KiB/s, done.
 Resolving deltas: 100% (2023/2023), done.
\end{commandline}
ここでは、cloneしつつdevelopブランチをcheckoutしてます。
developブランチを開発用として運用しているプロジェクトは
結構あるかと思いますが、このプロジェクトは
masterがJavaで書かれた全く別物で、
READMEにも
「masterのJavaコードは開発を中止していて、
現在developブランチにてQt5で書き直しているよ
\footnote{実は、developブランチの開発も止まっているという
哀しい現実が待ち構えています。}
」
となっております……。

コードの取得が完了したら、まずは以下の通りにして新規ディレクトリ中に
Makefileを作成します。
qmakeは、基本的にビルド用に作ったディレクトリ上に
.pro ファイルの設定からMakefileを生成してビルドを行なう形になるようです。
\begin{commandline}
 % cd tomighty
 % mkdir build
 % cd build
 % qmake -qt=qt5 ../src/tomighty.pro
 Info: creating stash file /tmp/tomighty/build/.qmake.stash
\end{commandline}
qmakeと同じように、ビルド用のディレクトリを掘ってMakefileを
生成するツールにはcmakeもあるそうで、
こちらもQtのアプリ用によく使われているようです。
qmakeの利点---と言いますか、cmakeとの使い分けは、
少し調べた感じだと Qt Creator
\footnote{\url{https://www.qt.io/ide/}}
と呼ばれるQtのIDEを使うかどうか、というのが一つ大きな理由としてありそうでした。
詳しくは調べきれてませんので、もっと他に何かあれば教えていただきたく。

ところで、qmakeにさらりと \verb|-qt=qt5| というオプションを使っていますが、
これは複数バージョンのQtを共存させるためにqtchooserという仕組みが
upstreamの時点から入っており、各コマンドを叩くときに
\verb|-qt| オプション、もしくは QT\_SELECT という環境変数で
バージョンを指定することができるようです。
各コマンドの名前のシンボリックリンクを作成するという、
組み込み系でよく使われるbusyboxみたいな
\footnote{私が組み込み畑の人間なのでこういう表現になってますが、
本当はもっと良い表現があるかもしれません……。}
仕組みのようです。
以下のコマンドを叩くとおもしろいかと。
 \begin{commandline}
% ls -l /usr/bin | grep qtchooser
  ...
lrwxrwxrwx 1 root   root           9 11月 11 00:29 qmake -> qtchooser*
  ...
-rwxr-xr-x 1 root   root       43728 11月 11 00:29 qtchooser*
  ...
 \end{commandline}

さて、ではmakeをしてみたのですが……、ベロベロとログを吐いた挙句、
以下のようにエラーが出ました……。
\begin{commandline}
 tomighty/build/test-runner/../core-tests//libtomighty-core-tests.so:
    `tmty::InMemoryPreferences::InMemoryPreferences(QObject*)' に対する定義されていない参照です
 tomighty/build/test-runner/../core-tests//libtomighty-core-tests.so:
    `tmty::FakeTimer::interval() const' に対する定義されていない参照です
\end{commandline}
「なんでかなぁ? こっちの環境の問題かなぁ?」と思いつつ、
以下のパッチを作成してビルドが通る事は確認しました。
\begin{commandline}
diff --git src/test-runner/test-runner.pro src/test-runner/test-runner.pro
index 44ac2b4..82bc4c8 100644
--- src/test-runner/test-runner.pro
+++ src/test-runner/test-runner.pro
@@ -29,2 +29,3 @@ win32:CONFIG(release, debug|release): LIBS += \
   -L$$OUT_PWD/../core-tests/release/ -ltomighty-core-tests \
+  -L$$OUT_PWD/../core-mock/release/ -ltomighty-core-mock \
   -L$$OUT_PWD/../ui-tests/release/ -ltomighty-ui-tests
@@ -35,2 +36,3 @@ else:win32:CONFIG(debug, debug|release): LIBS += \
   -L$$OUT_PWD/../core-tests/debug/ -ltomighty-core-tests \
+  -L$$OUT_PWD/../core-mock/debug/ -ltomighty-core-mock \
   -L$$OUT_PWD/../ui-tests/debug/ -ltomighty-ui-tests
@@ -41,2 +43,3 @@ else:unix: LIBS += \
   -L$$OUT_PWD/../core-tests/ -ltomighty-core-tests \
+  -L$$OUT_PWD/../core-mock/ -ltomighty-core-mock \
   -L$$OUT_PWD/../ui-tests/ -ltomighty-ui-tests
\end{commandline}

なお、コケたのはテスト用のビルドで、代わりに
\begin{commandline}
% make sub-app-all sub-core-all sub-ui-all
\end{commandline}
としてやれば、一応アプリ自体はビルドできます。
windowsのmingwとかだとエラーが出ないようです。
疑問を持ちつつも自分でパッチを書いてりしてdeb化の作業をすすめていたのですが、
よくよく見ると、githubのトップページのプロジェクトの説明欄に
(私にとって)残念な記載が……!!
\begin{quote}
 Old Tomighty repo. For the new repos, please refer to: tomighty-osx and tomighty-windows
\end{quote}
どうやら、今回Qtアプリとしてさわってはみたものの、
開発はObjective-Cで書かれたMac用とC\#で書かれたWindows用しか行なわれてい
ないようです。
\textbf{「``Fork me on GitHub''リボンでこのプロジェクトにリンクしといて、
それはないじゃーーーん」}
と思わなくもないのですが……。
ひとまず、deb化の練習には問題がない(?)ので、
次に進めていこうかと思います……。

\subsection{debの作成}

ようやくここからdebianパッケージ作成の開始です。
まずは \verb|dh_make| を実行するところからなのですが、
upstreamがGitHubでタグもまだ付けていてくれない場合はどうするのか…………
結局のところわかりませんでした。
gitのハッシュをバージョン名に入れているパッケージは数あれど、
パッケージによって結構表記が違うようで……。
どなたか教えてください……。

結局、今回は lua-torch-doc とか lua-torch-sundown パッケージ
(by Debian Science Maintainers) で使われている "0~(日付)-g(git hash)"
が一番しっくり来たので、以下のようにしました。
\texttt{0f673359} の部分がdeb化する前のupstreamの最終コミットのgit hash
といった感じです。
\begin{commandline}
% dh_make --createorig -p 'tomighty_0.0.0+git0f673359' -s -c apache
\end{commandline}

さて、ひとまずむりくりビルドだけ通したくて、
debian/rulesを以下のように記述してみました。
色々なターゲットを大胆に上書いています。
これで、一応ビルドは通ります……。
\begin{commandline}
#!/usr/bin/make -f

override_dh_auto_configure:
        mkdir build
        cd build; qmake -qt=qt5 ../src/tomighty.pro

override_dh_auto_build:
        cd build; make

override_dh_auto_clean:
        rm -rf build

override_dh_auto_install:
        cd build; make install INSTALL_ROOT=$(PWD)/debian/tomighty

%:
        dh $@
\end{commandline}

が、もちろん決め打ち部分が多すぎてかなりイケてないのは私でもわかります。
せっかくdhコマンドを使っているのにoverrideした上に、
普通にコマンドを叩いてしまっていますし……。
ということで、色々と調べていくうちに、
Common Debian Build System(CDBS)%
\footnote{\url{https://build-common.alioth.debian.org/cdbs-doc.html}}
に行きあたりました。
こちらは、debian/rulesの記述を色々とモジュール化して
再利用を狙ったものということで、ばっちりqmake用のmakeも存在してました。
で、CDBSを使って書き直したものが以下になります。
\begin{commandline}
QT_SELECT = qt5
DEB_QMAKE_ARGS = ../src/tomighty.pro

DEB_SRCDIR = $(CURDIR)/src
DEB_BUILDDIR = $(CURDIR)/build

cleanbuilddir::
        -rm -rf $(DEB_DESTDIR)

clean::
        -rm $(DEB_BUILDDIR)/files
        -rm $(CURDIR)/debian/tomighty.substvars

include /usr/share/cdbs/1/rules/debhelper.mk
include /usr/share/cdbs/1/class/qmake.mk
\end{commandline}
cleanまわりがまだ怪しいのですが、だいぶすっきり書けたのではないでしょうか。
%おそらく「今ならdhコマンドでいいのでは?」と言うツッコミもあるかと思いますが、
%残念ながらdhコマンドではqmakeは2017年2月現在addonとして存在しないようです。
%(kdeは違いそう?)
%
%\begin{commandline}
%% dh --list | column
%autoreconf              dkms                    perl-dbi                python3                 tex
%autotools-dev           dpatch                  perl-openssl            quilt                   xml-core
%bash-completion         gir                     pkgkde-symbolshelper    ruby
%build-stamp             gnome                   pypy                    scour
%buildinfo               kde                     python-support          sodeps
%cme-upgrade             kf5                     python2                 systemd
%\end{commandline}
…………と、そこまで書いてみから「もしかして……」と調べたら、
実はdhもしっかりqmakeに対応していました……。
ドキュメントは見当たらなかったのですが、
\verb|/usr/share/perl5/Debian/Debhelper/Buildsystem/qmake.pm|
を見て以下のように記載しました。

\begin{commandline}
#!/usr/bin/make -f

%:
	dh $@ --buildsystem=qmake --builddirectory=build --sourcedirectory=src

override_dh_auto_configure:
	(export QT_SELECT=qt5; dh_auto_configure -- $(CURDIR)/src/tomighty.pro)
\end{commandline}

overrideしちゃってるのが気になりますが、CDBSよりもさっくりと書けてしまい
ました……。
もし、このoverrideをもっと綺麗に消せる方法がありましたら、
教えていただければと思います(他力本願)。
ソースを眺めていると、ソースディレクトリから.proファイルを
拾ってくれそうなんですが、 \verb|--sourcedirectory| オプション付けても
うまく拾ってくれませんでした……。

あと、 \verb|dh_auto_configure|に ``\verb|export QT_SELECT=qt5|''
が並んでいる件ですが、
これをやっておかないとpbuilder環境でqtchooserが彷徨ってしまって
以下のようなエラーになってしまいました……。
 \begin{commandline}
  qmake: could not find a Qt installation of ''
 \end{commandline}
qt5-defaultパッケージを入れると直るという情報があったり、
でもQtのメンテナがqtX-defaultにビルド依存してくれるなという
情報もあったり
\footnote{\url{http://stackoverflow.com/questions/16607003/qmake-could-not-find-a-qt-installation-of}}
で、ちょっと正解はわかりませんでした……。
なお、Qt4に関しては、 \verb|--buildsystem| オプションを \verb|qmake| から
\verb|qmake_qt4| にしてあげればよさそうです……。


\subsection{gbpを使う}

upstreamがgitで管理されている場合、
gbp (git-buildpackage)
\footnote{\url{https://honk.sigxcpu.org/piki/projects/git-buildpackage/}}
を使うのがよいらしいので、
ひとまず今回のパッケージングでもgbpを使ってみることにしました。
とはいいつつも、ごめんなさい、
今回gbpについては全然使いこなせておりません。

とりあえず、gbpの設定をしたいと思います。
今回、前述した通りupstreamのbranchはdevelopとなっています。
debを作成するにあったっては、通常はdebianという名前でbranchを切って、
その中で \verb|dh_make| でdebianディレクトリを作ってあげるのが
良いようです。
今回は、先にdevelop branchで \verb|dh_make| しちゃってましたが、
特にコミットはしていないので \verb|git checkout -b debian|
でよろしかったかと。

続けて、gbpの設定ファイルを作ります。
ユーザ毎の設定は \~{}/.gbp.conf に、
deb用のリポジトリ毎の設定なら、 debian/gbp.conf につくるのがよさそうです。
今回は、以下のようにupstreamとdebianのbranchを設定しました。

\begin{commandline}
[DEFAULT]
upstream-branch=develop
debian-branch=debian
\end{commandline}

この状態で、未コミットの物がある場合は \verb|--git-ignore-new|
オプションを付けつつ \texttt{buildpackage} を実行すると、
orig.tar.gzを含めて作成してくれました。

\subsection{travis.debian.netを使う}

%upstreamのリポジトリがGitHubなのでforkして使ってますが、
せっかくdebを作れるようになったので、できればCIツールを使いたいと思い、
GitHubならTravis CI
\footnote{\url{https://travis-ci.org/}}
が有名なのかなと調べてみたところ、travis.debian.net
\footnote{\url{http://travis.debian.net/}}
というのがあるようです。
こちらは、ビルド依存しているパッケージを含めたDebianのDockerイメージを作成して
ビルドのテストを行なってくれるようです。
なお、gbpでビルドできる事が前提のようです……。

設定は比較的簡単で、
\begin{enumerate}
 \item Travis CIでテストしたいリポジトリのビルドを有効にし、
       ``Build only if .travis.yml is present'' である事を確認しておく。
 \item リポジトリのルートディレクトリにビルド設定ファイル .travis.yml を
       置く。
 \item debian/source/optionsに
       \verb|extend-diff-ignore = "^\.travis\.yml$"'|
       と書いておく
\end{enumerate}
となっております。
設定ファイルの.travis.ymlの中身が問題なわけですが、
私の場合はひとまず以下のようになりました。
\begin{commandline}
sudo: required
language: c++

env:
  global:
    - TRAVIS_DEBIAN_GIT_BUILDPACKAGE="gbp buildpackage --git-upstream-tag=6afa215"
    - TRAVIS_DEBIAN_DISTRIBUTION="sid"
    - TRAVIS_DEBIAN_EXTRA_REPOSITORY_GPG_URL=""
    - TRAVIS_DEBIAN_EXTRA_REPOSITORY=""

services:
  - docker

script:
  - wget -O- http://travis.debian.net/script.sh | sh -

branches:
  except:
    - /^debian\/\d/
\end{commandline}
upstreamの指定がうまくできず、
ひとまず \verb|--git-upstream-tag| オプションにて
無理矢理コミットのハッシュ値つっこんでます……。
いずれ、良い感じに直したい……。

ymlのファイルを見て、
「``\verb@wget | sh -@'' て!!」
と思われた方もいらっしゃると思いますが、
それについてはtravis.debian.netのFAQでも一応触れられていて、
「自分のマシンなら、こんなコマンドは実行すべきではないけど、
使い捨てのコンテナの中で実行されて、
生成された.debも使われないから良いんだ」
的な感じでまとめられています。
ひとまず、ここまで設定すると、
GitHubにコミットしたタイミングでTravisCIでビルドが走るようになりました。

\subsection{まとめ}

ということで、若干中途半端ではありますが、
ひとまずはここまでの内容で当初の課題のうち、
\begin{itemize}
 \item qmakeのプロジェクト向けなdebian/rulesの記述
 \item GitHub+TravisCIでもビルドのチェック
\end{itemize}
はできるようになりました。
もちろん、改善すべき点はまだ色々と残っていそうですし、
まだ解決に至っていない問題も多く、今後は
\begin{itemize}
 \item gbpの使い方 (特にtravis.debian.netとの連携部分)
 \item lintianの警告の修正
\end{itemize}
あたりの確認をしていかねばと思ってます。
ちなみに、2つ目は、tomightyのupstreamがポモドーロ・テクニック用の
共有ライブラリを作るので、
その辺りがまだ調査できておらず色々と怒られてます……。

なんにせよ、今回、随分とひさしぶりにdebを作ってみたのですが、
以前と比べて、
\begin{itemize}
 \item dhコマンドすごい……。
 \item tar.gzで配布されてるソフトでgbpはかなり便利そう。
 \item (今回は使わなかったですが)quiltでパッチの管理が楽になってる。
\end{itemize}
と感じました。
upstreamの開発が止まっているという根本的な問題があるので、
今回のtomightyのdebはどうするのかというのは
ちょっと悩み中なのですが、
今後、上記の未解決案件に加えて
ITPとかの手順も調べていけたらなぁと思っています。

ここまでお付き合い頂き、ありがとうございました。


\dancersection{今後の予定}{Debian JP}

\subsection{関西Debian勉強会 10周年記念回}
次回、第121回関西Debian勉強会は03月19(日)に開催予定です。

\subsection{東京エリアDebian勉強会}
次回、第149回東京エリアDebian勉強会は、03月10と11日に OSC 2017 Tokyo/Spring に参加予定です。

%
% 冊子にするために、4の倍数にする必要がある。
% そのための調整
\dancersection{メモ}{}
\mbox{}\newpage
\mbox{}\newpage
% \mbox{}\newpage

\pagebreak

\begin{center}
本資料のライセンスについて
\end{center}

本資料はフリー・ソフトウェアです。あなたは、Free Software
Foundation が公表したGNU GENERAL PUBLIC LICENSEの "バージョン2"もしくはそれ以降
が定める条項に従って本プログラムを再頒布または変更することができ
ます。

本プログラムは有用とは思いますが、頒布にあたっては、市場性及び特
定目的適合性についての暗黙の保証を含めて、いかなる保証も行ないま
せん。詳細についてはGNU GENERAL PUBLIC LICENSE をお読みください。

\begin{multicols}{2}
 \begin{fontsize}{6}{6}
 \begin{verbatim}
            GNU GENERAL PUBLIC LICENSE
               Version 2, June 1991

 Copyright (C) 1989, 1991 Free Software Foundation, Inc.
    51 Franklin St, Fifth Floor, Boston, MA  02110-1301  USA
 Everyone is permitted to copy and distribute verbatim copies
 of this license document, but changing it is not allowed.

                Preamble

  The licenses for most software are designed to take away your
freedom to share and change it.  By contrast, the GNU General Public
License is intended to guarantee your freedom to share and change free
software--to make sure the software is free for all its users.  This
General Public License applies to most of the Free Software
Foundation's software and to any other program whose authors commit to
using it.  (Some other Free Software Foundation software is covered by
the GNU Library General Public License instead.)  You can apply it to
your programs, too.

  When we speak of free software, we are referring to freedom, not
price.  Our General Public Licenses are designed to make sure that you
have the freedom to distribute copies of free software (and charge for
this service if you wish), that you receive source code or can get it
if you want it, that you can change the software or use pieces of it
in new free programs; and that you know you can do these things.

  To protect your rights, we need to make restrictions that forbid
anyone to deny you these rights or to ask you to surrender the rights.
These restrictions translate to certain responsibilities for you if you
distribute copies of the software, or if you modify it.

  For example, if you distribute copies of such a program, whether
gratis or for a fee, you must give the recipients all the rights that
you have.  You must make sure that they, too, receive or can get the
source code.  And you must show them these terms so they know their
rights.

  We protect your rights with two steps: (1) copyright the software, and
(2) offer you this license which gives you legal permission to copy,
distribute and/or modify the software.

  Also, for each author's protection and ours, we want to make certain
that everyone understands that there is no warranty for this free
software.  If the software is modified by someone else and passed on, we
want its recipients to know that what they have is not the original, so
that any problems introduced by others will not reflect on the original
authors' reputations.

  Finally, any free program is threatened constantly by software
patents.  We wish to avoid the danger that redistributors of a free
program will individually obtain patent licenses, in effect making the
program proprietary.  To prevent this, we have made it clear that any
patent must be licensed for everyone's free use or not licensed at all.

  The precise terms and conditions for copying, distribution and
modification follow.

            GNU GENERAL PUBLIC LICENSE
   TERMS AND CONDITIONS FOR COPYING, DISTRIBUTION AND MODIFICATION

  0. This License applies to any program or other work which contains
a notice placed by the copyright holder saying it may be distributed
under the terms of this General Public License.  The "Program", below,
refers to any such program or work, and a "work based on the Program"
means either the Program or any derivative work under copyright law:
that is to say, a work containing the Program or a portion of it,
either verbatim or with modifications and/or translated into another
language.  (Hereinafter, translation is included without limitation in
the term "modification".)  Each licensee is addressed as "you".

Activities other than copying, distribution and modification are not
covered by this License; they are outside its scope.  The act of
running the Program is not restricted, and the output from the Program
is covered only if its contents constitute a work based on the
Program (independent of having been made by running the Program).
Whether that is true depends on what the Program does.

  1. You may copy and distribute verbatim copies of the Program's
source code as you receive it, in any medium, provided that you
conspicuously and appropriately publish on each copy an appropriate
copyright notice and disclaimer of warranty; keep intact all the
notices that refer to this License and to the absence of any warranty;
and give any other recipients of the Program a copy of this License
along with the Program.

You may charge a fee for the physical act of transferring a copy, and
you may at your option offer warranty protection in exchange for a fee.

  2. You may modify your copy or copies of the Program or any portion
of it, thus forming a work based on the Program, and copy and
distribute such modifications or work under the terms of Section 1
above, provided that you also meet all of these conditions:

    a) You must cause the modified files to carry prominent notices
    stating that you changed the files and the date of any change.

    b) You must cause any work that you distribute or publish, that in
    whole or in part contains or is derived from the Program or any
    part thereof, to be licensed as a whole at no charge to all third
    parties under the terms of this License.

    c) If the modified program normally reads commands interactively
    when run, you must cause it, when started running for such
    interactive use in the most ordinary way, to print or display an
    announcement including an appropriate copyright notice and a
    notice that there is no warranty (or else, saying that you provide
    a warranty) and that users may redistribute the program under
    these conditions, and telling the user how to view a copy of this
    License.  (Exception: if the Program itself is interactive but
    does not normally print such an announcement, your work based on
    the Program is not required to print an announcement.)

These requirements apply to the modified work as a whole.  If
identifiable sections of that work are not derived from the Program,
and can be reasonably considered independent and separate works in
themselves, then this License, and its terms, do not apply to those
sections when you distribute them as separate works.  But when you
distribute the same sections as part of a whole which is a work based
on the Program, the distribution of the whole must be on the terms of
this License, whose permissions for other licensees extend to the
entire whole, and thus to each and every part regardless of who wrote it.

Thus, it is not the intent of this section to claim rights or contest
your rights to work written entirely by you; rather, the intent is to
exercise the right to control the distribution of derivative or
collective works based on the Program.

In addition, mere aggregation of another work not based on the Program
with the Program (or with a work based on the Program) on a volume of
a storage or distribution medium does not bring the other work under
the scope of this License.

  3. You may copy and distribute the Program (or a work based on it,
under Section 2) in object code or executable form under the terms of
Sections 1 and 2 above provided that you also do one of the following:

    a) Accompany it with the complete corresponding machine-readable
    source code, which must be distributed under the terms of Sections
    1 and 2 above on a medium customarily used for software interchange; or,

    b) Accompany it with a written offer, valid for at least three
    years, to give any third party, for a charge no more than your
    cost of physically performing source distribution, a complete
    machine-readable copy of the corresponding source code, to be
    distributed under the terms of Sections 1 and 2 above on a medium
    customarily used for software interchange; or,

    c) Accompany it with the information you received as to the offer
    to distribute corresponding source code.  (This alternative is
    allowed only for noncommercial distribution and only if you
    received the program in object code or executable form with such
    an offer, in accord with Subsection b above.)

The source code for a work means the preferred form of the work for
making modifications to it.  For an executable work, complete source
code means all the source code for all modules it contains, plus any
associated interface definition files, plus the scripts used to
control compilation and installation of the executable.  However, as a
special exception, the source code distributed need not include
anything that is normally distributed (in either source or binary
form) with the major components (compiler, kernel, and so on) of the
operating system on which the executable runs, unless that component
itself accompanies the executable.

If distribution of executable or object code is made by offering
access to copy from a designated place, then offering equivalent
access to copy the source code from the same place counts as
distribution of the source code, even though third parties are not
compelled to copy the source along with the object code.

  4. You may not copy, modify, sublicense, or distribute the Program
except as expressly provided under this License.  Any attempt
otherwise to copy, modify, sublicense or distribute the Program is
void, and will automatically terminate your rights under this License.
However, parties who have received copies, or rights, from you under
this License will not have their licenses terminated so long as such
parties remain in full compliance.

  5. You are not required to accept this License, since you have not
signed it.  However, nothing else grants you permission to modify or
distribute the Program or its derivative works.  These actions are
prohibited by law if you do not accept this License.  Therefore, by
modifying or distributing the Program (or any work based on the
Program), you indicate your acceptance of this License to do so, and
all its terms and conditions for copying, distributing or modifying
the Program or works based on it.

  6. Each time you redistribute the Program (or any work based on the
Program), the recipient automatically receives a license from the
original licensor to copy, distribute or modify the Program subject to
these terms and conditions.  You may not impose any further
restrictions on the recipients' exercise of the rights granted herein.
You are not responsible for enforcing compliance by third parties to
this License.

  7. If, as a consequence of a court judgment or allegation of patent
infringement or for any other reason (not limited to patent issues),
conditions are imposed on you (whether by court order, agreement or
otherwise) that contradict the conditions of this License, they do not
excuse you from the conditions of this License.  If you cannot
distribute so as to satisfy simultaneously your obligations under this
License and any other pertinent obligations, then as a consequence you
may not distribute the Program at all.  For example, if a patent
license would not permit royalty-free redistribution of the Program by
all those who receive copies directly or indirectly through you, then
the only way you could satisfy both it and this License would be to
refrain entirely from distribution of the Program.

If any portion of this section is held invalid or unenforceable under
any particular circumstance, the balance of the section is intended to
apply and the section as a whole is intended to apply in other
circumstances.

It is not the purpose of this section to induce you to infringe any
patents or other property right claims or to contest validity of any
such claims; this section has the sole purpose of protecting the
integrity of the free software distribution system, which is
implemented by public license practices.  Many people have made
generous contributions to the wide range of software distributed
through that system in reliance on consistent application of that
system; it is up to the author/donor to decide if he or she is willing
to distribute software through any other system and a licensee cannot
impose that choice.

This section is intended to make thoroughly clear what is believed to
be a consequence of the rest of this License.

  8. If the distribution and/or use of the Program is restricted in
certain countries either by patents or by copyrighted interfaces, the
original copyright holder who places the Program under this License
may add an explicit geographical distribution limitation excluding
those countries, so that distribution is permitted only in or among
countries not thus excluded.  In such case, this License incorporates
the limitation as if written in the body of this License.

  9. The Free Software Foundation may publish revised and/or new versions
of the General Public License from time to time.  Such new versions will
be similar in spirit to the present version, but may differ in detail to
address new problems or concerns.

Each version is given a distinguishing version number.  If the Program
specifies a version number of this License which applies to it and "any
later version", you have the option of following the terms and conditions
either of that version or of any later version published by the Free
Software Foundation.  If the Program does not specify a version number of
this License, you may choose any version ever published by the Free Software
Foundation.

  10. If you wish to incorporate parts of the Program into other free
programs whose distribution conditions are different, write to the author
to ask for permission.  For software which is copyrighted by the Free
Software Foundation, write to the Free Software Foundation; we sometimes
make exceptions for this.  Our decision will be guided by the two goals
of preserving the free status of all derivatives of our free software and
of promoting the sharing and reuse of software generally.

                NO WARRANTY

  11. BECAUSE THE PROGRAM IS LICENSED FREE OF CHARGE, THERE IS NO WARRANTY
FOR THE PROGRAM, TO THE EXTENT PERMITTED BY APPLICABLE LAW.  EXCEPT WHEN
OTHERWISE STATED IN WRITING THE COPYRIGHT HOLDERS AND/OR OTHER PARTIES
PROVIDE THE PROGRAM "AS IS" WITHOUT WARRANTY OF ANY KIND, EITHER EXPRESSED
OR IMPLIED, INCLUDING, BUT NOT LIMITED TO, THE IMPLIED WARRANTIES OF
MERCHANTABILITY AND FITNESS FOR A PARTICULAR PURPOSE.  THE ENTIRE RISK AS
TO THE QUALITY AND PERFORMANCE OF THE PROGRAM IS WITH YOU.  SHOULD THE
PROGRAM PROVE DEFECTIVE, YOU ASSUME THE COST OF ALL NECESSARY SERVICING,
REPAIR OR CORRECTION.

  12. IN NO EVENT UNLESS REQUIRED BY APPLICABLE LAW OR AGREED TO IN WRITING
WILL ANY COPYRIGHT HOLDER, OR ANY OTHER PARTY WHO MAY MODIFY AND/OR
REDISTRIBUTE THE PROGRAM AS PERMITTED ABOVE, BE LIABLE TO YOU FOR DAMAGES,
INCLUDING ANY GENERAL, SPECIAL, INCIDENTAL OR CONSEQUENTIAL DAMAGES ARISING
OUT OF THE USE OR INABILITY TO USE THE PROGRAM (INCLUDING BUT NOT LIMITED
TO LOSS OF DATA OR DATA BEING RENDERED INACCURATE OR LOSSES SUSTAINED BY
YOU OR THIRD PARTIES OR A FAILURE OF THE PROGRAM TO OPERATE WITH ANY OTHER
PROGRAMS), EVEN IF SUCH HOLDER OR OTHER PARTY HAS BEEN ADVISED OF THE
POSSIBILITY OF SUCH DAMAGES.

             END OF TERMS AND CONDITIONS

        How to Apply These Terms to Your New Programs

  If you develop a new program, and you want it to be of the greatest
possible use to the public, the best way to achieve this is to make it
free software which everyone can redistribute and change under these terms.

  To do so, attach the following notices to the program.  It is safest
to attach them to the start of each source file to most effectively
convey the exclusion of warranty; and each file should have at least
the "copyright" line and a pointer to where the full notice is found.

    <one line to give the program's name and a brief idea of what it does.>
    Copyright (C) <year>  <name of author>

    This program is free software; you can redistribute it and/or modify
    it under the terms of the GNU General Public License as published by
    the Free Software Foundation; either version 2 of the License, or
    (at your option) any later version.

    This program is distributed in the hope that it will be useful,
    but WITHOUT ANY WARRANTY; without even the implied warranty of
    MERCHANTABILITY or FITNESS FOR A PARTICULAR PURPOSE.  See the
    GNU General Public License for more details.

    You should have received a copy of the GNU General Public License
    along with this program; if not, write to the Free Software
    Foundation, Inc., 51 Franklin St, Fifth Floor, Boston, MA  02110-1301 USA


Also add information on how to contact you by electronic and paper mail.

If the program is interactive, make it output a short notice like this
when it starts in an interactive mode:

    Gnomovision version 69, Copyright (C) year  name of author
    Gnomovision comes with ABSOLUTELY NO WARRANTY; for details type `show w'.
    This is free software, and you are welcome to redistribute it
    under certain conditions; type `show c' for details.

The hypothetical commands `show w' and `show c' should show the appropriate
parts of the General Public License.  Of course, the commands you use may
be called something other than `show w' and `show c'; they could even be
mouse-clicks or menu items--whatever suits your program.

You should also get your employer (if you work as a programmer) or your
school, if any, to sign a "copyright disclaimer" for the program, if
necessary.  Here is a sample; alter the names:

  Yoyodyne, Inc., hereby disclaims all copyright interest in the program
  `Gnomovision' (which makes passes at compilers) written by James Hacker.

  <signature of Ty Coon>, 1 April 1989
  Ty Coon, President of Vice

This General Public License does not permit incorporating your program into
proprietary programs.  If your program is a subroutine library, you may
consider it more useful to permit linking proprietary applications with the
library.  If this is what you want to do, use the GNU Library General
Public License instead of this License.
 \end{verbatim}
 \end{fontsize}
\end{multicols}

\begin{center}
ソースコードについて
\end{center}

このプログラムは tex で記述されたものです。ソースコードは
\begin{center}
  \url{git://anonscm.debian.org/tokyodebian/monthly-report.git}
\end{center}
から取得できます。

\begin{center}
Debian オープンユーズロゴ ライセンス
\end{center}

\begin{multicols}{2}
 \begin{fontsize}{6}{6}
 \begin{verbatim}

Copyright (c) 1999 Software in the Public Interest
Permission is hereby granted, free of charge, to any person
obtaining a copy of this software and associated documentation
files (the "Software"), to deal in the Software without restriction,
including without limitation the rights to use, copy, modify, merge,
publish, distribute, sublicense, and/or sell copies of the Software,
and to permit persons to whom the Software is furnished to do so,
subject to the following conditions:

The above copyright notice and this permission notice shall be
included in all copies or substantial portions of the Software.

THE SOFTWARE IS PROVIDED "AS IS", WITHOUT WARRANTY OF ANY
KIND, EXPRESS OR IMPLIED, INCLUDING BUT NOT LIMITED TO THE
WARRANTIES OF MERCHANTABILITY, FITNESS FOR A PARTICULAR PURPOSE AND
NONINFRINGEMENT. IN NO EVENT SHALL THE AUTHORS OR COPYRIGHT HOLDERS
BE LIABLE FOR ANY CLAIM, DAMAGES OR OTHER LIABILITY, WHETHER IN
AN ACTION OF CONTRACT, TORT OR OTHERWISE, ARISING FROM, OUT OF OR
IN CONNECTION WITH THE SOFTWARE OR THE USE OR OTHER DEALINGS IN
THE SOFTWARE.
 \end{verbatim}
 \end{fontsize}
\end{multicols}

\printindex
%\cleartooddpage

 \begin{minipage}[b]{0.2\hsize}
  \rotatebox{90}{\fontsize{80}{80} {\gt 関西 Debian 勉強会} }
 \end{minipage}
 \begin{minipage}[b]{0.8\hsize}

 \vspace*{15cm}
 \rule{\hsize}{1mm}
 \vspace{2mm}
 \includegraphics[width=2cm]{image200502/openlogo-nd.eps}
 \noindent \Large \bfseries{Debian 勉強会資料}\\ \\
 \noindent \normalfont \debmtgyear{}年\debmtgmonth{}月\debmtgdate{}日 \hspace{5mm}  初版第1刷発行\\
 \noindent \normalfont 関西 Debian 勉強会 (編集・印刷・発行)\\
 \rule{\hsize}{1mm}
 \end{minipage}

\end{document}
%%% Local Variables:
%%% skk-kutouten-type: jp
%%% End:
