\begin{prework}{ ftake }
  \begin{enumerate}
  \item HackMD をインストールします
  \item Which do you prefer community management or development? How long can you use time for community activity in a day or a week?
  \end{enumerate}
\end{prework}

\begin{prework}{ yoshitake }
  \begin{enumerate}
  \item セッションのみですが、参加させていただきましたら幸いです。
  \item Good to see you again ;)
  \end{enumerate}
\end{prework}

\begin{prework}{ khibino }
  \begin{enumerate}
  \item cabal-debian の proposed-update への upload 準備
  \item (thinking...)
  \end{enumerate}
\end{prework}

\begin{prework}{ ghanshyamman }
  \begin{enumerate}
  \item Open source developer
  \item How is the Debian release model experience. Any feedback or key thing to share for other software.
  \end{enumerate}
\end{prework}

\begin{prework}{ yy\_y\_ja\_jp }
  \begin{enumerate}
  \item パッケージ更新,DDTSS
  \item Outreach -- are there any ideas for women and young people to attract Debian?
  \item init systems -- I use sysvinit instead of systemd. I feel some DDs have some intention to make developers/users uncomfortable who support other init systems than systemd. In my opinion it's not a good idea for developers' or users' diversity and universality. Around the GR period some DDs have left Debian. I think it's unfortunate. What do you think about this?
  \end{enumerate}
\end{prework}

\begin{prework}{ Roger Shimizu }
  \begin{enumerate}
  \item pkg maintainence of torbrowser-launcher
  \item I hope more infrastructure, such as buildd, ftp-master server can move to asia, or even japan! But how can local community help?
  \end{enumerate}
\end{prework}

\begin{prework}{ dictoss }
  \begin{enumerate}
  \item gccのpieオプションを調べて発表準備をする
  \item Please tell me popular sub project of debian. I dot't find sub project guide page in debian wiki.
  \end{enumerate}
\end{prework}

\begin{prework}{ koedoyoshida }
  \begin{enumerate}
  \item 未定
  \item (thinking...)
  \end{enumerate}
\end{prework}

\begin{prework}{ ysaito }
  \begin{enumerate}
  \item ドキュメント読み
  \item The DPL may largely be entangled in daily business to keep the project going. Do you have a personal dream or vision on what Debian should be like in a distant future?
  \end{enumerate}
\end{prework}

\begin{prework}{ masayukig }
  \begin{enumerate}
  \item Read the docs and setup the Debian openstack cloud image to my own ``cloud''
  \item What are challenges of the Debian community?
  \item Do you have any expectations for Japanese user group?
  \item What technologies are you interested in, recently?
  \end{enumerate}
\end{prework}

\begin{prework}{ BurnDuck (nabaua) }
  \begin{enumerate}
  \item tensorflow
  \item how to update debian 9
  \end{enumerate}
\end{prework}

\begin{prework}{ はしもとまさ }
  \begin{enumerate}
  \item GPU入りノートPCにDebianはインストールできるか?(をやってみるかも)
  \item How about in Yokohama ? (横浜の感想をぜひ。)
  \end{enumerate}
\end{prework}

\begin{prework}{ NOKUBI Takatsugu }
  \begin{enumerate}
  \item 次の勉強会の資料作成
  \item Do you know the age bracket of Debian Developeres and Maintaineres ? (今開発者の年齢層はどうなっているか。)
  \item Are young Debian Developeres and Maintaineres increasing ? (若い人は増えているのかどうか)
  \end{enumerate}
\end{prework}

\begin{prework}{ henrich }
  \begin{enumerate}
  \item ディスカッション
  \item What would you like to achieve in your DPL term?
  \end{enumerate}
\end{prework}
