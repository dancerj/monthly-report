\documentclass[cjk,dvipdfmx,10pt,compress,%
hyperref={bookmarks=true,bookmarksnumbered=true,bookmarksopen=false,%
colorlinks=false,%
pdftitle={第 67 回 関西 Debian 勉強会},%
pdfauthor={倉敷・のがた・佐々木・かわだ},%
%pdfinstitute={関西 Debian 勉強会},%
pdfsubject={資料},%
}]{beamer}

\title{第 67 回 関西 Debian 勉強会}
\subtitle{$\sim$発表資料$\sim$}
\author[かわだ てつたろう]{{\large\bf 倉敷・のがた・佐々木・かわだ}}
\institute[Debian JP]{{\normalsize\tt 関西 Debian 勉強会}}
\date{{\small 2012 年 12 月 23 日}}

%\usepackage{amsmath}
%\usepackage{amssymb}
\usepackage{graphicx}
\usepackage{moreverb}
\usepackage[varg]{txfonts}
\AtBeginDvi{\special{pdf:tounicode EUC-UCS2}}
\usetheme{Kyoto}
\def\museincludegraphics{%
  \begingroup
  \catcode`\|=0
  \catcode`\\=12
  \catcode`\#=12
  \includegraphics[width=0.9\textwidth]}
%\renewcommand{\familydefault}{\sfdefault}
%\renewcommand{\kanjifamilydefault}{\sfdefault}
\begin{document}
\settitleslide
\begin{frame}
\titlepage
\end{frame}
\setdefaultslide

\begin{frame}[fragile]
\frametitle{Agenda}

\tableofcontents

\end{frame}

\section{最近の Debian 関係のイベント}

\takahashi[40]{最近の Debian\\関係のイベント}

\begin{frame}[fragile]
  \frametitle{第 66 回関西 Debian 勉強会}
  \begin{itemize}
  \item 日時: 11 月 9 日、10 日
  \item 場所: 関西オープンソース 2012
  \end{itemize}
  \begin{block}{内容}
    \begin{itemize}
    \item「Debian 7.0 "Wheezy" の紹介」
    \item 発表者(司会?): 佐々木洋平
      \begin{itemize}
      \item 皆様, ご参加\&愛あるツッコミありがとうございました.
      \end{itemize}
    \end{itemize}
  \end{block}
\end{frame}

\begin{frame}[fragile]
  \frametitle{第 95 回 東京エリア Debian 勉強会}
  \begin{itemize}
  \item 日時: 12 月 15 日
  \item 場所: あんさんぶる荻窪
  \end{itemize}
  \begin{block}{内容}
    \begin{itemize}
    \item 「im-config」
    \item 「2012年度東京エリアDebian勉強会の振り返り」
    \item 「日本におけるDFSGの求める自由と2012年改正著作権法」
    \end{itemize}
  \end{block}
\end{frame}

\takahashi[50]{そんな\\こんなで}
\takahashi[120]{次}

\section{事前課題発表}

\takahashi[50]{事前課題}

\begin{frame}[fragile]
  \frametitle{事前課題}
  \begin{block}{今回の事前課題}
    \begin{description}
    \item[事前課題1] 関西Debian勉強会で今年印象に残った話と来年聞きたい話を教えてください。
    \item[事前課題2] DebianPolicy の6章に目を通しておいてください。
    \item[事前課題3] 忘年会は何が食べたいですか?
    \end{description}
  \end{block}
\end{frame}

\takahashi[50]{事前課題\\発表}

\begin{frame}
  \frametitle{山下尊也}
  \begin{enumerate}
  \item ごめんなさい. 今年は参加できていません
  \item 昨今のDebian関係の話とか i386, amd64 以外に Debian を入れてみた話とか聴きたいです.
  \end{enumerate}
\end{frame}

\begin{frame}
  \frametitle{川江}
  \begin{enumerate}
  \item 特にありません.
  \item 了解
  \item 焼き肉
  \end{enumerate}
\end{frame}

\begin{frame}
  \frametitle{ 佐々木洋平 }
  \begin{enumerate}
  \item 大統一, かな. よくできたと思います. 来年度も是非やりたいですね.
  \item 了解です.
  \item いつものモツ鍋屋って予約とれるの?
  \end{enumerate}
\end{frame}

\begin{frame}
  \frametitle{ のがたじゅん }
  \begin{enumerate}
  \item 今年印象に残ったのは勉強会に入らないかもしれないけど大統一と、LDAP周りはあまり話しをする事がないから残ったかな。
    来年は、パッケージを久々に作ろうと思ったら浦島太郎状態だったので今どきのパッケージの作り方とか?
  \item わかりました。
  \end{enumerate}
\end{frame}

\begin{frame}
  \frametitle{ かわだてつたろう }
  \begin{enumerate}
  \item 温泉合宿、大統一、Ralf さんの話しがよかったです。
    来年は月刊 Debian Policy と LDAP の話しの続きをやりたいですね。
  \end{enumerate}
\end{frame}

\begin{frame}
  \frametitle{ cuzic }
  \begin{enumerate}
  \item 初参加なので、分かりません。。。Debian も初心者なんで。
  \item 当日までにやっておきます。。。
  \end{enumerate}
\end{frame}

\begin{frame}
  \frametitle{ yyatsuo }
  \begin{enumerate}
  \item 前半はパッケージ系(やってみた)が多く、後半は LDAP などの認証系の話題が多かったのが印象的でした。他にも温泉合宿、大統一勉強会、Ralf さんの EDOS の話なんかがありましたね。
    来年は西田さんと酒井さんのパッケージの今後を見守りたいですね。
  \item 見ておきます
  \end{enumerate}
\end{frame}

\begin{frame}
  \frametitle{ lurdan }
  \begin{enumerate}
  \item 温泉合宿と大統一がよかったですねー。あと、Debian Policy の輪読や複数の人が新たにパッケージ作成に挑戦、などなど新しい試みもあったなぁとか。来年は wheezy も出るはずなので、またサイクルを戻して初心者講座とかするのがいいのかなぁ。
  \item ざっくりと見ておきました
  \end{enumerate}
\end{frame}

\begin{frame}
  \frametitle{ 大林 }
  \begin{itemize}
  \item 今回初めて参加するので「特になし」ということで
  \end{itemize}
\end{frame}

\begin{frame}
  \frametitle{ ワチ マサタカ }
  \begin{itemize}
  \item 初参加なので印象に残った話はわからないです。\\
    来年は、パッケージメンテナの人が仕事とどう両立させてるかを聞いてみたいです。
  \item 通しておきます。
  \end{itemize}
\end{frame}

\begin{frame}
  \frametitle{ 山城の国の住人 久保博 }
  \begin{itemize}
  \item 今年印象に残った話
    \begin{itemize}
    \item 「スクリプト言語 Konoha の Debianパッケージ化」\mbox{~}\\
      Konoha がライセンスに関して混沌としていたのが衝撃的でした。
    \item 「News from EDOS: finding outdated packages」\mbox{~}\\
      パッケージ管理の問題がコンピュータ科学の研究テーマに
      なっていることに感銘を受けました。
    \item 「Debian ではじめる Kerberos 認証」\mbox{~}\\
      Active Directory 連係以外で Kerberos が使われていることや、
      今でも開発され続けているのが意外でした。
    \item また、
      翻訳について盛り上がったことが印象に残りました。
    \end{itemize}
  \item 来年聞きたい話
    \begin{itemize}
    \item wheezy への移行について
    \item samba4 とどうつき合っていけばいいか
    \item 開発に関すること
    \end{itemize}
  \end{itemize}
\end{frame}

\begin{frame}
  \frametitle{ 清野陽一 }
  \begin{enumerate}
  \item あまり参加できなかったので…。来年はたくさん参加して色々な話を聞きたいです。
  \item はーい
  \end{enumerate}
\end{frame}

\takahashi[50]{そんな\\こんなで}
\takahashi[120]{次}

\section{Android端末(Asus Transformer TF201)へのDebianインストール奮闘記}
\takahashi[25]{Android端末\\{\large{(Asus Transformer TF201)}}への\\Debianインストール奮闘記\\by\\cuzic}

\takahashi[50]{そんな\\こんなで}
\takahashi[120]{次}

\section{月刊 Debian Policy 第7回 「パッケージ管理スクリプトとインストールの手順」}
\takahashi[25]{月刊 Debian Policy\\第7回 \\「パッケージ管理スクリプトとインストールの手順」\\by\\かわだ てつたろう}

\takahashi[50]{そんな\\こんなで}
\takahashi[120]{次}

\section{2012年の振り返りと2013年の企画}
\takahashi[30]{2012年の振り返りと\\2013年の企画\\by\\佐々木洋平\\(Debian JP Project)}

\takahashi[50]{そんな\\こんなで}
\takahashi[120]{次}

\section{今後の予定}
\begin{frame}[fragile]
\frametitle{今後の予定}

\begin{block}{第 68 回関西 Debian 勉強会}
\begin{itemize}
  \item 日時: 1 月 27 日(日)
  \item 会場: 国際奈良学セミナーハウス
    \begin{description}
    \item[お題その1]:
    \item[お題その2]:
    \item[お題その3]: 月間Debian Policy by
    \end{description}
\end{itemize}
\end{block}

\begin{block}{第 96 回東京エリア Debian 勉強会}
  \begin{itemize}
  \item 日時: 1 月 19 日(土)
  \end{itemize}
\end{block}

\end{frame}


\takahashi[50]{  }


\end{document}
%%% Local Variables:
%%% mode: japanese-latex
%%% TeX-master: t
%%% End:
