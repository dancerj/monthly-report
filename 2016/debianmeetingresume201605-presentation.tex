%; whizzy paragraph -pdf xpdf -latex ./whizzypdfptex.sh
%; whizzy-paragraph "^\\\\begin{frame}\\|\\\\emtext"
% latex beamer presentation.
% platex, latex-beamer でコンパイルすることを想定。 

%     Tokyo Debian Meeting resources
%     Copyright (C) 2012 Junichi Uekawa

%     This program is free software; you can redistribute it and/or modify
%     it under the terms of the GNU General Public License as published by
%     the Free Software Foundation; either version 2 of the License, or
%     (at your option) any later version.

%     This program is distributed in the hope that it will be useful,
%     but WITHOUT ANY WARRANTY; without even the implied warreanty of
%     MERCHANTABILITY or FITNESS FOR A PARTICULAR PURPOSE.  See the
%     GNU General Public License for more details.
%     You should have received a copy of the GNU General Public License
%     along with this program; if not, write to the Free Software
%     Foundation, Inc., 51 Franklin St, Fifth Floor, Boston, MA  02110-1301 USA

\documentclass[cjk,dvipdfmx,12pt]{beamer}
\usetheme{Tokyo}
\usepackage{monthlypresentation}

%  preview (shell-command (concat "evince " (replace-regexp-in-string "tex$" "pdf"(buffer-file-name)) "&")) 
%  presentation (shell-command (concat "xpdf -fullscreen " (replace-regexp-in-string "tex$" "pdf"(buffer-file-name)) "&"))
%  presentation (shell-command (concat "evince " (replace-regexp-in-string "tex$" "pdf"(buffer-file-name)) "&"))

%http://www.naney.org/diki/dk/hyperref.html
%日本語EUC系環境の時
\AtBeginDvi{\special{pdf:tounicode EUC-UCS2}}
%シフトJIS系環境の時
%\AtBeginDvi{\special{pdf:tounicode 90ms-RKSJ-UCS2}}

\newenvironment{commandlinesmall}%
{\VerbatimEnvironment
  \begin{Sbox}\begin{minipage}{1.0\hsize}\begin{fontsize}{8}{8} \begin{BVerbatim}}%
{\end{BVerbatim}\end{fontsize}\end{minipage}\end{Sbox}
  \setlength{\fboxsep}{8pt}
% start on a new paragraph

\vspace{6pt}% skip before
\fcolorbox{dancerdarkblue}{dancerlightblue}{\TheSbox}

\vspace{6pt}% skip after
}
%end of commandlinesmall

\title{東京エリアDebian勉強会}
\subtitle{第139回 2016年5月度}
\author{杉本 典充}
\date{2016年5月21日}
\logo{\includegraphics[width=8cm]{image200607/openlogo-light.eps}}

\begin{document}

\begin{frame}
\titlepage{}
\end{frame}

\begin{frame}{Agenda}
 \begin{minipage}[t]{0.45\hsize}
  \begin{itemize}
   \item 事前課題発表
   \item 最近あったDebian関連のイベント報告
	 \begin{itemize}
	 \item 第138回東京エリアDebian勉強会
	 \end{itemize}
  \end{itemize}
 \end{minipage} 
 \begin{minipage}[t]{0.45\hsize}
  \begin{itemize}
   \item Debian Trivia Quiz
   \item Rogerさん 「Buffalo Linkstation 向け Debian Installer」
   \item 今日の宴会場所
  \end{itemize}
 \end{minipage}
\end{frame}

\section{事前課題}
\emtext{事前課題}
{\footnotesize
 \begin{prework}{ koedoyoshida }
  \begin{enumerate}
  \item $B$"$j$^$9(B
  \item raspberry-pi 1B/2/3
  \item DebianJP miniconf2016JP/PyCon JP$B=`Hw(B
  \end{enumerate}
\end{prework}

\begin{prework}{ kenhys }
  \begin{enumerate}
  \item $B$"$j$^$9(B
  \item ($B2sEz$J$7(B)
  \item ITP$B$N:n6H$r:F3+$9$k(B
  \end{enumerate}
\end{prework}

\begin{prework}{ tyamada22 }
  \begin{enumerate}
  \item $B$"$j$^$;$s(B
  \item ($B2sEz$J$7(B)
  \item DDTSS $B$^$?$O(B po-templates$B$J$I$NK]Lu(B
  \end{enumerate}
\end{prework}

\begin{prework}{ dictoss }
  \begin{enumerate}
  \item $B$"$j$^$9(B
  \item $B9uH"(BHG$B!"%i%:%Y%j!<%Q%$(B2
  \item kfeebsd$B$H(Btilegx$B$N>pJs<}=8(B
  \end{enumerate}
\end{prework}

\begin{prework}{ Charles Plessy }
  \begin{enumerate}
  \item $B$"$j$^$9(B
  \item https://wiki.debian.org/Cloud/AmazonEC2DebianInstaller
  \item GPG$B%-!<(B \& mime-support $B%Q%C%1!<%8(B
  \end{enumerate}
\end{prework}

\begin{prework}{ Roger Shimizu }
  \begin{enumerate}
  \item $B$"$j$^$9(B
  \item ARM board$B$d!"(Bvirtualbox VM$B$J$I(B
  \item debian installer $B$K(B GNU/screen $BBP1~$K4X$9$k:n6H(B
  \end{enumerate}
\end{prework}

\begin{prework}{ takaswie }
  \begin{enumerate}
  \item $B$"$j$^$;$s(B
  \item ($B2sEz$J$7(B)
  \item ALSA$B%f!<%6!<%i%s%I%i%$%V%i%j$N%Q%C%A$r=q$/M=Dj$G$9!#(B
  \end{enumerate}
\end{prework}

\begin{prework}{ henrich }
  \begin{enumerate}
  \item $B$"$j$^$;$s(B
  \item ($B2sEz$J$7(B)
  \item administrators handbook$B$N::FI(B
  \end{enumerate}
\end{prework}

\begin{prework}{ $B$D$k(B }
  \begin{enumerate}
  \item $B$"$j$^$;$s(B
  \item ($B2sEz$J$7(B)
  \item Debian $B?7%a%s%F%J!<%,%$%I$rFI$`(B
  \end{enumerate}
\end{prework}

}

\section{イベント報告}
\emtext{イベント報告}

\begin{frame}{第138回東京エリアDebian勉強会 }

\begin{itemize}
\item 場所はサイボウズさんをお借りして開催しました
\item 参加者は14名
\item セミナ内容は、以下内容を発表しました
  \begin{enumerate}
  \item kenhysさん「Debian のインフラを借りるには 」
  \item wskokaさん「tilegx について」
  \item Adrianさん「Introduction to Debian Ports」
  \end{enumerate}
\item 残りの時間でhack timeを行い、成果発表をしました
\end{itemize} 
\end{frame}

\subsection{問題}

%; whizzy-master ../debianmeetingresume201311.tex
% $B0J>e$N@_Dj$r$7$F$$$k$?$a!"$3$N%U%!%$%k$G(B M-x whizzytex $B$9$k$H!"(Bwhizzytex$B$,MxMQ$G$-$^$9!#(B
%

\santaku
{LTS$B%5%]!<%H$,;O$^$C$?(BDebian 7 (wheezy)$B!#(BDebian 6 (squeeze)$B$+$i%5%]!<%H$9$k%"!<%-%F%/%A%c!<$,A}$($^$7$?!#A}$($?%"!<%-%F%/%A%c$N@5$7$$AH$_9g$o$;$O$I$l$G$7$g$&$+!)(B}
{armel $B$H(B armhf}
{armel $B$H(B arm64}
{mipsel $B$H(B ppc64el}
{A}
{wheezy$B$N(BLTS$B%5%]!<%H3+;OD>A0$K(Barmel/armhf$B$r(BLTS$B%5%]!<%H$KDI2C$9$k$+5DO@$,?J$_!"$a$G$?$/(Barmel/armhf$B$,(BLTS$B%5%]!<%H$5$l$k$3$H$K$J$j$^$7$?!#5DO@$O0J2<$N%9%l%C%I$X$I$&$>!#(B \url{https://lists.debian.org/debian-lts/2016/04/msg00045.html}}

\santaku
{Debian GNU/Linux$B$G(Bapt$B$K$h$kG[I[$,;O$^$C$?(Bzfs-dkms$B%Q%C%1!<%8!#$I$N%3%s%]!<%M%s%H$GDs6!$5$l$F$$$k$G$7$g$&$+!)(B}
{main}
{contrib}
{non-free}
{B}
{ZFS$B$N%=!<%9%3!<%I$O(BCDDL$B$H$$$&%i%$%;%s%9$G8x3+$5$l$F$$$k$?$a!"(BGPLv2$B$G$"$k(BLinux$B%+!<%M%k$N%=!<%9%3!<%I$K%^!<%8$7$F(B(linux$B$H$7$F(B)$B:FG[I[$9$k$3$H$,$G$-$J$$!"$H$5$l$F$-$?7P0^$,$"$j$^$9!#$D$$$K%i%$%;%s%9LdBj$,2r7h$5$l!"(Bcontrib$B$H$7$F(BZFS$B$N%=!<%9%3!<%I$rG[I[$7(Bdkms$B$K$F%S%k%I$7$FMxMQ$9$k7A$KMn$ACe$-$^$7$?!#>\:Y$O0J2<$N(Bweb$B%Z!<%8$H%Z!<%8Fb$N%j%s%/$r$I$&$>!#(B \url{https://bits.debian.org/2016/05/what-does-it-mean-that-zfs-is-in-debian.html}}


\section{Buffalo Linkstation 向け Debian Installer}
\emtext{Buffalo Linkstation 向け Debian Installer}

\section{Hack time}
\emtext{Hack time}

\begin{frame}{成果を記入下さい!}

  今回、Hack time時の成果を記録に残してみます。

\url{https://tokyodebianmeeting.titanpad.com/1}

に皆さんアクセス頂き(認証不要です)、各自成果を17:30までに記録するようにおねがいします。

\end{frame}
  
\section{今後のイベント}
\emtext{今後のイベント}
\begin{frame}{今後のイベント}
\begin{itemize}
\item 6/18 OSC 2016 北海道
  \begin{itemize}
  \item イベント展示
  \item セミナー「Debian Updates」岩松 信洋)
  \end{itemize}
\item 6/某日 東京エリアDebian勉強会
\end{itemize}
\end{frame}

\section{今日の宴会場所}
\emtext{今日の宴会場所}
\begin{frame}{今日の宴会場所}
未定
\end{frame}

\end{document}

;;; Local Variables: ***
;;; outline-regexp: "\\([ 	]*\\\\\\(documentstyle\\|documentclass\\|emtext\\|section\\|begin{frame}\\)\\*?[ 	]*[[{]\\|[]+\\)" ***
;;; End: ***
