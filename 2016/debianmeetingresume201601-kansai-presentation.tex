\documentclass[cjk,dvipdfmx,10pt,compress,%
hyperref={bookmarks=true,bookmarksnumbered=true,bookmarksopen=false,%
colorlinks=false,%
pdftitle={第 106 回 関西 Debian 勉強会},%
pdfauthor={倉敷・のがた・佐々木・かわだ},%
%pdfinstitute={関西 Debian 勉強会},%
pdfsubject={資料},%
}]{beamer}

\title{第 106 回 関西 Debian 勉強会}
\subtitle{$\sim$発表資料$\sim$}
\author[かわだ てつたろう]{{\large\bf 倉敷・のがた・佐々木・かわだ}}
\institute[Debian JP]{{\normalsize\tt 関西 Debian 勉強会}}
\date{{\small 2016 年 1 月 24 日}}

%\usepackage{amsmath}
%\usepackage{amssymb}
\usepackage{graphicx}
\usepackage{moreverb}
\usepackage[varg]{txfonts}
\AtBeginDvi{\special{pdf:tounicode EUC-UCS2}}
\usetheme{Kyoto}
\def\museincludegraphics{%
  \begingroup
  \catcode`\|=0
  \catcode`\\=12
  \catcode`\#=12
  \includegraphics[width=0.9\textwidth]}
%\renewcommand{\familydefault}{\sfdefault}
%\renewcommand{\kanjifamilydefault}{\sfdefault}
\begin{document}
\settitleslide
\begin{frame}
\titlepage
\end{frame}
\setdefaultslide

\begin{frame}[fragile]
  \frametitle{Disclaimer}
  \begin{itemize}
  \item 疑問、質問、ツッコミ、茶々、\alert{大歓迎}
  \item その場でインタラクティブにどうぞ
  \item ハッシュタグ \#kansaidebian
  \end{itemize}
\end{frame}

\begin{frame}[fragile]
\frametitle{Agenda}

\tableofcontents

\end{frame}

\section{最近の Debian 関係のイベント}

\takahashi[40]{最近の Debian\\関係のイベント}

\begin{frame}[fragile]
  \frametitle{第105回関西Debian勉強会}
  \begin{itemize}
  \item 日時: 12月27日(日)
  \item 場所: 福島区民センター
  \end{itemize}
  \begin{block}{内容}
    \begin{itemize}
    \item 「LibreOfficeの最近の動向とDebianでのLibreOfficeパッケージについて」
    \item 「2015年の振り返りと2016年の企画」
    \end{itemize}
  \end{block}
\end{frame}

\begin{frame}[fragile]
  \frametitle{第135回東京エリアDebian勉強会}
  \begin{itemize}
  \item 日時: 1月23日(土)
  \item 場所: イベント&コミュニティスペース dots.
  \end{itemize}
  \begin{block}{内容}
    \begin{itemize}
    \item 「 Debian 今年の半年分の計画を立ててみた 」
    \item 「 Debian で Linux Ftrace まわりをいじってみた 」
    \end{itemize}
  \end{block}
\end{frame}

\begin{frame}[fragile]
  \frametitle{Debian Project}
  \begin{itemize}
  \item Debian Project mourns the loss of Ian Murdock
  \item Debian Installer Stretch Alpha 5 release
  \item support for merged /usr in Debian
  \end{itemize}
\end{frame}

\takahashi[50]{そんな\\こんなで}
\takahashi[120]{次}

\takahashi[50]{事前課題}

\begin{frame}[fragile]
  \frametitle{事前課題}
  \begin{block}{今回の事前課題}
    事前課題はありませんでした。
  \end{block}
\end{frame}

\takahashi[50]{事前課題\\発表}

\begin{frame}
  \frametitle{ 矢吹 幸治 }
\end{frame}

\begin{frame}
  \frametitle{ Say-no }
\end{frame}

\begin{frame}
  \frametitle{ むんくさん }
\end{frame}

\begin{frame}
  \frametitle{ Yosuke OTSUKI }
\end{frame}

\begin{frame}
  \frametitle{ 川江 浩 }
\end{frame}

\begin{frame}
  \frametitle{ Katsuki Kobayashi }
\end{frame}

\begin{frame}
  \frametitle{ tsukudamayo }
\end{frame}

\begin{frame}
  \frametitle{ 佐々木洋平 }
\end{frame}

\begin{frame}
  \frametitle{ t3rkwd }
\end{frame}

\begin{frame}
  \frametitle{ lurdan }
\end{frame}

\begin{frame}
  \frametitle{ Yamada Yohei (山田 洋平) }
\end{frame}

\begin{frame}
  \frametitle{ 門 誠 }
\end{frame}

\takahashi[50]{そんな\\こんなで}
\takahashi[120]{次}

\section{GNUHurdのインストールしてみた。と、Xサーバの立ち上げに挑戦}
\takahashi[30]{GNUHurdのインストールしてみた。\\と、Xサーバの立ち上げに挑戦\\by\\川江 浩}

\section{VyOSを入れてAPを構築してみた。}
\takahashi[30]{VyOSを入れて\\APを構築してみた。\\by\\かわだてつたろう}

\section{今後の予定}
\begin{frame}[fragile]
  \frametitle{今後の予定}

  \begin{block}{第107回関西Debian勉強会}
    \begin{itemize}
    \item 日時: 2月28日(日)
    \item 場所: 福島区民センター
      \begin{block}{予定内容}
        \begin{itemize}
        \item 「systemd-networkd」
        \end{itemize}
      \end{block}
    \end{itemize}
  \end{block}

  \begin{block}{第136回東京エリアDebian勉強会}
    \begin{itemize}
    \item 日時: 2月26日(金)、27日(土)
    \item 場所: OSC 2016 Tokyo Spring
    \end{itemize}
  \end{block}

\end{frame}

\takahashi[50]{  }

\end{document}
%%% Local Variables:
%%% mode: japanese-latex
%%% TeX-master: t
%%% End:
