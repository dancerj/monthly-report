\documentclass[cjk,dvipdfmx,10pt,compress,%
hyperref={bookmarks=true,bookmarksnumbered=true,bookmarksopen=false,%
colorlinks=false,%
pdftitle={第 109 回 関西 Debian 勉強会},%
pdfauthor={倉敷・のがた・佐々木・かわだ},%
%pdfinstitute={関西 Debian 勉強会},%
pdfsubject={資料},%
}]{beamer}

\title{第 109 回 関西 Debian 勉強会}
\subtitle{$\sim$発表資料$\sim$}
\author[かわだ てつたろう]{{\large\bf 倉敷・のがた・佐々木・かわだ}}
\institute[Debian JP]{{\normalsize\tt 関西 Debian 勉強会}}
\date{{\small 2016 年 4 月 24 日}}

%\usepackage{amsmath}
%\usepackage{amssymb}
\usepackage{graphicx}
\usepackage{moreverb}
\usepackage[varg]{txfonts}
\AtBeginDvi{\special{pdf:tounicode EUC-UCS2}}
\usetheme{Kyoto}
\def\museincludegraphics{%
  \begingroup
  \catcode`\|=0
  \catcode`\\=12
  \catcode`\#=12
  \includegraphics[width=0.9\textwidth]}
%\renewcommand{\familydefault}{\sfdefault}
%\renewcommand{\kanjifamilydefault}{\sfdefault}
\begin{document}
\settitleslide
\begin{frame}
\titlepage
\end{frame}
\setdefaultslide

\begin{frame}[fragile]
  \frametitle{Disclaimer}
  \begin{itemize}
  \item 疑問、質問、ツッコミ、茶々、\alert{大歓迎}
  \item その場でインタラクティブにどうぞ
  \item ハッシュタグ \#kansaidebian
  \end{itemize}
\end{frame}

\begin{frame}[fragile]
\frametitle{Agenda}

\tableofcontents

\end{frame}

\section{最近の Debian 関係のイベント}

\takahashi[40]{最近の Debian\\関係のイベント}

\begin{frame}[fragile]
  \frametitle{第108回関西Debian勉強会}
  \begin{itemize}
  \item 日時: 3月27日(日)
  \item 場所: 福島区民センター
  \end{itemize}
  \begin{block}{内容}
    \begin{itemize}
    \item 「systemdに浸ってみた」
    \end{itemize}
  \end{block}
\end{frame}

\begin{frame}[fragile]
  \frametitle{Debian Project}
  \begin{itemize}
  \item Debian Project Leader Election 2016 Results
  \item Updated Debian \{7: 7.10, 8: 8.4\} released
  \item xscreensaver: please disable "This version of XScreenSaver is very old! Please upgrade!" message
  \item debhelper compat 10 ready for testing (debhelper/9.20160403)
  \item その他
    \begin{itemize}
    \item Overall bitrot, package reviews and fast(er) unmaintained package removals
    \item Time for a .changes file format 2.0?
    \end{itemize}
  \end{itemize}
\end{frame}

\takahashi[50]{そんな\\こんなで}
\takahashi[120]{次}

\section{事前課題}
\takahashi[50]{事前課題}

\begin{frame}[fragile]
  \frametitle{事前課題}
  \begin{block}{今回の事前課題}
    \begin{description}
    \item[事前課題1] ENIAC が作成された目的は何でしょうか?

      調べてみてください。
    \end{description}
  \end{block}
\end{frame}

\takahashi[50]{事前課題\\発表}

\begin{frame}
  \frametitle{ Syn }
  \begin{enumerate}
  \item アメリカ陸軍の依頼で、弾道計算のために作成されたコンピューター
  \end{enumerate}
\end{frame}

\begin{frame}
  \frametitle{ むんくさん }
  \begin{enumerate}
  \item かなり以前に雑誌の記事で読んだ記憶では、砲弾の弾道計算のため、だったと思
    います。
  \end{enumerate}
\end{frame}

\begin{frame}
  \frametitle{ t3rkwd }
  \begin{enumerate}
  \item 弾道計算。大砲の弾道表をはやく作成するため。
  \end{enumerate}
\end{frame}

\begin{frame}
  \frametitle{ murase\_syuka }
  \begin{enumerate}
  \item 大砲の弾道計算
  \end{enumerate}
\end{frame}

\begin{frame}
  \frametitle{ Yosuke OTSUKI }
  \begin{enumerate}
  \item 出題者なので、ここでネタばらしはしません。
  \end{enumerate}
\end{frame}

\begin{frame}
  \frametitle{ 佐々木洋平 }
  \begin{enumerate}
  \item 当初の目的は「弾道計算」ですよね、フォン・ノイマンの\sout{せい}おかげで、
    最初に計算されたのはマンハッタン計画用の原爆に関する計算だった様ですが。

    自分の専門分野に関係することとして、ENIAC では単純化した系(1層の順圧渦度方程
    式)を問いて「天気予報」が行なわれています。
    こっちは割と詳しく知っていますが、あいにくと弾道計算と原爆関連の計算について
    の詳細は知りません(知ってたら教えて下さい).
  \end{enumerate}
\end{frame}

\begin{frame}
  \frametitle{ 川江 浩 }
  \begin{enumerate}
  \item ENIAC(エニアック、Electronic Numerical Integrator and Computer)は、
    アメリカで開発された黎明期の電子計算機(コンピュータ)。チューリング完
    全でディジタル式だがプログラム内蔵方式とするにはプログラムのためのメモ
    リがごくわずかで、パッチパネルによるプログラミングは煩雑ではあったもの
    の、必ずしも専用計算機ではなく広範囲の計算問題を解くことができた
    by Wikipedia
  \end{enumerate}
\end{frame}

\begin{frame}
  \frametitle{ Hideo Ueno }
  \begin{enumerate}
  \item アメリカ陸軍の弾道計算用に作られた。真空管で構成されていて完全にデジタ
    ル方式でプログラムが実行された。プログラムはパネルで配線することで行わ
    れ、プログラムの変更の際には、手作業にて配線をやり直す必要があった。

    余談だが、IBMのシステム360が企業に出回るまでは、パンチカードの処理
    を行うために似たような配線式のシステムを使用していたと記憶している。
    (当時はまだ入社下手で、先輩の話を聞いただけで、動かない実物を一度見た
    記憶はあります)
  \end{enumerate}
\end{frame}


\takahashi[50]{そんな\\こんなで}
\takahashi[120]{次}

\section{OpenFOAM で数値流体解析}
\takahashi[30]{OpenFOAM\\で数値流体解析\\by\\Yosuke Otsuki}

\section{今後の予定}
\begin{frame}[fragile]
  \frametitle{今後の予定}

  \begin{block}{第110回関西Debian勉強会}
    \begin{itemize}
    \item 日時: 5月21日(土) or 22日(日)
    \item 場所: 福島区民センター
    \end{itemize}
  \end{block}

  \begin{block}{第139回東京エリアDebian勉強会}
    \begin{itemize}
    \item 日時: 5月21日(土)?
    \item 場所:
    \end{itemize}
  \end{block}

\end{frame}

\takahashi[50]{  }

\end{document}
%%% Local Variables:
%%% mode: japanese-latex
%%% TeX-master: t
%%% End:
