%; whizzy-master ../debianmeetingresume201311.tex
% 以上の設定をしているため、このファイルで M-x whizzytex すると、whizzytexが利用できます。
%

\santaku
{LTSサポートが始まったDebian 7 (wheezy)。Debian 6 (squeeze)からサポートするアーキテクチャーが増えました。増えたアーキテクチャの正しい組み合わせはどれでしょうか?}
{armel と armhf}
{armel と arm64}
{mipsel と ppc64el}
{A}
{wheezyのLTSサポート開始直前にarmel/armhfをLTSサポートに追加するか議論が進み、めでたくarmel/armhfがLTSサポートされることになりました。議論は以下のスレッドへどうぞ。 \url{https://lists.debian.org/debian-lts/2016/04/msg00045.html}}

\santaku
{Debian GNU/Linuxでaptによる配布が始まったzfs-dkmsパッケージ。どのコンポーネントで提供されているでしょうか?}
{main}
{contrib}
{non-free}
{B}
{ZFSのソースコードはCDDLというライセンスで公開されているため、GPLv2であるLinuxカーネルのソースコードにマージして(linuxとして)再配布することができない、とされてきた経緯があります。ついにライセンス問題が解決され、contribとしてZFSのソースコードを配布しdkmsにてビルドして利用する形に落ち着きました。詳細は以下のwebページとページ内のリンクをどうぞ。 \url{https://bits.debian.org/2016/05/what-does-it-mean-that-zfs-is-in-debian.html}}
