\documentclass[cjk,dvipdfmx,10pt,compress,%
hyperref={bookmarks=true,bookmarksnumbered=true,bookmarksopen=false,%
colorlinks=false,%
pdftitle={第 115 回 関西 Debian 勉強会},%
pdfauthor={倉敷・のがた・佐々木・かわだ・オオツキ},%
%pdfinstitute={関西 Debian 勉強会},%
pdfsubject={資料},%
}]{beamer}

\title{第 115 回 関西 Debian 勉強会}
\subtitle{$\sim$発表資料$\sim$}
\author[かわだ てつたろう]{{\large\bf 倉敷・のがた・佐々木・かわだ・オオツキ}}
\institute[Debian JP]{{\normalsize\tt 関西 Debian 勉強会}}
\date{{\small 2016 年 10 月 23 日}}

%\usepackage{amsmath}
%\usepackage{amssymb}
\usepackage{graphicx}
\usepackage{moreverb}
\usepackage[varg]{txfonts}
\AtBeginDvi{\special{pdf:tounicode EUC-UCS2}}
\usetheme{Kyoto}
\def\museincludegraphics{%
  \begingroup
  \catcode`\|=0
  \catcode`\\=12
  \catcode`\#=12
  \includegraphics[width=0.9\textwidth]}
%\renewcommand{\familydefault}{\sfdefault}
%\renewcommand{\kanjifamilydefault}{\sfdefault}
\begin{document}
\settitleslide
\begin{frame}
\titlepage
\end{frame}
\setdefaultslide

\begin{frame}[fragile]
  \frametitle{Disclaimer}
  \begin{itemize}
  \item 疑問、質問、ツッコミ、茶々、\alert{大歓迎}
  \item その場でインタラクティブにどうぞ
  \item ハッシュタグ \#kansaidebian
  \end{itemize}
\end{frame}

\begin{frame}[fragile]
\frametitle{Agenda}

\tableofcontents

\end{frame}

\section{最近の Debian 関係のイベント}
\takahashi[40]{最近の Debian\\関係のイベント}

\begin{frame}[fragile]
  \frametitle{第114回関西Debian勉強会}
  \begin{itemize}
  \item 日時: 9月25日(日)
  \item 場所: 福島区民センター
  \begin{block}{内容}
    \begin{itemize}
    \item 「Let's Encryptのススメ」
    \item 「初心者が初めてパッケージをつくってみた」
    \end{itemize}
  \end{block}
\end{itemize}
\end{frame}

\begin{frame}[fragile]
  \frametitle{第144回東京エリアDebian勉強会}
  \begin{itemize}
  \item 日時: 10月15日(土)
  \item 場所: 株式会社朝日ネット
  \end{itemize}
  \begin{block}{内容}
    \begin{itemize}
    \item 「preseedでDebianを自動インストールをしてみよう」
    \end{itemize}
  \end{block}
\end{frame}

\begin{frame}[fragile]
  \frametitle{Debian Project}
  \begin{itemize}
  \item Misc Developer News (\#42)
  \item backup server running out of space
  \end{itemize}
\end{frame}

\takahashi[50]{そんな\\こんなで}
\takahashi[120]{次}

\section{事前課題}
\takahashi[50]{事前課題}

\begin{frame}[fragile]
  \frametitle{事前課題}
  \begin{block}{今回の事前課題}
    \begin{enumerate}
    \item sbuildおよびdebciを動作させるために複数のlxcコンテナ(馬から落馬感?)を
      利用する予定です。lxcに限定しませんが、なんらかの仮想環境で最低限二つ以上
      のDebian unstable環境を動作させられる様にしておいて下さい。
    \end{enumerate}
  \end{block}
\end{frame}

\takahashi[50]{事前課題\\発表}

\begin{frame}
  \frametitle{ 佐々木 洋平 }
\end{frame}

\begin{frame}
  \frametitle{ Yosuke OTSUKI] }
\end{frame}

\begin{frame}
  \frametitle{ lurdan }
\end{frame}

\begin{frame}
  \frametitle{ TaisukeNishimori }
  \begin{itemize}
  \item 我が家のバックアップサーバがDebian
  \end{itemize}
\end{frame}

\begin{frame}
  \frametitle{ t3rkwd }
\end{frame}

\begin{frame}
  \frametitle{ ipv6waterstar }
\end{frame}

\begin{frame}
  \frametitle{ Katsuki Kobayashi }
\end{frame}

\takahashi[50]{そんな\\こんなで}
\takahashi[120]{次}

\section{sbuildとdebciを触ってみた。}
\takahashi[30]{sbuildとdebciを触ってみた。\\by\\佐々木 洋平}

\takahashi[50]{そんな\\こんなで}
\takahashi[120]{次}

\section{今後の予定}
\begin{frame}[fragile]
  \frametitle{今後の予定}

  \begin{block}{第116回関西Debian勉強会@KOF2016}
    \begin{itemize}
    \item 日時: 11月12日(土)
    \item 場所: 大阪南港ATC ITM 棟10F
    \end{itemize}
  \end{block}

  \begin{block}{第146回東京エリアDebian勉強会}
    \begin{itemize}
    \item 日時: 11月19日(土)
    \item 場所:
    \end{itemize}
  \end{block}

\end{frame}

\takahashi[50]{  }

\end{document}
%%% Local Variables:
%%% mode: japanese-latex
%%% TeX-master: t
%%% End:
