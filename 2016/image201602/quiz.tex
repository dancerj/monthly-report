%; whizzy-master ../debianmeetingresume201311.tex
% 以上の設定をしているため、このファイルで M-x whizzytex すると、whizzytexが利用できます。
%

\santaku
{2016/2/3にて、debtagのtag付けについての変更が流れました。以下のどれ?}
{tag付け廃止}
{tag付けについてユーザ認証付きにする}
{tagのレビューをさらに強固にする}
{B}
{debtagを編集できるサイトがいろいろリニューアルするというアナウンスが流れ、実際いくつも実行されたようです。まず、匿名によるtag付けの編集を廃止し、代わりにsso.debian.orgによるシングルサインオンでユーザ認証しないと編集出来ないようにしたとのこと。また、tag編集のURLが変更になっており、https://debtags.debian.org/となりました。他にもいろいろ変更が出ていますので、詳しくはhttps://lists.debian.org/debian-devel-announce/2016/02/msg00000.html 参照。}

\santaku
{2016/1/12にて、debian sidにphpの新しいバージョンを入れた件がアナウンスされました。どのバージョン?}
{php 7.0}
{php 5.6}
{phpって何?}
{A}
{php7.0がdebian sidにてリリースされました。ちなみに、php7の目玉機能は大幅な実効速度改善です。これで次期安定版バージョンであるstrechで、php7が利用できる見込みがとても高くなってきましたね!}

\santaku
{dbgsymパッケージですが、こちらを保管するミラー先はどこでしょう?}
{mirrors.debian.org}
{debug.mirrors.debian.org}
{ftp.jp.debian.org}
{B}
{debhelper 9.20151219以降にて、常にデバッグ用シンボルを収めたパッケージ(名前はパッケージ名-dbgsym)を生成するようになりました。この-dbgsymパッケージの保管先がアナウンスされ、http://debug.mirrors.debian.org/debian-debug/と、http://snapshot.debian.org/archive/debian-debug/になったようです。}

