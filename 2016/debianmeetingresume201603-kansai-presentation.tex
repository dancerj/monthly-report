\documentclass[cjk,dvipdfmx,10pt,compress,%
hyperref={bookmarks=true,bookmarksnumbered=true,bookmarksopen=false,%
colorlinks=false,%
pdftitle={第 108 回 関西 Debian 勉強会},%
pdfauthor={倉敷・のがた・佐々木・かわだ},%
%pdfinstitute={関西 Debian 勉強会},%
pdfsubject={資料},%
}]{beamer}

\title{第 108 回 関西 Debian 勉強会}
\subtitle{$\sim$発表資料$\sim$}
\author[かわだ てつたろう]{{\large\bf 倉敷・のがた・佐々木・かわだ}}
\institute[Debian JP]{{\normalsize\tt 関西 Debian 勉強会}}
\date{{\small 2016 年 3 月 27 日}}

%\usepackage{amsmath}
%\usepackage{amssymb}
\usepackage{graphicx}
\usepackage{moreverb}
\usepackage[varg]{txfonts}
\AtBeginDvi{\special{pdf:tounicode EUC-UCS2}}
\usetheme{Kyoto}
\def\museincludegraphics{%
  \begingroup
  \catcode`\|=0
  \catcode`\\=12
  \catcode`\#=12
  \includegraphics[width=0.9\textwidth]}
%\renewcommand{\familydefault}{\sfdefault}
%\renewcommand{\kanjifamilydefault}{\sfdefault}
\begin{document}
\settitleslide
\begin{frame}
\titlepage
\end{frame}
\setdefaultslide

\begin{frame}[fragile]
  \frametitle{Disclaimer}
  \begin{itemize}
  \item 疑問、質問、ツッコミ、茶々、\alert{大歓迎}
  \item その場でインタラクティブにどうぞ
  \item ハッシュタグ \#kansaidebian
  \end{itemize}
\end{frame}

\begin{frame}[fragile]
\frametitle{Agenda}

\tableofcontents

\end{frame}

\section{最近の Debian 関係のイベント}

\takahashi[40]{最近の Debian\\関係のイベント}

\begin{frame}[fragile]
  \frametitle{第107回関西Debian勉強会}
  \begin{itemize}
  \item 日時: 2月28日(日)
  \item 場所: 福島区民センター
  \end{itemize}
  \begin{block}{内容}
    \begin{itemize}
    \item 「勉強会資料の歩き方」
    \item 「周回遅れでDocker触ってみた」
    \end{itemize}
  \end{block}
\end{frame}

\begin{frame}[fragile]
  \frametitle{第138回東京エリアDebian勉強会}
  \begin{itemize}
  \item 日時: 3月5日(土)
  \item 場所: サイボウズ株式会社東京オフィス
  \end{itemize}
  \begin{block}{内容}
    \begin{itemize}
    \item 「How to become a Debian Developer」
    \item 「tilegxについて(仮)」
    \item 「Debianのインフラを借りるには」
    \end{itemize}
  \end{block}
\end{frame}

\begin{frame}[fragile]
  \frametitle{Debian Project}
  \begin{itemize}
  \item Renaming Iceweasel to Firefox
    \begin{itemize}
    \item MozillaはDebian固有のパッチがIceweasel/Firefoxの品質に影響ないと認識する
    \item Debian stableリリース中でもFirefox ESRのバージョンは更新する
    \item \#354622 の問題は解消され、FirefoxのロゴはDFSGに合致する
    \item jessieではメンテナンスのためIceweaselの名は残る
    \item Icedove/Thunderbirdについても同様の対応をすすめる
    \end{itemize}
  \item DebConf17 to be held in Montreal (CA)
  \item Bits from the Release Team: A Slightly Moveable Feast
    \begin{itemize}
    \item 2016/11/05 transitions freeze
    \item 2017/01/05 "Soft" freeze
    \item 2017/02/05 Full freeze
    \end{itemize}
  \item Archive changes
  \item Linux Kernel ABI report
  \end{itemize}
\end{frame}

\takahashi[50]{そんな\\こんなで}
\takahashi[120]{次}

\section{事前課題}
\takahashi[50]{事前課題}

\begin{frame}[fragile]
  \frametitle{事前課題}
  \begin{block}{今回の事前課題}
    \begin{description}
    \item[事前課題1] systemdで起動するJessie環境を御用意下さい。\\
      幾つか設定を変更することがありますので、VM環境の方が無難かもしれません。
    \end{description}
  \end{block}
\end{frame}

\takahashi[50]{事前課題\\発表}

\begin{frame}
  \frametitle{ lurdan }
  \begin{enumerate}
  \item 16時くらいに参上いたします……
  \end{enumerate}
\end{frame}

\begin{frame}
  \frametitle{ t3rkwd }
  \begin{enumerate}
  \item わかりました。sid環境になるとおもいます。
  \end{enumerate}
\end{frame}

\begin{frame}
  \frametitle{ むんくさん }
  \begin{enumerate}
  \item 承知しました。Note PC 持参します。
  \end{enumerate}
\end{frame}

\begin{frame}
  \frametitle{ 佐々木洋平 }
  \begin{enumerate}
  \item 無駄に、vagrant-kvm に挑戦してみたり。
  \end{enumerate}
\end{frame}

\begin{frame}
  \frametitle{ tsukudamayo }
  \begin{enumerate}
  \item VMの環境です。よろしくお願いします。
  \end{enumerate}
\end{frame}

\begin{frame}
  \frametitle{ 川江 浩 }
  \begin{enumerate}
  \item 了解。
  \end{enumerate}
\end{frame}

\begin{frame}
  \frametitle{ Kazuhiro NISHIYAMA }
  \begin{enumerate}
  \item VirtualBox で環境を用意しました。
  \end{enumerate}
\end{frame}

\begin{frame}
  \frametitle{ Say-no }
  \begin{enumerate}
  \item わかりました。
  \end{enumerate}
\end{frame}

\begin{frame}
  \frametitle{ hiroshi morimoto }
  \begin{enumerate}
  \item よろしくお願いします。
  \end{enumerate}
\end{frame}

\begin{frame}
  \frametitle{ Katsuki Kobayashi }
  \begin{enumerate}
  \item よろしくおねがいします。
  \end{enumerate}
\end{frame}

\takahashi[50]{そんな\\こんなで}
\takahashi[120]{次}

\section{systemdに浸ってみた}
\takahashi[30]{systemdに浸ってみた\\by\\佐々木 洋平}

\section{今後の予定}
\begin{frame}[fragile]
  \frametitle{今後の予定}

  \begin{block}{第109回関西Debian勉強会}
    \begin{itemize}
    \item 日時: 4月22日(土) or 23日(日)
    \item 場所: 福島区民センター
    \end{itemize}
  \end{block}

  \begin{block}{第139回東京エリアDebian勉強会}
    \begin{itemize}
    \item 日時: 4月16日(土)
    \item 場所:
    \end{itemize}
  \end{block}

\end{frame}

\takahashi[50]{  }

\end{document}
%%% Local Variables:
%%% mode: japanese-latex
%%% TeX-master: t
%%% End:
