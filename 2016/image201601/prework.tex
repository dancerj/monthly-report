\begin{prework}{ mkouhei }
  \begin{enumerate}
  \item Q.hack time に何をしますか?\\
    A. https://qa.debian.org/developer.php? login=mkouhei@palmtb.net のメンテ。
  \end{enumerate}
\end{prework}

\begin{prework}{ henrich }
  \begin{enumerate}
  \item Q.hack time に何をしますか?\\
    A. afdko関連のパッケージか、os-autoinstのパッケージあたりを進めておきたい
  \end{enumerate}
\end{prework}

\begin{prework}{ kenhys }
  \begin{enumerate}
  \item Q.hack time に何をしますか?\\
    A. porterboxがこのときまでに使える状態になっていたら\#770243、じゃなかったら別の何か。
  \end{enumerate}
\end{prework}

\begin{prework}{ wskoka }
  \begin{enumerate}
  \item Q.hack time に何をしますか?\\
    A. tilegxのdebian化
  \end{enumerate}
\end{prework}

\begin{prework}{ rosh }
  \begin{enumerate}
  \item Q.hack time に何をしますか?\\
    A. smali のパッケージングを再挑戦します。
  \end{enumerate}
\end{prework}

\begin{prework}{ dictoss }
  \begin{enumerate}
  \item Q.hack time に何をしますか?\\
    A. bluetoothテサリングの設定を調べて、raspberry piをネットワークへ接続できるようにする
  \end{enumerate}
\end{prework}

\begin{prework}{ y.y }
  \begin{enumerate}
  \item Q.hack time に何をしますか?\\
    A. Caff
  \end{enumerate}
\end{prework}

\begin{prework}{ yy\_y\_ja\_jp }
  \begin{enumerate}
  \item Q.hack time に何をしますか?\\
    A. DDTSS
  \end{enumerate}
\end{prework}

\begin{prework}{ 野島 }
  \begin{enumerate}
  \item Q.hack time に何をしますか?\\
    A. DDTSS、東京エリアdebian勉強会のレジメ作り。
  \end{enumerate}
\end{prework}




