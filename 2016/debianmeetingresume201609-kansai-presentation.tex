\documentclass[cjk,dvipdfmx,10pt,compress,%
hyperref={bookmarks=true,bookmarksnumbered=true,bookmarksopen=false,%
colorlinks=false,%
pdftitle={第 114 回 関西 Debian 勉強会},%
pdfauthor={倉敷・のがた・佐々木・かわだ},%
%pdfinstitute={関西 Debian 勉強会},%
pdfsubject={資料},%
}]{beamer}

\title{第 114 回 関西 Debian 勉強会}
\subtitle{$\sim$発表資料$\sim$}
\author[かわだ てつたろう]{{\large\bf 倉敷・のがた・佐々木・かわだ}}
\institute[Debian JP]{{\normalsize\tt 関西 Debian 勉強会}}
\date{{\small 2016 年 9 月 25 日}}

%\usepackage{amsmath}
%\usepackage{amssymb}
\usepackage{graphicx}
\usepackage{moreverb}
\usepackage[varg]{txfonts}
\AtBeginDvi{\special{pdf:tounicode EUC-UCS2}}
\usetheme{Kyoto}
\def\museincludegraphics{%
  \begingroup
  \catcode`\|=0
  \catcode`\\=12
  \catcode`\#=12
  \includegraphics[width=0.9\textwidth]}
%\renewcommand{\familydefault}{\sfdefault}
%\renewcommand{\kanjifamilydefault}{\sfdefault}
\begin{document}
\settitleslide
\begin{frame}
\titlepage
\end{frame}
\setdefaultslide

\begin{frame}[fragile]
  \frametitle{Disclaimer}
  \begin{itemize}
  \item 疑問、質問、ツッコミ、茶々、\alert{大歓迎}
  \item その場でインタラクティブにどうぞ
  \item ハッシュタグ \#kansaidebian
  \end{itemize}
\end{frame}

\begin{frame}[fragile]
\frametitle{Agenda}

\tableofcontents

\end{frame}

\section{最近の Debian 関係のイベント}
\takahashi[40]{最近の Debian\\関係のイベント}

\begin{frame}[fragile]
  \frametitle{第113回関西Debian勉強会}
  \begin{itemize}
  \item 日時: 8月28日(日)
  \item 場所: 福島区民センター
  \end{itemize}
\end{frame}

\begin{frame}[fragile]
  \frametitle{第143回東京エリアDebian勉強会}
  \begin{itemize}
  \item 日時: 9月17日(土)
  \item 場所: 株式会社朝日ネット
  \end{itemize}
  \begin{block}{内容}
    \begin{itemize}
    \item 「DEP5/Machine-readable debian/copyright 再考」
    \end{itemize}
  \end{block}
\end{frame}

\begin{frame}[fragile]
  \frametitle{Debian Project}
  \begin{itemize}
  \item Updated Debian 8: 8.6 released
  \item Declassifying debian-private再び
  \item Support for merged-/usr now in debootstrap; default for stretch?
  \item Introducing default-mysql-* metapackages
  \end{itemize}
\end{frame}

\takahashi[50]{そんな\\こんなで}
\takahashi[120]{次}

\section{事前課題}
\takahashi[50]{事前課題}

\begin{frame}[fragile]
  \frametitle{事前課題}
  \begin{block}{今回の事前課題}
    事前課題はありませんでした。
  \end{block}
\end{frame}

\takahashi[50]{事前課題\\発表}

\begin{frame}
  \frametitle{ lurdan }
\end{frame}

\begin{frame}
  \frametitle{ t3rkwd }
\end{frame}

\begin{frame}
  \frametitle{ むんくさん }
  \begin{enumerate}
  \item JOSM
  \end{enumerate}
\end{frame}

\begin{frame}
  \frametitle{ Yosuke OTSUKI }
  \begin{enumerate}
  \item 東京エリアDebian勉強会に参加
  \item Fortran trie ライブラリを探している
  \end{enumerate}
\end{frame}

\begin{frame}
  \frametitle{ 川江 浩 }
  \begin{enumerate}
  \item HUAWEI MateBook に Debian インストール
  \end{enumerate}
\end{frame}

\takahashi[50]{そんな\\こんなで}
\takahashi[120]{次}

\section{初心者が初めてパッケージをつくってみた}
\takahashi[30]{初心者が初めてパッケージをつくってみた\\by\\Yosuke Otsuki}

\takahashi[50]{そんな\\こんなで}
\takahashi[120]{次}

\section{Let's Encryptのススメ}
\takahashi[30]{Let's Encryptのススメ\\by\\かわだ てつたろう}

\takahashi[50]{そんな\\こんなで}
\takahashi[120]{次}

\section{今後の予定}
\begin{frame}[fragile]
  \frametitle{今後の予定}

  \begin{block}{第115回関西Debian勉強会}
    \begin{itemize}
    \item 日時: 10月23日(日)
    \item 場所: 福島区民センター
    \end{itemize}
  \end{block}

  \begin{block}{第145回東京エリアDebian勉強会}
    \begin{itemize}
    \item 日時: 10月15日(土)
    \item 場所:
    \end{itemize}
  \end{block}

\end{frame}

\takahashi[50]{  }

\end{document}
%%% Local Variables:
%%% mode: japanese-latex
%%% TeX-master: t
%%% End:
