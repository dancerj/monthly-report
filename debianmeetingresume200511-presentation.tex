%; whizzy document
% latex beamer presentation.
% platex, latex-beamer でコンパイルすることを想定。 

%     Tokyo Debian Meeting resources
%     Copyright (C) 2006 Junichi Uekawa

%     This program is free software; you can redistribute it and/or modify
%     it under the terms of the GNU General Public License as published by
%     the Free Software Foundation; either version 2 of the License, or
%     (at your option) any later version.

%     This program is distributed in the hope that it will be useful,
%     but WITHOUT ANY WARRANTY; without even the implied warranty of
%     MERCHANTABILITY or FITNESS FOR A PARTICULAR PURPOSE.  See the
%     GNU General Public License for more details.

%     You should have received a copy of the GNU General Public License
%     along with this program; if not, write to the Free Software
%     Foundation, Inc., 51 Franklin St, Fifth Floor, Boston, MA  02110-1301 USA


\documentclass[cjk,dvipdfmx]{beamer}
\usetheme{Warsaw}
%  preview (shell-command "xpdf debianmeetingresume200511-presentation.pdf&")
%  presentation (shell-command "xpdf -fullscreen debianmeetingresume200511-presentation.pdf&")

%http://www.naney.org/diki/dk/hyperref.html
%日本語EUC系環境の時
\AtBeginDvi{\special{pdf:tounicode EUC-UCS2}}
%シフトJIS系環境の時
%\AtBeginDvi{\special{pdf:tounicode 90ms-RKSJ-UCS2}}


\title[Debian 勉強会クイズ問題]{Debian勉強会クイズ}
\subtitle{2005年11月12日版}
\author{上川}
\date{2005年11月12日}

% 三択問題用
\newcounter{santakucounter}
\newcommand{\santaku}[5]{%
\addtocounter{santakucounter}{1}
\frame{\frametitle{問題\arabic{santakucounter}. #1}
%問題\arabic{santakucounter}. #1
\begin{itemize}
\item □ A #2\\
\item □ B #3\\
\item □ C #4\\
\end{itemize}
}
\frame{\frametitle{問題\arabic{santakucounter}. #1}
%問題\arabic{santakucounter}. #1
\begin{itemize}
\item □ A #2\\
\item □ B #3\\
\item □ C #4\\
\end{itemize}
\vfill{}
#5
}
}


\begin{document}
\frame{\titlepage{}}
\section{DWNQ}

%% debianmeetingresume200511.texから以下コピー
\subsection{2005年44号}

\url{http://www.debian.org/News/weekly/2005/44/}
にある11月1日版です。

\santaku{Nathanael Nerodeがi386のサポートについて何を宣言したか}
{そろそろi386CPUも世界から駆逐できたので、無くす}
{gccがi386サポートを復活させたので、etchでは真のi386マシンでも動くリリー
スが作れるかも知れない}
{i386向けのバイナリをとうとう生成できなくなってしまったので、今後は対応
はしないことになる。}
{B}

\santaku{Jay Berkenbiltがlibtoolの依存関係からパッケージの依存関係を計算するためのスクリプトを作
成したいという宣言を出した際に、問題になるだろうと指摘されたのは}
{multiarchの場合の.laファイルが複数あるケースをどう扱うか}
{思想的にそういうことはしてはならないことになっているので話題にあげるな}
{自動で依存関係を解決するなんてそんな危険なことをしてはならない}
{A}

\santaku{opensslの新しいバージョンがアップロードされたがその影響は}
{opensslに必要だったセキュリティー対応が取り込まれた}
{opensslがより安定したバージョンになった}
{結果として300以上のパッケージをリビルドする必要がある}
{C}

\santaku{berlinuxでは何が展示されていなかったか}
{nokia 770 で動いているDebian}
{Debianで制御している鉄道模型 }
{Debianを使った炊飯器}
{C}

\santaku{IETFのRFCについてSimon Josefssonは何をしようとしてい
るか}
{IETFからRFCを解放する運動を開始したい}
{RFCを利用しなくても世界がまわるようにあたらしい規格システムを提案、普及
推進}
{RFCのライセンスをフリーソフトウェアに使いやすいように変更するための署名
運動}
{C}

\santaku{openafs-module-sourceの機構で問題になっているのは何か}
{インストールしても動かないため、それをごまかすための機構}
{ユーザに使い方を説明するためのマニュアルが大きすぎること}
{アップグレードする際にモジュールを自動で構築するように求める機構}
{C}

\santaku{自動テストの結果を上流に還元する方法についてDaniel Jacobwitzは
何を提案したか}
{パッケージのビルド中に結果を出力するようにする}
{結果ログを上流に自動でメール送信するように設定する}
{頑張って手動で確認する}
{A}

\santaku{Debian packageで、postinstの処理がパッケージ毎に実行されてしま
うために遅い場合がある。
その処理を複数パッケージ分まとめて実行する方法にはなにがあるか}
{cygwinのsetup.exeを利用する}
{そんな方法はない。}
{zopeはaptのpost-installを利用している。}
{C}

\santaku{gnomeメタパッケージがgnome-gamesに依存していることに出た苦情は}
{政府機関ではゲームが禁止されているのでインストールされると困る。}
{gnome-gamesは大きすぎる}
{gnome-gamesはおもしろくないので、もっとおもしろいゲームを提供すべきだ。}
{A}

\subsection{2005年45号}
\url{http://www.debian.org/News/weekly/2005/45/}
にある11月8日版です。

\santaku{Linux-Info-Tagのブースではネットワーク障害があったが、どうやって対応し
たか}
{ノートパソコンの電源さえあればみんなハッピーだった}
{ノートパソコンにDebianのミラーがはいっていたので、ネットワークは特に問
題ではなかった。}
{通信衛星を利用してネットワークを仮稼働させたので問題にはならなかった}
{B}

\santaku{Robert Milanが発表したgingは何をするものか?}
{CDROMから起動するDebian GNU/kFreeBSD}
{CDROMから起動するWindows}
{CDROMから起動するHurd}
{A}

\santaku{ 次回の debconf の開催期間はいつか}
{2005年5月14日から22日}
{2007年5月14日から22日}
{2006年5月14日から22日}
{C}

\santaku{opensslを利用しているGPLのプログラムをgnutlsを利用するようにするには}
{気合いで書き直す}
{opensslを使い続けるほうが利点があるので、GPLのプログラムのライセンスを
変えてしまう}
{gnutls/openssl.hという互換レイヤーがあるのでそれを利用すればよい}
{C}

\santaku{\url{http://popcon.debian.org/}のトップページで見れない情報は何か}
{最近各アーキテクチャにおいてのpopconの利用者が急激に増えている}
{popconのバージョンがリリースされた際にインストールされている数がどう遷
移するか}
{どのパッケージが一番人気があるか}
{C}

\end{document}