\documentclass[cjk,dvipdfmx,10pt,compress,%
hyperref={bookmarks=true,bookmarksnumbered=true,bookmarksopen=false,%
colorlinks=false,%
pdftitle={第 84 回 関西 Debian 勉強会},%
pdfauthor={倉敷・のがた・佐々木・かわだ・八津尾},%
%pdfinstitute={関西 Debian 勉強会},%
pdfsubject={資料},%
}]{beamer}

\title{第 84 回 関西 Debian 勉強会}
\subtitle{$\sim$発表資料$\sim$}
\author[かわだ てつたろう]{{\large\bf 倉敷・のがた・佐々木・かわだ・八津尾}}
\institute[Debian JP]{{\normalsize\tt 関西 Debian 勉強会}}
\date{{\small 2014 年 5 月 25 日}}

%\usepackage{amsmath}
%\usepackage{amssymb}
\usepackage{graphicx}
\usepackage{moreverb}
\usepackage[varg]{txfonts}
\AtBeginDvi{\special{pdf:tounicode EUC-UCS2}}
\usetheme{Kyoto}
\def\museincludegraphics{%
  \begingroup
  \catcode`\|=0
  \catcode`\\=12
  \catcode`\#=12
  \includegraphics[width=0.9\textwidth]}
%\renewcommand{\familydefault}{\sfdefault}
%\renewcommand{\kanjifamilydefault}{\sfdefault}
\begin{document}
\settitleslide
\begin{frame}
\titlepage
\end{frame}
\setdefaultslide

\begin{frame}[fragile]
  \frametitle{Disclaimer}
  \begin{itemize}
  \item 疑問、質問、ツッコミ、茶々、\alert{大歓迎}
  \item その場でインタラクティブにどうぞ
  \item ハッシュタグ \#kansaidebian
\end{itemize}
\end{frame}

\begin{frame}[fragile]
\frametitle{Agenda}

\tableofcontents

\end{frame}

\section{最近の Debian 関係のイベント}

\takahashi[40]{最近の Debian\\関係のイベント}

\begin{frame}[fragile]
  \frametitle{第83回関西Debian勉強会}
  \begin{itemize}
  \item 日時: 4月27日(日)
  \item 場所: 福島区民センター
  \end{itemize}
  \begin{block}{内容}
    \begin{itemize}
    \item 「自宅サーバにKVMを導入してみよう」
    \item 「Notmuch Mail」
    \end{itemize}
  \end{block}
\end{frame}

\begin{frame}[fragile]
  \frametitle{第113回東京エリアDebian勉強会}
  \begin{itemize}
  \item 日時: 5月17日(土)
  \item 場所: 株式会社スクウェア・エニックス 会議室
  \end{itemize}
  \begin{block}{内容}
    \begin{itemize}
    \item 「debian と docker.io」
    \item もくもくの会
    \end{itemize}
  \end{block}
\end{frame}

\begin{frame}[fragile]
  \frametitle{Debian Project (1/3)}
  \begin{block}{}
    \begin{itemize}
    \item Code of Conduct
    \item systemd/GNOME Sprint
    \item リリースに向けて
    \item CTTE
    \end{itemize}
  \end{block}
\end{frame}

\begin{frame}[fragile]
  \frametitle{Debian Project (2/3)}
  \begin{block}{debian-devel}
    \begin{itemize}
    \item Call for help from KDE Team
    \item Ghostscript licensing changed to AGPL
    \item Removal of emacs23 from unstable/testing
    \item systemd-fsck?
    \item Media type vnd.debian.binary-package accepted by the IANA.
    \end{itemize}
  \end{block}
\end{frame}

\begin{frame}[fragile]
  \frametitle{Debian Project (3/3)}
  \begin{block}{debian-devel}
    \begin{itemize}
    \item Bug\#747596: ITP: liblxqt-mount
    \item Bug\#747597: ITP: lxqt-config
    \item Bug\#747598: ITP: lxqt-config-randr
    \item Bug\#747599: ITP: lxqt-common
    \item Bug\#747600: ITP: lxqt-about
    \item Bug\#747601: ITP: lxqt-globalkeys
    \item Bug\#747602: ITP: lxqt-notificationd
    \item Bug\#747603: ITP: lxqt-openssh-askpass
    \item Bug\#747604: ITP: lxqt-qtplugin
    \item Bug\#747605: ITP: pcmanfm-qt
    %% \item Bug\#747607: ITP: lxqt-policykit
    %% \item Bug\#747608: ITP: lxqt-session
    %% \item Bug\#747610: ITP: lxqt-panel
    %% \item Bug\#747611: ITP: lxqt-powermanagement
    %% \item Bug\#747613: ITP: lxqt-runner
    %% \item Bug\#747620: ITP: liblxqt

      …
    \end{itemize}
  \end{block}
\end{frame}


\takahashi[50]{そんな\\こんなで}
\takahashi[120]{次}

\section{事前課題発表}

\takahashi[50]{事前課題}

\begin{frame}[fragile]
  \frametitle{事前課題}
  \begin{block}{今回の事前課題}
    \begin{description}
    \item[事前課題1]
      もくもくの会で行なう作業、質問などの課題を用意して教えてください。
    \item[事前課題2]
      前回(第83回)の勉強会に参加された方は、前回の作業や課題がその後どう
      なったか結果を教えてください。
    \item[事前課題3]
      LT(ライトニングトーク) 歓迎です。何かお話したい方はタイトルを下さい。
    \end{description}
  \end{block}
\end{frame}

\takahashi[50]{事前課題\\発表}

\begin{frame}
  \frametitle{ かわだてつたろう }
  \begin{enumerate}
  \item 何か用意しておきます。
  \item 参加できませんでした。Notmuch Mail...
  \item いつものDebian Projectの話題を紹介します。
  \end{enumerate}
\end{frame}

\begin{frame}
  \frametitle{ murase\_{}syuka }
  \begin{enumerate}
  \item mrubyパッケージ更新
  \item 不参加
  \item 特に無し
  \end{enumerate}
\end{frame}

\begin{frame}
  \frametitle{ takata }
  \begin{enumerate}
  \item いつも質問ばかりですみません。

    Wake on LANで PCを起動する場合、暗号化LUKSパーティションのパスワードを安全に入力するにはどうするのがよいでしょう?

    マジックパケットを使った Wake on LANでは、パスワードをペイロードとして投げるような機能はないようにみえます。

    あらかじめ以下のような方法で LUKSパスワードの入力を省略するように設定しておき、Wake on LANではマジックパケットを投げつけて電源ONにするだけにする、とかでしょうか。

    \url{http://www.howtoforge.com/automatically-unlock-luks-encrypted-drives-with-a-keyfile}
  \end{enumerate}
\end{frame}

\begin{frame}
  \frametitle{ 佐々木洋平 }
  参加します。

  ruby 関連の移行がメイン、かな。
\end{frame}

\begin{frame}
  \frametitle{ 川江 }
  \begin{enumerate}
  \item とりあえず、HTML5でのサイトの作成と、自宅サーバの整備。
  \item 同上
  \end{enumerate}
\end{frame}

\begin{frame}
  \frametitle{ lurdan }
  \begin{enumerate}
  \item いくつか BTS たまってきてるのでそれを片す、もしくは webwml-git の件をもうちょっと
  \end{enumerate}

  所用のため途中で抜けます
\end{frame}

\begin{frame}
  \frametitle{ yyatsuo }
\end{frame}

\begin{frame}
  \frametitle{ Hiroyuki Nagata }
  \begin{enumerate}
  \item D言語をなんか書くかo2onのポーティング作業
  \item JaneCloneをITPしました
  \item とくになしです
  \end{enumerate}
\end{frame}

\takahashi[50]{そんな\\こんなで}
\takahashi[120]{次}

\section{もくもくの会}
\takahashi[30]{もくもくの会}

\takahashi[50]{そんな\\こんなで}
\takahashi[120]{次}

\section{今後の予定}
\begin{frame}[fragile]
\frametitle{今後の予定}

\begin{block}{第85回関西Debian勉強会}
  \begin{itemize}
  \item 日時: 6月22日(日) 13:30 -
  \item 場所: 福島区民センター
  \end{itemize}
\end{block}

\begin{block}{第114回東京エリアDebian勉強会}
  \begin{itemize}
  \item 日時: 6月14日(土)
  \item 場所: 株式会社スクウェア・エニックス セミナールーム
  \item 内容: 「GPG 秘密鍵取り扱い方法の提案」
  \end{itemize}
\end{block}

\end{frame}

\takahashi[50]{  }

\end{document}
%%% Local Variables:
%%% mode: japanese-latex
%%% TeX-master: t
%%% End:
