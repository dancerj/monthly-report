

\begin{prework}{ 原口秀康 }

\begin{enumerate}
\item デフォルトの2を使用しています。
通常のインストールをしてそのまま使用しているからです。
2-5全て同じなのに他のランレベルを使用する意味はわかっていません

\item apt-get apptitude、「パッケージの内容を検索」を使用します。
明確に名前が分かるものはapt-getやapptitudeで
こういうものがほしいと思うときは「パッケージの内容を検索」を使います。
キーワードが使われていないパッケージでも検索することが可能だからです
\end{enumerate}

\end{prework}



\begin{prework}{ キタハラ }
\begin{enumerate}

\item デフォルトで組み込まれる「init」、特に不都合がないため。
\item 最近は、「Google先生」に尋ねる事のほうが多い。

\end{enumerate}

\end{prework}



\begin{prework}{ yama1066 }
\begin{enumerate}

\item sysvinit
まだ移行してないから。
upstart に移行するメリットが議論できればいいなと思います。
\item dselect でひたすらゴリゴリ…、嘘です。apt-cache search でほげほげ。
\end{enumerate}

\end{prework}

\begin{prework}{ henrich }
\begin{enumerate}
\item sysvinit を使ってます。昔は initng を使ったこともありましたが…
面倒を避けるため、ですかね。
\end{enumerate}

\end{prework}



\begin{prework}{ koedoyoshida }

\begin{enumerate}
\item 安定志向なのでlenny標準のinitです。
仕事で使ってるマシンはupstartやinitが混在しています。

私は基本的にLinuxはサーバ使用です。
滅多に再起動しないのであまり恩恵を受けていません。

特にメーカー系のサーバ機は(再)起動時にSAS,FC等を含めたBIOSチェックだけで数分かかるのもざらなので、起動高速化のメリットは受けにくいというのが正直なところです。

init以外を使用することによる、起動時以外のメリットが有れば知りたいです。

\end{enumerate}

\end{prework}



\begin{prework}{ 鈴木崇文 }
\begin{enumerate}
\item sysvinit。
特にスピードを求めようと思ったことがないことと、サービス起動が非同期だと何か問題が起きたときに調査しにくいイメージがあるため。
とはいえ、UbuntuのマシンではUpstart使っているので、ただ流されているだけだと思います。
\end{enumerate}

\end{prework}



\begin{prework}{ 藤沢理聡(risou) }
\begin{enumerate}
\item 今使用しているのは sysvinit です。理由は、 lenny のデフォルトになっているからです……。
#このあたり、違いがよくわかってないので、わざわざ変更してないです。
\end{enumerate}

\end{prework}



\begin{prework}{ 村田信人 }

\begin{enumerate}
\item sysvinit。少し前にSqueezeをインストールしたらデフォルトだったから。
\item  apt-cache searchでざっくりと見当をつけてからapt-cache showで詳細を確認。
\end{enumerate}
\end{prework}

\begin{prework}{ akedon }

\begin{enumerate}
\item sysvinit です。現在、デフォルトなのとメカニズムが単純なのでこれで良いかなと思っています。
\item aptitude search 〜 と apt-file search 〜 を使っています。
\end{enumerate}
\end{prework}

\begin{prework}{ Hirotaka Kawata }

\begin{enumerate}
\item init。Debian 標準の init (ふつうの init)
\item aptitude search "keyword"
\end{enumerate}
\end{prework}

\begin{prework}{ opentaka }

\begin{enumerate}
\item デフォルトのinit。デフォルトで入っていたので。
\item aptitude search "package name"
\end{enumerate}
\end{prework}

\begin{prework}{ 松澤二郎 }
\begin{enumerate}
\item あまり意識していませんでしたが、デフォルトのものを使っています。sysvinit?
\item aptitude search hoge
\end{enumerate}
\end{prework}

\begin{prework}{ まえだこうへい }

\begin{enumerate}
\item sysvinit。テスト環境とかではupstartも試してますが、MacBookのSid環境とかでは切り替えるのまだ怖いですね。
\item apt-cache
\end{enumerate}
\end{prework}

\begin{prework}{ Yasunori Higashiyama }

\begin{enumerate}
\item sysvinit。特に困っていないのでそのまま。
\item webで検索かaptitude search
\end{enumerate}
\end{prework}

\begin{prework}{ 岩松 信洋 }

\begin{enumerate}
\item sysvinit です。Debianのデフォルトだからです。
\item aptitude で検索しています 。タグを使います。パッケージのインストー
      ルはapt ですが....。あとは、iceweaselやGoogle Chromeの検索ボックスで検索すること
      もあります。
\end{enumerate}
\end{prework}


\begin{prework}{ google-account@rolf.leggewie.biz }

今回もよろしくお願いします。

\end{prework}


