%; whizzy-master ../debianmeetingresume201201.tex
% 以上の設定をしているため、このファイルで M-x whizzytex すると、
% whizzytexが利用できます

\begin{prework}{上川 純一}

テーマ案です。

\begin{itemize}
 \item Android/iOSをDebianから使う (USB device driver, wifi, bluetooth,
       adb などなど)
 \item 電話でdebianを使う
 \item タブレットでDebianを使う
 \item Debianからクラウドサービスを使う (EC2 API とか)
 \item ウェブアプリケーション開発をDebianでする
 \item Debianのウェブブラウザ
 \item nodejs
 \item Debianでのjavascriptの開発・パッケージ
 \item clojure / Debian
\end{itemize}
\end{prework}

\begin{prework}{まえだこうへい}

\begin{itemize}
  \item DebianでOpenStack
  \item Python2, 3、両方に対応したパッケージ作成
  \item Pythonのテストツール
  \item Vyattaネタ(Debianベース、という意味で)
  \item webOSネタ(Ubuntuベース、という意味で)
\end{itemize}
\end{prework}

\begin{prework}{岩松 信洋}

\begin{itemize}
  \item DebianでAR
  \item DebianでArduino
  \item Debian で動画配信
  \item Debian専用ルール/方言をまとめてみた(Apacheの設定など)
\end{itemize}
\end{prework}

\begin{prework}{ koedoyoshida }

\preworksection{2012年の勉強会のテーマとして各月なにをするべきか提案してください。 }
予定は未定

\preworksection{自分がどのテーマを担当したいか提案してください。}
出来そうなことを出来る範囲で


\end{prework}

\begin{prework}{ dictoss(杉本 典充) }

\preworksection{2012年の勉強会のテーマとして各月なにをするべきか提案してください。}
トランジションの流れ(自分がよくわかってないので)、Debianインストーラ(kfreebsdだとよくコケるので要所をつかみたい)

\preworksection{自分がどのテーマを担当したいか提案してください。}
Debianインストーラ?
\end{prework}

\begin{prework}{ yamamoto }

\preworksection{2012年の勉強会のテーマとして各月なにをするべきか提案してください。}
聞きたいネタ、ランキング
1位 debhelper 9、 ついにリリース。ずばり徹底解説。
2位 みんな、新しいフリーソフトウェア、どこで見つけてる?
3位 ここが出る!DD 試験傾向と対策。

\preworksection{自分がどのテーマを担当したいか提案してください。}
むー、実に難しい課題ですなw
\end{prework}

\begin{prework}{ 野島 貴英 }

\preworksection{2012年の勉強会のテーマとして各月なにをするべきか提案してください。}
 2月: Desktop KDE、3月:Desktop GNOME、4月:Desktop その他、5月: WEBとクラウド環境、6月: DBとKVS、7月:オフィススイート環境、 8月:組み込み用途と変わったデヴァイス、9月:学術系ソフト、10月:スマホアプリ開発、11月:マルチメディア/CG、12月: 作ったものLT大会 と、キャッチーなテーマをいってみるテスト。あ、もちろん全部Debianな話で。

\preworksection{自分がどのテーマを担当したいか提案してください。}
 GNOMEと、マルチメディアかな...

\end{prework}

\begin{prework}{ 野首 }

\preworksection{2012年の勉強会のテーマとして各月なにをするべきか提案してください。}
VCSを使うbuildpackage(svn-buildpackage, git-buildpackage等)の話

\preworksection{自分がどのテーマを担当したいか提案してください。}
Windows Severとの連携 (krb5認証とか、LDAP検索とか)

\end{prework}
