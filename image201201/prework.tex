%; whizzy-master ../debianmeetingresume201201.tex
% 以上の設定をしているため、このファイルで M-x whizzytex すると、
% whizzytexが利用できます

\begin{prework}{上川 純一}

テーマ案です。

\begin{itemize}
 \item Android/iOSをDebianから使う (USB device driver, wifi, bluetooth,
       adb などなど)
 \item 電話でdebianを使う
 \item タブレットでDebianを使う
 \item Debianからクラウドサービスを使う (EC2 API とか)
 \item ウェブアプリケーション開発をDebianでする
 \item Debianのウェブブラウザ
 \item nodejs
 \item Debianでのjavascriptの開発・パッケージ
 \item clojure / Debian
\end{itemize}
\end{prework}

\begin{prework}{まえだこうへい}
\begin{itemize}
  \item DebianでOpenStack
  \item Python2, 3、両方に対応したパッケージ作成
  \item Pythonのテストツール
  \item Vyattaネタ(Debianベース、という意味で)
  \item webOSネタ(Ubuntuベース、という意味で)
\end{itemize}
\end{prework}

\begin{prework}{岩松 信洋}
\begin{itemize}
  \item DebianでAR
  \item DebianでArduino
  \item Debian で動画配信
  \item Debian専用ルール/方言をまとめてみた(Apacheの設定など)
\end{itemize}
\end{prework}

% TODO: paste prework here.
