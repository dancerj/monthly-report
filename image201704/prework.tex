\begin{prework}{ yy\_y\_ja\_jp }
  \begin{enumerate}
  \item DDTSS
  \item 最近はfirefoxの拡張パッケージを新しいバージョンに入れ替えるぐらいです
  \end{enumerate}
\end{prework}

\begin{prework}{ nabaua }
  \begin{enumerate}
  \item その日の考えます。
  \item カスタマイズはないです。
  \end{enumerate}
\end{prework}

\begin{prework}{ 大神祐真 }
  \begin{enumerate}
  \item QEMU上で動作する自作のOSをDebianパッケージ化してみたいと思います。
  \item カスタマイズとは呼ばないかもしれませんが、最新のメインラインカーネルの機能を使ってみたく、Linusのツリーをcloneしてmake-kpkgでdeb生成、インストールしていました。
  \end{enumerate}
\end{prework}

\begin{prework}{ dictoss }
  \begin{enumerate}
  \item RFAになっているsnmpttのupstreamおよびパッケージの中身の調査
  \item backportsで問題ないため、既存パッケージのカスタマイズはしていない。
  \end{enumerate}
\end{prework}

\begin{prework}{ Roger Shimizu }
  \begin{enumerate}
  \item パッケージメインテインとか、バグ潰しなどするつもり
  \item 現在特にありません
  \end{enumerate}
\end{prework}

\begin{prework}{ hatochan }
  \begin{enumerate}
  \item debianインストールまたはlive-helpをごにょごにょ
  \item firmware関連のパッケージを試行錯誤
  \end{enumerate}
\end{prework}

\begin{prework}{ kenhys }
  \begin{enumerate}
  \item パッケージのメンテナンス
  \item 独自にメンテするのが面倒なので基本はしない 必要ならフィードバックしてパッケージ側に反映してもらう
  \end{enumerate}
\end{prework}

\begin{prework}{ Marc Dequènes (Duck) }
  \begin{enumerate}
  \item BSPとかメンター作業など
  \item backportとかpending fixなどのためにカスタマイズをよくします。
  \end{enumerate}
\end{prework}

\begin{prework}{ koedoyoshida }
  \begin{enumerate}
  \item Debian関係作業、PyConJP関係作業
  \item 最近は無し
  \end{enumerate}
\end{prework}

\begin{prework}{ takaswie }
  \begin{enumerate}
  \item libhinawa の新リリースの準備作業
  \item network-manager。

    dpkg-divert で/usr/lib/NetworkManager/conf.d/

    10-globally-managed-devices.conf を退避。

    https://bugs.launchpad.net/ubuntu/+source/

    network-manager/+bug/1638842 の問題のため。
  \end{enumerate}
\end{prework}
