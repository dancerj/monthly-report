% Created 2011-09-25 日 16:25
\documentclass[10pt,final,c,dvipdfmx,cjk,colorlinks=false]{beamer}
\usepackage[utf8]{inputenc}
\usepackage[T1]{fontenc}
\usepackage{fixltx2e}
\usepackage{graphicx}
\usepackage{longtable}
\usepackage{float}
\usepackage{wrapfig}
\usepackage{soul}
\usepackage{textcomp}
\usepackage{marvosym}
\usepackage{wasysym}
\usepackage{latexsym}
\usepackage{amssymb}
\usepackage{hyperref}
\tolerance=1000
\AtBeginSection[]{\begin{frame}<beamer>\frametitle{Outline}\tableofcontents[currentsection]\end{frame}}
\providecommand{\alert}[1]{\textbf{#1}}

\title{git-buildpackage (again)}
\author{佐々木洋平/Youhei SASAKI}
\date{\footnotesize{2011-08-28}}

\AtBeginDvi{\special{pdf:tounicode EUC-UCS2}}
\hypersetup{pdftitle={VCS-buildpackage, Git 編},bookmarks}
\usetheme{Kyoto}
\usepackage[varg]{txfonts}
\usefonttheme{professionalfonts}
%\renewcommand{\kanjifamilydefault}{\gtdefault}
\setbeamertemplate{itemize/enumerate subbody end}{\vspace{0.5\baselineskip}}
\renewcommand{\maketitle}{\settitleslide\begin{frame}\titlepage\end{frame}\setdefaultslide}
\begin{document}

\maketitle

\begin{frame}
\frametitle{Outline}
\setcounter{tocdepth}{3}
\tableofcontents
\end{frame}

\section{はじめに}
\label{sec-1}
\begin{frame}[fragile,squeeze]
\frametitle{ }
\label{sec-1-1}

\end{frame}
\takahashi[50]{遅れて}
\takahashi[50]{大変}
\takahashi[70]{申し訳ありません}
\begin{frame}{謝罪ばっかで大変申し訳ありません}
\begin{center}
{\huge{河田さんありがとう!!}}
\end{center}
\end{frame}
\begin{frame}[fragile,squeeze]
\frametitle{話の枕}
\label{sec-1-2}


\begin{itemize}
\item 山下さんの bzr 編に引き続き、Git を使う場合のお話
\item 基本, 前回および前々回の続き(?)
\begin{itemize}
\item 前々々回(2011年6月, 第48回): \texttt{git-buildpackage}
\item 前回(2011年8月, 第50回): パッケージ作成入門
\end{itemize}
\item 幾つかコマンド/パッケージが追加されていて, それを試してみたり.
\end{itemize}
\end{frame}
\begin{frame}[fragile,squeeze]
\frametitle{前提とする知識と目的}
\label{sec-1-3}
\begin{block}{前提知識}
\label{sec-1-3-1}

\begin{itemize}
\item source package についてのある程度の知識
\item Git に関して, 特に tag と branch についてのある程度の知識
\begin{itemize}
\item 適宜質問して下さい
\end{itemize}
\end{itemize}
\end{block}
\begin{block}{今日の目的}
\label{sec-1-3-2}

\begin{itemize}
\item 何をしているのか/何ができるのか/できないのか
\begin{itemize}
\item 特に source format 3.0 (quilt) との連携
\end{itemize}
\end{itemize}
\end{block}
\end{frame}
\begin{frame}[fragile,squeeze]
\frametitle{パッケージ作成作業のサイクル(復習)}
\label{sec-1-4}


\begin{itemize}
\item 通常、パッケージ作成は
\begin{enumerate}
\item upstream のソースを取得
\item (場合によっては) non-free な部分を除いたりして、
\item \texttt{./debian} ディレクトリ以下を作成/更新して、
\item 場合によっては upstream のソースにパッチを当てて、
\item ソース/バイナリ パッケージをビルド
\end{enumerate}
\end{itemize}

という事を行ないます。
\end{frame}
\begin{frame}[fragile,squeeze]
\frametitle{典型的なリポジトリレイアウト}
\label{sec-1-5}


じゃあ, どういう風に Git で作業するのか, というと

\begin{commandline}
  $ git branch
  * master             <-- debian/ 入りのフルソース
    pristine-tar       <-- orig.tar.{gz,bz2} のバイナリデルタ
    upstream           <-- debian/ 無し(upstream)のソース
\end{commandline}

という構成のリポジトリ内で作業します.
\end{frame}
\begin{frame}[fragile,squeeze]
\frametitle{Git で Debian パッケージを管理するには?}
\label{sec-1-6}


\begin{itemize}
\item 以下のパッケージが提供されています
\begin{itemize}
\item git-buildpackage
\item git-dpm
\item gitpkg
\item topgit
\end{itemize}
\item パッケージ毎にそれぞれ特徴があります
\begin{itemize}
\item 他の vcs-buildpackage と同じ様に使うなら
    \texttt{git-buildpackage} を使うと良い
\end{itemize}
\end{itemize}
\end{frame}
\section{upstream ソースを import するには?}
\label{sec-2}
\begin{frame}[fragile,squeeze]
\frametitle{upstream ソースを import するには?}
\label{sec-2-1}


\begin{itemize}
\item 既存の source package を import する時は以下の通りに
\end{itemize}
\begin{commandline}
 $ git-import-dsc [some dsc files]
\end{commandline}
\begin{itemize}
\item じゃあ, upstream の tarball や VCS から
  始めるには?
\end{itemize}
\end{frame}
\begin{frame}[fragile,squeeze]
\frametitle{upstream ソースを import するには?}
\label{sec-2-2}


\begin{itemize}
\item simple な場合は
\begin{itemize}
\item tarball を展開して import \newline
    or
\item upstream の VCS を import
\end{itemize}
\item 調整が必要な場合
\begin{itemize}
\item non-dfsg-free な部分を削除してから import
\end{itemize}
\item 気をつける必要があるのは
\begin{itemize}
\item \alert{\textbf{正確な tarball の copy を保存}}: \texttt{pristine-tar}
\end{itemize}
\end{itemize}
\end{frame}
\begin{frame}[fragile,squeeze]
\frametitle{pristine-tar ?}
\label{sec-2-3}


\begin{itemize}
\item upstream の tarball を import する際にバイナリデルタを保存
\item \texttt{upstream} ブランチから tarball(\texttt{.orig.tar.gz}) を
  生成する際に、checksum の等しい tarball を生成
\begin{itemize}
\item バイナリデルタは \texttt{pristine-tar} ブランチに保存される.
\item import 時に忘れた場合は以下の様に\ldots{}
\end{itemize}
\end{itemize}
\begin{commandline}
  $ pristine-tar commit foobar.tar.gz [upstream-ref]
  $ pristine-tar checkout ../foobar.tar.gz
\end{commandline}

\begin{itemize}
\item ごく稀に駄目な tarball がある(らしい)けれど, 多くの場合上手く動作
\end{itemize}
\end{frame}
\begin{frame}[fragile,squeeze]
\frametitle{upstream の VCS から import する}
\label{sec-2-4}


\begin{itemize}
\item \alert{\textbf{tarballも一緒にimportすること!!}}
\begin{itemize}
\item 特定の branch に upstream のコミット履歴を import する
\end{itemize}
\item その Top に tarball を import する
\begin{itemize}
\item import したコミット履歴が diff として管理される
\begin{itemize}
\item コミット履歴 $\to$ quilt パッチとして扱える!!
\end{itemize}
\end{itemize}
\item import したらタグを打っておくべき
\begin{itemize}
\item git-buildpackage 等で便利
\end{itemize}
\end{itemize}
\end{frame}
\begin{frame}[fragile,squeeze]
\frametitle{upstream の VCS から import する(1)}
\label{sec-2-5}


幸運にも upstream が Git だったら

\begin{commandline}
$ git remote add upstream-repos [url]
$ git fetch upstream-repos
$ git co upstream && git merge  [upstream tag]
\end{commandline}

で ok
\end{frame}
\begin{frame}[fragile,squeeze]
\frametitle{upstream の VCS から import する(2)}
\label{sec-2-6}

\begin{itemize}
\item Subversion: 初回
\end{itemize}
\begin{commandline}
$ git-svn init [url]
$ git svn fetch
$ git log ref/remotes/git-svn
$ git checkout -b upstream refs/remotes/git-svn
$ git push origin upstream:upstream
\end{commandline}
\begin{itemize}
\item Subversion: 二回目以降
\end{itemize}
\begin{commandline}
$ git config --remove-section svn-remote.svn 1>/dev/null 2>&1
$ git svn init [url]
$ git show-ref origin/upstream > \
   `git rev-parse-git-dir`/refs/remotes/git-svn
\end{commandline}

\ldots{}面倒ですねぇ\ldots{}
\end{frame}
\begin{frame}[fragile,squeeze]
\frametitle{tarball を import するツール(1)}
\label{sec-2-7}
\begin{block}{\texttt{git-import-orig}}
\label{sec-2-7-1}

\begin{itemize}
\item \texttt{git-buildpackage} パッケージで提供
\begin{itemize}
\item simple に tarball を import
\item (Optionつければ) pristine-tar も実行
\item (あれば)現状の master ブランチへ自動で merge して
\item タグも打ってくれる
\end{itemize}
\end{itemize}
\end{block}
%% 一番 simple
\label{sec-2-7-2}


必要な事は全てやってくれるので, これで十分な事が多い
\end{frame}
\begin{frame}[fragile,squeeze]
\frametitle{tarball を import するツール(2)}
\label{sec-2-8}
\begin{block}{\texttt{git-dpm import-new-upstream}}
\label{sec-2-8-1}

\begin{itemize}
\item \texttt{git-dpm}: \alert{git Debian package manager} パッケージで提供
\item 動作は git-import-orig とほぼ同じ
\item manager と言うだけあって, 多機能
\begin{itemize}
\item VCS の履歴との対応づけ
\item \texttt{patch-queue}ブランチ(後述)
        の生成/管理, など
\end{itemize}
\end{itemize}
\end{block}
\end{frame}
\begin{frame}[fragile,squeeze]
\frametitle{調整が必要な場合(1)}
\label{sec-2-9}


\begin{itemize}
\item upstream ブランチで non-dfsg-free な部分を削除/調整
\item new upstream version として merge/commit
\item tarball として repack した後に import/タグ打ち
\end{itemize}

\begin{commandline}
$ git checkout upstream
$ git merge -s recursive -X theirs [upstream tag]
\end{commandline}

もしくは

\begin{commandline}
$ git status -s | egrep '^(DU|UA| U|UD)' | cut -c4- | \
    xargs git rm --ignore-unmatch DUMMY$$
$ git commit
\end{commandline}

とか?
\end{frame}
\begin{frame}[fragile,squeeze]
\frametitle{調整が必要な場合(2)}
\label{sec-2-10}


\begin{itemize}
\item tarball 展開して修正/再 tarball 化して, それを import する, でも良い
\item \texttt{uscan} で repack する, 等の script があるなら,
  それで tarball 生成してから import, とか.
\end{itemize}
\end{frame}
\section{\texttt{./debian} をガシガシ書く/修正する}
\label{sec-3}
\begin{frame}[fragile,squeeze]
\frametitle{\texttt{./debian} をガシガシ書く/修正する}
\label{sec-3-1}


\begin{commandline}
  $ git branch
  * master             <-- debian/ 入りのフルソース
    pristine-tar       <-- orig.tar.{gz,bz2} のバイナリデルタ
    upstream           <-- debian/ 無し(upstream)のソース
\end{commandline}

\begin{itemize}
\item upstream のソースは全て Git リポジトリ内に
\begin{itemize}
\item \texttt{upstream} ブランチ
\end{itemize}
\item \texttt{./debian} での変更は \texttt{master} ブランチで
\begin{itemize}
\item 全ての変更は \texttt{master} 内で行なう
\item 何をしたのかは \texttt{git log} で容易に追跡できる
\begin{itemize}
\item patch も作成しやすい
\end{itemize}
\end{itemize}
\end{itemize}
\end{frame}
\section{patch を扱うには?}
\label{sec-4}
\begin{frame}[fragile,squeeze]
\frametitle{source format 3.0 (quilt)}
\label{sec-4-1}


\begin{itemize}
\item source format 3.0 (quilt)
\begin{itemize}
\item upstream への変更点は quilt パッチとして
    \texttt{./debian/patches} 以下に格納
\end{itemize}
\item VCS とどう仲良くするか/管理するか?
\end{itemize}
\end{frame}
\begin{frame}[fragile,squeeze]
\frametitle{単純な方法}
\label{sec-4-2}


\begin{itemize}
\item Git の事は忘れて quilt だけで.
\item 複雑な事は何もない
\item 特に得することも無いけれど
\end{itemize}
\end{frame}
\begin{frame}[fragile,squeeze]
\frametitle{git を使う場合(1)}
\label{sec-4-3}


\begin{itemize}
\item quilt を忘れて git だけで
\begin{itemize}
\item \texttt{master} ブランチでの変更,
\item \texttt{upstream} ブランチとの差分 \newline
\end{itemize}
$\to$ 全て patch として \texttt{debian/patches} 以下に抽出
\begin{itemize}
\item これらを \texttt{master} ブランチにコミットしていく
\begin{itemize}
\item \texttt{dpkg-source} で自動的に apply/unpatch される
\end{itemize}
\end{itemize}
\item 変更点の track が面倒
\item VCS っぽくない
\end{itemize}
\end{frame}
\begin{frame}[fragile,squeeze]
\frametitle{patch ブランチで}
\label{sec-4-4}


\begin{itemize}
\item パッチを track するための branch を用意
\begin{itemize}
\item \texttt{patch-queue} ブランチ
\item \texttt{master} は \texttt{./debian} だけを管理
\item ソースへの変更点は別途ブランチを作成
\item 1 パッチ/1 ブランチ or 1パッチ/1コミット
\end{itemize}
\item パッチを track するブランチの内容を quilt に export
\item export した patch 群を \texttt{master} へ取り込む
\item quilt への export を行なうには
  コミット履歴が\alert{綺麗}でないといけない
\begin{itemize}
\item 1 パッチ/1コミット
\item squash !! squash !! squash !!
\end{itemize}
\item 今のところ 1-way rebase
\end{itemize}
\end{frame}
\begin{frame}[fragile,squeeze]
\frametitle{patch も Git で管理}
\label{sec-4-5}
\begin{block}{topgit: a Git patch queue manager}
\label{sec-4-5-1}

\begin{itemize}
\item コミット履歴をブランチとして管理
\item パッチ間の依存関係も管理
\item 便利だけど, やりすぎな気もしないでもない
\item \texttt{patch-queue} ブランチから quilt へ export したパッチは
  \texttt{master} ブランチにコミットしておく必要がある
\end{itemize}
\end{block}
\end{frame}
\begin{frame}[fragile,squeeze]
\frametitle{\texttt{patch-queue} の内容を \texttt{master} へ(1)}
\label{sec-4-6}


\begin{itemize}
\item \texttt{git-dpm} だと自動的に行なわれる
\item \texttt{gitpkg}, \texttt{gbp-pq}:
\begin{itemize}
\item \texttt{git merge} の wrapper
\end{itemize}
\item \texttt{dpkg-source} は自動で patch を生成/適応する
\begin{itemize}
\item source format 3.0 (quilt) から
\end{itemize}
\end{itemize}
\end{frame}
\begin{frame}[fragile,squeeze]
\frametitle{\texttt{patch-queue} の内容を \texttt{master} へ(2)}
\label{sec-4-7}
\begin{block}{gbp-pq}
\label{sec-4-7-1}

\begin{itemize}
\item \texttt{git-buildpackage} で提供
\item \texttt{git format-patch} の wrapper
\item 1パッチ/1コミット, として patch を生成/取り込み
\item \texttt{master} を rebase
\end{itemize}
\end{block}
%% 使い方
\label{sec-4-7-2}

\begin{commandline}
$ git checkout master ; git branch -D patch-queue
$ quilt pop -a
$ gbp-pq import
 ... 作業 ...
$ git checkout master ; gbp-pq export
\end{commandline}
\end{frame}
\begin{frame}[fragile,squeeze]
\frametitle{\texttt{patch-queue} の内容を \texttt{master} へ(3)}
\label{sec-4-8}
\begin{block}{git-dpm での patch の取り扱い}
\label{sec-4-8-1}

\begin{itemize}
\item パッチは一つのブランチで管理
\item 1パッチ/1コミット
\item パッチは \texttt{master} ブランチに merge されたままで管理
\item パッチが当たった \texttt{upstream} ブランチを rebase
\item プライベートブランチの SHA1 ハッシュを
      \texttt{./debian/.git-dpm} に保存
\end{itemize}
\end{block}
\end{frame}
\begin{frame}[fragile,squeeze]
\frametitle{\texttt{patch-queue} の内容を \texttt{master} へ(4)}
\label{sec-4-9}
\begin{block}{gitpkg の quilt export hook}
\label{sec-4-9-1}

\begin{itemize}
\item 1パッチ/1コミット, などという制限は無い
\item \texttt{debian/source/git-patches} に設定を書く
\end{itemize}
    \begin{commandline}
    upstream/[UPSTREAM_REF]...patche-queue1/[DEBIAN_REF1]
    upstream/[UPSTREAM_REF2]...topic1/[DEBIAN_REF2]
    \end{commandline}
\begin{itemize}
\item パッチはコミットされない
\item tag は再生成される
\end{itemize}
\end{block}
\end{frame}
\section{source package の生成}
\label{sec-5}
\begin{frame}[fragile,squeeze]
\frametitle{source package の生成}
\label{sec-5-1}
\begin{block}{git-dpm}
\label{sec-5-1-1}

\begin{itemize}
\item ビルド用の特定のコマンドは無い(\texttt{dpkg-source -b} とか)
\end{itemize}
\end{block}
\begin{block}{gitpkg}
\label{sec-5-1-2}

\begin{itemize}
\item pristine-tar, \texttt{upstream} ブランチから tarball を生成し
      source package をビルド
\end{itemize}
\end{block}
\begin{block}{git-buildpackage}
\label{sec-5-1-3}

\begin{itemize}
\item default. バイナリパッケージも作成する
\item \texttt{git-pbuilder}: pbuilder/cowbuilder を呼び出せる
\item タグを打ったり.
\end{itemize}
\end{block}
\end{frame}
\section{まとめ(?)}
\label{sec-6}
\begin{frame}[fragile,squeeze]
\frametitle{まとまってませんが\ldots{}}
\label{sec-6-1}


\begin{itemize}
\item git-buildpackage が一番簡単/便利/移行コストも低い
\begin{itemize}
\item 既に幾つかの project が Git でパッケージを管理している
\item workflow もわかりやすい
\end{itemize}
\item git-dpm/gitpkg
\begin{itemize}
\item workflow/patch-queue の自由度は高い(けれど, 複雑になりがち)
\item git-dpm
\begin{itemize}
\item コマンドが多くて, ちょっと敷居が高い(かも)
\item 一番「git らしく」作業できる(らしい)
\end{itemize}
\item gitpkg
\begin{itemize}
\item hook での拡張/カスタマイズが容易.
\item リポジトリのレイアウトも固定されていない
\end{itemize}
\end{itemize}
\end{itemize}
\end{frame}

\end{document}