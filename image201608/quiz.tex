%; whizzy-master ../debianmeetingresume201311.tex
% 以上の設定をしているため、このファイルで M-x whizzytex すると、whizzytexが利用できます。
%

\santaku
{unstableにてGCCがTransitionされました。さて、バージョンは何にTransitionされたでしょうか}
{5.4}
{6.1}
{7.0}
{B}
{debian-devel-announceにてGCC-6へTransitionされました。GCC-6ではC++14が実装されています。GNU Webサイトの「Porting to GCC 6」の記事を確認して新GCCへ移行してください。メール:「Transition news: GCC 6 enabled by default」\url{https://lists.debian.org/debian-devel-announce/2016/08/msg00001.html}}

\santaku
{unstalbeにてperlがTransitionされました。さて、バージョンは何にTransitionされたでしょうか}
{5.22}
{5.24}
{6.0}
{B}
{2016年5月にリリースされたperl-5.24へTransitionされました。2015年クリスマスにはPerl-6リリースの大ニュースがありましたが、Debianで採用されるには先が長そうです。メール:「transition: perl」\url{https://bugs.debian.org/cgi-bin/bugreport.cgi?bug=830200}}

\santaku
{Linux Kernelにおいて次のLTS(=LongTermSupport)とするバージョンの候補が上がってきました。さてバージョンは何でしょうか}
{4.8}
{4.9}
{4.10}
{B}
{stretchのリリース計画ではlinux-4.10を採用する予定としたためフリーズ時期を遅らせることにしていました。LTSなカーネルのリリースが早いからといってすでにアナウンス済のフリーズ時期が早まることはないと思いますが、注視しておいたほうがいいかもしれません。メール:「Linux 4.9 will be next LTS」\url{https://lists.debian.org/debian-release/2016/08/msg00147.html}}
