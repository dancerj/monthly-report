%; whizzy document
% latex beamer presentation.
% platex, latex-beamer でコンパイルすることを想定。 

%     Tokyo Debian Meeting resources
%     Copyright (C) 2006 Junichi Uekawa

%     This program is free software; you can redistribute it and/or modify
%     it under the terms of the GNU General Public License as published by
%     the Free Software Foundation; either version 2 of the License, or
%     (at your option) any later version.

%     This program is distributed in the hope that it will be useful,
%     but WITHOUT ANY WARRANTY; without even the implied warranty of
%     MERCHANTABILITY or FITNESS FOR A PARTICULAR PURPOSE.  See the
%     GNU General Public License for more details.

%     You should have received a copy of the GNU General Public License
%     along with this program; if not, write to the Free Software
%     Foundation, Inc., 51 Franklin St, Fifth Floor, Boston, MA  02110-1301 USA


\documentclass[cjk,dvipdfmx]{beamer}
\usetheme{Warsaw}
%  preview (shell-command "xpdf debianmeetingresume200512-presentation.pdf&")
%  presentation (shell-command "xpdf -fullscreen debianmeetingresume200512-presentation.pdf&")

%http://www.naney.org/diki/dk/hyperref.html
%日本語EUC系環境の時
\AtBeginDvi{\special{pdf:tounicode EUC-UCS2}}
%シフトJIS系環境の時
%\AtBeginDvi{\special{pdf:tounicode 90ms-RKSJ-UCS2}}


\title[Debian 勉強会クイズ問題]{Debian勉強会クイズ}
\subtitle{2005年12月10日版}
\author{上川}
\date{2005年12月10日}

% 三択問題用
\newcounter{santakucounter}
\newcommand{\santaku}[5]{%
\addtocounter{santakucounter}{1}
\frame{\frametitle{問題\arabic{santakucounter}. #1}
%問題\arabic{santakucounter}. #1
\begin{itemize}
\item □ A #2\\
\item □ B #3\\
\item □ C #4\\
\end{itemize}
}
\frame{\frametitle{問題\arabic{santakucounter}. #1}
%問題\arabic{santakucounter}. #1
\begin{itemize}
\item □ A #2\\
\item □ B #3\\
\item □ C #4\\
\end{itemize}
\vfill{}
#5
}
}


\begin{document}
\frame{\titlepage{}}
\section{DWNQ}

%% debianmeetingresume200512.texから以下コピー
\subsection{2005年46号}
\url{http://www.debian.org/News/weekly/2005/46/}
にある11月15日版です。

\santaku
{Debian armebの進捗はどうか}
{やっとgcc/glibc/binutilsが移植された}
{ほとんどのパッケージが移植されている}
{まだ起動もしていない}
{B}

\santaku
{DevJamでJavaの現状について議論があった。その際の認識はどうだったか}
{まだフリーなjavaで全てを実装できていないので、動かないものがある}
{フリーなJavaは充分利用できる状況で、それだけで全てが充足できる。}
{フリーなJavaは全く利用出来ない状態}
{A}

\santaku
{Clam Antivirus について Marc Haberが発表したのは}
{15分毎に更新を確認して、あたらしくなっていたら自動で
volatile.debian.net にアップロードする}
{更新は手動で確認して、メンテナが暇なときにアップデートする。新しいデー
タを常に欲しい人は、頑張って自分でアップデートすること。}
{データ量が多いため、更新はしないので、各自がんばって更新してください。}
{A}

\santaku
{debian-installer etch betaが出ました。Joey Hessがこんなに時間がかかったことに
ついて言明したのは}
{めんどくさかったので放置していたので、こんなに時間がかかりました}
{10位の項目についてそれぞれで3日づつ遅延要因になるため、一月くらいは
遅れるはめになる}
{ちゃんとハックできる人が参加していないので、コードの品質が下がったため、
こんなに時間がかかりました。}
{B}

\santaku
{SugarCRMはMPL1.1をベースとしたライセンスで配布されている。そのライセン
スはフリーだろうか}
{MPLはMozillaのライセンスなので、その時点でフリーだ}
{ウェブページにフリーソフトだ、と書いてあるので、フリーだ。}
{改変した場合に名前を利用できないことになっているので、名前を変更すれば
よいだろう}
{C}

\santaku
{Debconfの発表資料をDFSGフリーにしようという提案についてAnthony Townsが
した反論は}
{MLでのスレッドなどDFSGフリーでないコンテンツは多数ある。全てがそうである必要はない。}
{ライセンスなんてつけるだけ無駄なので、つけないほうがよいでしょう。}
{あらゆるものはDFSGフリーどころか、全部GPLであるべきなので、GPL以外のラ
イセンスは考えるのもおこがましい。}
{A}

\santaku
{Gabor Gombas さんが、複数の-devパッケージがconflictすることについて苦情
を出した。その対応は}
{-devパッケージがインストールできないのは問題なので、上流のやっている内
容を改変して共存できるようにするのがよい}
{opensslとgnutlsをまぜるほうがライセンス的に適切なので、両方がリンクされ
たパッケージを作る}
{includeファイルのパスなどは開発用のAPIの一部であり、同じパスを利用する
複数の-devパッケージはconflictして当然だ。}
{C}

\santaku
{ping が Linux 専用である点についての議論で、FreeBSDやHurdでも動作させる
ためにパッチを適用することに対してはどういう意見が出たか}
{今後のDebianの一貫性を維持するためにはするべきだ}
{pingなんてBSD上でははやらないのでなくしてもよい}
{あきらかにforkしているため、メンテナンスが大変になる}
{C}

\subsection{2005年47号}
\url{http://www.debian.org/News/weekly/2005/47/}
にある11月22日版です。

\santaku
{Matthias Klose がg++について発表したのは何か}
{g++は今後D言語用のコンパイラによって置き換えられるので、C++なんて古い言
語をつかうのはもうやめろ}
{g++のメモリアロケータが変わるため、またg++ で生成されたライブラリの
ABIが変更になる}
{g++は最適化するために今後はマクロの展開処理を省略する。そのために文法が
若干変更になる}
{B}

\santaku
{Anthony Towns が -private メーリングリストについて提案したのは}
{3年たったら一般公開する}
{存在自体を抹消する}
{即時公開メーリングリストにする}
{A}

\santaku
{Branden RobinsonがDPLについて何ができるかという説明文を発表した。その条
文はいくつあるか}
{3}
{10}
{120}
{B}

\santaku
{Enrico Ziniが発表した新しい検索エンジンでは何をもってパッケージを検索できるか}
{2chの過去ログ情報を用いて検索}
{debtags情報を使って検索}
{popconの利用頻度情報を使って検索}
{B}

\santaku
{Ian Jacksonが提案したのは何か}
{パッケージの自動テストのためのスクリプトインタフェース}
{パッケージを受け入れるときのための基準}
{パッケージの品質をあげるための魔法}
{A}

\santaku
{Christopher Berg が発表した、メンテナ向けのパッケージ一覧ページの新機能
でないのは}
{パッケージがどれくらい人気あるのかということを確認できる}
{パッケージがどれくらいよい品質なのかが確認できる}
{一覧で確認できるパッケージを任意に追加できる}
{B}

\santaku
{PHPライセンスについてSteve Langasek の考えは}
{PHPを使うこと自体がまず問題だ}
{PHP自体については問題ないが、PHP以外にそのライセンスを適用するのには問
題がある}
{PHPライセンスは本当にDFSGフリーなのかどうかはグレーだ}
{B}

\subsection{2005年48号}
\url{http://www.debian.org/News/weekly/2005/48/}
にある11月29日版です。

\santaku
{Freetype に関して何が起きる、とSteve Langasekは宣言したか}
{誰も使っていないので、パッケージを削除する}
{ABIに変更があったので、5のパッケージが移行する必要がある}
{ABIに変更があったので、600のパッケージが移行する必要がある。}
{C}

\santaku
{sbuildの最新版はバージョンが 1.0-1 のパッケージに対してのbinary NMU番号をどうつけてくれるようになったか}
{1.0-1+b1}
{1.0-1.1}
{1.0-1.0.1}
{A}

\santaku
{Frank K\"uster は、パッケージのconffileへの変更の反映を管理者が拒否し、その結
果 postinst が失敗になることについて、問題ないだろう、と質問した。それに
対しての Petter Reinholdtsen の対応は}
{そういうエラーは管理者が拒否するのが問題なので、管理者を日勤教育するべきだ}
{そのような問題は存在しない}
{そういう場合には、設定ファイルを動作に必須なものとローカルで管理者がオー
バライドする部分とに分離することを提案する}
{C}

\santaku
{vserverは何をするものか}
{chrootなどの技術を応用し、複数の仮想サーバコンテキストを作成してくれて、Linux上で複数のサーバを仮
想的に提供できる}
{サインは}
{サーバの統合管理のためのツール}
{A}

\subsection{2005年49号}
\url{http://www.debian.org/News/weekly/2005/49/}
にある12月6日版です。

\santaku
{Manoj SrivastavaがGRの議論期間を宣言した。今回の議論は何についてか}
{-private メーリングリストの一般公開について}
{-devel メーリングリストの秘密化について}
{-mentors メーリングリストの会員制化について} 
{A}

\santaku
{テンポラリディレクトリについての議論があり、ユーザ毎にテンポラリディレ
クトリを持つことがよいのではないかという結論が出た。
ユーザ毎にテンポラリディレクトリを持つ際にその機能を実装してくれるのは}
{/etc/profile でテンポラリディレクトリの作成}
{initスクリプトでのディレクトリの作成}
{pam-tmpdir というPAMモジュール }
{C}

\santaku
{C++のメモリアロケータの移行でまだ移行できていない、ということでさらしあ
げになった日本の開発者は}
{mhattaさんと土屋さん}
{gniibeさんと鵜飼さん}
{えとーさんと岩松さん}
{A}

\santaku
{パッケージがどのバージョン(unstable, stable, testing)用に作成されたのか
を確認する簡単な方法がないか、という質問に対しての Marc Brockschmidt の
回答は何だったか}
{パッケージのバージョン番号を見ればわかる}
{パッケージのchangelogを見ると、どのバージョン用にビルドしたのか、という
ことは確認できる。}
{Debianのパッケージはほとんど全てが一旦はunstableにあったことがあり、
testingとstableに入るため、パッケージがどれ用につくられるというものでは
ない。}
{C}


\end{document}