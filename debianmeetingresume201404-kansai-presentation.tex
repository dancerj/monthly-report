\documentclass[cjk,dvipdfmx,10pt,compress,%
hyperref={bookmarks=true,bookmarksnumbered=true,bookmarksopen=false,%
colorlinks=false,%
pdftitle={第 82 回 関西 Debian 勉強会},%
pdfauthor={倉敷・のがた・佐々木・かわだ・八津尾},%
%pdfinstitute={関西 Debian 勉強会},%
pdfsubject={資料},%
}]{beamer}

\title{第 83 回 関西 Debian 勉強会}
\subtitle{$\sim$発表資料$\sim$}
\author[かわだ てつたろう]{{\large\bf 倉敷・のがた・佐々木・かわだ・八津尾}}
\institute[Debian JP]{{\normalsize\tt 関西 Debian 勉強会}}
\date{{\small 2014 年 4 月 27 日}}

%\usepackage{amsmath}
%\usepackage{amssymb}
\usepackage{graphicx}
\usepackage{moreverb}
\usepackage[varg]{txfonts}
\AtBeginDvi{\special{pdf:tounicode EUC-UCS2}}
\usetheme{Kyoto}
\def\museincludegraphics{%
  \begingroup
  \catcode`\|=0
  \catcode`\\=12
  \catcode`\#=12
  \includegraphics[width=0.9\textwidth]}
%\renewcommand{\familydefault}{\sfdefault}
%\renewcommand{\kanjifamilydefault}{\sfdefault}
\begin{document}
\settitleslide
\begin{frame}
\titlepage
\end{frame}
\setdefaultslide

\begin{frame}[fragile]
  \frametitle{Disclaimer}
  \begin{itemize}
  \item 疑問、質問、ツッコミ、茶々、\alert{大歓迎}
  \item その場でインタラクティブにどうぞ
  \item ハッシュタグ \#kansaidebian
\end{itemize}
\end{frame}

\begin{frame}[fragile]
\frametitle{Agenda}

\tableofcontents

\end{frame}

\section{最近の Debian 関係のイベント}

\takahashi[40]{最近の Debian\\関係のイベント}

\begin{frame}[fragile]
  \frametitle{第82回関西Debian勉強会}
  \begin{itemize}
  \item 日時: 3月23日(日)
  \item 場所: 福島区民センター
  \end{itemize}
  \begin{block}{内容}
    \begin{itemize}
    \item 「Debian で楽しむ 3D プリンティング」
    \item もくもくの会
    \end{itemize}
  \end{block}
\end{frame}

\begin{frame}[fragile]
  \frametitle{第112回東京エリアDebian勉強会}
  \begin{itemize}
  \item 日時: 4月19日(土)
  \item 場所: 株式会社スクウェア・エニックス 会議室
  \end{itemize}
  \begin{block}{内容}
    \begin{itemize}
    \item 「Golang アプリケーション Debian パッケージ」
    \item もくもくの会
    \end{itemize}
  \end{block}
\end{frame}

\begin{frame}[fragile]
  \frametitle{Debian Project}
  \begin{itemize}
  \item Debian Project Leader Election 2014
  \item Code of Conduct
  \item Long term support for Debian 6.0 Announced
  \end{itemize}
\end{frame}

\takahashi[50]{そんな\\こんなで}
\takahashi[120]{次}

\section{事前課題発表}

\takahashi[50]{事前課題}

\begin{frame}[fragile]
  \frametitle{事前課題}
  \begin{block}{今回の事前課題}
    \begin{description}
    \item[事前課題1]
      エミュレータ、仮想化ソフト等を問わず、自分が普段使っているOSと違う
      「OS、もしくはソフト」を使ったことがあるか否か。使用したことのある
      方はその「不満な点を一つ以上上げてください」を教えてください。
    \item[事前課題2]
      もくもくの会で行なう作業、質問などの課題を用意して教えてください。
    \item[事前課題3]
      前回(第82回)の勉強会に参加された方は、前回の作業や課題がその後どう
      なったか結果を教えてください。
    \item[事前課題4]
      LT(ライトニングトーク) 歓迎です。何かお話したい方はタイトルを下さい。
    \end{description}
  \end{block}
\end{frame}

\takahashi[50]{事前課題\\発表}

\begin{frame}
  \frametitle{ のがたじゅん }
  \begin{enumerate}
  \item Windows。ライセンスが面倒。複数用意しづらい。Linux的作業環境を整えようとすると面倒。
  \end{enumerate}
\end{frame}

\begin{frame}
  \frametitle{ takata }
  \begin{enumerate}
  \item 不満な点
    \begin{enumerate}
    \item 仮想エミュレータ全般(kvm, VirtualBox, VMware, etc.)
      環境を立ち上げるのに一手間必要。
      仮想化環境によってはUSBデバイスなどの物理デバイスが使えない場合がある。
    \item chroot環境
      お手軽な反面、仮想化は不十分。
    \item LXC(Linux Containers)
      Debianの場合、設定が少し面倒(Ubuntuに比べて)。
    \end{enumerate}
  \item 第80回勉強会に参加したときの結果について簡単にご報告します。
    \begin{itemize}
    \item 無事、Windows8ノート機(UEFI+GPT)に Debianがインストールできました。
    \item 結局、Windows BootManagerからブートするのはあきらめて、直接Grubから起動することとしました。
    \end{itemize}
  \end{enumerate}
\end{frame}


\begin{frame}
  \frametitle{ 木下 }
  \begin{enumerate}
  \item 使用経験:あり

    \begin{itemize}
    \item VMwarePlayer
      \begin{itemize}
      \item ネットワークデバイスの設定がやりにくくなった。
      \item モッサリ感がある。
      \end{itemize}
    \item VMwareServer ※最近は使わなくなったので割愛
    \item VirtualBox
      \begin{itemize}
      \item 設定によるのかもしれませんが、空きメモリを一気に奪われてしまう。
      \end{itemize}
    \item Eucalyptus
      \begin{itemize}
      \item VMではありますが、リソースが物理マシンに依存
        物理CPU:2、物理MEM:2GBとした場合、VMwarePlayerやVirtualBoxでは上記構成のVMを複数稼働可能ですが(実運用としてはどうかと思う節もありますが・・・)、Eucalyptusは上記構成のVMは1台のみしか稼働できないようです。
      \item インスタンスの記録を手動で行う必要がある
      \end{itemize}
    \end{itemize}

  \item もくもくの会で行なう作業

    \begin{enumerate}
    \item DistCCの調査、研究
    \item Eucalyptus(プライベートクラウドとして)の調査、研究
    \item クリッドコンピューティング関連の調査、研究
      \begin{itemize}
      \item GlobusToolkitで何ができる?
        AndroidOSのクロスコンパイルで使えたら嬉しいかも。
      \end{itemize}
    \item Debian7 on PANDABOARDの調査、研究
      \begin{itemize}
      \item WiFiモジュール(On Board:TI製)の有効化
      \item GPUデバイスドライバの有効化
      \end{itemize}
    \end{enumerate}

  \item 前回(第82回)の勉強会の結果

    \begin{enumerate}
    \item クリッドコンピューティング関連の調査、研究
      \begin{itemize}
      \item GlobusToolkitで何ができる?→AndroidOSのクロスコンパイルで使えたら嬉しいかも。
      \end{itemize}
      \begin{description}
      \item[実績] 保留
      \end{description}
    \item Eucalyptus(プライベートクラウドとして)の調査・研究
      \begin{description}
      \item[実績] インスタンスの起動に成功→起動できなかった原因は、KVMが動作に必要な物理マシンのBIOS設定に誤りがあった。
      \item[課題] 現状、インスタンスを終了させると、今までの結果をすべて忘れてしまうので、インスタンスの記録が必要ですが、現在記録はできるようになりましたが、記録したインスタンスでの起動に失敗してしまう。
      \end{description}
    \item Debian7 on PANDABOARDの調査・研究
      \begin{itemize}
      \item WiFiモジュール(On Board:TI製)の有効化
      \item GPUデバイスドライバの有効化
      \end{itemize}
      \begin{description}
      \item[実績] 保留
      \end{description}
    \end{enumerate}
  \end{enumerate}
\end{frame}

\begin{frame}
  \frametitle{ 山城の国の住人 久保博 }
  \begin{enumerate}
  \item あります。
    \begin{description}
    \item[Windows 7] 自分が普段使っている OS になりりつつあること。
    \item[Windows XP] セキュリティ修正がリリースされなくなったこと。
    \item[iOS] 操作するのにキーボードがないこと。
    \item[Android] おなじくキーボードがないこと。
    \item[その他組込機器] 自分ではどうしようもないこと
    \end{description}
  \item Xen のお勉強。望むらくは xl コマンドで domU を作って起動したい。
  \item 前回は fcm2 のbug \# 647440の修正バッチを bug 報告に対して投げましたが、なしのつぶてです。
  \end{enumerate}
\end{frame}

\begin{frame}
  \frametitle{ yyatsuo }
  \begin{enumerate}
  \item %
    \begin{description}
    \item[組込のリアルタイムOS] μITRON系、VxWorks等
    \item[不満] OSそのものに不満はありません
    \end{description}
    \begin{description}
    \item[NW機器のOS] IOS、ScreenOS等
    \item[不満] 微妙にUnix系のコマンドが使えたり使えなかったりする
    \end{description}
  \item %
    \begin{itemize}
    \item fcitx-skk のパッケージ修正
    \item kernel-handbook の日本語訳の品質向上
    \end{itemize}
  \item %
    \begin{itemize}
    \item fcitx-skk のパッケージング
    \item RFS して岩松さんにスポンサーになってもらった
    \end{itemize}
  \end{enumerate}
\end{frame}

\begin{frame}
  \frametitle{ 川江 }
  \begin{enumerate}
  \item ここ、2年ほどXenを使ってました。Xenは準仮想化がデフォルトなので、Widowsなどの仕様の違うOSを使えないのが不満です。
  \item HTMの仕様を調べてみる
  \item spiceでの接続には成功しました。実演する予定です。
  \end{enumerate}
\end{frame}

\begin{frame}
  \frametitle{ lurdan }
  \begin{enumerate}
  \item あります。vagrant いいですね。proxmox イケてますね。
  \item webwml-sync の検討をします
  \item python-social-auth は RFS 済み
  \end{enumerate}
\end{frame}

\begin{frame}
  \frametitle{ Hiroyuki Nagata }
  \begin{enumerate}
  \item 自作の2ちゃんブラウザ、JaneCloneをパッケージとしてDebianに提出する方法など聞く
  \item JaneCloneを久しぶりにDebianでビルドしてパッケージを作る
  \item o2onの移植作業を進める
  \end{enumerate}
\end{frame}

\begin{frame}
  \frametitle{ Kiwamu Okabe }
  \begin{enumerate}
  \item ATS関連のパッケージ作業
  \item なし
  \item Debianと関連が低いですが、ATS言語 http://jats-ug.metasepi.org/ についてならいくつかしゃべれます
  \end{enumerate}
\end{frame}

\begin{frame}
  \frametitle{ Hideaki Oose }
  \begin{enumerate}
  \item kfreebsdでjdk1.7の環境構築
  \item kfreebsdでオンボード無線LANによるDHCP接続に成功しました。
  \item ifconfigによるwlandevのcreateと、wpa\_supplicantによるdriverのloadについてLTします。
  \end{enumerate}
\end{frame}

\takahashi[50]{そんな\\こんなで}
\takahashi[120]{次}

\section{自宅サーバにKVMを導入してみよう}
\takahashi[30]{自宅サーバに\\KVMを導入してみよう\\by\\川江 浩}

\takahashi[50]{そんな\\こんなで}
\takahashi[120]{次}

\section{Notmuch Mail}
\takahashi[30]{Notmuch Mail\\by\\David Bremner}

\takahashi[50]{そんな\\こんなで}
\takahashi[120]{次}

\section{もくもくの会}
\takahashi[30]{もくもくの会}

\takahashi[50]{そんな\\こんなで}
\takahashi[120]{次}

\section{今後の予定}
\begin{frame}[fragile]
\frametitle{今後の予定}

\begin{block}{第84回関西Debian勉強会}
  \begin{itemize}
  \item 日時: 5月25日(日) 13:30 -
  \item 場所: 福島区民センター
  \end{itemize}
\end{block}

\begin{block}{第113回東京エリアDebian勉強会}
  \begin{itemize}
  \item 日時: 5月17日(土)
  \item 場所、内容: 未定
  \end{itemize}
\end{block}

\end{frame}

\takahashi[50]{  }

\end{document}
%%% Local Variables:
%%% mode: japanese-latex
%%% TeX-master: t
%%% End:
