%; whizzy paragraph -pdf xpdf -latex ./whizzypdfptex.sh
%; whizzy-paragraph "^\\\\begin{frame}"
% latex beamer presentation.
% platex, latex-beamer でコンパイルすることを想定。 

%     Tokyo Debian Meeting resources
%     Copyright (C) 2009 Junichi Uekawa
%     Copyright (C) 2009 Nobuhiro Iwamatsu

%     This program is free software; you can redistribute it and/or modify
%     it under the terms of the GNU General Public License as published by
%     the Free Software Foundation; either version 2 of the License, or
%     (at your option) any later version.

%     This program is distributed in the hope that it will be useful,
%     but WITHOUT ANY WARRANTY; without even the implied warreanty of
%     MERCHANTABILITY or FITNESS FOR A PARTICULAR PURPOSE.  See the
%     GNU General Public License for more details.

%     You should have received a copy of the GNU General Public License
%     along with this program; if not, write to the Free Software
%     Foundation, Inc., 51 Franklin St, Fifth Floor, Boston, MA  02110-1301 USA

\documentclass[cjk,dvipdfmx,12pt]{beamer}
\usetheme{Tokyo}
\usepackage{monthlypresentation}

%  preview (shell-command (concat "evince " (replace-regexp-in-string
%  "tex$" "pdf"(buffer-file-name)) "&")) 
%  presentation (shell-command (concat "xpdf -fullscreen " (replace-regexp-in-string "tex$" "pdf"(buffer-file-name)) "&"))
%  presentation (shell-command (concat "evince " (replace-regexp-in-string "tex$" "pdf"(buffer-file-name)) "&"))

%http://www.naney.org/diki/dk/hyperref.html
%日本語EUC系環境の時
\AtBeginDvi{\special{pdf:tounicode EUC-UCS2}}
%シフトJIS系環境の時
%\AtBeginDvi{\special{pdf:tounicode 90ms-RKSJ-UCS2}}

\newenvironment{commandlinesmall}%
{\VerbatimEnvironment
  \begin{Sbox}\begin{minipage}{1.0\hsize}\begin{fontsize}{8}{8} \begin{BVerbatim}}%
{\end{BVerbatim}\end{fontsize}\end{minipage}\end{Sbox}
  \setlength{\fboxsep}{8pt}
% start on a new paragraph

\vspace{6pt}% skip before
\fcolorbox{dancerdarkblue}{dancerlightblue}{\TheSbox}

\vspace{6pt}% skip after
}
%end of commandlinesmall

\title{DDTP 及び DDTSS の紹介}
\subtitle{東京エリア・関西合同 Debian 勉強会} % 第 218 回
\author{Norimitsu SUGIMOTO (杉本 典充) \\dictoss@live.jp}
\date{2023-02-18}
\logo{\includegraphics[width=8cm]{image-assets/openlogo-light.eps}}

\begin{document}

\section{表紙}

\begin{frame}
  \titlepage{}
\end{frame}


\section{目次}

\begin{frame}{アジェンダ}
  \begin{itemize}
  \item 自己紹介
  \item DDTP
  \item DDTSS
  \item まとめ
  \item 参考資料
  \end{itemize}
\end{frame}


\section{自己紹介}

\begin{frame}{自己紹介}
  \begin{itemize}
  \item Norimitsu SUGIMOTO (杉本 典充)
  \item dictoss@live.jp
  \item Twitter: @dictoss
  \item Debianを使い始めたのは3.1 sargeがtestingの頃
  \item 仕事はソフトウェア開発者をやってます
  \item python と Django の組み合わせで使うことが多いです
  \end{itemize}
\end{frame}


\section{DDTP}
\emtext{DDTP}

\begin{frame}{DDTP}
  \begin{itemize}
  \item The Debian Description Translation Project
  \item Debian パッケージの説明文の翻訳を取り組むプロジェクト
  \item \url{https://www.debian.org/international/l10n/ddtp.ja.html}
  \item 翻訳状況の統計情報、翻訳作業の支援ツールを提供
  \item Debian パッケージの翻訳を行うと以下の場所などに翻訳文が表示される
    \begin{itemize}
    \item \url{https://packages.debian.org/ja/}
    \item apt show packagename
    \end{itemize}
  \end{itemize}
\end{frame}

\begin{frame}{DDTP}
  \begin{itemize}
  \item パッケージ説明文翻訳の言語別の進捗
    \begin{itemize}
    \item \url{https://ddtp.debian.org/stats/stats-sid.html}
    \item Required、Important、Standard、Optional、Extra の区分は Debian パッケージの優先度\footnote{\url{https://www.debian.org/doc/manuals/debian-faq/pkg-basics.ja.html\#priority}}
      \begin{itemize}
      \item Required、Important、Standard のパッケージは利用者が多いため優先的に翻訳するとよい
      \end{itemize}
    \item Popcon500、PopconRank は popularity-contest\footnote{\url{https://popcon.debian.org/}} パッケージで収集されたパッケージの利用状況のランキング
      \begin{itemize}
      \item Popcon500 のパッケージは利用者が多いため優先的に翻訳するとよい
      \end{itemize}
    \item 表のセルにマウスを乗せて少し待つとツールチップが表示され、未翻訳のパッケージ名が表示される
    \end{itemize}
  \end{itemize}
\end{frame}


\section{DDTSS}
\emtext{DDTSS}

\begin{frame}{DDTSS}
  \begin{itemize}
  \item Debian パッケージの説明文の翻訳とレビューを行う Web サイト
    \begin{itemize}
    \item Debian Distributed Translation Server Satellite    
    \item \url{https://ddtp.debian.org/ddtss/index.cgi/xx}
    \item sid のパッケージが作業対象
    \item 翻訳を行い、3 人のレビューが OK になると反映される
    \item アカウントの登録が必要
    \end{itemize}
  \end{itemize}
\end{frame}


\section{そのほかの翻訳ツール}
\emtext{そのほかの \\ 翻訳ツール}

\begin{frame}{Weblate}
  \begin{itemize}
  \item \url{https://hosted.weblate.org/}
    \begin{itemize}    
    \item プロジェクト横断でドキュメントの翻訳作業を支援する Web サイト
    \end{itemize}      
  \item Debian では以下のドキュメントの翻訳作業が行える
    \begin{itemize}
    \item \url{https://hosted.weblate.org/projects/debian-installer/}
    \item \url{https://hosted.weblate.org/projects/debian-installation-guide/}
    \item \url{https://hosted.weblate.org/projects/debian-reference/}
    \item \url{https://hosted.weblate.org/projects/debian-handbook/}
    \item \url{https://hosted.weblate.org/projects/debexpo/}
    \end{itemize}
  \end{itemize}
\end{frame}


\begin{frame}{リリースノート}
  \begin{itemize}
  \item 基本的な情報 \url{https://wiki.debian.org/ReleaseNotes}
  \item 翻訳ファイル置き場
    \begin{itemize}
    \item \url{https://salsa.debian.org/ddp-team/release-notes/}
    \item 翻訳 web サイトはないため、po ファイルを修正してマージリクエストを出します
    \end{itemize}
  \item 翻訳状況
    \begin{itemize}
    \item Debian 11 bullseye版 \url{https://www.debian.org/releases/bullseye/statistics.html}
    \item Debian testing版 \url{https://www.debian.org/releases/testing/statistics.html}
    \end{itemize}
  \end{itemize}
\end{frame}
  
\section{まとめ}

\begin{frame}[containsverbatim]{まとめ}
  \begin{itemize}
  \item DDTP の取り組みを紹介しました
  \item DDTSS の Webサイト \url{https://ddtp.debian.org/ddtss/index.cgi/xx}
  \item 2023 年の bookworm リリースまでにできるだけ翻訳作業を進めましょう
  \end{itemize}
\end{frame}


%\section{参考文献}
%\emtext{参考文献}
%
%\begin{frame}{参考文献}
%
%\begin{itemize}
%\item xx
%\end{itemize}
%
%\end{frame}

\end{document}


;;; Local Variables: ***
;;; outline-regexp: "\\([ 	]*\\\\\\(documentstyle\\|documentclass\\|emtext\\|section\\|begin{frame}\\)\\*?[ 	]*[[{]\\|[]+\\)" ***
;;; End: ***
