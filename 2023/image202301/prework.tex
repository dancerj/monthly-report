\begin{prework}{ dictoss }
  \begin{enumerate}
  \item あります
  \item bookwormリリースに向けた翻訳、オフラインイベントへの出展
  \end{enumerate}
\end{prework}

\begin{prework}{ yosuke\_san }
  \begin{enumerate}
  \item ありません
  \item 1) pandas-datareader に新機能を実装したい 2) 分散システム関連で発表したい(未定)
  \end{enumerate}
\end{prework}

\begin{prework}{ NOKUBI Takatsugu (knok) }
  \begin{enumerate}
  \item あります
  \item 安定版に対応する, 変わったハードウェア/環境でDebianを動かす
  \end{enumerate}
\end{prework}

\begin{prework}{ kenhys }
  \begin{enumerate}
  \item あります
  \item mentors.d.nの日本語版を利用可能にする、fabre.d.nの使い勝手を向上させる
  \end{enumerate}
\end{prework}

\begin{prework}{ ipv6waterstar }
  \begin{enumerate}
  \item あります
  \item To be thinking
  \end{enumerate}
\end{prework}

\begin{prework}{ yy\_y\_ja\_jp }
  \begin{enumerate}
  \item あります
  \item ・日本語特に日本語入力周りのデバッグ ・DDTSSなどでの翻訳
  \end{enumerate}
\end{prework}

\begin{prework}{ daisukeokaoss }
  \begin{enumerate}
  \item ありません
  \item LinuxアプリケーションのIntelのIntrinsic関数をRISC-Vにポートするためのテストフレームワークのアイディアを一個でも出したい。今まで出したアイディアは以下に。\url{https://qiita.com/daisukeokaoss}
  \end{enumerate}
\end{prework}

\begin{prework}{ yukiyam99999999999999999999999 }
  \begin{enumerate}
  \item あります
  \item 検討中
  \end{enumerate}
\end{prework}
