\begin{prework}{ 鈴木崇文 }

gnomeでdesktopを利用しています。
工夫というほどではないですが、便利なツールとしてKDE系のアプリやEBViewを使っています。
リモートデスクトップ(RDP) & VNC 用には、KRDC を使い、スクリーンショット用には KSnapshot を使っています。
あとは英辞郎を購入して、EBViewでいつでも翻訳できるようにしています。
\end{prework}

\begin{prework}{ dictoss(杉本 典充) }

低性能マシンはstartx+icewm、中高性能マシンはgdm+xfce4かgnomeと使い分けています。KDEは最近使ってないです。(KDEは重そうなイメージがある)
カスタマイズしているのは、ページャの個数を6つに増やしている、複数のターミナルを重ならないように同時起動するシェルスクリプトを作り一発で画面をターミナルで埋め尽くせるようにしています。(昔タイル型ウィンドウマネージャ使えばいいのに、とか突っ込まれました)
\end{prework}

\begin{prework}{ yamamoto }

メインに使っているのは、録画サーバも兼ねた Debian squeeze (amd64) です。
外出時は気分次第で、sid の i386 ネットブックと sid の amd64 ノート PC を選んでいます。
どれも、特に何の変てつもない、ただの KDE 環境です。

デスクトップとしての見た目は、壁紙すらデフォルトのままで、改造してませんが、機体としては家の LAN にぶら下がったマシン間を「何か(?)」が行き来する、魔改造スクリプトがいくつか仕込んであり、ものぐさな私にはとっても快適です。
\end{prework}

\begin{prework}{ 野島 貴英 }

GNOME 3.2.2をDebianで利用しています。unstableではもの足りず、experimentalからupgradeして引っ張ってきてます。特にクールなカスタマイズは何もしてないですが、gnome-shellがjavascriptなど解釈できるということから、将来ちょっとしたガジェットぐらい作ってみたいなーと思うこの頃です。あと、gxconsole( \url{http://gnomefiles.org/content/show.php/gxconsole?content=132145} )がGNOME3.2.2になっても、やっぱり欲しかったので、GNOME 3.2.2用に移植したい...
\end{prework}

\begin{prework}{ 日比野 啓 }

普段はタイル型ウィンドウマネージャの XMonad を使っています。
マルチディスプレイに対するサポートが使いやすくてプレゼンのときにも便利で気にいっています。
最近、趣味のプログラミングのメインで使っている言語が Haskell なので、
Haskellでカスタマイズできることも魅力です。
gnome-session との組合せも使ってみましたが、なぜか sid ではうまく動かなくて残念。
あと、画面の上下が狭くなるのが嫌なので、なんとかする方法が知りたい。

\end{prework}
