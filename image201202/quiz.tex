%; whizzy-master ../debianmeetingresume201201.tex
% 以上の設定をしているため、このファイルで M-x whizzytex すると、whizzytexが利用できます。
%

\santaku
{Lenny のセキュリティサポートが終わったのはいつ?}
{2012/02/06}
{2012/02/07}
{2012/02/08}
{A}
{}

\santaku
{2012/01/28 に更新された Squeeze のヴァージョンは?}
{6.0.4}
{6.1.0}
{20120128}
{A}
{}

\santaku
{Debian Game チームが 2/25から2/26まで行うイベントは何か?}
{どれだけの Windows のゲームが Wine 上で動作するか検証するパーティ}
{Debian で提供されているゲームパッケージのスクリーンショットを撮りまくるパーティ}
{Debian で提供されているゲームを48時間連続プレイするパーティ}
{B}
{}

\santaku
{Wheezy で採用される Linux カーネルヴァージョンは?}
{2.6.39}
{3.2}
{4.0}
{B}
{}

\santaku
{pts.debian.org で表示されるようになった情報は?}
{パッケージメンテナが誕生日の日は「おめでとう」と出る。}
{パッケージを乗っ取ろうとしている人の情報}
{パッケージ Transition 情報}
{C}
{}

\santaku
{アクセプトされたDEPは?}
{DEP 3}
{3 DEP }
{DEP DEP DEP}
{A}
{DEP3 は Patch tagging guideline. Debian Enhancement Proposals}


