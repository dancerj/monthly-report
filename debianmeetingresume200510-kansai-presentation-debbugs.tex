%; whizzy document
% latex beamer presentation.
% platex, latex-beamer でコンパイルすることを想定。 


%     Tokyo Debian Meeting resources
%     Copyright (C) 2006 Junichi Uekawa

%     This program is free software; you can redistribute it and/or modify
%     it under the terms of the GNU General Public License as published by
%     the Free Software Foundation; either version 2 of the License, or
%     (at your option) any later version.

%     This program is distributed in the hope that it will be useful,
%     but WITHOUT ANY WARRANTY; without even the implied warranty of
%     MERCHANTABILITY or FITNESS FOR A PARTICULAR PURPOSE.  See the
%     GNU General Public License for more details.

%     You should have received a copy of the GNU General Public License
%     along with this program; if not, write to the Free Software
%     Foundation, Inc., 51 Franklin St, Fifth Floor, Boston, MA  02110-1301 USA

\documentclass[cjk,dvipdfmx]{beamer}
\usetheme{Warsaw}
% latex-beamerではhyperref は動かない?
%http://www.naney.org/diki/dk/hyperref.html
%日本語EUC系環境の時
\AtBeginDvi{\special{pdf:tounicode EUC-UCS2}}
%シフトJIS系環境の時
%\AtBeginDvi{\special{pdf:tounicode 90ms-RKSJ-UCS2}}

% previewするキーバインド
% (defun dancer-debianmeetingresume-debbugs-preview () "Preview" (interactive) (compile "make") (shell-command "xpdf debianmeetingresume200510-kansai-presentation-debbugs.pdf"))

%\def\UrlBreaks{\do\.\do\@\do\\\do\/\do\!\do\_\do\|\do\%\do\;\do\>\do\]%
% \do\)\do\,\do\?\do\'\do\+\do\=\do\#\do\&}%


\title{Debian勉強会資料}
\author{上川 純一}
\date{2005年10月29日}

\begin{document}
%% debianmeetingresume200510-kansai.texから以下コピー

\frame{\titlepage}

\section{debbugsの内部構造}
\section{debbugs概要}

\begin{frame}
\begin{itemize}
 \item 55000以上の現在アクティブなバグ報告
 \item 231000のアーカイブされたバグ報告
 \item 毎週1000以上の新規のバグ報告
 \item リアルタイムでバグ報告をウェブページにどんどん反映
\end{itemize}
\end{frame}


\begin{frame}
 \begin{itemize}
 \item インタフェース: 開発者がメールで操作できるようになっており、誰で
       もウェブで閲覧できるようになっている。
 \item パッケージベース: バグ報告をパッケージ別に高速に管理する必要があ
       る
 \item スケーラビリティー: 大量のバグ報告に対応できる必要がある
 \item 即時性: 現在のバグの状態をすぐに報告してくれる必要があり、バグの
       状態が変更されたらすぐに反映される必要がある
 \item 安定性: 継続して動作する必要がある。新規の機能がどんどん追加さ
       れたとしても。
 \item 公開: 議論の内容にDebianコミュニティー全体として参加できるように、
       永続的な公開記録として保存される必要がある。
 \end{itemize}
\end{frame}

\section{データ形式}

\begin{frame}
 \begin{itemize}
 \item /org/bugs.debian.org/spool
       \begin{itemize}
	\item incoming/
	\item db-h/
	\item archive/
	\item index.db -- index.db.realtimeへのシンボリックリンク
	\item index.archive -- index.archive.realtimeへのシンボリックリンク
	\item nextnumber
       \end{itemize}
 \end{itemize}
\end{frame}


\subsection{incoming}

\begin{frame}
 \begin{itemize}
 \item T receiveによってうけとられた
 \item S SPAM確認待ち
 \item R SPAM確認中
 \item I SPAMチェック通った
 \item G service か process スクリプトを通った
 \item P process中
 \end{itemize}
\end{frame}

\begin{frame}
 
 \begin{itemize}
 \item B: 通常のバグ報告。submit@ 1234@
 \item M: -maintonly メーリングリストに投げない
 \item Q: BTSに登録しない。-quiet
 \item F: アップストリームにフォーワード -forwarded
 \item D: バグ終了 -done
 \item U: サブミッターにメール -submitter
 \item R: ユーザのリクエスト用インタフェース request@
 \item C: デベロッパーの制御用インタフェース control@
 \end{itemize}
\end{frame}

\subsection{StatusとSummary}

\begin{frame}
 Status
 \begin{itemize}
 \item バグ報告者のメールアドレス
 \item 時間(秒)
 \item サブジェクト
 \item 元のメールのメッセージID
 \item バグがアサインされているパッケージ
 \item タグ
 \item closeした人のメールアドレス
 \item 上流のメールアドレスかURL(forwardされたばあい)
 \item マージされているバグ番号
 \item severity
 \end{itemize}
\end{frame}

\begin{frame}
 Summary
 \begin{itemize}
 \item Format-Version: このファイル形式のバージョン
 \item Submitter: バグ報告者のメールアドレス
 \item Date: 時間(秒)
 \item Subject: サブジェクト
 \item Message-ID: 元のメールのメッセージID
 \item Package: バグがアサインされているパッケージ
 \item Tags: タグ
 \item Done: closeした人のメールアドレス
 \item Forwarded-To: 上流のメールアドレスかURL(forwardされたばあい)
 \item Merged-With: マージされているバグ番号
 \item Severity: severity
 \item Owner: バグの所有者
 \end{itemize}
\end{frame}

\subsection{logファイル}

\begin{frame}
 \begin{itemize}
 \item kill-init: まだ一行も処理していません
 \item incoming-recv: 07: あとにgoがくる、Received:行
 \item autocheck: 01: X-Debian-Bugs-..: までの無視されている行、
       autowaitが次に来る
 \item html: 06: 生で表示すべきHTML
 \item recips: 02: メールの受取人、04で分割されている
 \item go: 05: メールの文書
 \item go-nox: X: メールの文書、Xではじまる行
 \item kill-end: 03: メッセージの終り。
 \item autowait: go-noxがあとにくる、空行まで無視されるその他の情報。
 \end{itemize}
\end{frame}

\subsection{Indexファイル}

\begin{frame}
\begin{itemize}
 \item パッケージ
 \item バグ番号
 \item 時間
 \item ステータス
 \item メールアドレス
 \item severity
 \item 例: pbuilder 317998 1121196782 open [Junichi Uekawa <dancer@netfort.gr.jp>] normal
\end{itemize}
\end{frame}

\section{コード形式}

\frame{設定ファイルは全て/etc/debbugs にあります。}

\subsection{コアのスクリプト}

\begin{frame}
 メールの処理部分
 \begin{itemize}
  \item errorlib: ライブラリ
  \item receive: MTAからメールを受信する
  \item spamscan: 受信メールをSPAMチェックする
  \item processall: process と service にメールを分配する
  \item process: バグメールを処理する
  \item service: control@ と report@ メールを処理
  \item expire: closeされてから28日過ぎたバグをエキスパイア処理する
  \item rebuild: indexファイルをリビルド
  \item 15 分に一回cronで動作
 \end{itemize}
\end{frame}

\subsection{CGIスクリプト}

\begin{frame}
 ウェブインタフェース
 \begin{itemize}
 \item bugreport.cgi: バグレポートを一つ表示
 \item pkgreport.cgi: パッケージやサブミッタなどでサマリを作成する
 \item pkgindex.cgi: パッケージやseverityに対して数を表示
 \item common.pl: ライブラリとして利用
 \end{itemize}
\end{frame}


\subsection{ハックするには}

\begin{frame}
 \begin{itemize}
  \item ソースはCVS
  \item  merkel.debian.orgの/org/bugs.debian.org に複製がある
 \end{itemize}
\end{frame}

\section{そして何がおきたか}

\subsection{バージョントラッキング}

\begin{frame}
 \begin{itemize}
  \item close バグ番号 バージョン
  \item reassign バグ番号 パッケージ バージョン
  \item found バグ番号 バージョン
  \item 'Source-Version: バグ番号' タグが追加
  \item \url{http://bugs.debian.org/cgi-bin/pkgreport.cgi?pkg=cowdancer&version=0.4}と
\url{http://bugs.debian.org/cgi-bin/pkgreport.cgi?pkg=cowdancer&version=0.5}
  \item Summaryファイルにも、Found-In: cowdancer/0.4,
	Fixed-In: cowdancer/0.5
 \end{itemize}
\end{frame}


\subsection{ユーザタグ}

\begin{frame}
 \begin{itemize}
  \item user aj@azure.humbug.org.au
  \item usertag 18733 + good-reasons-to-run-for-dpl
  \item usertag 18733 + still-cant-believe-it-finally-got-fixed
  \item usertag 62529 + your-days-are-numbered
  \item \url{http://bugs.debian.org/cgi-bin/pkgreport.cgi?pkg=dlisp;users=dancer@debian.org}
  \item \url{http://bugs.debian.org/cgi-bin/pkgreport.cgi?tag=ignore-for-now;users=dancer@debian.org}
 \end{itemize}
\end{frame}

\subsection{バグ購読}

\begin{frame}
 \begin{itemize}
  \item {\tt バグ番号-subscribe@bugs.debian.org} にメールを出す
 \end{itemize}
\end{frame}

\subsection{バグブロッカー}

\begin{itemize}
 \item  block 保留中のバグ番号 by 原因のバグ番号
 \item  unblock 保留中のバグ番号 by 原因のバグ番号
\end{itemize}

\subsection{mindays maxdays}
\begin{itemize}
 \item \url{http://bugs.debian.org/cgi-bin/pkgreport.cgi?maint=dancer@debian.org&maxdays=90}
 \item \url{http://bugs.debian.org/cgi-bin/pkgreport.cgi?maint=dancer@debian.org&mindays=90}
\end{itemize}

\end{document}
