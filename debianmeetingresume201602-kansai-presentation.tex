\documentclass[cjk,dvipdfmx,10pt,compress,%
hyperref={bookmarks=true,bookmarksnumbered=true,bookmarksopen=false,%
colorlinks=false,%
pdftitle={第 107 回 関西 Debian 勉強会},%
pdfauthor={倉敷・のがた・佐々木・かわだ},%
%pdfinstitute={関西 Debian 勉強会},%
pdfsubject={資料},%
}]{beamer}

\title{第 107 回 関西 Debian 勉強会}
\subtitle{$\sim$発表資料$\sim$}
\author[かわだ てつたろう]{{\large\bf 倉敷・のがた・佐々木・かわだ}}
\institute[Debian JP]{{\normalsize\tt 関西 Debian 勉強会}}
\date{{\small 2016 年 2 月 28 日}}

%\usepackage{amsmath}
%\usepackage{amssymb}
\usepackage{graphicx}
\usepackage{moreverb}
\usepackage[varg]{txfonts}
\AtBeginDvi{\special{pdf:tounicode EUC-UCS2}}
\usetheme{Kyoto}
\def\museincludegraphics{%
  \begingroup
  \catcode`\|=0
  \catcode`\\=12
  \catcode`\#=12
  \includegraphics[width=0.9\textwidth]}
%\renewcommand{\familydefault}{\sfdefault}
%\renewcommand{\kanjifamilydefault}{\sfdefault}
\begin{document}
\settitleslide
\begin{frame}
\titlepage
\end{frame}
\setdefaultslide

\begin{frame}[fragile]
  \frametitle{Disclaimer}
  \begin{itemize}
  \item 疑問、質問、ツッコミ、茶々、\alert{大歓迎}
  \item その場でインタラクティブにどうぞ
  \item ハッシュタグ \#kansaidebian
  \end{itemize}
\end{frame}

\begin{frame}[fragile]
\frametitle{Agenda}

\tableofcontents

\end{frame}

\section{最近の Debian 関係のイベント}

\takahashi[40]{最近の Debian\\関係のイベント}

\begin{frame}[fragile]
  \frametitle{第106回関西Debian勉強会}
  \begin{itemize}
  \item 日時: 1月24日(日)
  \item 場所: 福島区民センター
  \end{itemize}
  \begin{block}{内容}
    \begin{itemize}
    \item 「GNUHurdのインストールしてみた。と、Xサーバの立ち上げに挑戦」
    \item 「VyOSを入れてAPを構築してみた」
    \end{itemize}
  \end{block}
\end{frame}

\begin{frame}[fragile]
  \frametitle{第136回東京エリアDebian勉強会}
  \begin{itemize}
  \item 日時: 2月13日(土)
  \item 場所: イベント&コミュニティスペース dots.
  \end{itemize}
  \begin{block}{内容}
    \begin{itemize}
    \item 「Debian GNU/Linux上での省電力方法について」
    \item 「libhinawaというライブラリをDebianプロジェクトにITP/RFSした話」
    \end{itemize}
  \end{block}
\end{frame}

\begin{frame}[fragile]
  \frametitle{Debian Project}
  \begin{itemize}
  \item Debian Policy 3.9.7.0 released
  \item Debtags consolidation
  \item Debian 6.0 Long Term Support reaching end-of-life
  \item The Debian Administrator's Handbook, Debian Jessie from Discovery to Mastery (Japanese version)
  \end{itemize}
\end{frame}

\takahashi[50]{そんな\\こんなで}
\takahashi[120]{次}

\section{事前課題}
\takahashi[50]{事前課題}

\begin{frame}[fragile]
  \frametitle{事前課題}
  \begin{block}{今回の事前課題}
    事前課題はありませんでした。
  \end{block}
\end{frame}

\takahashi[50]{事前課題\\発表}

\begin{frame}
  \frametitle{ 矢吹 幸治 }
  \begin{itemize}
  \item 低いところから高いところ、組織運営まで
  \item 2016年、C言語はどう書くべきか

    \url{http://postd.cc/how-to-c-in-2016-1/}
    \url{http://postd.cc/how-to-c-in-2016-2/}

  \item 電子メール自衛

    \url{https://emailselfdefense.fsf.org/ja/}
  \end{itemize}
\end{frame}

\begin{frame}
  \frametitle{ t3rkwd }
  \begin{itemize}
  \item Squeeze LTS が終わるので VyOS をリプレイス
  \end{itemize}
\end{frame}

\begin{frame}
  \frametitle{ むんくさん }
  \begin{itemize}
  \item アマチュア無線局を開局した
  \item cqrlog
  \end{itemize}
\end{frame}

\begin{frame}
  \frametitle{ Yosuke OTSUKI }
  \begin{itemize}
  \item 仮想化環境構築

    ネットワークまわりの設定
  \item First Osaka Linux Meetup

    \url{http://www.meetup.com/ja-JP/osakalinux/events/229032054/}
  \end{itemize}
\end{frame}

\begin{frame}
  \frametitle{ 川江 浩 }
  \begin{itemize}
  \item GNUHurd X起動
  \end{itemize}
\end{frame}

\begin{frame}
  \frametitle{ Yamada Yohei (山田 洋平) }
  \begin{itemize}
  \item OSコードリーディング
  \item Window Manager
  \end{itemize}
\end{frame}

\begin{frame}
  \frametitle{ lurdan }
  \begin{itemize}
  \item 最近はWindows
  \item Vagrant から docker へ
  \end{itemize}
\end{frame}

\begin{frame}
  \frametitle{ Hideo Ueno }
  \begin{itemize}
  \item Debian 3.0 Woody から
  \end{itemize}
\end{frame}

\begin{frame}
  \frametitle{ Masahiro Yamada }
  \begin{itemize}
  \item Debian 3.1 Sargeはインストールしてみた、最近は Ubuntu
  \item uboot
  \item GPGキーサイン
  \end{itemize}
\end{frame}


\takahashi[50]{そんな\\こんなで}
\takahashi[120]{次}

\section{勉強会資料の歩き方}
\takahashi[30]{勉強会資料の歩き方\\by\\かわだてつたろう}

\begin{frame}
  \frametitle{ 読むには }
  \begin{itemize}
  \item PDF

    \url{http://tokyodebian.alioth.debian.org/pdf/}
  \item 東京エリアDebian勉強会

    \url{https://tokyodebian.alioth.debian.org/}
  \item 関西Debian勉強会

    \url{https://wiki.debian.org/KansaiDebianMeeting/}
  \item あんどきゅめんてっど でびあん

    \url{https://tokyodebian.alioth.debian.org/undocumenteddebian.html}
  \end{itemize}
\end{frame}

\begin{frame}
  \frametitle{ 関西Debian勉強会のTips }
  \begin{itemize}
  \item HTML版

    kozo2さんによる作成公開
  \item トピック一覧

    各年12月号参照
  \end{itemize}
\end{frame}

\begin{frame}
  \frametitle{ ソースは }
  \begin{itemize}
  \item TeX で作成
  \item ライセンスはGPL-2 or later
  \item Aliothで公開

    http://anonscm.debian.org/cgit/tokyodebian/monthly-report.git/
  \end{itemize}
\end{frame}

\begin{frame}[containsverbatim]
  \frametitle{ ビルド }
  \begin{commandline}
$ sudo apt install texlive-lang-japanese texlive-latex-extra \
              texlive-generic-recommended texlive-fonts-recommended \
              ghostscript-x lv
$ git clone https://anonscm.debian.org/git/tokyodebian/monthly-report.git
$ cd monthly-report
$ cp git-pre-commit.sh .git/hooks/pre-comit
$ make
  \end{commandline}
%$
\end{frame}

\begin{frame}
  \frametitle{ 関西Debian勉強会の資料 }
  \begin{itemize}
  \item debianmeetingresumeYYYYMM-kansai.tex
  \end{itemize}
  \begin{block}{よく使うコマンド}
    \begin{itemize}
    \item dancersection
    \item commandline
    \item debianbug
    \end{itemize}
  \end{block}
\end{frame}

\begin{frame}
  \frametitle{ まとめ }
  \begin{itemize}
  \item 手元でいつでも資料を参照
  \item あとは発表するだけ
  \end{itemize}
\end{frame}

\section{周回遅れでDocker触ってみた}
\takahashi[30]{周回遅れでDocker触ってみた\\by\\倉敷 悟}

\section{今後の予定}
\begin{frame}[fragile]
  \frametitle{今後の予定}

  \begin{block}{第108回関西Debian勉強会}
    \begin{itemize}
    \item 日時: 3月27日(日)
    \item 場所: 福島区民センター
      \begin{block}{予定内容}
        \begin{itemize}
        \item 「systemd-networkd」
        \item 「Open FOAM」
        \end{itemize}
      \end{block}
    \end{itemize}
  \end{block}

  \begin{block}{第137回東京エリアDebian勉強会}
    \begin{itemize}
    \item 日時: 3月5日(土)
    \item 場所: サイボウズ株式会社
      \begin{itemize}
      \item 「How to become a Debian Developer」
      \end{itemize}
    \end{itemize}
  \end{block}

\end{frame}

\takahashi[50]{  }

\end{document}
%%% Local Variables:
%%% mode: japanese-latex
%%% TeX-master: t
%%% End:
