%; whizzy-master ../debianmeetingresume200904.tex
% 以上の設定をしているため、このファイルで M-x whizzytex すると、whizzytexが利用できます。

\begin{prework}{上川純一}
\preworksection{私のDebianワークフロー}

\begin{itemize}
 \item メールでバグレポートを受け取る
 \item コードを直す・パッチを git am で適用
 \item pdebuild-normal スクリプトを実行、cowbuilder --update, cowbuilder
       --build が実行され、一連のインストール・実行テストスクリプトが実
       行される。成功したら pending ディレクトリにパッケージが移動される。
 \item pending ディレクトリを確認、debsignで署名、dput でアップロード
\end{itemize}

\preworksection{こう改善したい}

全アーキテクチャでのビルドとテストを自動化したい。

\end{prework}

\begin{prework}{まえだこうへい}
\preworksection{私のDebianワークフロー}

ganttprojectを初めてITPしてから止まったまま。

\preworksection{こう改善したい}

家庭と仕事に影響されずにパッケージメンテナンスできるようにしていきたい。

\end{prework}


% \begin{prework}{名前}
% \preworksection{私のDebianワークフロー}
% \preworksection{こう改善したい}
% \end{prework}
% 以下に同様のテンプレートで追加する。

\begin{prework}{小川伸一郎}
\preworksection{私の作業環境について}
会社では Ubuntu 8.04.1 Desktop をインストールしたデスクトップPCで,
家では Ubuntu 8.10 をインストールした Thinkpad X61 を使って,
開発や日々の業務などをこなしています.
全然 Debian じゃないのですが,Ruby on Rails なので,
Ruby の Version があわないので,Ubuntu 使っています.

GW中に Thinkpad に Lenny 入れる予定です.
会社のサーバ群も,Debian にしたいなと,いろいろ模索中です.

\end{prework}

\begin{prework}{山本浩之}
\preworksection{私のDebianワークフロー}

パッケージ化したいソフトウェアを見つけたら、まず自分自身用の野良パッケージを作り、試します。次に大雑把にライセンスを確認し、良さげなら、自分に喝を入れるため ITP します(笑)。それからコードなど、技術的な検討に入ります (ここで挫折したものもいくつあるのやら…)。さらにコピーライトやライセンスの精査をし、debian/copyright を完成させます。次に私にとってとても難関の(笑)英語のドキュメントをつけて、pbuilder でビルドします。最後に mentors.debian.net へのアップロードと mentors@org ML、および debian-develop@jp ML にメールを投げてスポンサー探しをします。以上。

\preworksection{こう改善したい}

みんなが使っている文字コードや locale を UTF-8 に統一したい。

\end{prework}

\begin{prework}{やまだたくま}
\preworksection{私のDebianワークフロー}

 \begin{enumerate}
  \item DDTSS (ja) で Pending review の項目を順番に選びます。
  \item doc/(パッケージ名) フォルダを作成し、原文 (英語) と日本語訳のコ
	ピーのテキストファイルを作成します。
  \item 翻訳ソフトで英日翻訳を実行します。
  \item 用語とその日本語訳を確認して、対訳リストを作ります。
  \item 文章の内容を確認するため、オンラインマニュアルやパッケージ関連ファ
	イルを参照します。
  \item 使用例や用語 (訳語) の使用頻度を調査して、訳語を選びます。
  \item 原文を先頭から順番に手動で再翻訳します。
  \item Debian JP の文書作成/翻訳ルールを守っているか確認します。
  \item debian-doc ML へ査読依頼します。
  \item 査読完了後に DDTSS (ja) へ登録します。
 \end{enumerate}

 \begin{itemize} 
  \item 作業は、複数の場所、複数の PC で行なっています。
  \item 作業ファイルは、Mercurial で同期管理しています。
  \item オンラインマニュアルやパッケージのファイルは、VMware 上の Debian
	(sid) または ssh で Debian サーバに接続して確認作業しています。
 \end{itemize}

\preworksection{こう改善したい}

 \begin{itemize} 
  \item 対訳表の収録語を増やして、訳語の確認時間を短縮したい。
  \item Debian JP の文書作成/翻訳ルールの確認作業を自動化したい。
 \end{itemize}

\end{prework}

\begin{prework}{中尾圭佐}
\preworksection{私のDebianワークフロー}
私はDebianに貢献しているわけではないで、Debian開発の開発工程はもっていま
 せんので、普段の作業工程を記述します。

\begin{itemize}
 \item まず、必要とされている機能を見付けます。見つけ方は、ボスから指示があった
 り、手作業でやっていていらついたとき、簡単な作業でも毎日やっていること
 に気付いたとき等によく見付かります。

 \item どうやったら、その機能が実現できるかを考えます。個人的にこの段階が一番楽しい
 です。

 \item この機能が本当に必要か考えます。

 \item 必要ならば、本当に実装して良いか考えます。私がいる職場は、放射線が出
 たり、100kVの高電圧がかかっていたりするので、放射線管理上問題がないか安
 全上問題がないか検討します。

 \item 一番楽しい段階が終ったこともあり、本当に私が実装すべきか考えます。

 \item 私が実装すべきという結論が出た場合、ぶつぶつ文句を言いながら、実装し
 ます。

 \item 時々Debugします。テストファーストとか、自動化はできていませんが、ユニットテストを行います。

 \item だいたいできたら、職場のみんなに見せびらかします。そうすると色々コメ
 ントが出てくるので、それに対応します。運用が大変なら、修正します。
\end{itemize}


\preworksection{こう改善したい}
改善したいことは、私以外の誰かが、書いたコードをメンテナンスできるように
することです。そのためには:
\begin{itemize}
 \item バージョン管理システムを導入する
 \item ユニットテストの自動化を徹底する
 \item ドキュメント、資料、コメントをちゃんと残す
\end{itemize}
が必要だと考えています。

私の職場では、コードの所有権のようなものが心理的に存在しています。バージョン管理システムを導入する事で
所有者のコードを残しつつ、所有者以外の人がコードを取得でき、また修正する事ができます。これにより
コードの所有権の意味を曖昧にすることができあます。

ユニットテストの自動化を実現する事で、コードの所有者以外の人がコードを修正した場合、
その修正が他に影響を及ぼさない事を確認する事ができます。このことは他人のコードを修正すことの
心理的障壁を下げることにつながります。

さらにドキュメントを残す事で、トラブルの時、修正する時に非常に有用な情報を提供する事ができます。
時が進み、開発時何を考えていたか、当時どのような必要があってこのような実装したか、このような記録は、
どこを捨ててどこを残すかという判断に必要な情報になります。

私がいなくなっても、私が書いたコードやシステムがちゃんと維持でき、状況に
応じてメンテナンスできるように、以上のことを改善したいと考えています。
\end{prework}

\begin{prework}{あけど}
\preworksection{私のDebianワークフロー}

Debian上で作業することが少ないなと思っています。
せいぜい管理しているサーバのファイアウォールルールを手直しするくらいなので、
手元のメインマシンがMac OS X(10.5.6)ということもあり、Debianなデスクトップ環境を殆ど使ってません。
Debianのデスクトップ環境はDebian勉強会の事前課題に使う程度なので(いろんな環境に慣れるという意味で)
もっと使う様にするにはDebian勉強で標準的な環境のemacsを使うのがいいかなと思います。

\preworksection{こう改善したい}

上記を踏まえて、勉強がてらemacsを使う様にしてみます。

\end{prework}

\begin{prework}{藤沢理聡}
\preworksection{私のDebianワークフロー}

パッケージをメンテナンスしたり、といったDebianへの貢献は
まったくできていないのですが、Debian上でスクリプトを
書くことは結構あります。

フローにすると、
\begin{enumerate}
\item 仕事とかしてて、こういうのがあったらなあ、と思う
\item 思いついたことを実現する仕組みを考えてみる
\item 実現できそうなら、実際に使用する環境やユーザの範囲を考える
\item 具体的に作るもののイメージができたら、エディタを起動する
\item 適当に書いて、とりあえず動くものを作る
\item 規模が小さければ、ホワイトボックステストをする
\item 自分以外に使う人がいれば、とりあえず試してもらう
\item 動くものを見て、新たに出てきた要望に答える
\item 飽きたら開発終了
\end{enumerate}

Debianである必然性のないワークフローになりました。
この情報は一体何の役に立つんだろう、と自問自答。

\preworksection{こう改善したい}

たいてい2つ目のステップで「やっぱいらないか」と思って
終了してしまうのを改善したい。

ワークフローとして改善すべきは、
\begin{itemize}
\item 他人と共同で開発することに向いていない
\item 誰かが作ったものの改良や、ある程度作ってから誰かに引き継ぎ、に向い
      ていない
\end{itemize}
ことかな、と自分では思っています。
\end{prework}

\begin{prework}{日比野}
\preworksection{私のDebianワークフロー}

Debianのワークフローかどうかわかりませんが、Debianも利用している私の会社でのワークフローを紹介します。

 \begin{enumerate}
  \item 自動化できそうな機能やDebian化することでインストールが楽になりそうな機能を見つける - たとえば
  \begin{itemize}
   \item バッチジョブのログを取りながら経過や結果を監視し、問題があるようならアラームをいろんな方法で投げるperl script
   \item 複数のアーキテクチャやDebianのバージョンに対して内製のパッケージをbuildする cowbuilder や pbuilder の wrapper
   \item 社内のDebianで運用するサーバーのインストーラースクリプト
  \end{itemize}
  \item 利用者になりうる人や他の開発者の意見を取り入れながら、機能をプログラム化する
  \item まとまった規模になったらDebian化を行なう
  \item 適当な区切りで履歴管理システムにタグを打ち、内製のpackage spoolにリリースする
 \end{enumerate}

\preworksection{こう改善したい}

内製パッケージのビルドやDebianのインストール、ネットワークの設定の自動化をすすめたい。

\end{prework}
