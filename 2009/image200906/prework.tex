%; whizzy-master ../debianmeetingresume200906.tex
% 以上の設定をしているため、このファイルで M-x whizzytex すると、whizzytexが利用できます。

\begin{prework}{川本健太郎}
\preworksection{適用した主要な方法}

2009/06の資料の「DDTSSの翻訳作業の紹介」を参考に、
DDTSS にユーザー登録し、新しい翻訳・修正・レビューを3つずつ行いました。


\preworksection{発見した課題}

\begin{itemize}
 \item ``Accept with changes'' すると、レビュープロセスがやり直しになるので、ちょっとした修正 (句読点の追加など) ができない
 \item (その前の稿との差分だけではなく) 最新稿も見たい
 \item 翻訳の基準がわからないので、Submitして良いのか自信が持てない
 \item DDTSSのサーバ証明書が不正
 \item Aliasのルールが登録時とログイン時とで異なる (登録時は ``at least 4 letter long'' でログイン時は ``at least 6 letter long'')
\end{itemize}

3 点目は心理的な問題ですが、
どのレベルの翻訳を求められているのかわからないので、
Submit をためらってしまいました。
(とはいえ、最終的には Submit しましたが。)
誤訳がないのは当然としても、専門用語の訳や、
日本語としての自然さなど、翻訳の質が気になってしまいます。

\preworksection{提案する理想像(ツールとか)、共有したい情報}

上記課題の 1 点目の対策として、
``Change and restart review process'' と ``Accept with minor changes'' との
2 つが分かれていれば良いと思います。
「わずかな修正でも、レビュープロセスをやり直す」というのは、
厳密性を保つためには必要ですが、
``The number of translations pending review has gotten quite large.'' という状況で、
句読点一つ直すたびに、また 3 人のレビューワが必要になるのは現実的ではないと思います。
逆に、レビューワがちょっとした修正をあきらめることで、
ドキュメントがちょっと読みにくくなるのは残念です。


\end{prework}

\begin{prework}{まえだこうへい}

\preworksection{適用した主要な方法}

ここでの ``方法''が何を意図しているのかが分からないので、ツールの使い方で
 はなく、翻訳のやり方という観点で書きます。翻訳のやり方自体は、翻訳通信 別冊『仁
 平和夫小論集 翻訳のコツ』
 \footnote{\url{http://homepage3.nifty.com/hon-yaku/tsushin/bn/200209SAp2.pdf}}
 を参考にしました。コツは色々あるようですが、いくつも意識するのは難しい
 ので、次の三点だけは意識するようにしました。

\begin{itemize}
 \item 直訳にはしないで、日本語としてわかりやすい文章を心がける。
 \item 長文で分かりづらければ、短文に分割してみる。
 \item 形容詞が連発されている部分は意図的に訳さない。
\end{itemize}

\preworksection{発見した課題}

 まず、システム的なこと。

 ログインしなくても レビュー、修正できてしまったので、ID を作っていたに
 も関わらず、cokkie に残った情報でアクセスしているのかと思いそのまま
 Accept すると、ID が IP アドレスになってしまうというちょっとマヌケな自
 体に。


 もう一つは、専門用語について。元々大学が生物学科でしたので、生物関連の
 専門用語に限定して話をします。学生時分に、一番困ったのは英語の文書しか
 ないことではなくて、変な翻訳のされかたをしている用語、特にカタカナに翻
 訳されている用語です。

 専門用語を調べるのには、通常、専門用語の辞書を使います。生物だと生物学
 全般、生化学、系統分類学、分子生物学、などなど、各分野で専門の辞書があ
 りますが、変に翻訳されていると、索引から引くことができません。ですので、
 専門用語を調べるときは、原文(英語か、学名で使われるラテン語)で調べる
 のが基本です。

\preworksection{提案する理想像(ツールとか)、共有したい情報}

 前者については、DDTSSにログインしないと変更できないようにリクエストを出
 すのがよいのでしょうか…。


 後者については、あけどさんが今回お話、提言してくださると思いますが、一般
 人向けに翻訳はするものの、原文は括弧書きなどで後ろに残しておくのが実際
 に使うユーザ(その分野の専門家)には親切だと思います。

\end{prework}

\begin{prework}{明渡忠郎}

\preworksection{適用した主要な方法}

 前月の DDTSS の翻訳作業を踏まえ、訳文作成の前に対象パッケージについて
 事前に調査した上で翻訳する方法を取ってみました。
 レビューでは、分からない部分が出てきた時点でググるなりして確認すると
 いうオンデマンド的な方法を取ってみました。

\preworksection{発見した課題}

 パッケージの詳細について翻訳できる程度に理解するには単純に調べる程度で
 は調査は不足で、一通りざっと翻訳してからでないと正確な翻訳を作成するの
 は難しいように思いました。
 レビューについては適用した方法で問題なくレビューできそうです。
 翻訳の日本語としての品質が一定していないようにも思いますので、その点の
 分かりやすいガイドラインが必要ではないかと思います。

\preworksection{提案する理想像(ツールとか)、共有したい情報}

 個人的な理想を言えば OmegaT のようなツールを手軽に利用できて、翻訳辞書
 を全員で共有できるという仕組みがあればどうでしょうか。
 後は今回の資料にも書いてあるのですが、便利なツールが色々あるのでそういっ
 た情報も共有できると良いかなと思います。

\end{prework}

\begin{prework}{あらきやすひろ}

\preworksection{適用した主要な方法}

とくにないかなあ。ひたすら翻訳。

\preworksection{発見した課題}

パッケージ名いきなりじゃなくて、パッケージの所属するsectionがかかれていると楽な
のになあ、と思いました。
reviewがおもいのほか簡単というか、べつに Debian Developer がひとりも関係しなくてもいいんですね。
ある意味おどろきかも。

\preworksection{提案する理想像(ツールとか)、共有したい情報}

超むかし(1997ころ)にこの手の作業をしたときとはぜんぜん違いますね。
でも、あいてる時間にやりたいときもあるのでオンラインではなくバッチ処理できる仕組
ものこってるとうれしいかなと。

\end{prework}

\begin{prework}{日比野 啓}

dnswalk, gobjc-4.2, gzip, jlex 他、文章が短いもの多数をレビューしました。
cl-swank, cl-uffi, cmucl を翻訳してみました。

\preworksection{適用した主要な方法}

先月と同じくDDTSSのWebからレビューと翻訳を行なってみました。
Pending reviewが多数ある状態が作業効率を悪くしていると感じたので、
意図的に文章が短かいものを片っぱしからレビューしてみました。

\preworksection{発見した課題}

Pending review をつぶしていくときに、
レビューできそうなものかを確認する作業の効率があまりよくないと
感じました。一度見てあきらめたものかどうかを忘れてしまうので。

\preworksection{提案する理想像(ツールとか)、共有したい情報}

pendingの一覧にマークを付けられるようにするといいかもしれません。

あきらかに意味が間違ってしまっている訳の修正を行なったときに、
リリースを迅速に行なうための調整が必要なんではないかと感じました。
ポリシーがはっきりしていないような気もしますが、
まずはあきらかな間違いを正したものを配布することを第一目的として、
レビューを通してリリースし、そのあとでもう一度より細かい修正を
行なうのが良いように思います。
自戒もこめてですが、コメント欄とdebian-doc MLを活用しましょう。

\end{prework}

\begin{prework}{キタハラ}

\preworksection{作業に適用した主要な方法}

Web上の「エキサイト翻訳」と「英辞郎」を利用した。

\preworksection{発見した課題}

「DDTSS login」すると「ddtp.debian.net は不正なセキュリティ
証明書を使用しています。」って出るのですが・・・。(ってのはもちろん
ネタで、ぱっと思いつかないので)DDTSS自体の日本語化、ということで。

\preworksection{方法の改善案の提案}

基本的に「翻訳する人」が使うツールなので、英語が読めること前提で
良いのでしょうが、英語が読めなくても「日本語として読みやすいか?」を
レビューするだけならば参加してくれる人が多くいそうな気がする。
DDTSSが日本語化されていれば、そういう人達にとっては参入障壁が低く
なると思います。(それとは別に、翻訳の支援ツールが翻訳されていない
のは、何となく「紺屋の白袴」な気がする・・・。)
\end{prework}

\begin{prework}{なかおけいすけ}
\preworksection{作業に適用した主要な方法}
事前に公開されていた今月のPDF資料を参考に、DTSSにアカウントを作り、主
 にGNU R関連の翻訳とレビューを2件づつ行いました。
 ツールはブラウザと紙の辞書で行いました。

\preworksection{発見した課題}
GNU Rは統計ソフトなので、解析手法と思われる専門用語を訳してよいものか
 迷いました。物理屋としては、読んでいる専門書はほとんど英文なので、専門
用語は英語のままでもかまわないと思います。実は日本語のほうを知らないとい
 うこともままあります。

\preworksection{方法の改善案}
専門用語に関しては、手っ取り早く \url{http://www.alc.co.jp} でひいてし
 まいがちですが、物理であれば、物理学用語辞典や理化学辞典で調べるべきで
 しょう。とはいえ、ボランティアベースの翻訳プロジェクトでは、そのような専門辞書に
 アクセスすることができない場合も多いと思うので、その時はあえてその専門
 用語は原文のままでよいと思います。可能であれば、そのプログラムを使いた
 い専門家が訳するべきでしょう。
\end{prework}

\begin{prework}{高橋 ``masaka'' 正和}

\preworksection{適用した主要な方法}

 はじめてのDDTSSをやってみました。

 新規で翻訳するほうは、perlのモジュールをやってみました。元のdescription
 がPerlモジュールのPODからサマっているので、PODの翻訳をするperldoc.jpか
 ら訳文をもってきてサマるという手抜き。ちょっと改変を入れたりしました。

 レビューのほうについては、こんな(↓)感じで言い回しをいじってみました。

\begin{itemize}
 \item 指示代名詞が多いと、つながりが一瞬わらりづらいので、できるだけな
       くした
 \item 原文の一対一対応だと回りくどい言い回しになる部分があるので、削った
\end{itemize}

\preworksection{発見した課題}

はじめてのレビューは、どこまでいじっちゃっていいのかわからなくて、ドキ
 ドキでした。特に、どこまでくだいた言い回しにしちゃっていのかとか。

 でも、ツッコミはまだしも、OKを出すのはもっとドキドキでした ><

 あと、perldoc.jpの訳文のライセンスは、元のPerlモジュールのライセンスと
 解釈したけど、それでいいのか確信は持てないです。

 言葉遊びだらけのdescriptionは、どう訳すか決めかねたので、見送りました。
 具体的には、libyaml-perlのshort descriptionが「YAML Ain't Markup
 Language (tm)」とか。

\preworksection{提案する理想像(ツールとか)、共有したい情報}

\begin{itemize}
 \item 先人の文例を調べられるとうれしいです。対訳を英文で検索、とか。理
       想は翻訳メモリみたいなの
 \item upstreamに対して翻訳しているプロジェクトがすでにある場合、そこと
       話がついてるとやりやすいかも
\end{itemize}

\end{prework}

\begin{prework}{山本 浩之}

\preworksection{適用した主要な方法}
英和辞書と Cross Translation を用いて、ひたすら。

\preworksection{発見した課題}

\begin{itemize}
 \item コメントがついていれば、それぞれの翻訳者・査読者の考え方が分かって良い。
 \item 一つ前の稿より、現時点の稿のほうを英文の近くに持ってきてもらいたい。(特にレビューの時)
 \item 機械的改行されてしまう。
 \item ログインに失敗していても、編集できてしまった。
 \item DDTSS のサーバ証明書がおれおれ証明書。
\end{itemize}

\preworksection{提案する理想像(ツールとか)、共有したい情報}
一行の文字数の制限を明示してもらいたい。(あと何文字とか)

\end{prework}


\begin{prework}{吉野与志仁}

\preworksection{適用した主要な方法}

2009年5月、6月の資料を参考に、DDTSSで翻訳とレビューを行いました。
\preworksection{発見した課題}

\begin{itemize}
 \item 自分が翻訳or修正した(ownerの)ものをさらに修正すると、自分の変更だ
       けdiffがとられることがある
 \item かなり前にfetchされていて、既に原文や(DDTSS以外により)訳が更新され
       ていることがある
 \item spammerに荒らされて訳やコメントが消されたりすることがある
 \item ownerのものでコメント部分のみが変更されても、開かないと気づかない
       
\end{itemize}

\preworksection{方法の改善案}
 \begin{itemize}
  \item 自分より前の人の変更からdiffがとられるといいのかもしれません。た
	だ、既にDBに訳があった場合は、それとのdiffもあるといいかもしれません。
  \item ほぼ毎回 Force fetching\dots にチェックしてfetchしていましたが、
	各ページにボタンがあると使いやすいと思います。
  \item IPからの書き込みを制限したほうがいいのかもしれません。
  \item Message for youにcommentの変更が通知されてもいいかもしれません。
 \end{itemize}
 
\end{prework}
