%; whizzy-master ../debianmeetingresume200911.tex
% 以上の設定をしているため、このファイルで M-x whizzytex すると、whizzytexが利用できます。
\begin{prework}{まえだこうへい}
\preworksection{普段行っている統計処理}

 現場にいるときはあまり必要としなかったのですが、昨年度までのプリセールス
 のときは、プロビジョニングする上でのCPUやメモリのリソース使用率、見積り
 試算、売上試算、現状の企画の仕事では、新規企画との現状サー ビスの利用量、
 料金の対比などを行う際に統計を行ったりしています。

 が、Excelが遅いんですよね。5列2万行程度の計算や、グラフのプロットを行う
 と、もう落ちる落ちる。OSのレスポンスも非常に悪化して、昨年やってた一番
 酷い例では、5分に1度、Excelが落ちて、データ復旧してはグラフ作ってまた落
 ちて、なんてときに、もうExcelだけでなくスプレッドシートは嫌だなと思いま
 した。

 結果が分かっている統計データをプロットするだけならスプレッドシートでも
 良いのかもしれませんが、値のコピーだけをすると、あとで見直したときに、
 このデータはどれから算出したのだったのかを把握できなくなります。でもセ
 ルの参照を多用すると前述の問題は頻発しやすくなります。すると、データ解
 析したりグラフ描画はもうスプレッドシートは止めたいな、というのがきっか
 けで gnuplot を使い始めました。これはこれで便利ですが今回のネタ発表する
 のに当たって、GNU Rを使ってみたら、これはさらに便利ですね。仕事でもっと
 活用しようと思います。


 プライベートは、今回のネタのサンプルデータのような光熱費や、家計簿の統計
 を取るのに OOo を使ってましたが、これもやっぱりデータを出すのが面倒なの
 で、この機会に変更しようと思ってます。OOo自体も止めようかな。
\end{prework}


\begin{prework}{やまねひでき}
\preworksection{「普段行っている統計処理の内容とその処理で行っているハック」}

いや、全く統計処理なるものをやっていないのです…どなたかどういう場面で役立つか、とか
もし初歩から知りたい場合はこれを見ろ!など示唆いただけると大変助かりますです、はい。

\end{prework}

\begin{prework}{日比野 啓}
\preworksection{普段行っている統計処理の内容とその処理で行っているハック}

 普段、統計処理のようなことをほとんどやらないのですが、2,3年前に、
 自分の開発効率を計測するために、OpenOffice.org の表計算機能で
 システム開発にかけた時間を細かく計測してみたことがあります。
 やむをえず手作業になっている本来自動化できそうな部分あり、
 libspreadsheet-writeexcel-perl あたりで何とかできないか気になっていました。
 また機会があれば試してみたいと思っています。
 Rのような処理系も今日をきっかけに使いこなせるようにしておきたいところです。

\end{prework}

\begin{prework}{本庄弘典}
\preworksection{普段行っている統計処理}
Perlを使用してログを集計する程度です。Perlからgnuplotを使用したりはします。ハックはありません。

\end{prework}

\begin{prework}{なかおけいすけ}
\preworksection{普段やっている統計処理の内容}
 職業柄、主に時系列解析と相関解析を行っています。また機器類の校正、精度
 測定のために線形解析も行っています。

 私の職場は高電圧や大電力な装置がそこら中にあるので、時々放電が起こりま
 す。放電によって機器が破損することがあるので、放電箇所を特定しなければ
 なりません。
 放電しそうなところにカメラを置いて録画監視しているのですが、いつ
 放電が起こるかわかりません。
 放電が起こるのを今か今かとずっと見ているわけにはいかないので、録画した
 画像を数値化して分布を見ることで、放電した時刻と場所を特定しています。

\preworksection{統計処理で行っているハック}
 特にハックして使っていないのですが、データを処理した過程を記録すること
 が重要なので、できるだけスクリプトにしています。
\end{prework}

\begin{prework}{吉田@板橋}
\preworksection{普段やっている統計処理の内容}
本来の統計処理という意味ではcalcやExcelで済ませて
しまい、最近はマクロ等を作ることもあまり無く、
せいぜいlookup系やソルバー止まりでまったり使っています。
あとはスクリーニングやご家庭内システム監視をするのに
rubyやpythonまたシェルスクリプトを
使ってますが、かなり監視対象機器に依存している&単純計算が
主なので、統計という程の内容ではないですね。
小ネタとしてログイン時にディスク空き容量がピンチか自動的に
確認できるように下記のようなスクリプトを仕込んでいます。

\begin{commandline}
$ tail .bash\_profile

#Yellow
echo -e "\033[1;33m"
df | grep "9.%" && sleep 15
echo -e "\033[0m"

#Red
echo -e "\033[1;31m"
df | grep "100%" && sleep 30
echo -e "\033[0m"
\end{commandline}

要するに容量が90\%以上使われているパーティションがあると
ログインがそこで15〜30秒止まり、なにが溢れそうか強制的に
確認します。

まあ、統計かというと微妙ですが、
まあ、パーセンテージを使っていると言うことで(笑)

\end{prework}


\begin{prework}{emasaka}
\preworksection{普段行っている統計処理}
統計処理というほどのことはやってません。Excelで集計するぐらいでしょうか。
\end{prework}

\begin{prework}{araki}
\preworksection{araki}

統計ですか。。ぜんぜん使ってないですね。はるか昔にS-langを使うmakefileを書いたけ
ど、それを呼ぶことが7月にあったのが最後。でも勉強はしたいです。というわけでよろ
しく。

\end{prework}

\begin{prework}{あけど ただお}
\preworksection{普段行っている統計処理とハック}
 普段は特に決まった統計処理をしているわけではないので、比較的最近に行っ
 た統計処理(っぽい事)について述べます。

 個人運営サイトのメールサーバの管理を引き受けているものがあり、そこでの
 迷惑メールの着信状況の推移を調べるのにログファイルを grep や sed 等でテ
 キスト処理を行い最後は wc -l にて行数をカウントするスクリプトを仕込んでい
 ました。

 それと関連してトラフィック統計を vnstat で取って月毎の総量変化を見たりし
 ました。またこの結果から iptables にフィルタルールを追加したりもしてみ
 ました。

 結果として、いくつかポイントになるフィルタ設定によりサーバ全体の総トラ
 フィック量が10分の1以下になりました。
\end{prework}
