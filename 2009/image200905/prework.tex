%; whizzy-master ../debianmeetingresume200905.tex
% 以上の設定をしているため、このファイルで M-x whizzytex すると、whizzytexが利用できます。

\begin{prework}{上川純一}

\preworksection{適用した主要な方法}

DDTSS をブラウザで眺めて、翻訳をレビューしてみました。
パッケージのdescriptionだけで分からない部分についてはバグレポートをしま
した。
不明点は debian-doc@jp メーリングリストに質問として投稿しました。

\preworksection{発見した課題}

レビューをして分かったことですが、翻訳の作業で単語の翻訳まではできている
のですが、文章全体として係り受けがおかしく、意味が通っていないものがありました。
また、英語の句読点をそのまま日本語の句読点に置き換えており、そのままでは日本語として
文章が長すぎるものもありました。翻訳作業を直訳作業とすると読みにくい
Descriptionができあがってしまうと思われます。

\preworksection{提案する理想像(ツールとか)、共有したい情報}

\begin{itemize}
 \item Description に対してすでにバグレポートが投稿されているかどうかの
       チェックするツール。

 \item 用語集にすでに掲載されている用語を簡単にウェブブラウザからチェック
       できるgreasemonkey 。

 \item DDTSS で行ったレビュー・コメントなどが debian-doc@jp メーリングリ
       ストに反映すること。
\end{itemize}

\end{prework}

\begin{prework}{まえだこうへい}
\preworksection{適用した主要な方法}
事前課題まで手が回らずじまい。だけだと芸がないので、現在友人と進めている
 CouchDBのWebサイトの翻訳について話します。
\begin{enumerate}
 \item CouchDBのWebサイトをGoogle Docsでまるごと取り込む。1ページにつき、
       翻訳者は限定。他は査読し、一文ずつ翻訳する。直訳ではなく、日本語
       としてわかりやすい文章を重視。
 \item Webサイト、Wikiをある程度の比率まで翻訳を進める。
 \item CouchDBの開発者向けのMLに公開したい旨を相談する。
 \item DebianのCouchDBのパッケージメンテナでもある、開発者からpo4aを使う
       こと、他にもいろいろ課題(システム管理、アップロードの方法、翻訳
       の陳腐化など)があるけどそれを考慮しないといけないよと、アドバイ
       スをもらう。
 \item PO形式でパッチ投げて反映自体はお願いし、更新はなんらかの手段で確
       認して随時翻訳していく旨を伝える。
 \item 今後の方針をどうするかを友人と相談して、今ココ。
\end{enumerate}

\preworksection{発見した課題}
WebサイトはXHTMLになっていないので、まず変換するところから始めないといけ
 ないことと、Wikiは更新が頻繁にありすぎだから、対象から外した方が良いよ
 ね、と友人と相談したところ。翻訳がそもそもやりたいことではなかったので。

\preworksection{提案する理想像(ツールとか)、共有したい情報}
最初の翻訳だけ行ったら、後は手離れする(誰かに引き継ぐ)のが理想ではある
 ものの、協力者を増やさないといけない。そもそもDebianパッケージになって
 いるので、ソースパッケージに含まれるドキュメントなら、Debian-docで翻訳
 して、そっち経由でアップストリームに反映してもらう、というのも手だなと
 いうことを検討中。
\end{prework}


\begin{prework}{明渡忠郎}
\preworksection{適用した主要な方法}
手入力とネット上の英和辞典、必要に応じて Google、wikipedia、その他専門サイトを参照
 
\preworksection{発見した課題}
DNA解析ツールが多くあり、専門用語がまちまちで
それの統一若しくは共通化が必要だと思いました。

\preworksection{提案する理想像(ツールとか)}
専門用語については辞書ページを用意するか、基準にする専門用語のサイトを決めるのが良いかと思います。

共有したい情報
Debian パッケージには数多くの DNA 解析ツールがあって、専門用語の翻訳にあたって注意深く作業する必要があると思います。
DNA 解析について予備知識として下記サイトが参考になるかと思います。

タンパク質/核酸のアライメント解析 
\footnote{\url{http://www.icot.or.jp/ARCHIVE/Museum/SOFTWARE/GIP/gene_alignment.html}
 ※文字コードが ISO-2022-JP なのにエンコード情報は Shift-JIS になっています。}

Debian JP 文書作成の指針 について DDTSS for ja にリンクがあった方が良いかと思います。

\end{prework}

\begin{prework}{日比野 啓}
DDTSSでocamlとlibperl4caml-ocamlのDescriptionを翻訳してみました。
レビューはnasmとlibobjc2をやってみました。

\preworksection{適用した主要な方法}

DDTSSのWebインターフェースから翻訳とレビューを行なってみました。

\preworksection{発見した課題}

自分が翻訳を行なった内容が更新されたときや
コメントが付いたときに、再び見にいかないと更新を知ることが
できないのが気になりました。

\preworksection{提案する理想像(ツールとか)、共有したい情報}

翻訳した人やレビューした人に更新を通知する機能と
それをコントロールする機能が欲しいです。
\end{prework}

\begin{prework}{小川 伸一郎}
 以下,やってみた感想など.
 \preworksection{登録}
 まず登録時の Alias と言うのが直感でわかりませんでした.
 \preworksection{翻訳}
 翻訳は以下の2つのパッケージについてやってみました.
 \begin{itemize}
  \item wgerman-medical
  \item libbio-ruby
 \end{itemize}
 思ったのは,いきなり全てを訳さなければならないのは辛いんじゃないかと言
 うこと.少しずつ翻訳できたり,もしくは途中結果を保存できるた方がいいん
 じゃないかと思いました.

 あとは,最初にどう何をすればいいのかがわかりにくと言うこと.
 わかった当たり前なのかも知れませんが,幅広く協力者を求めるのなら,何か
 しら情報(HowToとか?)あった方がいい気がします.
 \preworksection{レビュー}
 いきなり 'Accept as is' と書かれるのが怖かった.ちゃんと,
 ``'Accept as is' means you agree with this translation.''
 と書かれてはいるんですが,どうにも「えっ,いいのか?」と言う思いが抜け
 きれませんでした.ので,レビューは見ただけです.

 で,multimail を見たときの感想ですが,diff をもう少しわかりやすく表示で
 きないかと思いました.
 さすがに前後にあると,頭の中でいろいろ組み立てないと読めないので,
 上下に異なる部分だけ文字長を併せて表示できたりすると,わかりやすいかな
 と.

 「思った」ばかりの感想になってしまいましたが,こんな感じでした.
\end{prework}

\begin{prework}{やまだたくま}

\preworksection{適用した主要な方法}

DDTSS より原文と訳文を入手して、翻訳支援ツール OmegaT またはテキスト
エディタで翻訳しました。複数のオンライン版の辞書、ペーパー版の辞書、翻訳
ソフト、対象ソフトウェア、Web 検索、書籍の調査を併用しました。
レビュー依頼を debian-doc メーリングリストに投稿しました。
情報提供依頼を YLUG メーリングリストに投稿しました。

\preworksection{発見した課題}

Deiban JP 関連の翻訳のシステムと運用についての情報収集の難易度が高い
状態でした。ドキュメントが整備されていない情報や、数年間アップデートされて
いないページがありました。

要求される訳の品質が不明なため、作業時間や作業手順の見積りが困難でした。

訳の品質安定に必要な執筆基準の整備状況が不十分でした。

\preworksection{提案する理想像(ツールとか)、共有したい情報}

\begin{itemize}
 \item 作業手順ページのアップデート。

 \item 執筆基準と用語集のアップデート。

 \item 執筆基準適合テストツールや翻訳支援ツール用ライブラリの提供ができ
       るとうれしいでしょう。
\end{itemize}

\end{prework}

\begin{prework}{やまねひでき}

\preworksection{適用した主要な方法}

DDTSS をブラウザで眺めて、翻訳&レビュー…したような記憶が微かにあります。

\preworksection{発見した課題}

一人だと寂しすぎる。というのはさておき、全体像と現状の把握が難しく、一体どこからこの山を
登ればいいのかよく分からない状態だった記憶が。

\preworksection{提案する理想像(ツールとか)、共有したい情報}

\begin{itemize}
 \item 見やすいページ構成が欲しい。素朴過ぎて何が何やら分からんのは勘弁。せめてpo-debconfのページ(http://www.debian.org/international/l10n/po-debconf/ja)レベルのは欲しい。
 \item 「一ユーザが読んで理解できる」文章を心がけること。当たり前なんですが、難しいので改めて(逆に言うと、査読はユーザにしてもらう仕組みが欲しいところです)。
\end{itemize}

\end{prework}

\begin{prework}{山本浩之}

\preworksection{適用した主要な方法}

DDTSS をブラウザで眺めて、Cross translation の翻訳結果を参考にしながら、翻訳とレビューをしてみました。

\preworksection{発見した課題}

翻訳に時間がかかりすぎたせいか、なぜかロックがはずれて他の人と重なったみたいです。やっと翻訳したものが消えてしまった時は泣けた。(T\_T)

\preworksection{提案する理想像(ツールとか)、共有したい情報}

\begin{itemize}
 \item なんか使い方が良く分からなかったので、使い方のドキュメントがあると良いです。
 \item コメント欄になにを書けば良いのかも良く分からなかったです。みんなは何を書いているんだろう?
\end{itemize}

\end{prework}

