%; whizzy-master ../debianmeetingresume200912.tex
% 以上の設定をしているため、このファイルで M-x whizzytex すると、whizzytexが利用できます。

\begin{prework}{上川純一}
\preworksection{2009年を振り返って自分は何をしたか}

2009年、半分くらいしかDebian勉強会には参加してません。

DDTSS をこそこそやってましたが、気づいたら今年はDDTSSのサーバがとまった
りといろいろと起きたようです。

Debianパッケージをぼちぼちとメンテナンスしていました。
月に一回以上は活動できていたと思います。

新しい開発の方向としては、 qemubuilder をいろいろといじっていました。
ARM向けの qemu をいじっているのが中心です。
これがしたかったのは Android 携帯をゲットしたからで、
Android アプリもいくつか書きました。

\preworksection{2009年を振り返ってDebian関連で世間では何が起こったか}

ARMの携帯できるLinuxデバイスが一気に増えた気がします。
日本初のAndroid携帯HT03Aが発売され、
SharpのNetwalker がリリースされました。

Intel CPUの高速化はそこまですすんでいない感じですが、4コアのCPUがリリー
スされて良い感じに盛り上がっていると思います。

\end{prework}

\begin{prework}{吉田俊輔}
\preworksection{2009年を振り返って自分は何をしたか}
Debian関係だとDDTSSでレビューを中心に180件程度やっていたようです。
DDTSSは休んだりやったり、まったりやっています。
イベント関連では2月にLinux conf.auに行って、秋はJLSにボランティア参加、
KOFに行って関西Debianの方とキーサインをしたりしていました。
あとは例年通り?あんどきゅめんてっどでびあんのゴーストエディター兼売り子として、
某所で勉強会資料をまとめて本にして頒布しています。
\preworksection{2009年を振り返ってDebian関連で世間では何が起こったか}
やはりLennyリリースが一番でしょう。
基本的にstableを使うチキンなので、メジャーバージョンアップは大きな
ニュースです。
最近のDebianは定期的にリリースされるので各種ソフトもバージョンが
割と新しく、便利に感じています。
おかげで使いたいソフトのために、sid等からバックポートしたり、
そのほかtarボールからビルドする必要等も少なくなり、またその場合の手間も
昔(woodyの頃)に比べてかなり減っておりありがたいです。
\end{prework}

\begin{prework}{まえだこうへい}
\preworksection{2009年を振り返って自分は何をしたか}

ヨメに Debian (Sid)を使わせ始めた、Hack Cafe, Debian JP Project の選挙管
 理委員、Debian 勉強会の運営や、はじめての ITP, DebConf9 参加、はじめて
 の執筆等、ちょこちょこやってきましたが、やろうと思っていたことはなかな
 かできていないのが現状です。(ITPしっ放しにしてしまっているし…。)

来年は、もうすこしマルチタスクで(要領良く)動けるようにしたい、というか
しないといかんなぁと思う年の瀬でした。

\preworksection{2009年を振り返ってDebian関連で世間では何が起こったか}

OpenBlockS 600 が出ましたね。OpenBlockS 266 ほど面白みはないですが、スペッ
 クが上がっても消費電力が低い(5W)のは良いですね。
\end{prework}

\begin{prework}{やまねひでき}
\preworksection{2009年を振り返って自分は何をしたか}
以下で幾つかの項目に分けて振り返ってみます。

\preworksubsection{パッケージの作成と維持}
主にフォントのパッケージについて追加を実施しました。以下が追加されたパッケージです。

\begin{itemize}
 \item ttf-umefont - 「梅フォント」のパッケージです
 \item ttf-umeplus - PCLinuxOSの標準フォント umeplus のパッケージです。
 \item ttf-ipafont/ttf-ipafont-jisx0208/otf-ipafont - IPAフォントです。
 \item ttf-monapo - monaフォントとIPAフォントの合成フォントです
 \item ttf-sawarabi-gothic -「さわらびゴシック」フォントです。
 \item ttf-misaki - 「美咲フォント」です。
 \item ttf-kanjistrokeorders -「漢字の書き順」フォントです。日本語学習者に人気です。
\end{itemize}

あとは ITP したまま作業が止まっていたり reject されて再調整ができていないのが幾つかありますので、
それを片付けていければと思います。

\preworksubsection{翻訳作業}
ウェブについては、Developers News の査読や翻訳作業、それから po-debconf の更新と新規翻訳作業を実施しています。
残念ながら i18n.debian.net サーバが正常に稼働していないので、po-debconf の進捗状況は昨年と比べてどの程度上がっているのかは不明ですが、後退はしていないはずです(現状で 70\% 以上は作業している)。

あとは upstream で何かできれば、と思い、ここ数日ではあるものの GNOME の翻訳更新作業に手を出してみています。
2.30 リリースの頃 (2010/3 かな?) に慌てずに査読までできれば良いですね。

\preworksubsection{イベント/広報系な活動}
Debian/Ubuntu活動関連で記事を少し書かせていただいたり、対談に参加させてもらったりしていました。

\begin{itemize}
 \item Software Design 2009/07 特集「初心者にやさしく,ベテランも満足!「Debian GNU/Linux 5.0(Lenny)をお勧めする理由」
 \item Software Design 2009/10 「Debian GNU/Linuxカンファレンス「Debconf9」取材レポート Debian開発者が集うスペイン14日間」
 \item ThinkIT「TOMOYO Linux徹底解剖 第3回:DebianでTOMOYO Linuxを使う」
 \item ascii.jp「行っとけ!Ubuntu道場」
 \item アスキーメディアワークス「週刊アスキー別冊 さくさくUbuntu!」
\end{itemize}

また、イベントに幾つか参加しました。今年は体調が悪かったこともあり、昨年より少なめです (KOFとか行けなかった…)

\begin{itemize}
 \item Linux Consortium 10 Years Event !!「オープンソースデスクトップの未来」(2009/01)
 \item 第17回オープンソーステクノロジー勉強会@GREE Labs(2009/04)
 \item Ubuntu 9.04 オフラインミーティング (2009/04)
 \item Open Source Conference Tokyo/Fall (2009/11)
\end{itemize}

上記の中で、実質的に発表したのは GREE Labs だけでしたので、来年はもう少し体調を整えて、ネタをかませれば、と思います。


\preworksubsection{Debconf9 参加}
ようやく Debconf に参加することができました。が、とりあえず行っただけで終わってるので、
もう少し実績を積むことと英語でのコミュニケーションを改善できればと思っています。

\preworksection{2009年を振り返ってDebian関連で世間では何が起こったか}
うーん、世間というとそんなにインパクトは与えられていないかなーと個人的には感じています。
\end{prework}

\begin{prework}{本庄弘典}
\preworksection{2009年を振り返って自分は何をしたか}

仕事が忙しく勉強会も欠席しがちでした。
明日から頑張る。

\preworksection{2009年を振り返ってDebian関連で世間では何が起こったか}

SmartQ5が出ました。
\end{prework}

\begin{prework}{キタハラ}
\preworksection{2009年を振り返って自分は何をしたか}

直接 Debian でなく申し訳ないのですが・・・。
この秋、実家に ADSL 引きまして、 Ubuntu インストールしたノート PC を
置いてきました。ネットサーフィン専用機ですが、得に文句もなく普通に
使っているようです。debian 系 linux ユーザを一人増やしたということで。

\preworksection{2009年を振り返ってDebian関連で世間では何が起こったか}

同件多数かも知れませんが、 Lenny リリースですかね。
世間へのインパクトは・・・、私のまわりではあまりなかったような?!
\end{prework}

\begin{prework}{日比野 啓}
\preworksection{2009年を振り返って自分は何をしたか}

\begin{itemize}

\item Debian勉強会でCやPerl以外の言語にも注目してもらおうと、OCamlやCommon Lispの布教活動をしました。
\item Debian勉強会でCommon Lispネタの発表をしました。

\end{itemize}

\preworksection{2009年を振り返ってDebian関連で世間では何が起こったか}

\begin{itemize}

\item lennyリリース。会社のマシンもアップデートが必要になって大変忙しかった。
\item Debian勉強会の宴会でのヨタ話もきっかけの一つになって、日本では初のOCaml特化イベント、 OCaml Meeting Tokyo 2009で開催。みなさんに感謝。

\end{itemize}

\end{prework}

\begin{prework}{岩松 信洋}
\preworksection{2009年を振り返って自分は何をしたか}
\begin{itemize}
\item Debian Developerになった。\\
Debian Developerになったので、スポンサーを行うようにした。
いままでやってもらったことをやるように心がけるようにしている。

\item Debian JP Project Leader になった。\\
なぜか DJPL なっている。

\item Debian SH4 Port 復活\\
Debconf9にいって 他のPorterと作業ができたのが大きい。
この成果物を利用した製品も出たようです。

\item 子供が生まれた\\
子供が生まれたので、生活が一変した。

\end{itemize}

\preworksection{2009年を振り返ってDebian関連で世間では何が起こったか}
\begin{itemize}
\item Lenny がリリースされた
\item GPG鍵 4096R への移行
\item glibc から eglibc への移行
\end{itemize}

\end{prework}

\begin{prework}{吉野与志仁}
\preworksection{2009年を振り返って自分は何をしたか}
 DDTSSは途中サーバが落ちていて滞っていましたが Translations: 165,
 Reviews: 206 くらいやっていたようです.

\preworksection{2009年を振り返ってDebian関連で世間では何が起こったか}
 やっぱりLenny releaseでしょうか... 来年のkFreeBSD official 化(によ
 る盛り上がり)に期待したいですね.
\end{prework}
