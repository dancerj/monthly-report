\begin{prework}{ henrich }
  \begin{enumerate}
  \item Q.hack time に何をしますか?\\
    A. バグレポートを読んで、直せそうなところを整理します。
  \end{enumerate}
\end{prework}

\begin{prework}{ wskoka }
  \begin{enumerate}
  \item Q.hack time に何をしますか?\\
    A. tilegxのdebian化
  \end{enumerate}
\end{prework}

\begin{prework}{ rosh }
  \begin{enumerate}
  \item Q.hack time に何をしますか?\\
    A. 自分の担当するパッケージをメインテインする (wide-dhcpv6とadjtimex)
  \end{enumerate}
\end{prework}

\begin{prework}{ tai }
  \begin{enumerate}
  \item Q.hack time に何をしますか?\\
    A. もくもくとパッケージングリハビリとbugs.d.oの落穂ひろいをします。
  \end{enumerate}
\end{prework}

\begin{prework}{ koedoyoshida }
  \begin{enumerate}
  \item Q.hack time に何をしますか?\\
    A. 予定は未定
  \end{enumerate}
\end{prework}

\begin{prework}{ dictoss }
  \begin{enumerate}
  \item Q.hack time に何をしますか?\\
    A. kfreebsd関連作業
  \end{enumerate}
\end{prework}

\begin{prework}{ nametake }
  \begin{enumerate}
  \item Q.hack time に何をしますか?\\
    A. windowsタブレットにdebianインストール
  \end{enumerate}
\end{prework}

\begin{prework}{ yy\_y\_ja\_jp }
  \begin{enumerate}
  \item Q.hack time に何をしますか?\\
    A. DDTSS
  \end{enumerate}
\end{prework}

\begin{prework}{ かじ }
  \begin{enumerate}
  \item Q.hack time に何をしますか?\\
    A. 前回参加できなかったので、今回はwaylandで簡単なサンプルを作ってみたい。\\
参考にしたいのは\\
http://d.hatena.ne.jp/devm33/20140414/1397473785
  \end{enumerate}
\end{prework}

\begin{prework}{ 野島 }
  \begin{enumerate}
  \item Q.hack time に何をしますか?\\
    A. DDTSSとか
  \end{enumerate}
\end{prework}



