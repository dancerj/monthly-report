\documentclass[cjk,dvipdfmx,10pt,%
hyperref={bookmarks=true,bookmarksnumbered=true,bookmarksopen=false,%
colorlinks=false,%
pdftitle={第 64 回 関西 Debian 勉強会},%
pdfauthor={倉敷・のがた・佐々木・かわだ},%
%pdfinstitute={関西 Debian 勉強会},%
pdfsubject={資料},%
}]{beamer}

\title{第 64 回 関西 Debian 勉強会}
\subtitle{{\scriptsize 資料}}
\author[かわだ てつたろう]{{\large\bf 倉敷・のがた・佐々木・かわだ}}
\institute[Debian JP]{{\normalsize\tt 関西 Debian 勉強会}}
\date{{\small 2012 年 9 月 23 日}}

%\usepackage{amsmath}
%\usepackage{amssymb}
\usepackage{graphicx}
\usepackage{moreverb}
\usepackage[varg]{txfonts}
\AtBeginDvi{\special{pdf:tounicode EUC-UCS2}}
\usetheme{Kyoto}
\def\museincludegraphics{%
  \begingroup
  \catcode`\|=0
  \catcode`\\=12
  \catcode`\#=12
  \includegraphics[width=0.9\textwidth]}
%\renewcommand{\familydefault}{\sfdefault}
%\renewcommand{\kanjifamilydefault}{\sfdefault}
\begin{document}
\settitleslide
\begin{frame}
\titlepage
\end{frame}
\setdefaultslide

\begin{frame}[fragile]
\frametitle{Agenda}

\tableofcontents

\end{frame}

\section{最近の Debian 関係のイベント}

\takahashi[40]{最近の Debian\\関係のイベント}

\begin{frame}[fragile]
\frametitle{第 63 回関西 Debian 勉強会}

\begin{itemize}
\item 日時: 8 月 26 日
\item 場所: 福島区民センター
\end{itemize}
\begin{block}{内容}
  \begin{itemize}
  \item「Debian ではじめる Kerberos 認証」
  \item 「News from EDOS: finding outdated packages」
  \end{itemize}
\end{block}
\end{frame}

\begin{frame}[fragile]
  \frametitle{第 92 回 東京エリア Debian 勉強会}
  \begin{itemize}
  \item 日時: 9 月 8 日
  \item 場所: OSC2012 Tokyo/Fall
  \end{itemize}
  \begin{block}{内容}
    \begin{itemize}
    \item 「次期安定版 Debian 7.0 "Wheezy" の紹介」
    \end{itemize}
  \end{block}
\end{frame}

\begin{frame}[fragile]
  \frametitle{第 0 回 Debian パッケージング道場@楽天}
  \begin{itemize}
  \item 日時: 9 月 22 日
  \item 場所: 楽天
  \end{itemize}
  \begin{block}{内容}
    \begin{itemize}
    \item Debian パッケージ作成
    \end{itemize}
  \end{block}
\end{frame}

\takahashi[50]{そんな\\こんなで}
\takahashi[120]{次}

\section{事前課題発表}

\takahashi[50]{事前課題}

\begin{frame}[fragile]
\frametitle{事前課題}

\begin{block}{今回の事前課題}
  \begin{description}
  \item[事前課題1] Debian を使ってやってみたいことを教えてください。
  \end{description}
\end{block}

\end{frame}

\takahashi[50]{事前課題\\発表}

\begin{frame}
  \frametitle{ 川江 }
  kvmによるWebServer等々の作成
\end{frame}

\begin{frame}
  \frametitle{ かわだてつたろう }
  そろそろ開発などを。
\end{frame}

\begin{frame}
  \frametitle{ yyatsuo }
  Debian は手段ではなく目的なので特に無し。

  強いて言うならば free の精神を広めること。
\end{frame}

\begin{frame}
  \frametitle{ 岡野孝悌 }

翻訳とかしたいですね。

でもって、Debian 関係の翻訳に関する情報をまとめて発表とかしたいですね。

ていうか、かわださんによると近いうちにやることになってますよ。やばい、やばいよ! まずは DDTP に参加だー!
\end{frame}

\begin{frame}
  \frametitle{ 木下 }
  \begin{itemize}
  \item 複数のDebianマシンで分散コンパイル

    →できたらいいな的なレベルです

    現在AndroidOS開発に携わっているのですが、

    Android2.3以降、ターゲットボードのソースをクリーンビルドすると、

    数時間かかるので、これを複数のDebianマシンで分散コンパイルして

    時間短縮できればいいな的な・・・
  \item DTM等DAW環境として使えたら等・・・
  \item 今更ながらですが、VR等、サイバースペースサーバみたいなシステムとして・・・

    ※この辺、ドシロートです。
  \end{itemize}
\end{frame}

\begin{frame}
  \frametitle{ lurdan }
Debian の開発

遅刻します。16時過ぎくらいかな?
\end{frame}

\begin{frame}
  \frametitle{ 清野陽一 }
自由でオープンな社会の発展
\end{frame}

\begin{frame}
  \frametitle{ 山城の国の住人 久保博 }
ソフトウェア開発作業、翻訳作業、Twitter, 年賀状作りなど PC でやる作業全般。そのための環境作りをぼちぼちやって行きたいです。

最近、Xen の準仮想化環境作りに挑戦して、失敗しています。これもやってみたいことの一つです。
\end{frame}

\takahashi[50]{そんな\\こんなで}
\takahashi[120]{次}

\section{clangによるパッケージビルド}
\takahashi[25]{clangによる\\パッケージビルド\\by\\かわだ てつたろう}

\takahashi[50]{そんな\\こんなで}
\takahashi[120]{次}

\section{月刊 Debian Policy 第6回 「文書」}
\takahashi[25]{月刊 Debian Policy\\第6回 「文書」\\by\\岡野孝悌}

\takahashi[50]{そんな\\こんなで}
\takahashi[120]{次}

\section{今後の予定}
\begin{frame}[fragile]
\frametitle{今後の予定}

\begin{block}{第 65 回関西 Debian 勉強会}
\begin{itemize}
  \item 日時: 10 月 28 日(日)
  \item 会場: 港区民センター
\end{itemize}
\end{block}

\begin{block}{第 66 回関西 Debian 勉強会}
\begin{itemize}
  \item 日時: 11 月 9 日(金)、10 日(土)
  \item 関西オープンソース2012
\end{itemize}
\end{block}

\begin{block}{東京エリア Debian 勉強会}
  \begin{itemize}
  \item 日時: 10 月 20 日(土)
  \end{itemize}
\end{block}

\end{frame}


\takahashi[50]{  }


\end{document}
%%% Local Variables:
%%% mode: japanese-latex
%%% TeX-master: t
%%% End:
