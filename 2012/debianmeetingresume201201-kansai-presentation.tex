\documentclass[cjk,dvipdfmx,10pt,%
hyperref={bookmarks=true,bookmarksnumbered=true,bookmarksopen=false,%
colorlinks=false,%
pdftitle={第 55 回 関西 Debian 勉強会@Debian 温泉合宿},%
pdfauthor={倉敷・のがた・佐々木・かわだ},%
%pdfinstitute={関西 Debian 勉強会},%
pdfsubject={資料},%
}]{beamer}

\title{第 55 回 関西 Debian 勉強会}
\subtitle{{\small{資料}}}
\author[かわだ てつたろう]{{\large\textbf{かわだてつたろう}}}
\institute[Debian JP]{{\normalsize\texttt{関西Debian勉強会}}}
\date{{\small 2012 年 1 月 28 日}}

%\usepackage{amsmath}
%\usepackage{amssymb}
\usepackage{graphicx}
\usepackage{moreverb}
\usepackage[varg]{txfonts}
\AtBeginDvi{\special{pdf:tounicode EUC-UCS2}}
\AtBeginSection[]{\begin{frame}<beamer>\frametitle{Agenda}\tableofcontents[currentsection]\end{frame}}
\usetheme{Kyoto}
\def\museincludegraphics{%
  \begingroup
  \catcode`\|=0
  \catcode`\\=12
  \catcode`\#=12
  \includegraphics[width=0.9\textwidth]}
%\renewcommand{\familydefault}{\sfdefault}
%\renewcommand{\kanjifamilydefault}{\sfdefault}
\begin{document}
\settitleslide
\begin{frame}
\titlepage
\end{frame}
\setdefaultslide

\begin{frame}[fragile]
\frametitle{Agenda}
\tableofcontents
\end{frame}

\section{Debian温泉合宿}

\takahashi[50]{というわけで}

\takahashi[70]{Debian\\温泉合宿}

\takahashi[120]{通称}

\takahashi[110]{Deb泉}

\takahashi[50]{開催です!}

\takahashi[50]{諸注意\\連絡事項}

\takahashi[50]{そんな\\こんなで}

\takahashi[50]{事前課題発表}

\begin{frame}[fragile]
\frametitle{事前課題}
\begin{block}{今回の事前課題}
  \begin{description}
  \item 取り組む課題を考えてきてください。 
  \end{description}
\end{block}
\end{frame}

\takahashi[50]{参加者の回答\\兼自己紹介}

\begin{frame}[fragile]
  \frametitle{ かわだてつたろう }
  \begin{enumerate}
  \item パッケージ作成のおさらい
  \item DDR を読む
  \item 勉強会の予定
  \item 人の作業を盗む
  \end{enumerate}
\end{frame}

\begin{frame}[fragile]
  \frametitle{ 清野陽一 }
  \begin{enumerate}
  \item まだ考えてないので当日までに決めますが、多分Debianを使って実際のGISの解析をやると思います。
  \end{enumerate}
\end{frame}

\begin{frame}[fragile]
  \frametitle{ 甲斐正三 }
  \begin{enumerate}
  \item Deianにおける組込み環境の研究
    \begin{itemize}
    \item Cortex-mx関連
    \end{itemize}
  \end{enumerate}
\end{frame}

\begin{frame}[fragile]
  \frametitle{ 佐々木洋平 }
  \begin{enumerate}
  \item Debian Ruby In Wheezy Transition をアレコレ
  \item NM の進捗
  \item Developer Reference の査読
  \item 自作のパッケージ群のライセンス整理と ITP
  \end{enumerate}
\end{frame}

\begin{frame}[fragile]
  \frametitle{ kozo2 }
  \begin{enumerate}
  \item 12月に保留していたt-codeのDebianパッケージ作成
  \item t-codeのメンテナさんと連絡を取る、あわよくばメンテ権限をもらう
  \item debhelperの使い方を学ぶ
  \item quiltの使い方を学ぶ
  \end{enumerate}
\end{frame}

\begin{frame}[fragile]
  \frametitle{ Y.YATSUO }
  \begin{enumerate}
  \item Debianセキュア化ガイドの翻訳
  \item debhelperのコマンドを眺める
  \item 息抜きにpythonで遊ぶ
  \end{enumerate}
\end{frame}

\begin{frame}[fragile]
  \frametitle{ lurdan }
  \begin{enumerate}
  \item ITPしている mha-mysql を仕上げて RFS する
  \item hyperestraier を更新する
  \item 多分まだ終わってないので NM のメール返信を続ける
  \end{enumerate}
\end{frame}

\takahashi[50]{そんな\\こんなで}
\takahashi[70]{Let's Hacking!}

\takahashi[50]{そんな\\こんなで}
\takahashi[120]{次}

\takahashi[50]{成果発表}

\takahashi[50]{そんな\\こんなで}
\takahashi[120]{次}

\begin{frame}[fragile]
\frametitle{今後の予定}

\begin{block}{第 56 回関西 Debian 勉強会}
  \begin{itemize}
  \item 日時: 2012年 2 月 26日(日)
  \item 会場: 大阪福島区民センター
  \item 内容:
    \begin{enumerate}
    \item 「Autofs と PAM--chroot で作る、Debian GNU/Linux 6.0 の安全なマルチユーザー環境」
    \item 月刊:Debian Policy
    \end{enumerate}
  \end{itemize}
\end{block}
\end{frame}

\takahashi[50]{  }

\end{document}
%%% Local Variables:
%%% mode: japanese-latex
%%% TeX-master: t
%%% End:
