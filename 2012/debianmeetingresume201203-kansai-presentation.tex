\documentclass[cjk,dvipdfmx,12pt,%
hyperref={bookmarks=true,bookmarksnumbered=true,bookmarksopen=false,%
colorlinks=false,%
pdftitle={第 57 回 関西 Debian 勉強会},%
pdfauthor={倉敷・のがた・河田・佐々木},%
%pdfinstitute={関西 Debian 勉強会},%
pdfsubject={資料},%
}]{beamer}

\title{第 57 回 関西 Debian 勉強会}
\subtitle{{\scriptsize 資料}}
\author[佐々木 洋平]{{\large\bf 倉敷・のがた・河田・佐々木}}
\institute[Debian JP]{{\normalsize\tt 関西 Debian 勉強会}}
\date{{\small 2012 年 3 月 25 日}}

%\usepackage{amsmath}
%\usepackage{amssymb}
\usepackage{graphicx}
\usepackage{moreverb}
\usepackage[varg]{txfonts}
\AtBeginDvi{\special{pdf:tounicode EUC-UCS2}}
\usetheme{Kyoto}
\def\museincludegraphics{%
  \begingroup
  \catcode`\|=0
  \catcode`\\=12
  \catcode`\#=12
  \includegraphics[width=0.9\textwidth]}
%\renewcommand{\familydefault}{\sfdefault}
%\renewcommand{\kanjifamilydefault}{\sfdefault}
\begin{document}
\settitleslide
\begin{frame}
\titlepage
\end{frame}
\setdefaultslide

\begin{frame}[fragile]
\frametitle{Agenda}

\tableofcontents

\end{frame}

\section{最近の Debian 関係のイベント}


\takahashi[40]{最近の Debian\\関係のイベント}

\begin{frame}[fragile]
\frametitle{第 56 回関西 Debian 勉強会}

\begin{itemize}
\item 日時: 2 月 26 日
\item 於: 大阪福島区民センター
\end{itemize}

\begin{block}{内容}
  \begin{itemize}
  \item Autofs と pam\_chroot で作るマルチユーザ環境
  \item 月刊(?) t-code
  \item 月刊 Debian Policy
  \end{itemize}
\end{block}
ネタ出しは随時行なっております! 皆様よろしく!!
\end{frame}

\begin{frame}[fragile]
  \frametitle{第 86 回 東京エリア Debian 勉強会}
  \begin{itemize}
  \item  日時: 3月 17 日
  \item 於: OSC 2012 Tokyo Spring
  \end{itemize}
  \begin{block}{内容}
    \begin{itemize}
    \item ブース展示
    \item セッション: 「Apache2/HTTP サーバから始めるDebian」
      \begin{itemize}
      \item 『Webサーバとして世界で一番採用されているLinuxディストリビュー
        ションであるDebianのApache2について「ユーザ視点」から語る勉強会』
      \end{itemize}
    \end{itemize}
  \end{block}
\end{frame}

\takahashi[50]{そんな\\こんなで}
\takahashi[120]{次}

\section{事前課題発表}

\takahashi[50]{事前課題}

\begin{frame}[fragile]
\frametitle{事前課題}

\begin{block}{今回の事前課題}
  \begin{description}
  \item[事前課題1] Debian Policy の第5章を読んできて下さい。
  \item[事前課題2] 何か一つパッケージを取得して
    debian/control ファイルを読んできてください。
    そのパッケージの control ファイルの内容について簡単に、
    わからない箇所は調べて、当日紹介して下さい。
  \item[事前課題3] 勉強会開催をどのような媒体、方法で告知を出してもらいた
    いですか。
  \end{description}
\end{block}

\end{frame}

\takahashi[50]{事前課題\\発表}

\begin{frame}[fragile]
\frametitle{ 榎真治 }
\begin{center}
  (無回答)
\end{center}
\end{frame}

\begin{frame}[fragile]
\frametitle{ 山下康成 }
\begin{center}
  げほげほっ!
\end{center}
\end{frame}

\begin{frame}[fragile]
\frametitle{ 川江 }
  \begin{enumerate}
  \item 読んできます
  \item 努力します
  \item 今のままでいいと思いますが、個人的な事情として「多忙」な人が気軽
    に『参加』できるような「方法」があればありがたいです。
  \end{enumerate}
\end{frame}

\begin{frame}[fragile]
\frametitle{ かわだてつたろう }
  \begin{enumerate}
  \item 読んでおきます。
  \item cpp の control を読みました。
    \begin{itemize}
    \item sid:amd64 で "Multi-Arch: allowd" となっているのはこのパッケージだけのようです。どのような場合に指定するとよいのか今一つわかっていません。
    \end{itemize}
  \item Debian JP のサイト。。。
  \end{enumerate}
\end{frame}

\begin{frame}[fragile]
\frametitle{ kozo2 }
  \begin{enumerate}
  \item 読みます
  \item t-codeの読みます
  \item 媒体,方法 twitter, ML 位しか思い浮かばんとです
  \end{enumerate}
\end{frame}

\begin{frame}[fragile]
\frametitle{ 山田 洋平 }
  \begin{enumerate}
  \item 当日までに読んで来ます。
  \item 当日までに読んで来ます。
  \item カレンダーのイベント
  \end{enumerate}
\end{frame}

\begin{frame}[fragile]
\frametitle{ 酒井 忠紀 }
\begin{enumerate}
  \item ざっくり読みました
  \item ruby1.8 の debian/controlファイルをざっくり読みました。
    今回、発表させて頂く Konoha に関連することですが、
    debian/controlファイルに関する疑問点を以下に記述します。(次のスライドへ)
  \item 現状のままでもよいと思いますが、ATNDなどを使用すると、より多くの人の目に留まるかもしれません。
\end{enumerate}
\end{frame}

\begin{frame}[fragile]
\frametitle{ 酒井 忠紀 }
\begin{description}
\item[Maintainer]
  emailアドレスの名前にピリオドが含まれていると問題が
  あるようですが、以下以外であれば、問題ない認識で
  よいでしょうか?
  \begin{itemize}
  \item 連続してピリオドを使用している
  \item '@' の直前にピリオドを使用している
  \end{itemize}
\item[Priority]
  新規パッケージで他と競合しないものは、optional で
  よい認識で問題ないでしょうか?
\item[Architecture]
  ここで指定するアーキテクチャが、実際に自動ビルドされて
  Debianアーカイブに置かれるアーキテクチャになる認識でよいで
  しょうか?
\end{description}
\end{frame}

\begin{frame}[fragile]
\frametitle{ 山城の国の住人 久保博 }
  \begin{itemize}
  \item はい、読みます
  \item 当日までに何とか。
  \item 発表に関係の深い単語のハッシュタグをつけて twitter で
  \end{itemize}
\end{frame}

\begin{frame}[fragile]
\frametitle{ よしだともひろ }
  \begin{itemize}
  \item Debian JP Projectの日本語訳に一通り目を通しました。
  \item e2wmのdebian/controlを見ました
  \item Debian JPのメーリングリストでよいと思います。
  \end{itemize}
\end{frame}

\begin{frame}[fragile]
\frametitle{ のがたじゅん }
  OSC愛媛から日曜の朝、鈍行で帰るのでたどり着けませーん。
  \begin{itemize}
  \item 前はmixiにも告知を流してたけど、今ならぐぐたすかfacebookか…といっても、あまりフィットする感じではないのでtwitterぐらいなのかなぁと思ったり。
  \end{itemize}
\end{frame}

\begin{frame}[fragile]
\frametitle{ yyatsuo }
  3. 今のままで特に不都合無いです
\end{frame}

\begin{frame}[fragile]
\frametitle{ 佐々木洋平 }
  \begin{itemize}
  \item 了解しました。
  \item {\scriptsize{\tt{git.debian.org:/git/pkg-ruby-extras/pkg-ruby-extras.git}}}
    以下にある control はひととおり目を通しています。
  \item twitter の通知の自動化はともかくとして、他になにか無いかなぁ.
    (人力ではなくて、適当に自動化できることが前提だけれど). 告知と、締切前の再告知とかできると良いのだけれど.
    新入生獲得(?)に繋がりそうな他のアイデアないかねぇ.
  \end{itemize}
\end{frame}

\takahashi[50]{そんな\\こんなで}
\takahashi[120]{次}

\section{Konoha の Debianパッケージ化について}

\takahashi[25]{Konoha の Debianパッケージ化について\\by\\酒井 忠紀}

\takahashi[50]{そんな\\こんなで}
\takahashi[120]{次}

\section{月刊 t-code パッケージ修正}

\takahashi[25]{月刊 t-code パッケージ修正 \\ by \\西田 孝三}

\section{月刊 Debian Policy 第2回 「Controlファイルについて 」}
\takahashi[25]{月刊 Debian Policy 第2回 「Controlファイルについて 」 \\by\\ 八津尾 雄介}

\section{新年度からのネタ/スケジュールに関して}

\takahashi[25]{新年度からのネタ/スケジュールに関して \\by\\司会: 佐々木洋平}
\takahashi[25]{$<$閑話休題$>$}
\takahashi[50]{pxdvi のパッケージができましたYo!!}

\begin{frame}[fragile]
\frametitle{pxdvi: build}

unstable 使いは是非テストして下さい

\begin{commandline}
  $ aptitutte install git-buildpackage
  $ gbp-clone http://dennou-k.gfd-dennou.org/member/uwabami/tmp/pxdvi.git
  $ cd pxdvi
  $ git-buildpackage
\end{commandline}
\end{frame}

\takahashi[25]{$<$/閑話休題$>$}

\begin{frame}[fragile]
  \frametitle{ネタ/スケジュール/告知}
  \begin{itemize}
  \item ネタ
    \begin{itemize}
    \item 配布資料参照
    \end{itemize}
  \item スケジュールと告知
    \begin{itemize}
    \item ML: 一週間前, 締切直前.
      \begin{itemize}
      \item そもそも締切って? 印刷の都合だけ?
      \item OSC-Kansai や Ubuntu-JP はどうだろう?
      \end{itemize}
    \item Web告知, twitter. ついでに IT 勉強会は?
    \end{itemize}
  \end{itemize}
\end{frame}

\takahashi[50]{そんな\\こんなで}
\takahashi[120]{次}

\begin{frame}[fragile]
\frametitle{今後の予定}


\begin{block}{第 58 回関西 Debian 勉強会}
\begin{itemize}
  \item 日時: 4 月 22 日
  \item 会場: 福島区民センター
  \item 内容: 月刊Debian Policy, 月刊t-code(?), ライセンスと著作権(?)
\end{itemize}
\end{block}


\end{frame}



\takahashi[50]{  }


\end{document}
%%% Local Variables:
%%% mode: japanese-latex
%%% TeX-master: t
%%% End:
