\documentclass[cjk,dvipdfmx,10pt,%
hyperref={bookmarks=true,bookmarksnumbered=true,bookmarksopen=false,%
colorlinks=false,%
pdftitle={第 61 回 関西 Debian 勉強会},%
pdfauthor={倉敷・のがた・かわだ・佐々木},%
%pdfinstitute={関西 Debian 勉強会},%
pdfsubject={資料},%
}]{beamer}

\title{第 61 回 関西 Debian 勉強会}
\subtitle{{\scriptsize 資料}}
\author[かわだ てつたろう]{{\large\bf 倉敷・のがた・かわだ・佐々木}}
\institute[Debian JP]{{\normalsize\tt 関西 Debian 勉強会}}
\date{{\small 2012 年 7 月 22 日}}

%\usepackage{amsmath}
%\usepackage{amssymb}
\usepackage{graphicx}
\usepackage{moreverb}
\usepackage[varg]{txfonts}
\AtBeginDvi{\special{pdf:tounicode EUC-UCS2}}
\usetheme{Kyoto}
\def\museincludegraphics{%
  \begingroup
  \catcode`\|=0
  \catcode`\\=12
  \catcode`\#=12
  \includegraphics[width=0.9\textwidth]}
%\renewcommand{\familydefault}{\sfdefault}
%\renewcommand{\kanjifamilydefault}{\sfdefault}
\begin{document}
\settitleslide
\begin{frame}
\titlepage
\end{frame}
\setdefaultslide

\begin{frame}[fragile]
\frametitle{Agenda}

\tableofcontents

\end{frame}

\section{最近の Debian 関係のイベント}

\takahashi[40]{最近の Debian\\関係のイベント}

\begin{frame}[fragile]
\frametitle{大統一 Debian 勉強会}

\begin{itemize}
\item 日時: 6 月 23 日
\end{itemize}

\begin{block}{内容}
  この後詳しく
\end{block}
\end{frame}

\begin{frame}[fragile]
  \frametitle{第 90 回 東京エリア Debian 勉強会}
  \begin{itemize}
  \item  日時: 7 月 21 日
  \end{itemize}
  \begin{block}{内容}
    \begin{itemize}
    \item MacBook Air に Debian
    \item Debconf12 について語るの二本立てでした。
    \end{itemize}
  \end{block}
\end{frame}

\takahashi[50]{そんな\\こんなで}
\takahashi[120]{次}

\section{事前課題発表}

\takahashi[50]{事前課題}

\begin{frame}[fragile]
\frametitle{事前課題}

\begin{block}{今回の事前課題}
  \begin{description}
  \item[事前課題1] Linux(Unix?)における login 時のユーザ認証の流れについて調べて(復習して)おいて下さい。\\
    大統一 Debian 勉強会の西山さんの発表資料などを参考にしてください。

  \item[事前課題2] man 5 ldif などを参考に ldif ファイル形式のフォーマットを説明してください。\\
    何か一つでかまいませんのでディレクティブ、エントリの扱い方などの具体例をあげて説明してください。\\
    ldif.5.gz は ldap-utils パッケージで提供されています。
  \end{description}
\end{block}

\end{frame}

\takahashi[50]{事前課題\\発表}

\begin{frame}
  \frametitle{ のがたじゅん }
  (無回答)
\end{frame}


\begin{frame}
  \frametitle{ 佐々木洋平 }
  とりあえず発表で。
\end{frame}

\begin{frame}
  \frametitle{ 川江 }
  (無回答)
\end{frame}

\begin{frame}
  \frametitle{ かわだてつたろう }
  \begin{enumerate}
  \item 西山さんの資料を読みました。
  \item フォーマットは次の形式。base64 エンコードと URL の場合が異なる。\\
    dn: $<$distinguished name$>$\\
    $<$nattrdesc$>$: $<$attrvalue$>$\\
    $<$attrdesc$>$:: $<$base64-encoded-value$>$\\
    $<$attrdesc$>$:$<$ $<$URL$>$\\
    
    エントリを変更するには changetype ディレクティブ、属性には modify, add, delete, modrdn を指定する。modrdn は識別名を変更するのに使用する。\\
    changetype: $<$[modify|add|delete|modrdn]$>$
  \end{enumerate}
\end{frame}

\begin{frame}
  \frametitle{ 岡野孝悌 }
  申し訳ありませんが間に合いませんでした。

  (ldif(5)とRFC 2849は読みましたがそもそもLDAPがさっぱりわからん)
\end{frame}

\begin{frame}
  \frametitle{ 甲斐正三 }
  回答無しで済みません。

  極力勉強会までにみておきます。
\end{frame}

\begin{frame}
  \frametitle{ lurdan }
  \begin{enumerate}
  \item login/ssh/xdm → NSS → PAM → (module いろいろ)
  \item
    cn=lurdan,ou=People,dc=debian,dc=or,dc=jp (ノード名)\\
    changetype: add (どういう変更をするのか)\\
    objectClass: posixAccount (エントリ全体の意味付け)\\
    uid: 1000\\
    homeDirectory: /home/lurdan\\
    shell: /usr/bin/zsh\\
    ...\\
    ...
  \end{enumerate}
\end{frame}

\begin{frame}
  \frametitle{ 山城の国の住人 久保博 }
  \begin{enumerate}
  \item 西山さんの発表資料を読みました。
  \item ldif ディレクトリオブジェクトを記述するファイル形式です。テキスト
    形式の一種で、テキストエディタで編集できます。複数のエントリが並び、
    エントリ間は空行で区切られます。ldif のエントリは、ディレクトリオブジェ
    クトを表します。エントリの欄は 属性名と属性値をコロンで区切った行が並
    んだ形をしています。最初に現れる属性名 dn は、一意にそのオブジェクト
    を特定する、階層的な名前です。その以降、その他の属性の名前とその値が
    並びます。同じ属性名の行が複数現れても構いません。特別な属性とし
    て、objectClass があり、オブジェクトのスキーマを表します。一つのエン
    トリは、複数の objectClass を持てます。
  \end{enumerate}
\end{frame}

\begin{frame}
  \frametitle{ daisuke\_oka@nanosoftware.biz }
  (無回答)
\end{frame}

\begin{frame}
  \frametitle{ kino }
  \begin{enumerate}
  \item ldif ファイル形式のフォーマット
    dn: (distinguished name) と\\
    attrdesc: attrvalue  からなる属性の組み合わせで一つのエントリーを記述。\\
    objectClass: がよくわかってません。
  \item どれを使ってよいのかよくわかってません。
  \end{enumerate}
\end{frame}

\takahashi[50]{そんな\\こんなで}
\takahashi[120]{次}

\section{大統一 Debian 勉強会の報告 by Debian JP}
\takahashi[25]{大統一 Debian 勉強会の報告 \\by\\ Debian JP}

\takahashi[50]{そんな\\こんなで}
\takahashi[120]{次}

\section{Debian で作る LDAP サーバ by 佐々木 洋平}
\takahashi[25]{ Debian で作る LDAP サーバ \\by\\ 佐々木 洋平}

\takahashi[50]{そんな\\こんなで}
\takahashi[120]{次}

\section{月刊 Debian Policy 第5回 「ソースパッケージ」 by 甲斐 正三}
\takahashi[25]{月刊 Debian Policy \\第5回「ソースパッケージ」 \\by\\ 甲斐 正三}

\takahashi[50]{そんな\\こんなで}
\takahashi[120]{次}

\section{今後の予定}
\begin{frame}[fragile]
\frametitle{今後の予定}

\begin{block}{OSC2012 Kansai@Kyoto}
\begin{itemize}
  \item 日時: 8 月 3 日(金)、4 日(土)
  \item 会場: 京都リサーチパーク
\end{itemize}
\end{block}

\begin{block}{第 63 回関西 Debian 勉強会}
\begin{itemize}
  \item 日時: 8 月 26 日(日)
  \item 会場: 福島区民センター
\end{itemize}
\end{block}

\begin{block}{福岡 Debian 勉強会}
  \begin{itemize}
  \item 日時: 7 月 28 日(土)
  \end{itemize}
\end{block}

\end{frame}


\takahashi[50]{  }


\end{document}
%%% Local Variables:
%%% mode: japanese-latex
%%% TeX-master: t
%%% End:
