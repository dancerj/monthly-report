%; whizzy-master ../debianmeetingresume201112.tex
% 以上の設定をしているため、このファイルで M-x whizzytex すると、whizzytexが利用できます。
%
% ちなみに、クイズは別ブランチで作成し、のちにマージします。逆にマージし
% ないようにしましょう。
% (shell-command "git checkout quiz-prepare")

\santaku
{11月終わり頃にルートファイルシステムの構造について議論を呼んでます。内容は?}
{/userを作る}
{/bin,/sbin,/libの実体を/usr以下に移動して、代わりにシンボリックリンクにする}
{/etcの実体を/usr以下に移動して、代わりにシンボリックリンクにする}
{B}
{他主要ディストリビューションが採用検討中...}

\santaku
{sun-java6がDebianパッケージとして配布できなくなりました。代わりにDebianで推奨されるJavaは?}
{openjdk}
{gcj-jdk}
{coco-java}
{A}
{残念だ>○racle}

\santaku
{11/19に長らく活動を停止していたパッケージチームが復活宣言をしました。どれでしょう?}
{CORBA packaging team}
{Ham-radio packaging team}
{SDL packaging team}
{C}
{これからも頑張って欲しいですね}

\santaku
{10/28〜30でMiniDebconf2011が開かれました。どこの国でしょう?}
{ニカラグア}
{インド}
{フランス}
{B}
{来年は日本がいいなぁ}

\santaku
{Wheezyフリーズの為のBSPが各国で開かれました。ドイツとどこ?}
{フランス}
{ニカラグア}
{ポーランド}
{C}
{Wheezyのフリーズは2012/6なので、開発作業はお早めに}

\santaku
{armhf がunstableにはいったのはいつか}
{2011-11-24 1952}
{2013-11-24 1952}
{2001-11-24 1952}
{A}
{dinstall mirror pulse の時間です。}

\santaku
{s390x がunstableにはいったのはいつか}
{2011-11-25 0152}
{2013-11-25 0152}
{2001-11-25 0152}
{A}
{dinstall mirror pulse の時間です。}

\santaku
{1/17にaliothになにがおきたか}
{vasks.debian.orgが起動しなくなった}
{wagner.debian.orgが起動しなくなった}
{SOPAの抗議をはじめた}
{A}
{}

\santaku
{NM process のNMは何を意味することになったか}
{New Maintainer}
{New Member}
{New Moemoe}
{B}
{New MaintainerからNew Memberに切り替わりました}

\santaku
{REVUになにがおきるといっているか}
{universeを拡大}
{Debianを必要なくする}
{mentors.debian.netに統合}
{C}
{}

\santaku
{トレードマークについての連絡先は}
{trademark@debian.org}
{trade@debian.net}
{iwamatsu@debian.org}
{A}
{}

\santaku
{win32-loader.exeの新機能は}
{Debian GNU/Hurdのインストール}
{Debian GNU/kFreeBSDのインストール}
{Debian GNU/Linuxのインストール}
{A}
{win32-loader.exeはWindowsで起動するとDebian-installerを起動できるように
してくれるツール。今回はHurdもインストールできるようになりました。}

\santaku
{wiki.debian.orgのlaunchpadバグ対応を利用するにはどのタグを使うか}
{UbuntuBug}
{DebianBug}
{Hoge}
{A}
{}

\santaku
{dh-execとはなにか}
{実行可能な設定ファイルの出力を使う仕組み}
{どんなものでも実行する仕組み}
{実行、実行、実行}
{A}
{}

\santaku
{Derivatives Census \url{http://wiki.debian.org/Derivatives/Census}には
なにがかいてあるか}
{Debianの正当な後継者の一覧}
{Debianからの派生物の一覧}
{Debianをdisってる人の一覧}
{B}
{}

\santaku
{\url{http://debtags.debian.net/}のリニューアルでは何をしたか}
{DjangoとjQueryでの書き直し}
{Debianベースでの再実装}
{ocamlで実装しなおした}
{A}
{}

\santaku
{Debianの監査役としてがんばっているのは誰か}
{Nobuhiro Iwamatsu}
{Stefano Zacchiroli}
{Martin Michlmayr}
{C}
{}

\santaku
{kassiaとlisztはいくらするのか}
{10,000USD}
{100万円}
{11'792.9 EUR}
{C}
{}

\santaku
{Portland BSPで使ったsbuildインスタンスはいくらしたか}
{70USD}
{700USD}
{7000USD}
{A}
{}

\santaku
{Lenny のセキュリティサポートが終わったのはいつ?}
{2012/02/06}
{2012/02/07}
{2012/02/08}
{A}
{}

\santaku
{2012/01/28 に更新された Squeeze のヴァージョンは?}
{6.0.4}
{6.1.0}
{20120128}
{A}
{}

\santaku
{Debian Game チームが 2/25から2/26まで行うイベントは何か?}
{どれだけの Windows のゲームが Wine 上で動作するか検証するパーティ}
{Debian で提供されているゲームパッケージのスクリーンショットを撮りまくるパーティ}
{Debian で提供されているゲームを48時間連続プレイするパーティ}
{B}
{}

\santaku
{Wheezy で採用される Linux カーネルバージョンは?}
{2.6.39}
{3.2}
{4.0}
{B}
{}

\santaku
{pts.debian.org で表示されるようになった情報は?}
{パッケージメンテナが誕生日の日は「おめでとう」と出る。}
{パッケージを乗っ取ろうとしている人の情報}
{パッケージ Transition 情報}
{C}
{}

\santaku
{アクセプトされたDEPは?}
{DEP 3}
{3 DEP }
{DEP DEP DEP}
{A}
{DEP3 は Patch tagging guideline. Debian Enhancement Proposals}

\santaku
{今年度のDebian JP会長は誰か?}
{Kouhei Maeda}
{Nobuhiro Iwamatsu}
{Junichi Uekawa}
{A}
{前年度に引き続き前田さんが信任されました。またよろしくお願いします。}


\santaku
{Debian.org DPL 選挙は誰が立候補したか}
{Stefano Zacchiroli}
{Nobuhiro Iwamatsu}
{Kouhei Maeda}
{A}
{前年度に引き続きZacchiroliさん頑張ってます。}

\santaku
{Debconf12のsuponsordな参加の締切りはいつ}
{4月末}
{5月15日}
{5月末}
{B}
{うっかりすると過ぎてしまうので、気をつけましょう。また、UTCなのかニカラグアのローカルタイムなのかちょっとわからないので、5/15に登録開始するのは避けた方がよいかも。本家アナウンス\url{http://debconf12.debconf.org/}参照。}

\santaku
{大統一Debian勉強会での発表の公募(CFP)はいつが締切り?}
{もう過ぎた}
{4月末}
{4月22日}
{C}
{大統一Debian勉強会で発表できるチャンスです。締切りは忘れずに。}

\santaku
{experimental版のapacheのパッケージのバージョンはいくつ?}
{2.3}
{2.4.0}
{2.4.2}
{C}
{3/22にアナウンスがありました。upstream側も2.4.2です。最新版ですね。BUG見つけましたら、BUG Report書きましょう。}

\santaku
{Debian Eduはまたの名を何というでしょう?}
{emdebian}
{Scientific Linux}
{Skolelinux}
{C}
{Debian Eduとは教育機関向けに作られたDebianベースのディストリビューションの事です。昔はSkolelinuxという名前で開発されていた物だそうです。正式なDebianのサブプロジェクトです。先日新しいバージョンがアナウンスされました。}

\santaku
{3/30にDebian Projectが加盟した団体の名前は?}
{OSC}
{OSI}
{ETF}
{B}
{これでまたFree Softwareとして磨きがかかりました。OSIはOpen Source Initiativeの略だそうです。}

\santaku
{Debian Projectのgobbyサーバーとしてgobby.debian.orgがアナウンスされました。ところでgobbyって何?}
{旧ソ連で開発された諜報活動用ソフトウェア}
{churroの代替サーバ}
{エディタの名前}
{C}
{DebconfのBOF会場でよく使われていますエディタです。サーバを介する事により、複数人で同時に1つの文章を同時に編集できます。Debconfでは、リアルタイムに議事と議事録がBOF参加者によってどんどん編集されていく様はおもしろいです。ちなみにchurroとはi18n.debian.orgの事です。}

\santaku
{Debian installer 7.0 alpha 1 のリリース日は}
{5/13}
{6/13}
{4/13}
{A}
{Wheezyのインストーラーのアルファリリースが出たので皆さん試してください。}

\santaku
{Cyril Brulebが6月にWheezyをフリーズすると発表したが、Transitionの締め切りはいつだといっているか}
{5月13日}
{6月10日}
{5月20日}
{C}
{Transitionするなら5月20日までにバグをファイルしておけとのこと。Transitionというのはざっくりというと多数のパッケージが相互に依存しているような変更。例えば、ライブラリのABIが変わるだとか。}
