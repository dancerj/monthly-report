%; whizzy-master ../debianmeetingresume201211.tex
% 以上の設定をしているため、このファイルで M-x whizzytex すると、whizzytexが利用できます。
%

\santaku
{FTP master にあたらしく参加したのは}
{iwamatsu}
{ansgar}
{bdale}
{B}
{Ansgar が新しく参加しました。mhy, joerg, ansgar の三人体制に}

\santaku
{pdiffで何が改善されたか}
{最大2つのDiffをダウンロードすれば良いように変更になった}
{一日10個づつDiffを生成するようになった}
{Diffってなにそれおいしいの?}
{A}
{apt-get update の遅さがマシになりますね。}

\santaku
{CTTE 573745 で何が決定されたか}
{Mattias Klose クビ}
{python 終了のお知らせ}
{みんな仲良くしようね}
{C}
{python のメンテナのコミュニケーション不足についての議論は結局みんな仲良
くしましょうという結論になりましたね。}

\santaku
{新しくFront Deskのメンバーになったのは}
{Kouhei Maeda}
{Iwamatsu}
{Jonathan Wiltshire}
{C}
{4人になりました:
 Bernd Zeimetz      (bzed)
 Enrico Zini        (enrico)
 Jan Hauke Rahm     (jhr)
 Jonathan Wiltshire (jmw)
}

\santaku
{debian installer 7.0 beta3 の新機能ではないのはどれか}
{ipv6}
{UEFI}
{grub2}
{C}
{}

\santaku
{Debconf13 はどこで開催されるか}
{スイス}
{日本}
{中国}
{A}
{}

\santaku
{debian-cloudは何をするメーリングリストか}
{人をけむにまくため}
{クラウドサービスでDebianを利用する}
{エアリスー}
{B}
{}

\santaku
{Official Logo が変更されたのはなぜか}
{古いOfficial LogoがDFSG Freeじゃなかったから}
{時代に合わなくなってきたから}
{DPLの趣味}
{A}
{}

\santaku
{codesearch.debian.net は何をするサービスか}
{正規表現でソースを検索できる}
{ソースコードクレクレ}
{ブログサービス}
{A}
{}
