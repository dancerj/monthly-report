%; whizzy-master ../debianmeetingresume201211.tex
% 以上の設定をしているため、このファイルで M-x whizzytex すると、whizzytexが利用できます。
%

\begin{prework}{ koedoyoshida }

\begin{itemize}
 \item Wheezyインストーラのパーティション構成時のデータ消去待ち時間。
 暗号化パーティションを作ろうとすると果てしなく待たされる。
 \item 
 Wheezyの壁紙。いけてない感じでOSCとかの展示で適当なものを選ぶのが面倒で結局squeezeやDebianの(過去の)ジェネリックなものを選ぶことに...
 \item 
 デバッグシンボルを含んだバイナリがない。
 以前大統一で岩松さんが発表していた話が進んでいるとうれしい。
\end{itemize}

\end{prework}

\begin{prework}{ キタハラ }

私が使用する範囲ではありません。

\end{prework}

\begin{prework}{ MATOHARA }

あまり思いつかないですが、人と話をしているとき以下のようなことを言われたことがあります。
\begin{itemize}
 \item 
 Debianは規定値の設定がいけてないので設定を沢山いじらないと運用出来なくて工数が無駄に掛かる
 \item 
 Debianはインストールが難しい
\end{itemize}
具体例を聞けなかったのですが、設定については千差万別なのでその人にとって
 向いていなかったからと言ってダメかというとそうではないと思います。
しかし、openSUSE のYaST は一元的に管理できて便利そうだなとは思います。
インストールが難しいというのも昔のイメージなのか現在のことを言っているの不明なのですが、デスクトップ向けのディストリビューションに比べると選択肢が多いので難しく感じられるのかもしれません。

\end{prework}

\begin{prework}{ 鈴木崇文 }

現在は改善しているかもしれませんが、一部kernelのパッケージによってはdbgパッケージが無いものがあったりして、困った経験がありました。
\end{prework}

\begin{prework}{ 野島 貴英 }

Debianでいけていない機能/実装といわれると、

\begin{itemize}
 \item  ifupdownパッケージ
 \item  Solaris10以上でいうところのFMDとか、SVC欲しい。
 \item  インストーラで'/'全部BTFSというの選択可能でしたっけ?
 \item  WEB絡みで、最新WEB開発関係一式のパッケージリポジトリというのが欲しい気がする(WEBシステムで流行りものやら、良く使われていそうなバージョンのソフトに特化したリポジトリ。それならexperimentalしか存在しないとかでもイイ!)
 \item  Debianとはちょっとズレてるかもしれませんが、synapticは...iTunesみたいになってほしー
\end{itemize}

と言いたい放題言ってみた。
\end{prework}

\begin{prework}{ 上川純一 }

Android 用のADBコマンドとかが標準で入っていると嬉しいなぁ。
\end{prework}

\begin{prework}{ yamamoto }

Debian の理想と信念は大好きなんですが、少し気になる点もあります。

例えば Debian-Installer には non-free で頒布されている firmware パッケージが含まれていない点などです。non-free のパッケージは、様々な理由で non-free に分類されているわけですが、その一つにテキストのソースが存在しないため、とかいう理由もあります。

別に「non-free にするな」とか主張したいわけではないのですが、頒布の制限の無いパッケージまで「non-free だから」と、収録を拒絶するのは少々やりすぎではないかと考えています。
\end{prework}

\begin{prework}{ 野首 }

\begin{itemize}
 \item パッケージのリストアにdpkg --get-selectionsはちょっと微妙
 \item aptitude-run-state-bundleはいまいち用途がわからない
 \item stableにたまに使い物にならないパッケージがある
       \begin{itemize}
	\item 古すぎるからとか(例: lxc)
       \end{itemize}

 \item パッケージのトランザクションが欲しい
       \begin{itemize}
	\item 一度インストールしてみておかしかったらrevert
	\item 問題なければcommitみたいな
       \end{itemize} 
\item multiarchは導入して本当に良かったのか?
\end{itemize}

\end{prework}

\begin{prework}{ 日比野 啓 }

使いこみが足りてないのかもしれませんが、
履歴管理システムに保存されているtreeを
debianパッケージとして build したり
install したりするときに便利なツールが
あまり無いのかもと思いました。

\end{prework}

\begin{prework}{ dictoss(杉本 典充) }

iptablesのコマンド引数が他のOSと違うような感じがする。そのためiptablesの初心者がwebで調べたコマンドを実行しても構文エラーではじかれて辛い。
\end{prework}
