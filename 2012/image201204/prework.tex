\begin{prework}{ koedoyoshida }
\begin{enumerate}
\item  Debianで触ったことのあるウェブアプリケーションフレームワーク一つを簡潔に紹介してください。\\
Greasemonkey
\item 近々作成してみたいと思っているウェブアプリケーションを教えてください。\\
なし 
\end{enumerate}
\end{prework}

\begin{prework}{ noritadak }
\begin{enumerate}
\item Ruby on Rails。でもかなり昔です。
\item Web アプリケーションと言えるのか微妙ですが、ブラウザ上で使えるシェルを個人的に作りたいです。
\end{enumerate}
\end{prework}

\begin{prework}{ dictoss(杉本 典充) }
\begin{enumerate}
\item pythonのDjangoはなかなかいい感じですがpython3に現状非対応というところ。cakephpもほんの少し試しましたがよくわらかず挫折。
\item 自宅の本を管理するシステムがほしい。また、ほしいジャンルの本を自動で探してくれるボットつきがいいな。
\end{enumerate}
\end{prework}

\begin{prework}{ henrich }
\begin{enumerate}
\item ウェブアプリケーションフレームワーク、一つも触ったことがございません…
\item 作成してみたいというのは特には無いですが、将来的にはDjango辺りを触ってみたいとは思っています。
\end{enumerate}
\end{prework}

\begin{prework}{ ikeomasa }
\begin{itemize}
\item django
\end{itemize}
\end{prework}

\begin{prework}{ MATOHARA }
\begin{enumerate}
\item Debianで触ったことのあるウェブアプリケーションフレームワーク一つを簡潔に紹介してください。 \\
フレームワークは利用したことがありません.Perl のCatalyst を使ってみたいなと思っています.
\item 近々作成してみたいと思っているウェブアプリケーションを教えてください。\\
最近写真のexif 情報を編集するものを作りたいなと思っています.\\
#アップロードした写真のexif 情報を表示してカメラのシリアル,撮影者,位置情報などを選択して削除してダウンロード.
\end{enumerate}
\end{prework}

\begin{prework}{ beatenavenue }
\begin{enumerate}
\item pukiwikiもwebアプリに入りますでしょうか・・・。グループウェアみたいな使い方ができないかなと思って試していましたが、仲間うちでwiki文法の評判が非常に悪く今は触っていません。
\item 作成したいwebアプリは今のところありません。
\end{enumerate}
\end{prework}

\begin{prework}{ 本庄 }
\begin{enumerate}
\item CakePHPを使ったことがあります。pearに依存しないということで選択しました。
\item 最近、テレビ番組をチェックする自分用の適当なスクリプトを作成しました。
\end{enumerate}
\end{prework}

\begin{prework}{ yamamoto }
\begin{enumerate}
\item 全く触ったことがありません。
\item ウェブアプリケーションって、なんすかね?おいしいの?
\end{enumerate}
\end{prework}

\begin{prework}{ まえだこうへい }
\begin{enumerate}
\item Sinatra 使ってます。DebianじゃなくてUbuntu ですが…。
\item 近々、というか1でOSインストール用ツールの開発やってます。
\end{enumerate}
\end{prework}

\begin{prework}{ 鈴木崇文 }
\begin{enumerate}
\item 書籍にのっていたサンプルを動作させるためにpylonsを少しだけ使ったことがあります。現在はpyramidを代わりに使用することが推奨されているみたいです。ところで、フレームワークを使ってweb開発する場合はどうやってdebianパッケージを利用すべきなんでしょうか。いままではパッケージを使用せずに直接ダウンロードして使っていました。
\item 位置情報を使ったWebアプリを作ってみたいです。
\end{enumerate}
\end{prework}

\begin{prework}{ emasaka }
\begin{enumerate}
\item Bash on Rails作ったわー 3年ぐらい前に作ったわー(ミサワ)\\
(と課題に書くために参加します)
\item 仕事でちょっとしたWebアプリを提案中
\end{enumerate}
\end{prework}

\begin{prework}{ 大森 俊秀 }
\begin{enumerate}
\item django
\item かんたん週報アプリ
\end{enumerate}
\end{prework}
\begin{prework}{ 野島 貴英 }
\begin{enumerate}
\item 古い話ですが、PerlのCatalystぐらいでしょうか?(今だとmojioliciousなんだろうか...。)紹介については...べ、勉強してきますー。
\item サイト巡回して何かデータ取ってきて自動的に分類/似ているものを探す個人利用専用のサイトを作ってみたいなぁ...あ、これだとウェブフレームワークよりも、機械学習とかのテーマ...ごふっ...
\end{enumerate}
\end{prework}
