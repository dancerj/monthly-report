%; whizzy-master ../debianmeetingresume201204.tex
% 以上の設定をしているため、このファイルで M-x whizzytex すると、whizzytexが利用できます。
%

\santaku
{今年度のDebian JP会長は誰か?}
{Kouhei Maeda}
{Nobuhiro Iwamatsu}
{Junichi Uekawa}
{A}
{前年度に引き続き前田さんが信任されました。またよろしくお願いします。}


\santaku
{Debian.org DPL 選挙は誰が立候補したか}
{Stefano Zacchiroli}
{Nobuhiro Iwamatsu}
{Kouhei Maeda}
{A}
{前年度に引き続きZacchiroliさん頑張ってます。}

\santaku
{Debconf12のsuponsordな参加の締切りはいつ}
{4月末}
{5月15日}
{5月末}
{B}
{うっかりすると過ぎてしまうので、気をつけましょう。また、UTCなのかニカラグアのローカルタイムなのかちょっとわからないので、5/15に登録開始するのは避けた方がよいかも。本家アナウンス\url{http://debconf12.debconf.org/}参照。}

\santaku
{大統一Debian勉強会での発表の公募(CFP)はいつが締切り?}
{もう過ぎた}
{4月末}
{4月22日}
{C}
{大統一Debian勉強会で発表できるチャンスです。締切りは忘れずに。}

\santaku
{experimental版のapacheのパッケージのバージョンはいくつ?}
{2.3}
{2.4.0}
{2.4.2}
{C}
{3/22にアナウンスがありました。upstream側も2.4.2です。最新版ですね。BUG見つけましたら、BUG Report書きましょう。}

\santaku
{Debian Eduはまたの名を何というでしょう?}
{emdebian}
{Scientific Linux}
{Skolelinux}
{C}
{Debian Eduとは教育機関向けに作られたDebianベースのディストリビューションの事です。昔はSkolelinuxという名前で開発されていた物だそうです。正式なDebianのサブプロジェクトです。先日新しいバージョンがアナウンスされました。}

\santaku
{3/30にDebian Projectが加盟した団体の名前は?}
{OSC}
{OSI}
{ETF}
{B}
{これでまたFree Softwareとして磨きがかかりました。OSIはOpen Source Initiativeの略だそうです。}

\santaku
{Debian Projectのgobbyサーバーとしてgobby.debian.orgがアナウンスされました。ところでgobbyって何?}
{旧ソ連で開発された諜報活動用ソフトウェア}
{churroの代替サーバ}
{エディタの名前}
{C}
{DebconfのBOF会場でよく使われていますエディタです。サーバを介する事により、複数人で同時に1つの文章を同時に編集できます。Debconfでは、リアルタイムに議事と議事録がBOF参加者によってどんどん編集されていく様はおもしろいです。ちなみにchurroとはi18n.debian.orgの事です。}


