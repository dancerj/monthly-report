
\begin{prework}{ やまだ }

最近新しいカーネルを自分でビルドすることが再び
増えてきたので、その周りでごそごそやってます。
\begin{enumerate}
\item gitでタグを打って
\item そのタグをバージョンにして
\item ビルドして
\item テストVMに導入して
\item ついでに配布サーバに設置
\end{enumerate}

の自動化とかやりました。

興味というか次に調べたいのは *-module-source な
カーネルモジュールパッケージの作り方とかDKMSの使い方。
普通のパッケージと違う点が多々ありそう。
\end{prework}

\begin{prework}{ 野首 }

ホームのMHフォルダを外から見れるようuw-imapdを入れました。インストールし
 てdebconfに答えるだけでSSL readyなimap環境になりました。
\end{prework}

\begin{prework}{ MATOHARA }

2010年11月の勉強会資料を見ながらnilfs2 をバックアップディスクに設定して
 みました.
リサイズ機能も来たようなので試してみたいです.
- [PATCH 0/4] nilfs2 resize support -- Linux NILFS Development
\url{http://www.spinics.net/lists/linux-nilfs/msg00869.html}
\end{prework}

\begin{prework}{ hattorin }

バングラデシュでDebianインストールして、ネットワークの監視系ツールをいろ
 いろとセットアップしてきました。今はDebian on SqueezeでTokyoTyrantの
 KeyValueを使い、特定の問題を早く計算するためにネットワークを使って計算
 を早くするような仕組み作りをしています。
\end{prework}

\begin{prework}{ キタハラ }

実家でプリンタの設定したり、
スキャナーの設定に失敗してました。

\end{prework}

\begin{prework}{ koedoyoshida }

最近Debianで自分がやったこと
\begin{itemize}
\item ようやく、メイン環境をSqueezeにupdate
wide-dhcpがupdateできずにはまったが、下記を見て解決。
\url{http://www.flcl.org/~takasugi/tdiary-org/?date=20061023}
\item OSC仙台に参加、出展。
\end{itemize}


\end{prework}

\begin{prework}{ emasaka }

sidでちょっとはまったとこについて、GitHubでupstreamに簡単なパッチをpull
 requestしたら、そのバージョンが数日後にsidに降りてきました。
\end{prework}

\begin{prework}{ dictoss(杉本 典充) }

最近やっていることはkfreebsdを常用していること、Debianを開発環境として
 gtkアプリを試作していること。
今後はkfreebsdでIS03を使ってテザリング、klinuxよりkfreebsdの方がおすすめ
 といえる有利分野を見つけることをやりたい。
\end{prework}

\begin{prework}{ なかおけいすけ }

興味があること:DebianLive
最近ネカフェで一夜を明かしたのですが、ネカフェのPCにコンパイラが入ってな
 くて困ったので、USBやSDカードにインストールしたDebianを持ち歩いていると
 ハッピーになれるのではないかと。
\end{prework}

\begin{prework}{ 吉野(yy\_y\_ja\_jp) }

バグレポートとDDTSS/DDTPぐらいでしょうか.
\end{prework}

\begin{prework}{ henrich }

いくつかパッケージやメッセージをルーティンの更新しました。

\end{prework}

\begin{prework}{ Osamu MATSUMOTO }

\begin{itemize}
\item インフラ,サーバ管理の自動化\\
 自動インストール,構成管理,コンフィグ投入, 監視のまでを
 ラフに綺麗に繋ぎたい。Debian的な良い組み合わせあったら教えてください。
 (cobbler+ puppet/cfengine+ nagios + なんかwebcgi的な)
\end{itemize}

\end{prework}

\begin{prework}{ まえだこうへい }
\begin{itemize}
\item *diagシリーズ(http://blockdiag.com/)のdebパッケージ化中。
\item さくらのVPSを先月契約して、lxc \& Squeezeで開発\&検証環境に。
\item あらきさんメンテナンスのDebianのAMI使ってAWSでごにょごにょと。
\end{itemize}
\end{prework}

\begin{prework}{ yamamoto }

\begin{itemize}
\item 最近Debianで自分がやったこと\\
ポチポチと公式パッケージのリビルドをしました。
\item 興味のあること\\
移植。
\end{itemize}

\end{prework}

\begin{prework}{ 岩松 信洋 }

\begin{itemize}
\item SH4 buildd のメンテ。
\item libpng15 の experimental へのアップロードとtransition作業。
\item スポンサーのパッケージチェック、アップロードなど。
\end{itemize}

\end{prework}

\begin{prework}{ 野島 貴英 }

\begin{itemize}

\item pythonでgnome-notifyめがけて再生中のデータのメタデータ送る「できるだ
 け簡単にできる」totemのプラグイン書いてみた。
 \url{http://d.hatena.ne.jp/nozzy123nozzy/20110502/1304322969}
\item sidのalsa-lib,alsa-utils,alsa-driverの解析中。
(bluetoothヘッドフォン出力相手に、alsaのみで、演奏中の音をcaptureしたい
 し、音が出る複数のアプリの音をmixして出力したいっ)
\item sidのlinux-image-2.6.39-2-686-paeにて、multi-stateなusb装置にejectコ
 マンド発行すると高い確率でOopsする件のデバッグを誰かがパッチ出すまでや
 り中。(いったい誰だ、dptぶっ壊すのは...)

\end{itemize}

\end{prework}

\begin{prework}{ 上川純一 }
\begin{itemize}
\item xslt のツールチェインをいろいろといじってみたり。
\item Javascript のコードを書いてみたり。
\end{itemize}
\end{prework}
