\documentclass[cjk,dvipdfmx,12pt,%
hyperref={bookmarks=true,bookmarksnumbered=true,bookmarksopen=false,%
colorlinks=false,%
pdftitle={第 44 回 関西 Debian 勉強会},%
pdfauthor={倉敷・のがた・佐々木},%
%pdfinstitute={関西 Debian 勉強会},%
pdfsubject={資料},%
}]{beamer}

\title{第 44 回 関西 Debian 勉強会}
\subtitle{{\scriptsize 資料}}
\author[佐々木 洋平]{{\large\bf 倉敷・のがた・佐々木}}
\institute[Debian JP]{{\normalsize\tt 関西 Debian 勉強会}}
\date{{\small 2011 年 2 月 27 日}}

%\usepackage{amsmath}
%\usepackage{amssymb}
\usepackage{graphicx}
\usepackage{moreverb}
\usepackage[varg]{txfonts}
\AtBeginDvi{\special{pdf:tounicode EUC-UCS2}}
\usetheme{Kyoto}
\def\museincludegraphics{%
  \begingroup
  \catcode`\|=0
  \catcode`\\=12
  \catcode`\#=12
  \includegraphics[width=0.9\textwidth]}
%\renewcommand{\familydefault}{\sfdefault}
%\renewcommand{\kanjifamilydefault}{\sfdefault}
\begin{document}
\settitleslide
\begin{frame}
\titlepage
\end{frame}
\setdefaultslide

\begin{frame}[fragile]
\frametitle{Agenda}

\tableofcontents

\end{frame}

\section{最近の Debian 関係のイベント}


\takahashi[40]{最近の Debian\\関係のイベント}

\takahashi[50]{squeeze}
\takahashi[50]{いよいよ\\リリース?}
\begin{frame}[fragile]
\frametitle{squeeze いよいよリリース?}

\begin{itemize}
\item されました。
\item リリース日程: 2 月 6 日 (JST)
\end{itemize}
\end{frame}

\begin{frame}[fragile]
\frametitle{第 43 回関西 Debian 勉強会}

\begin{itemize}
\item 日時: 1 月 23 日
\item 於: 大阪港区民センター
\end{itemize}

\begin{block}{内容}
  \begin{itemize}
  \item Debian GNU/kFreeBSD で便利に暮らすための Tips
  \item バグ報告はバグのためだけじゃないよ
  \end{itemize}
\end{block}
ネタ出しは随時行なっております! 皆様よろしく!!
\end{frame}

\begin{frame}[fragile]
  \frametitle{東京エリア Debian 勉強会}
  \begin{itemize}
  \item 第 73 回: 2010/02/19 開催. \\ \qquad Hadoop と Debian Games Team、Squeeze 話
  \end{itemize}
\end{frame}


\section{事前課題発表}


\takahashi[50]{事前課題}


\begin{frame}[fragile]
\frametitle{今回の事前課題}

\begin{block}{ ついに Squeeze がリリースされました。そんなわけで Squeeze になって }
     \begin{itemize}
      \item よかったこと
      \item うれしかったこと
      \item (些細な事だけど)こんな所が変わった!
      \item いきなりナニカ踏んだ!!
     \end{itemize}
     なんて事柄を、なにか一つご報告下さい。
\end{block}

\end{frame}

\takahashi[50]{事前課題\\発表}

\begin{frame}[fragile]
  \frametitle{ 甲斐正三 }
  シリアル通信(ttyUSB0)が一般ユーザーでは使えなくなってあせっています。
\end{frame}

\begin{frame}[fragile]
  \frametitle{ 山下康成 }
  rc が変わってはまりました。
  LSB ヘッダ必須とか、updaterc.d 必須とか、、
\end{frame}

\begin{frame}[fragile]
  \frametitle{ 八津尾 }
  リリースされて本当によかった
\end{frame}

\begin{frame}[fragile]
  \frametitle{ 山下尊也 }
  \begin{description}
  \item {よかったこと} \\
    全体的にバージョンがあがったことで、比較的新しいソフトが動かせるようになりました。
    すいません。普段からsidなので、まったく思いつかなかったです...
  \end{description}
\end{frame}

\begin{frame}[fragile]
  \frametitle{ のがたじゅん }
    \begin{description}
  \item {よかったこと} \\
    デスクトップテーマのSpace Funがイカす!
    起動が早くなった。
    backportsが正式にDebianになったので新しいパッケージが欲しい人にbackportsを使ったら?と言いやすくなった。
  \item {残念だったこと} \\
    Squeezeが残念、ではなく自分の取り組みで残念だったことですが
    SqueezeのDebian Liveに時間が取れなかった事が心残りです。Wheezey
    がんばります。
  \end{description}
\end{frame}

\begin{frame}[fragile]
  \frametitle{ 古川竜雄 (frkwtto@gmail.com) }
  すいません。まだインストールしてません。
  ごめんなさい。
\end{frame}

\begin{frame}[fragile]
  \frametitle{ 川江 }
  とにかく、リリース「おめどとう」です。
  25日1時55分の時点でまだ、AirにSqueezeをインストールできてませんが、日曜までになんとかインストールするつもりです。
\end{frame}

\begin{frame}[fragile]
  \frametitle{ 木下達也 }
  \begin{description}
  \item {よかったこと} \\
    リリースされたこと
  \end{description}
\end{frame}

\begin{frame}[fragile]
  \frametitle{ 山田 洋平 }
  dwm のバージョンが 4.7 から 5.8 に上がり、仕様も新しくなりました。
  Wikipedia にも載っていますがステータスバーに表示する仕方が変わったりなど。
  でも今見たらパッケージの説明文が古いままです。
  バグ報告しなきゃですかねこれは。
\end{frame}

\begin{frame}[fragile]
  \frametitle{ 松澤二郎}
  \begin{description}
  \item {うれしかったこと} \\
    Space Funがすてき
  \end{description}
\end{frame}

\begin{frame}[fragile]
  \frametitle{ occult.zzz (立川勝宣) }
  \begin{description}
  \item {よかった事} \\
    初めから二画面のドライバーが入っており使いやすかった。
    lenyの時は、ドライバーを探すのに入れすぎで
    何度もシステムを潰してしまい。
    結果それが、学ぶ一つになりました。
    やりはじめたばかりで、
    疑問多数なので、ぜひ参加させてください。
    よろしくお願い致します。
  \end{description}
\end{frame}
\begin{frame}[fragile]
  \frametitle{ 水野源 }
  SheevaPlugでずっとtestingなsqueezeを使っていたので、正式リリースされて嬉しいです。
  今回の発表資料を作る際にsqueezeのpbuilderでubuntu環境を作ろうとして、キーリングが自動で渡らないので構築に失敗する問題踏みましたょ。
\end{frame}

\begin{frame}[fragile]
  \frametitle{ かわだてつたろう }
  \begin{itemize}
  \item first point release が三月に予定されている
  \item dropbox が入らなかった
  \item nodejs が入らなかった
  \end{itemize}
\end{frame}

\begin{frame}[fragile]
  \frametitle{ lurdan }
  何台か lenny からのアップグレードをしていますが、単機能サーバばかりなせいか、表紙抜けするくらいあっさりと完了しています。
  dependency boot への移行がやたら失敗するくらいかな? 別に有り難くもないのでこれは気にしてません。
\end{frame}

\begin{frame}[fragile]
  \frametitle{ 中川舟(しゅう) }
  \begin{description}
  \item {よかったこと} \\
    すいません、今のところ見当たりません。。。
  \item {うれしかったこと} \\
    これも今のところは見当たりません。。。
  \item {こんな所が変わった!} \\
    レガシーなネットワーク接続の方法が先進的な設定方法に変わったと思います。(GNOMEのNetworkManagerからの固定IPアドレスの設定or/etc/NetworkManager/以下のファイルに設定の要が記述される。)
  \item {いきなりナニカ踏んだ!!} \\
    無線でつまずきました。
  \end{description}
\end{frame}

\takahashi[50]{そんな\\こんなで}
\takahashi[120]{次}

\section{pbuilder を使ってみよう by 水野源}

\takahashi[25]{pbuilder を使ってみよう \\by\\水野源}

\takahashi[50]{そんな\\こんなで}
\takahashi[120]{次}

\section{Squeeze の変更点をみんなで見てみよう}

\takahashi[25]{Squeeze の変更点をみんなで見てみよう \\ by \\ Debian JP}

\takahashi[50]{そんな\\こんなで}
\takahashi[120]{次}

\begin{frame}[fragile]
\frametitle{今後の予定}


\begin{block}{第 45 回関西 Debian 勉強会}
\begin{itemize}
  \item 日時: 3 月 27 日
  \item 会場: 港区民センター
  \item 内容: 未定
  \begin{itemize}
    \item OSC 神戸の事, そろそろ決めようよ
  \end{itemize}
\end{itemize}
\end{block}


\end{frame}



\takahashi[50]{  }


\end{document}
%%% Local Variables:
%%% mode: japanese-latex
%%% TeX-master: t
%%% End:
