%; whizzy-master ../debianmeetingresume2011-natsu.tex
% 以上の設定をしているため、このファイルで M-x whizzytex すると、whizzytexが利用できます。
%
% ちなみに、クイズは別ブランチで作成し、のちにマージします。逆にマージし
% ないようにしましょう。
% (shell-command "git checkout quiz-prepare")

\santaku
{DACA ってなんですか?}
{Debian Admin Coaching Association}
{Debian を使うと (A)あの子と (C)クリスマスに (A)アレができるかもしれない}
{Debian's Automated Code Analysi project}
{C}
{解説}

\santaku
{DebConf11 はいつ開催されるでしょう}
{2011/6/24 - 30}
{2011/7/24 - 30}
{2011/8/24 - 30}
{B}
{解説}

\santaku
{squeeze の Linux カーネルは一味違います。何が違うでしょう。}
{non-free firmware を排除した}
{Tux くんを排除した}
{一つのカーネルイメージでkFreeBSDとLinuxを提供します}
{A}
{解説}

\santaku
{2010/12/17 の時点でRCハグはいくつあるでしょう。}
{3} 
{83}
{183}
{C}
{解説}

\santaku
{New Maintianer フロントデスクに追加されたメンバーは?}
{Xavier Oswald}
{Enrico Zini}
{Kenshi Muto}
{A}
{解説}

\santaku
{RCバグの現状ははどこで確認できるか}
{\url{http://bugs.debian.org/release-critical/}}
{\url{http://localhost/}}
{\url{http://debianmeeting.appspot.com/}}
{A}
{解説}

\santaku
{Debian勉強会予約システムのURLはどれか}
{\url{http://www.2ch.net/}}
{\url{http://atnd.org/events/}}
{\url{http://debianmeeting.appspot.com/}}
{C}
{解説}

\santaku
{events@debian.orgはどこと統合されたか}
{merchants@debian.org}
{hoge@debian.org}
{fuga@debin.org}
{A}
{解説}

\santaku
{antiharassment@debian.org のうらにいないのは誰か}
{Amaya Rodrigo Sastre}
{Patty Langasek}
{Kouhei Maeda}
{C}
{解説}

\santaku
{現在いくつかのSprintが開催され、企画されている。現在企画すらされていな
いsprintはどれか}
{-www sprint}
{security sprint}
{tokyo sprint}
{C}
{解説}

\santaku
{DACAはどこを見ればよいか}
{\url{http://qa.debian.org/daca/}}
{\url{http://daca.debian.org/}}
{\url{file:/tmp}}
{A}
{解説}

\santaku
{DEP は何の略か}
{Debian Enhancement Proposal}
{Device Enhancement Protocol}
{でっぷ}
{A}
{解説}

\santaku
{DEP5で提案されているdebian/copyrightの機械可読形式はどういうものか}
{S式}
{RFC822風}
{XML}
{B}
{解説}

\santaku
{Debian 6.0 がリリースされたのはいつか?}
{2011/02/05}
{2011/02/06}
{2011/02/07}
{A}
{解説}

\santaku
{今回のリリースでサポートされなくなったアーキテクチャは?}
{m68k, ppc64}
{alpha, arm, hppa}
{blackfin, microblaze}
{B}
{解説}

\santaku
{www.debian.org のデザインが変更されました。何年同じデザインを使っていたか?}
{11年}
{12年}
{13年}
{C}
{解説}

\santaku
{次のリリースコードネームは何か?}
{rc}
{spell} % Mr.Spell
{wheezy}
{C}
{解説}

\santaku
{今度行われるFTP-master会議で議論されない予定は?}
{パッケージ自動サイン機構}
{クロスコンパイルサポート}
{md5sumとshaXsumをとっぱらう}
{B}
{解説}

\santaku
{Lars Wirzeniusが提案した次のリリースゴール目標は?}
{UTF-8で動かないパッケージをなくす}
{Debianで動いている人工衛星を打ち上げます}
{全パッケージをLLVMでビルドします}
{A}
{解説}

\santaku
{2011年度 Debian JP 会長は誰でしょうか?}
{前田 耕平}
{岩松 信洋}
{荒木 靖宏}
{A}
{解説}

\santaku
{2011年度 DPL は誰でしょうか?}
{Kurt Roeckx}
{Kenshi Muto}
{Stefano Zacchiroli}
{C}
{解説}

\santaku
{Debian Policy 3.9.2 で追加されていない項目はどれか}
{Debianアカウントをメンテナアドレスに追加する必要があります。}
{アーキテクチャ依存のライブラリ等は DEB\_HOST\_MULTIARCH で取得した値を利
用する必要があります。}
{全てのパッケージはVCSで管理する必要があります。}
{C}
{解説}

\santaku
{DebConf chairsに指名されたのは誰?}
{Gunnar Wolf}
{Maeda Kouhei}
{Junichi Uekawa}
{A}
{解説}

\santaku
{ries.debian.org で取得できるようになったデータは?}
{debian.org の稼働状況データ}
{debian.org のMLデータをgzipで固めた物}
{dak のデータ}
{C}
{解説}

\santaku
{/run が消された理由は?}
{/go のほうがよくね?という人が現れた。}
{initscripts がまだサポートしてない!}
{/run を入れたのはパッケージングミスです。}
{B}
{解説}

\santaku
{HPPA と alpha の移転先はどこでしょうか?}
{buildd.debian.or.jp}
{buildd.debian-ports.org}
{www.buildd.net}
{B}
{解説}

\santaku
{linuxカーネル 2.6.39がDebianに入ることによって起きる変更は?}
{i386-bigmemがi386-paeになった}
{amd64がi386になった}
{i386はamd64のマルチバイナリになった}
{A}
{解説}

\santaku
{Qt3パッケージが削除されない理由は?}
{Qt3ユーザによる哀願のため}
{LSB 4.1がQt3を必要としているため}
{削除の仕方がわからない}
{B}
{解説}

\santaku
{Debianのサーバに追加された機能は?}
{ログインしているユーザをIRCに流す機能}
{RFC1149 の実装}
{DNSSEC}
{C}
{解説}
