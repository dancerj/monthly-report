\documentclass[cjk,dvipdfmx,10pt,compress,%
hyperref={bookmarks=true,bookmarksnumbered=true,bookmarksopen=false,%
colorlinks=false,%
pdftitle={第 85 回 関西 Debian 勉強会},%
pdfauthor={倉敷・のがた・佐々木・かわだ・八津尾},%
%pdfinstitute={関西 Debian 勉強会},%
pdfsubject={資料},%
}]{beamer}

\title{第 85 回 関西 Debian 勉強会}
\subtitle{$\sim$発表資料$\sim$}
\author[かわだ てつたろう]{{\large\bf 倉敷・のがた・佐々木・かわだ・八津尾}}
\institute[Debian JP]{{\normalsize\tt 関西 Debian 勉強会}}
\date{{\small 2014 年 6 月 22 日}}

%\usepackage{amsmath}
%\usepackage{amssymb}
\usepackage{graphicx}
\usepackage{moreverb}
\usepackage[varg]{txfonts}
\AtBeginDvi{\special{pdf:tounicode EUC-UCS2}}
\usetheme{Kyoto}
\def\museincludegraphics{%
  \begingroup
  \catcode`\|=0
  \catcode`\\=12
  \catcode`\#=12
  \includegraphics[width=0.9\textwidth]}
%\renewcommand{\familydefault}{\sfdefault}
%\renewcommand{\kanjifamilydefault}{\sfdefault}
\begin{document}
\settitleslide
\begin{frame}
\titlepage
\end{frame}
\setdefaultslide

\begin{frame}[fragile]
  \frametitle{Disclaimer}
  \begin{itemize}
  \item 疑問、質問、ツッコミ、茶々、\alert{大歓迎}
  \item その場でインタラクティブにどうぞ
  \item ハッシュタグ \#kansaidebian
\end{itemize}
\end{frame}

\begin{frame}[fragile]
\frametitle{Agenda}

\tableofcontents

\end{frame}

\section{最近の Debian 関係のイベント}

\takahashi[40]{最近の Debian\\関係のイベント}

\begin{frame}[fragile]
  \frametitle{第84回関西Debian勉強会}
  \begin{itemize}
  \item 日時: 5月25日(日)
  \item 場所: 福島区民センター
  \end{itemize}
  \begin{block}{内容}
    \begin{itemize}
    \item もくもくの会
    \item キーサイン、systemd ...
    \end{itemize}
  \end{block}
\end{frame}

\begin{frame}[fragile]
  \frametitle{第114回東京エリアDebian勉強会}
  \begin{itemize}
  \item 日時: 6月14日(土)
  \item 場所: 株式会社スクウェア・エニックス 会議室
  \end{itemize}
  \begin{block}{内容}
    \begin{itemize}
    \item 「GPG 秘密鍵取り扱い方法の提案」
    \item もくもくの会
    \end{itemize}
  \end{block}
\end{frame}

\begin{frame}[fragile]
  \frametitle{Debian Project}
  \begin{block}{}
    \begin{itemize}
    \item MATE 1.8 has now fully arrived in Debian
    \item Debian 6 debuts its long term support period
    \item New project goal: Get rid of Berkeley DB (post jessie)
    \end{itemize}
  \end{block}
\end{frame}

\takahashi[50]{そんな\\こんなで}
\takahashi[120]{次}

\section{事前課題発表}

\takahashi[50]{事前課題}

\begin{frame}[fragile]
  \frametitle{事前課題}
  \begin{block}{今回の事前課題}
    \begin{description}
    \item[事前課題1]
      もくもくの会で行なう作業、質問などの課題を用意して教えてください。
    \item[事前課題2]
      前回(第84回)の勉強会に参加された方は、前回の作業や課題がその後どう
      なったか結果を教えてください。
    \item[事前課題3]
      LT(ライトニングトーク) 歓迎です。何かお話したい方はタイトルを下さい。
    \item[事前課題4]
      DebianでInitとしてsystemdが動作する環境を用意してきて下さい。

      仮想環境でかまいません。unstableをお使いの方はsystemd-sysvパッケー
      ジを導入するとsysvinit-coreがreplaceされてInitとしてsystemdが動作
      するようになります。
    \end{description}
  \end{block}
\end{frame}

\takahashi[50]{事前課題\\発表}

\begin{frame}
  \frametitle{ かわだてつたろう }
  \begin{enumerate}
  \item uim-skkを使用しているのですが、
    \begin{itemize}
    \item chromiumでtitanpadにかな入力できない
    \item xmonad+gnucacheでかな入力できない
    \end{itemize}
    のをなんとかしたい
  \end{enumerate}
\end{frame}

\begin{frame}
  \frametitle{ 佐々木洋平 }
  申し込みわすれてたわー。

  \begin{enumerate}
  \item jekyll の依存が newqueue から unstable に落ちたので、ようやく
    new upstream を upload できます。
  \end{enumerate}

\end{frame}

\begin{frame}
  \frametitle{ 木下 (1/3)}
  \begin{enumerate}
  \item
    \begin{enumerate}
    \item Eucalyptus(プライベートクラウドとして)の調査・研究
    \item グリッドコンピューティング関連の調査・研究
      \begin{itemize}
      \item GlobusToolkitで何ができる?

        →AndroidOSのクロスコンパイルで使えたら嬉しいかも。
      \end{itemize}
    \item Debian7 on PANDABOARDの調査・研究
      \begin{itemize}
      \item WiFiモジュール(On Board:TI製)の有効化
      \item GPUデバイスドライバの有効化
      \end{itemize}
    \end{enumerate}
  \end{enumerate}
\end{frame}

\begin{frame}
  \frametitle{ 木下 (2/3)}
  \begin{enumerate}
    \setcounter{enumi}{1}
  \item ※前回(第84回)は欠席だった為、第83回の内容になります。
    \begin{enumerate}
    \item Eucalyptus(プライベートクラウドとして)の調査・研究

      実績:インスタンスの保存と自動化に成功
      \begin{itemize}
      \item 保存したインスタンスを起動させるとネットワーク接続が絶たれてしまっていた原因

        →ファイル:/etc/udev/rules.d/70-persistent-net.rulesにNICデバイス情報が残っていると
        新たにNICデバイス情報が登録されてしまい、NIC番号が変更され、通信不可となっていた。

        ※Eucalyptus関係者の方のお話では、「Debian用ツールのバグかも」とのこと。
        インスタンス保存にクリアすることで解決した。
      \item インスタンスをOSイメージファイル化し、ハイパーバイザ上にバックアップする

        スクリプトを作成し、crontabで定期的に起動できるようになった。
      \end{itemize}
    \end{enumerate}
  \end{enumerate}
\end{frame}

\begin{frame}
  \frametitle{ 木下 (3/3)}
  \begin{enumerate}
    \setcounter{enumi}{1}
  \item ※前回(第84回)は欠席だった為、第83回の内容になります。
    \begin{enumerate}
      \setcounter{enumii}{1}
    \item DistCCの調査・研究

      実績:上記プライベートクラウドシステムを用いて複数のVMで分散コンパイルさせる
      環境構築に成功。
      \begin{itemize}
      \item ビルドマシン(distccクライアント)とVM(distccサーバ)4台で
        PANDABOARD(OS:Android4.0.4)用FWを分散コンパイルさせ、
        コンパイル時間の短縮が可能となった。
      \end{itemize}
      課題:AndroidOSの場合、JAVAのコンパイル箇所がかなり存在するので、
      この部分が複数マシンで分散コンパイルできるようになれば、
      かなりのコンパイル時間の短縮となりそうなので、このあたりについて再調査必要。
    \item グリッドコンピューティング関連の調査・研究

      実績:保留
    \item Debian7 on PANDABOARDの調査・研究

      実績:保留
    \end{enumerate}
  \end{enumerate}
\end{frame}

\begin{frame}
  \frametitle{ takata (1/3)}
  \begin{enumerate}
  \item 課題:LUKSパーティションのパスワード入力を省略する件(未解決)

    keyfileを登録してみましたが、依然として swapパーティションに関して
    は起動時にパスワードを聞かれます。ググってみると同様の問題がいくつ
    かヒットするようなので、もくもくの会の時間でもう少し調べてみたいと
    思っています。

    stop crypttab asking for password for swap

    \url{http://askubuntu.com/questions/43432/stop-crypttab-asking-for-password-for-swap}
  \end{enumerate}
\end{frame}

\begin{frame}
  \frametitle{ takata (2/3)}
  \begin{enumerate}
  \item 課題:LUKSパーティションのパスワード入力を省略する件(未解決)
    ちなみに、"/"パーティションの keyfileでトラブりました。
    間違えて追加しすぎてしまった key\_{}slotを cryptsetup luksRemoveKey
    で削除しようとしたのですが、パスワードを登録した key\_{}slotを誤っ
    て削除してしまい、cryptsetup luksAddKeyでキーを登録できなくなるばか
    りかパスワードが解除できなくなってしまい(No key available with this passphrase.)、
    一瞬冷や汗が出ました。幸い、keyfileを設定した Key Slotが残っていたので、
    cryptsetup luksAddKey --key-file=...で回避でき何とか事なきを得ましたが、
    Key Slotを削除する場合はくれぐれもご注意ください。
  \end{enumerate}
\end{frame}

\begin{frame}
  \frametitle{ takata (3/3)}
  \begin{enumerate}
    \setcounter{enumi}{1}
  \item インターネットからの22番ポート(ssh)への不正アクセスについて

    教えていただいたとおり、fail2banを適用することで不正アクセスに対し
    て有効にブロックできているようです。ありがとうございました。
  \end{enumerate}
\end{frame}

\begin{frame}[containsverbatim]
  \frametitle{ 西山和広 }
  \begin{enumerate}
  \item ansible でのサーバー設定を進めたいです。
  \item 前回不参加です。
  \item blog の記事から\verb+^+Xg の話を予定しています。
  \item 用意しておきます。
  \end{enumerate}
\end{frame}

\begin{frame}
  \frametitle{ yyatsuo }
  \begin{enumerate}
  \item kernel-handbook の日本語訳をそろそろなんとかします

    あとは某雑誌の記事査読とか
  \item fcitx-skk new que に入りました!
    (new que で止まってるとも言います)
  \item 仕事の愚痴なら溜まってますよ?
  \item もう入ってる
  \end{enumerate}
\end{frame}

\begin{frame}
  \frametitle{ 榎真治 }

  Debian wheezyでAnthyを使っていますが、日本語入力の変換効率(候補が期
  待したような順番で並んでくれないなど)がいまひとつと感じていますので、
  環境を変更してみたいと考えています。

  お勧めの方法はありますか?

  mozcを試すとしたらuim,ibusなどのどれがお勧めですか?
\end{frame}

\begin{frame}
  \frametitle{ 坂本 貴史 }
  \begin{enumerate}
  \item 英語ドキュメントをTexフォーマットで作成しているので、その続きを書きます。
  \item 前回は参加していません
  \item 「Linuxのドライバメンテナになった体験記」というタイトルで短いセッションをする予定です
  \end{enumerate}
\end{frame}

\begin{frame}
  \frametitle{ lurdan }
  \begin{enumerate}
  \item 手持ちパッケージの bug squash
  \item webwml-git は CI できるようにはなったけど、運用が煩雑なので再検討中です
  \end{enumerate}
\end{frame}

\begin{frame}
  \frametitle{ 川江 }
  \begin{enumerate}
  \item emacsにて、HTML5形式のサイトの作成。で、web-mode.elがdebianの
    パッケージにないのですが、これって「anthy-el」のようにパッケージに
    できます? というか「.el」ファイルというのはそもそも何?
  \end{enumerate}
\end{frame}

\begin{frame}
  \frametitle{ Hiroyuki Nagata }
  \begin{enumerate}
  \item RFSのやり方、GPG鍵の交換 … 今度はある程度準備をしてくるつもり
  \item GPG鍵を作成しました、Mac Book ProにDebian jessieをインストールしました
  \item 今月はむりかもしれませんがDebianでルータ構築とか
  \end{enumerate}
\end{frame}

\takahashi[50]{そんな\\こんなで}
\takahashi[120]{次}

\section{Linuxのドライバメンテナになった体験記}
\takahashi[25]{Linuxの\\ドライバメンテナになった\\体験記\\by\\坂本貴史}

\takahashi[50]{そんな\\こんなで}
\takahashi[120]{次}

\section{Debian での systemd とのつきあい方}
\takahashi[25]{Debianでの\\systemd とのつきあい方\\by\\佐々木洋平}

\takahashi[50]{そんな\\こんなで}
\takahashi[120]{次}

\section{もくもくの会}
\takahashi[30]{もくもくの会}

\takahashi[50]{そんな\\こんなで}
\takahashi[120]{次}

\section{今後の予定}
\begin{frame}[fragile]
\frametitle{今後の予定}

\begin{block}{第86回関西Debian勉強会}
  \begin{itemize}
  \item 日時: 7月27日(日) 13:30 -
  \item 場所: 福島区民センター
  \end{itemize}
\end{block}

\begin{block}{第115回東京エリアDebian勉強会}
  \begin{itemize}
  \item 日時: 7月19日(土)
  \item 場所: 株式会社スクウェア・エニックス 応接11
  \item 内容: 「簡易ヲレヲレタイルサーバを作った」
  \end{itemize}
\end{block}

\end{frame}

\takahashi[50]{  }

\end{document}
%%% Local Variables:
%%% mode: japanese-latex
%%% TeX-master: t
%%% End:
