%; whizzy paragraph -pdf xpdf -latex ./whizzypdfptex.sh
%; whizzy-paragraph "^\\\\begin{frame}"
% latex beamer presentation.
% platex, latex-beamer でコンパイルすることを想定。 

%     Tokyo Debian Meeting resources
%     Copyright (C) 2009 Junichi Uekawa
%     Copyright (C) 2009 Nobuhiro Iwamatsu

%     This program is free software; you can redistribute it and/or modify
%     it under the terms of the GNU General Public License as published by
%     the Free Software Foundation; either version 2 of the License, or
%     (at your option) any later version.

%     This program is distributed in the hope that it will be useful,
%     but WITHOUT ANY WARRANTY; without even the implied warreanty of
%     MERCHANTABILITY or FITNESS FOR A PARTICULAR PURPOSE.  See the
%     GNU General Public License for more details.

%     You should have received a copy of the GNU General Public License
%     along with this program; if not, write to the Free Software
%     Foundation, Inc., 51 Franklin St, Fifth Floor, Boston, MA  02110-1301 USA

\documentclass[cjk,dvipdfmx,12pt]{beamer}
\usetheme{Tokyo}
\usepackage{monthlypresentation}

%  preview (shell-command (concat "evince " (replace-regexp-in-string "tex$" "pdf"(buffer-file-name)) "&")) 
%  presentation (shell-command (concat "xpdf -fullscreen " (replace-regexp-in-string "tex$" "pdf"(buffer-file-name)) "&"))
%  presentation (shell-command (concat "evince " (replace-regexp-in-string "tex$" "pdf"(buffer-file-name)) "&"))

%http://www.naney.org/diki/dk/hyperref.html
%日本語EUC系環境の時
\AtBeginDvi{\special{pdf:tounicode EUC-UCS2}}
%シフトJIS系環境の時
%\AtBeginDvi{\special{pdf:tounicode 90ms-RKSJ-UCS2}}

\title{東京エリアDebian勉強会 \\quilt でporting してみた}
\subtitle{}
\author{杉本 典充 dictoss@live.jp}
\date{2011年12月17日}
\logo{\includegraphics[width=8cm]{image200607/openlogo-light.eps}}

\begin{document}

\frame{\titlepage{}}

%\section{}

\begin{frame}
 \frametitle{アジェンダ}
 \begin{itemize}
  \item Debian パッケージのフォーマット
  \item quilt パッケージの紹介
  \item Debian GNU/kFreeBSDとporting
  \item quiltを使ってportingパッチの作成
  \item 質疑応答
 \end{itemize}
\end{frame}

\begin{frame}{自己紹介}
 \begin{itemize}
  \item 杉本 典充 (SUGIMOTO Norimitsu)
  \item Twitter: @dictoss
  \item しがないソフトウェア開発者
  \item Debian Userであり、FreeBSD Userでもある
  \item 両者が合わさったDebian GNU/kFreeBSDがおもしろそうなので、最近使っている
 \end{itemize}
\end{frame}

\emtext{Debianパッケージのフォーマット}

\begin{frame}{Debianパッケージのフォーマット}
 \begin{itemize}
  \item man dpkg-source(1)に書いてあります
  \item 3.0(native)
   \begin{itemize}
    \item 最初の頃からあるFormat 1.0(native)から圧縮形式が増えたもの。
    \item packagename-version.tar.ext
    \item packagename-version.dsc
   \end{itemize}
  \item 3.0(quilt)
   \begin{itemize}
    \item upstreamから持ってきてパッチなど修正を加えたパッケージはこの形式。
    \item packagename-upstreamversion.orig.tar.ext
    \item packagename-upstreamversion.orig-component.tar.ext(任意)
    \item packagename-debianversion.debian.tar.ext
    \item packagename-debianversion.dsc
   \end{itemize}
 \end{itemize}
\end{frame}

\begin{frame}{quiltコマンド}
 \begin{itemize}
  \item 3.0(quilt)で使用しているパッチ管理ツール
  \item debianプロジェクトオリジナルではなく、以下プロジェクトの成果物をパッケージしたものです
  \item https://savannah.nongnu.org/projects/quilt
  \item debian/patches/series、又はdebian/patches/debian.seriesにパッチが積まれます
  \item devscriptsの依存で一緒に入らないので別途「apt-get install quilt」
 \end{itemize}
\end{frame}

\emtext{Debian GNU/kFreeBSD}

\begin{frame}{Debian GNU/kFreeBSDとは}
 \begin{itemize}
  \item FreeBSDカーネルを使って動作するDebian
  \item kfreebsd-i386、kfreebsd-amd64がアーキテクチャ名
  \item もちろんaptが使えます
  \item ドライバやデバイスはやっぱりFreeBSD(kldstatとか、/dev/ad4s1とか)
  \item ZFSが使える
  \item debootstrapでlinux-i386バイナリをインストールしてchrootするとlinuxバイナリが動く
 \end{itemize}
\end{frame}

\begin{frame}{portingって何?}
 \begin{itemize}
  \item 様々なアーキテクチャで動作させるように対応すること(armelやmipsに対応するなど)
  \item カーネルやCPUの違いを意識する必要がある
  \item http://www.debian.org/ports/
  \item http://www.debian.org/ports/kfreebsd-gnu/
  \item http://glibc-bsd.alioth.debian.org/porting/
 \end{itemize}
\end{frame}

\begin{frame}{kfreebsdにおけるporting作業}
 \begin{itemize}
  \item 具体的な作業は、http://glibc-bsd.alioth.debian.org/porting/PORTING
  \item unameチェック
  \item debian/controlでカーネルの違いなのか、CPUの違いなのか
  \item libtool、aclocal.m4など(automake、autoconf周り)
  \item \_\_FreeBSD\_kernel\_\_、\_\_FreeBSD\_kernel\_versionといったマクロ名に対応すること(FreeBSD本家は\_\_FreeBSD\_\_を使っている)
  \item FreeBSDのデバイス管理はdevfsを使っていますので対応が必要
  \item FreeBSDなので存在しないシグナルがあります
  \item linux限定の共有ライブラリ名をハードコードしている部分の除去
 \end{itemize}
\end{frame}

\begin{frame}
\begin{center}
\LARGE{つまりDebian GNU/kFreeBSDで使えるようにするにはpatchをあてる必要がある}
\end{center}
\end{frame}

\begin{frame}
\begin{center}
\LARGE{そこでquilt登場}
\end{center}
\end{frame}


\begin{frame}{porting作業}
 \begin{itemize}
  \item 対象パッケージはicewmというウィンドウマネージャです
  \item icewmのタスクバーにある電源状態通知アイコンがkfreebsdだと、表示されない
  \item linux-i386、linux-amd64では表示されるのは確認済み
  \item portingすれば、kfreebsdでも表示できるはず
 \end{itemize}
\end{frame}

\begin{frame}{kfreebsdで動かない原因の追求}
 \begin{itemize}
  \item "grep -nr \_\_FreeBSD\_\_ *"、"grep -nr \_\_linux\_\_"を実行してみる
  \item 原因はマクロ名が違うせいで、linuxカーネル時の処理が実行されていた
  \item これを直せば、動くはず!!
%  \item #if defined(\_\_FreeBSD\_\_) || defined(\_\_FreeBSD\_kernel\_\_)
 \end{itemize}
\end{frame}

\begin{frame}{quiltコマンドでpatch操作}
 \begin{itemize}
  \item debian/patches/seriesに管理しているパッチファイルが積まれています
  \item quilt new kfreebsd\_porting\_aapm
  \item 新しいパッチファイルを追加できました
 \end{itemize}
\end{frame}

\begin{frame}{quiltコマンドでpatch操作}
 \begin{itemize}
  \item quilt add src/aapm.h
  \begin{itemize}
   \item 修正するファイルをquiltコマンドで登録します
  \end{itemize}
  \item quilt edit src/aapm.h
  \begin{itemize}
   \item ファイルを修正します
  \end{itemize}
  \item quilt refresh
  \begin{itemize}
   \item kfreebsd\_porting\_aapmに出力します
  \end{itemize}
  \item ファイルが複数あるので、add、edit、refreshを繰り返します。
 \end{itemize}
\end{frame}

\begin{frame}{ビルドできるか確かめる}
 \begin{itemize}
  \item dchコマンドでchangelogを追加
  \item debuild -uc -usコマンドでパッケージをビルドします
 \end{itemize}
\end{frame}

\begin{frame}{パッケージがビルドできました}
 \begin{itemize}
  \item インストールして動作確認「\$ sudo dpkg -i icewm*.deb」 
  \item 電源状態通知アイコンが表示されました
  \item reportbugコマンドでpatchを送信しておきます
  \item http://bugs.debian.org/cgi-bin/bugreport.cgi?bug=650395
 \end{itemize}
\end{frame}

\begin{frame}{参考情報}
 \begin{itemize}
  \item 東京エリアDebian 勉強会2007 年01 月号「パッチ管理ツールquilt の使い方」小林儀匡
  \item 東京エリアDebian 勉強会2010 年03 月号「dpkg ソース形式''3.0(quilt)''」吉野与志仁
  \item man dpkg-source(1), man quilt(1)
  \item http://wiki.debian.org/Projects/DebSrc3.0
  \item http://wiki.debian.org/Debian\_GNU/kFreeBSD
  \item http://glibc-bsd.alioth.debian.org/porting
 \end{itemize}
\end{frame}


\begin{frame}{質疑応答}
\begin{center}
\Large なにか質問はありますか?
\end{center}
\end{frame}

\end{document}

;;; Local Variables: ***
;;; outline-regexp: "\\([ 	]*\\\\\\(documentstyle\\|documentclass\\|emtext\\|section\\|begin{frame}\\)\\*?[ 	]*[[{]\\|[]+\\)" ***
;;; End: ***
