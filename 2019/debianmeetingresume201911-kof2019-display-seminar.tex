\documentclass[cjk,dvipdfmx,10pt,compress,%
hyperref={bookmarks=true,bookmarksnumbered=true,bookmarksopen=false,%
colorlinks=false,%
pdftitle={第 132 回 関西 Debian 勉強会},%
pdfauthor={かわだ},%
%pdfinstitute={関西 Debian 勉強会},%
pdfsubject={資料},%
}]{beamer}

%\usepackage{amsmath}
%\usepackage{amssymb}
\usepackage{graphicx}
\usepackage{moreverb}
\usepackage[varg]{txfonts}
\AtBeginDvi{\special{pdf:tounicode EUC-UCS2}}
\usetheme{Kyoto}
\def\museincludegraphics{%
  \begingroup
  \catcode`\|=0
  \catcode`\\=12
  \catcode`\#=12
  \includegraphics[width=0.9\textwidth]}
%\renewcommand{\familydefault}{\sfdefault}
%\renewcommand{\kanjifamilydefault}{\sfdefault}
\begin{document}
\settitleslide
%\begin{frame}
%\titlepage
%\end{frame}
\setdefaultslide

\begin{frame}[fragile]
  \frametitle{セミナー: 関西 Debian 勉強会}
  \begin{itemize}
  \begin{huge}
  \item Debian の最新情報をお届けします
    \begin{itemize}
    \begin{Large}
    \item 2019 年 7 月にリリースされた Debian 10 の情報
    \item Debian 11 の開発情報
    \end{Large}
    \end{itemize}
  \item 日時: 2019/11/09 (土) 15:00 @ ショーケース 1
  \item 毎月、最終日曜日に勉強会を開催しています
  \end{huge}
  \begin{Large}
  \item 次回は 11 月 26 日を予定
  \end{Large}
  \end{itemize}
\end{frame}

\end{document}
%%% Local Variables:
%%% mode: japanese-latex
%%% TeX-master: t
%%% End:
