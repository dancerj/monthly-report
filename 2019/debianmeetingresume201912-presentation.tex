%; whizzy paragraph -pdf xpdf -latex ./whizzypdfptex.sh
%; whizzy-paragraph "^\\\\begin{frame}\\|\\\\emtext"
% latex beamer presentation.
% platex, latex-beamer でコンパイルすることを想定。 

%     Tokyo Debian Meeting resources
%     Copyright (C) 2012 Junichi Uekawa

%     This program is free software; you can redistribute it and/or modify
%     it under the terms of the GNU General Public License as published by
%     the Free Software Foundation; either version 2 of the License, or
%     (at your option) any later version.

%     This program is distributed in the hope that it will be useful,
%     but WITHOUT ANY WARRANTY; without even the implied warreanty of
%     MERCHANTABILITY or FITNESS FOR A PARTICULAR PURPOSE.  See the
%     GNU General Public License for more details.
%     You should have received a copy of the GNU General Public License
%     along with this program; if not, write to the Free Software
%     Foundation, Inc., 51 Franklin St, Fifth Floor, Boston, MA  02110-1301 USA

\documentclass[cjk,dvipdfmx,12pt]{beamer}
\usetheme{Tokyo}
\usepackage{monthlypresentation}

%  preview (shell-command (concat "evince " (replace-regexp-in-string "tex$" "pdf"(buffer-file-name)) "&")) 
%  presentation (shell-command (concat "xpdf -fullscreen " (replace-regexp-in-string "tex$" "pdf"(buffer-file-name)) "&"))
%  presentation (shell-command (concat "evince " (replace-regexp-in-string "tex$" "pdf"(buffer-file-name)) "&"))

%http://www.naney.org/diki/dk/hyperref.html
%日本語EUC系環境の時
\AtBeginDvi{\special{pdf:tounicode EUC-UCS2}}
%シフトJIS系環境の時
%\AtBeginDvi{\special{pdf:tounicode 90ms-RKSJ-UCS2}}

\newenvironment{commandlinesmall}%
{\VerbatimEnvironment
  \begin{Sbox}\begin{minipage}{1.0\hsize}\begin{fontsize}{8}{8} \begin{BVerbatim}}%
{\end{BVerbatim}\end{fontsize}\end{minipage}\end{Sbox}
  \setlength{\fboxsep}{8pt}
% start on a new paragraph

\vspace{6pt}% skip before
\fcolorbox{dancerdarkblue}{dancerlightblue}{\TheSbox}

\vspace{6pt}% skip after
}
%end of commandlinesmall

\title{東京エリアDebian勉強会}
\subtitle{2019年12月度} %第181回
\author{杉本典充 / dictoss@live.jp}
\date{2019年12月22日}
\logo{\includegraphics[width=8cm]{image200607/openlogo-light.eps}}

\begin{document}

\begin{frame}
\titlepage{}
\end{frame}

% 町屋文化センター
\section{会場の諸注意}
%\emtext{会場の諸注意}

\begin{frame}{会場の諸注意}
\begin{itemize}
\item 建物の関係ない部屋には立ち入らないでください
\item 危険又は不潔な物品、動物等は持ち込まないでください
\item ポスター、看板、旗等を掲げたり張り付けたりしないでください
\item 飲酒または薬物の使用はお控えください
\item 許可なく物品の販売その他の営業行為はしないでください
\item 勉強会終了後は、部屋の後片付け、現状復帰のご協力をお願いいたします
\end{itemize} 
\end{frame}

\begin{frame}{Agenda}
 \begin{minipage}[t]{0.45\hsize}
  \begin{itemize}
  \item 最近あったDebian関連のイベント報告
    \begin{itemize}
    \item 2019年10月度 東京エリアDebian勉強会
    \item OSC 2019 Tokyo/Fall
    \end{itemize}
  \item 事前課題発表
  % \item Debian Trivia Quiz
  \end{itemize}
 \end{minipage}
 \begin{minipage}[t]{0.45\hsize}
   \begin{itemize}
   \item セミナー発表
     \begin{itemize}
     \item Debian 10 busterでnftablesを使ってみる
     \end{itemize}
  \item 今後のDebianにほしい機能や仕組みを整理する
  \end{itemize}
 \end{minipage}
\end{frame}

\section{イベント報告}
\emtext{イベント報告}

\begin{frame}{2019年10月度 東京エリアDebian勉強会}
\begin{itemize}
\item 2019年10月19日に荒川区立町屋文化センターの会議室で開催
\item 参加者は5名
\item セミナー「Debian GNU/kFreeBSD セットアップガイド 2019年版」
\item 学習会「月間 Debian Policy」
\end{itemize} 
\end{frame}

\begin{frame}{OSC 2019 Tokyo/Fall}
\begin{itemize}
\item 2019年11月23日に明星大学で開催
\item イベント全体の参加者は11/23(土)  約500名、11/24(日) 約330名
\item セミナー「Debian Updates」 Debian 10 busterのリリース情報を説明
\item ブース展示を行い、イベント参加者を交流しました
\end{itemize} 
\end{frame}


\section{事前課題}

\emtext{事前課題}

{\footnotesize
 \begin{prework}{ koedoyoshida }
  \begin{enumerate}
  \item $B;H$C$?$3$H$O$"$j$^$;$s(B
  \item ($B2sEz$J$7(B)
  \end{enumerate}
\end{prework}

\begin{prework}{ NOKUBI Takatsugu (knok) }
  \begin{enumerate}
  \item $B;H$C$?$3$H$O$"$j$^$;$s(B
  \item Cosmo Communicator$B$G(BDebian$B$,F0$/$=$&$G$9$,C/$+Gc$$$^$9$+(B?
  \end{enumerate}
\end{prework}

\begin{prework}{ Kouhei Maeda (mkouhei) }
  \begin{enumerate}
  \item $B;H$C$?$3$H$O$"$j$^$;$s(B
  \item ($B2sEz$J$7(B)
  \end{enumerate}
\end{prework}

\begin{prework}{ yy\_y\_ja\_jp }
  \begin{enumerate}
  \item $B;H$C$?$3$H$O$"$j$^$;$s(B
  \item ($B2sEz$J$7(B)
  \end{enumerate}
\end{prework}

\begin{prework}{ dictoss }
  \begin{enumerate}
  \item $B;H$C$?$3$H$,$"$j$^$9(B
  \item $B%\!<%I(BPC$B7O$N>pJs!J(BRaspberry Pi$B!"(BPINE64$B!K!"%3%s%F%J$N%2%9%H(BOS$B$H$7$F(Bdebian$B$O$I$NDxEY8~$$$F$$$k$N$+$ND4::!&8+2r(B
  \end{enumerate}
\end{prework}

}

%\subsection{問題}
%%; whizzy-master ../debianmeetingresume201211.tex
% $B0J>e$N@_Dj$r$7$F$$$k$?$a!"$3$N%U%!%$%k$G(B M-x whizzytex $B$9$k$H!"(Bwhizzytex$B$,MxMQ$G$-$^$9!#(B
%

\santaku
{DebConf13 $B$N3+:ECO$H3+:EF|$O!)(B}
{$BF|K\!"El5~ET(B 6$B7n(B20$BF|(B}
{$B%K%+%i%0%"(B $B%^%J%0%"(B 7$B7n(B8-14$BF|(B}
{$B%9%$%9!"%t%)!<%^%k%-%e(B 8$B7n(B11-18$BF|(B}
{C}
{$B%K%+%i%0%"$O(BDebConf12$B$N3+:ECO$G$9!#(B
DebConf13$B$O%9%$%9$N%-%c%s%WCO$G3+:E$G$9!#(B
6/20$B$O3'$5$sM=Dj$r6u$1$F$*$-$^$7$g$&!#(B}

\santaku
{$B@$3&$N(BWeb$B%5!<%P$G:G$b?M5$$N$"$k(BLinux $B%G%#%9%H%j%S%e!<%7%g%s(B(W3Techs$BD4$Y(B)$B$O!)(B}
{CentOS}
{Debian}
{Ubuntu}
{B}
{\url{http://w3techs.com/technologies/history_details/os-linux}$B$K7k2L$N%0%i%U$,$"$j$^$9!#(B
$B8=:_(B Linux $B$r;HMQ$7$F$$$k(B web $B%5!<%P$N(B 32.9\% $B$,(B Debian $B$rMxMQ$7$F$*$j!"$=$N3d9g$O8=:_$bA}2C$rB3$1$F$$$k$=$&$G$9!#(B}

\santaku
{Ben Hutchings $B$5$s$,<!4|(B Debian $B0BDjHG$H0l=o$K=P2Y$5$l$k(B Linux $B%+!<%M%k$K(B (3.2 $B7ONs$N(B mainline $B$K$OL5$$(B) $BDI2C5!G=$,Ek:\$5$l$kM=Dj$G$"$k$H=R$Y$F$$$^$9!#(B
$BB?$/$NDI2CE@$NCf$K4^$^$l$J$$$b$N$O2?!)(B}
{PREEMPT\_RT}
{Hyper-V guest drivers$B$N6/2=(B}
{ARM64/AArch64$B%"!<%-%F%/%A%c%5%]!<%H(B}
{C}
{Ben Hutchings $B$5$s$O(BDebian $B%+!<%M%k%A!<%`$N%a%s%P!<$G$"$j!"(Bkernel.org $B$N(B 3.2.y $B0BDjHG7ONs$N%a%s%F%J$G$9!#(BHyper-V guest drivers$B$O(Bmainline kernel$B$G(B3.2$B$K$b4^$^$l$F$$$^$9$,!"$h$j2~A1$5$l$?(B3.4$B$+$i$N=$@5$,F3F~$5$l$^$9!#(B
PREEMPT\_RT$B$O%O!<%I%j%"%k%?%$%`$r<B8=$9$k$?$a$N(BPatch$B!"(B
linux-image-rt-amd64 , linux-image-rt-686-pae $B$N(Bmetapackage$B$G;HMQ$G$-$^$9!#(B
$B?7$7$$(BARM 64$B%S%C%H%"!<%-%F%/%A%c%5%]!<%H$O(Bmainline kernel 3.7$B$+$i(B}

\santaku
{Wookey$B$5$s$,%"%J%&%s%9$7$?(Balpha$BHG$N(BDebian port arm64 image$B$O!)(B}
{Debian/Ubuntu port image}
{Debian/KFreeBSD port image}
{Debian/GnuHurd port image}
{A}
{self-bootstrapp(non x86)$BBP1~$H$N$3$H$G$9!#(B\url{http://wiki.debian.org/Arm64Port}$B$G%9%F!<%?%9$,3NG'$G$-$^$9!#(B}

\santaku
{700,000$BHVL\$N%P%0$,Js9p$5$l$?F|$rEv$F$k(B700000thBugContest$B$N7k2L$,=P$^$7$?!#$=$NM=A[F|$HJs9pF|$O!)(B}
{$BM=A[F|(B:2013/02/04$B!"Js9pF|(B:2013/02/14}
{$BM=A[F|(B:2013/02/07$B!"Js9pF|(B:2013/02/14}
{$BM=A[F|(B:2013/02/14$B!"Js9pF|(B:2013/02/07}
{C}
{$B:G$b6a$$(B2013/02/14$B$rM=A[$7$?(BChristian Perrier$B$5$s$,Ev$F$^$7$?!#7k2L$O(B\url{http://wiki.debian.org/700000thBugContest}$B$G8x3+$5$l$F$$$^$9!#(B
$B$^$?!"(B800,000/1,000,000$BHVL\$N%P%0$,Js9p$5$l$kF|$rEv$F$k%3%s%F%9%H(B\url{http://wiki.debian.org/800000thBugContest}$B$b3+:E$5$l$F$$$^$9!#(B}

\santaku
{master.debian.org$B$,?7$7$$5!3#$K0\9T$5$l$^$7$?!#$3$l$O2?$N%5!<%P$G$7$g$&$+(B $B!)(B}
{@debian.org$B$N%a!<%k%5!<%P(B}
{$B%Q%C%1!<%8$N%^%9%?!<%5!<%P(B}
{$B%Q%C%1!<%8$N%9%]%s%5!<(B(mentor)$B$rC5$9%5!<%P(B}
{A}
{$B8E$$%5!<%P$O%G%#%9%/>c32Ey$,$"$C$?$N$G!"<wL?$HH=CG$5$l!"%G!<%?$,B;<:$9$kA0$K?7$7$$%5!<%P$K0\9T$5$l$^$7$?!#(Bftp-master.debian.org$B$O(BDebian$B$N(B official package $B%j%]%8%H%j$G$9!#%Q%C%1!<%8$N%9%]%s%5!<(B(mentor)$B$rC5$9$N$O(Bmentors.debian.net$B!#(B }

\santaku
{pbuilder$B$K(Bclang support$B$,DI2C$5$l$^$7$?!#C/$,=q$$$?%Q%C%A$G$7$g$&$+!)(B}
{Sylvestre Ledru}
{Junichi Uekawa}
{Hideki Yamane}
{C}
{Debian$B$N(BClang$B%5%]!<%H$OCe!9$H?J$s$G$$$^$9!#(B}

\santaku
{DPN - 2013$BG/(B3$B7n(B4$BF|9f$K<h$j>e$2$i$l$?F|K\$N%$%Y%s%H$O(B}
{Open Source Conference 2013 Tokyo/Spring}
{Open Source Conference 2013 Hamamatu}
{Open Source Conference 2013 Tokushima}
{A}
{\url{http://henrich-on-debian.blogspot.jp/2013/02/open-source-conference-2013-tokyospring.html} $B>\:Y$O8e$[$I!#(B}



\section{セミナー1}
\emtext{Debian 10 busterでnftablesを使ってみる}

\section{ディスカッション}
\emtext{今後のDebianにほしい機能や仕組みを整理する}

%\section{Hack time}
%\emtext{Hack time}

%\begin{frame}{成果を記入下さい!}
%\end{frame}
  
\section{今後のイベント}
\emtext{今後のイベント}

\begin{frame}{今後のイベントの予定}
  \begin{itemize}
  \item 2020/1/11(土) ディストリビューション開発もくもく会 2020年1月
  \item 2020/1/18(土) 東京エリアDebian勉強会(予定)
  \item 2020/2 の定例の東京エリアDebian勉強会はお休み
  \item 2020/2/22(土) OSC 2020 Tokyo/Spring @ 駒沢大学
    \begin{itemize}
    \item セミナー及びブースを出展を予定
    \item 参加は2/22(土)のみ
    \end{itemize}
  \item 2020/3/21(土) 東京エリアDebian勉強会(予定)
  \item 勉強会では発表者を随時募集中です。杉本またはメーリングリストでご連絡ください。
  \end{itemize}
\end{frame}

\end{document}

;;; Local Variables: ***
;;; outline-regexp: "\\([ 	]*\\\\\\(documentstyle\\|documentclass\\|emtext\\|section\\|begin{frame}\\)\\*?[ 	]*[[{]\\|[]+\\)" ***
;;; End: ***
