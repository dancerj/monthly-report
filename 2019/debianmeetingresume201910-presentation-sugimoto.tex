%; whizzy paragraph -pdf xpdf -latex ./whizzypdfptex.sh
%; whizzy-paragraph "^\\\\begin{frame}"
% latex beamer presentation.
% platex, latex-beamer でコンパイルすることを想定。 

%     Tokyo Debian Meeting resources
%     Copyright (C) 2009 Junichi Uekawa
%     Copyright (C) 2009 Nobuhiro Iwamatsu

%     This program is free software; you can redistribute it and/or modify
%     it under the terms of the GNU General Public License as published by
%     the Free Software Foundation; either version 2 of the License, or
%     (at your option) any later version.

%     This program is distributed in the hope that it will be useful,
%     but WITHOUT ANY WARRANTY; without even the implied warreanty of
%     MERCHANTABILITY or FITNESS FOR A PARTICULAR PURPOSE.  See the
%     GNU General Public License for more details.

%     You should have received a copy of the GNU General Public License
%     along with this program; if not, write to the Free Software
%     Foundation, Inc., 51 Franklin St, Fifth Floor, Boston, MA  02110-1301 USA

\documentclass[cjk,dvipdfmx,12pt]{beamer}
\usetheme{Tokyo}
\usepackage{monthlypresentation}

%  preview (shell-command (concat "evince " (replace-regexp-in-string
%  "tex$" "pdf"(buffer-file-name)) "&")) 
%  presentation (shell-command (concat "xpdf -fullscreen " (replace-regexp-in-string "tex$" "pdf"(buffer-file-name)) "&"))
%  presentation (shell-command (concat "evince " (replace-regexp-in-string "tex$" "pdf"(buffer-file-name)) "&"))

%http://www.naney.org/diki/dk/hyperref.html
%日本語EUC系環境の時
\AtBeginDvi{\special{pdf:tounicode EUC-UCS2}}
%シフトJIS系環境の時
%\AtBeginDvi{\special{pdf:tounicode 90ms-RKSJ-UCS2}}

\newenvironment{commandlinesmall}%
{\VerbatimEnvironment
  \begin{Sbox}\begin{minipage}{1.0\hsize}\begin{fontsize}{8}{8} \begin{BVerbatim}}%
{\end{BVerbatim}\end{fontsize}\end{minipage}\end{Sbox}
  \setlength{\fboxsep}{8pt}
% start on a new paragraph

\vspace{6pt}% skip before
\fcolorbox{dancerdarkblue}{dancerlightblue}{\TheSbox}

\vspace{6pt}% skip after
}
%end of commandlinesmall

\title{Debian GNU/kFreeBSD \\ セットアップガイド \\ 2019年版}
\subtitle{東京エリアDebian勉強会} %第179回
\author{Norimitsu Sugimoto (杉本 典充) \\dictoss@live.jp}
\date{2019-10-19 ver1.01}
\logo{\includegraphics[width=8cm]{image200607/openlogo-light.eps}}

\begin{document}

\section{表紙}

\begin{frame}
  \titlepage{}
\end{frame}


\section{目次}

\begin{frame}{アジェンダ}
  \begin{itemize}
  \item 自己紹介
  \item Debian Ports と Debian GNU/kFreeBSD
  \item Debian GNU/kFreeBSD のインストール
  \item Debian GNU/kFreeBSD のセットアップ
  \item Debian の開発ツールのセットアップ
  \item その他の機能の紹介
  \item まとめ
  \item 参考資料
  \end{itemize}
\end{frame}


\section{自己紹介}

\begin{frame}{自己紹介}
  \begin{itemize}
  \item Norimitsu Sugimoto (杉本 典充)
  \item dictoss@live.jp
  \item Twitter: @dictoss
  \item Debian使って15年以上。使い始めはsargeの頃。
  \item 仕事はソフトウェア開発者をやってます
  \item pythonとDjangoの組み合わせで使うことが多いです
  \end{itemize}
\end{frame}


\section{Debian Ports と Debian GNU/kFreeBSD}
\emtext{Debian Ports と Debian GNU/kFreeBSD}

\subsection{Debian Ports と Debian GNU/kFreeBSD}
\begin{frame}{Debian PortsとDebian GNU/kFreeBSD}
  \begin{itemize}
  \item Debian Ports
    \begin{itemize}
    \item \url{https://www.debian.org/ports/}
    \item 様々なCPUアーキテクチャやカーネルを移植して動作させるプロジェクト
    \item 公式リリースと非公式リリースのものがある
    \end{itemize}
  \item Debian GNU/kFreeBSD
    \begin{itemize}
    \item \url{https://wiki.debian.org/Debian_GNU/kFreeBSD}
    \item FreeBSD カーネル をもつ Debian
    \item Debian Ports の1つで非公式版
    \item Debianにおけるアーキテクチャ名は kfreebsd-i386 と kfreebsd-amd64
    \end{itemize}
  \end{itemize}
\end{frame}

\subsection{Debian GNU/kFreeBSD のリリースの歴史}
\begin{frame}{Debian GNU/kFreeBSD のリリースの歴史}
  \begin{itemize}
  \item 2011年2月6日、Debian 6 (squeeze) で stable 版をリリース(テクノロジープレビュー扱い)
  \item 2013年5月4日、Debian 7 (wheezy) で stable 版をリリース
  \item 2014年9月、Debian 8 (jessie) でリリースするアーキテクチャの選定においてリリース対象外に決定\footnote{\url{https://lists.debian.org/debian-devel-announce/2014/09/msg00002.html}}
    \begin{itemize}
    \item jessie-kfreebsd 版という裏バージョンあり
    \end{itemize}
  \item 2018年5月31日、Debian 7 (wheezy)のLTSが終了
    \begin{itemize}
    \item Debian GNU/kFreeBSD の stable 版はサポート終了
    \end{itemize}
  \item Debian 10 (buster) のリリースに向けて ftp.debian.org の整理を実施
  \item 2019年5月25日、Debian GNU/kFreeBSDは Debian Ports(\url{https://www.ports.debian.org/}) へ移転
  \end{itemize}
\end{frame}

\subsection{Porterbox}
\begin{frame}{Porterbox}
  \begin{itemize}
  \item Debian Projectでは多くのアーキテクチャを開発しているため、移植作業用のサーバがある
    \begin{itemize}
    \item porterbox といい、一時的に借りることができる\footnote{\url{https://wiki.debian.org/PorterBoxHowToUse}}\footnote{借りる手順をまとめた記事はこちら。\url{https://tokyodebian-team.pages.debian.net/pdf2016/debianmeetingresume201603.pdf}}
    \end{itemize}
  \item lemon.debian.net
    \begin{itemize}
    \item kfreebsd-amd64 版の porterbox
    \item \url{https://db.debian.org/machines.cgi?host=lemon}
    \end{itemize}
  \end{itemize}
\end{frame}


\subsection{Debian GNU/kFreeBSD 固有の Debian パッケージ}
\begin{frame}{Debian GNU/kFreeBSD固有の\\Debian パッケージ}
  \begin{itemize}
  \item kfreebsd-image パッケージ
  \item zfsutils パッケージ    
  \item freebsd-utils パッケージ
  \item freebsd-net-tools パッケージ
  \item freebsd-smbfs パッケージ
  \item freebsd-ppp パッケージ
  \item pf パッケージ
  \end{itemize}
\end{frame}


\section{Debian GNU/kFreeBSD のインストール}
\emtext{Debian GNU/kFreeBSD のインストール}

\subsection{インストールイメージの入手}
\begin{frame}{インストールイメージの入手}
  \begin{itemize}
  \item \url{https://www.debian.org/devel/debian-installer/} に置いてあるインストーラは動かない
  \item 非公式版のインストーラがあり、そちらを使うこと\footnote{\url{http://jenkins.kfreebsd.eu/jenkins/view/cd/job/debian-cd_sid_kfreebsd-amd64/ws/build/debian-unofficial-kfreebsd-amd64-NETINST-1.iso}}
  \item 通常は kfreebsd-amd64 版を選択すること
  \item ISOイメージから CD/DVDメディアを作成する
  \item PC を CD/DVDメディアから起動してインストールを進める
  \item インストーラは日本語の表示ができないため LANG=C で進める
  \end{itemize}
\end{frame}

\subsection{パーティション構成とファイルシステム}
\begin{frame}{パーティション構成とファイルシステム}
  \begin{itemize}
  \item MBR形式の場合
    \begin{itemize}
    \item root パーティションは基本パーティションにする必要あり
      \begin{itemize}
      \item Debian GNU/Linux や windows と デュアルブートする場合は基本パーティション数4つまでの制限あり
      \end{itemize}
    \item ファイルシステムは UFS を選択するのが無難
    \item ZFS を利用する場合は kfreebsd-amd64 版を選択すること
    \end{itemize}
  \item GPT形式の場合
    \begin{itemize}
    \item 試していないため不明
    \end{itemize}
  \end{itemize}
\end{frame}

\subsection{ミラーサーバの指定}
\begin{frame}{ミラーサーバの指定}
  \begin{itemize}
  \item Debian Installer デフォルトの \url{http://deb.debian.org/debian/} にパッケージは置いていない
  \item Debian GNU/kFreeBSD のパッケージは Debian Ports のサーバにありますがすべてそろっているわけではない
  \item インストールでは、ミラーを利用しない設定でインストールを進める
  \item ``Standard system utilities'' のみをインストールしてインストールを終える
  \end{itemize}
\end{frame}


\section{Debian GNU/kFreeBSD のセットアップ}
\emtext{Debian GNU/kFreeBSD のセットアップ}

\subsection{有線LAN}
\begin{frame}{有線LAN}
  \begin{itemize}
  \item インタフェース名は NIC のベンダーによって変わる (例:em0, re0)
  \item 設定ファイルは Debian GNU/Linux と同じ /etc/network/interfaces
    \begin{itemize}
    \item allow-hotplug は Linux の udev の機能のため利用できない
    \item auto em0 としておくと外出先でも PC起動時に DHCP でアドレスを取得するため不便
    \end{itemize}
  \item \# ifup em0={network-name} は利用可能
  \item ip コマンドはないため、ifconfig コマンドを利用すること
  \end{itemize}
\end{frame}

\subsection{aptの設定}
\begin{frame}[containsverbatim]{aptの設定}

\begin{itemize}
\item インストール時にミラーサーバを指定しなかったため /etc/apt/sources.list に中身がない状態
\item /etc/apt/sources.list を以下に設定\footnote{\url{https://lists.debian.org/debian-bsd/2019/08/msg00005.html}}
\end{itemize}

\begin{commandlinesmall}
deb-src http://ftp.se.debian.org/debian/ \
    sid main contrib non-free
deb http://ftp.ports.debian.org/debian-ports \
    sid main
deb http://ftp.ports.debian.org/debian-ports \
    unreleased main
deb-src http://ftp.ports.debian.org/debian-ports \
    unreleased main
deb http://ftp.ports.debian.org/debian-ports \
    experimental main
\end{commandlinesmall}
\end{frame}

\begin{frame}[containsverbatim]{aptの設定}

\begin{itemize}
\item apt-get を実行すると GPGキーエラーが発生する
\item keyring をインストールすると解消
\end{itemize}
  
\begin{commandlinesmall}
# wget http://ftp.jp.debian.org/debian/pool/main/d/
debian-ports-archive-keyring/
debian-ports-archive-keyring_2018.12.27_all.deb

# dpkg -i debian-ports-archive-keyring_2018.12.27_all.deb
\end{commandlinesmall}

\begin{itemize}
\item 他のkeyringも更新します
\end{itemize}

\begin{commandlinesmall}
# apt-get update
# apt-get install debian-keyring debian-archive-keyring  
\end{commandlinesmall}

\end{frame}


\subsection{locale の設定}
\begin{frame}[containsverbatim]{locale の設定}

C.UTF-8 のままがよい場合はスキップ
  
\begin{commandlinesmall}
# dpkg-reconfigure locales

 -> 日本語の場合は ja_JP.UTF-8 を選択する
\end{commandlinesmall}

\end{frame}


\subsection{sshd のインストール}
\begin{frame}[containsverbatim]{sshd のインストール}

\begin{commandlinesmall}
# apt-get install openssh-server
\end{commandlinesmall}

\end{frame}


\subsection{無線LAN}
\begin{frame}[containsverbatim]{無線 LAN}

無線LANデバイスに応じたファームウェアをインストール(以下はIntelの場合)
  
\begin{commandlinesmall}
# wget http://ftp.se.debian.org/debian/pool/non-free/f/firmware-nonfree/
firmware-iwlwifi_20190717-2_all.deb
# dpkg -i firmware-iwlwifi_20190717-2_all.deb
# reboot
\end{commandlinesmall}

無線LANデーモンのwpasupplicantをインストール

\begin{commandlinesmall}
# apt-get install wpasupplicant
\end{commandlinesmall}

\end{frame}

\begin{frame}[containsverbatim]{無線 LAN}
  \begin{itemize}
  \item 2つのインタフェース
    \begin{itemize}
    \item iwn0  : 物理インタフェース (Intelの場合)
    \item wlan0 : 論理インタフェース
    \end{itemize}
  \item wlan0 の生成
  \end{itemize}
\begin{commandlinesmall}      
# ifconfig wlan create wlandev iwn0      
\end{commandlinesmall}

  \begin{itemize}
  \item 無線LANのSSIDとパスワードを設定
  \end{itemize}

\begin{commandlinesmall}      
# wpa_passphrase apname1 appassword > wpa_apname1.conf
\end{commandlinesmall}

  \begin{itemize}
  \item 無線LANアクセスポイントへ接続し、IPアドレスを取得
  \end{itemize}

\begin{commandlinesmall}      
# wpa_supplicant -i wlan0 -c ./wpa_apname1.conf
# dhclient wlan0
\end{commandlinesmall}

\end{frame}

\begin{frame}[containsverbatim]{無線 LAN}
\begin{itemize}
  \item つながらないときのポイント
    \begin{itemize}
    \item 2.4GHz帯のアクセスポイントを利用する
    \item デュアルチャネル接続を無効にする\footnote{ThinkPad X220とNEC製Atermの組み合わせだと無効にしないとアクセスポイント接続後、すぐに切れてしまう事象が発生しました。}
    \end{itemize}
  \end{itemize}

\begin{commandlinesmall}
# ifconfig wlan0 -ht40
\end{commandlinesmall}

\end{frame}


\subsection{音の再生}
\begin{frame}[containsverbatim]{音の再生}
  \begin{itemize}
  \item FreeBSDは OSS (Open Sound System) を利用
  \item 最近のPCの多くは snd\_hda.ko ドライバを利用するためおおよそ音が出る
  \item MP3ファイルの再生
  \end{itemize}

\begin{commandlinesmall}
# apt-get install mpg123
# mpg123 ${mp3ファイルのパス}
\end{commandlinesmall}

\end{frame}

\subsection{電源関連}
\begin{frame}[containsverbatim]{電源関係}
  \begin{itemize}
  \item CPUクロック制御は powerd で処理する
  \item CPUクロックの確認はsysctl を使う
  \end{itemize}

\begin{commandlinesmall}
$ sysctl dev.cpu.0.freq
dev.cpu.0.freq: 800
\end{commandlinesmall}

  \begin{itemize}
  \item バッテリー残量取得は acpiconf を使う
  \end{itemize}

\begin{commandlinesmall}
# /usr/sbin/acpiconf -i 0
\end{commandlinesmall}

  \begin{itemize}
  \item サスペンド、ハイバーネートは未確認
  \end{itemize}
\end{frame}


\subsection{KMSの有効化}
\begin{frame}[containsverbatim]{KMSの有効化}
  \begin{itemize}
  \item KMS = kernel mode settings
  \item KMS を有効にするとコンソール画面の解像度が上がる
  \item X Window System を使う場合は KMS を有効にすると性能がよくなる
  \item KMS は kernel module をロードすると有効になる
  \end{itemize}

\begin{commandlinesmall}
# kldunload i915
# kldload i915kms
\end{commandlinesmall}

  \begin{itemize}
  \item PC起動時に自動ロードする設定
  \end{itemize}

\begin{commandlinesmall}
# vi /etc/modules
i915kms
\end{commandlinesmall}

\end{frame}


\subsection{X Window System}
\subsubsection{X Window System用ドライバ}
\begin{frame}[containsverbatim]{X Window System用ドライバ}
  \begin{itemize}
  \item ThinkPad X220 の Intel内蔵グラフィックの場合は以下のドライバをインストール
  \end{itemize}
  
\begin{commandlinesmall}
# apt-get install xserver-xorg-video-intel
\end{commandlinesmall}

  \begin{itemize}
  \item ThinkPad X220 で安定動作させるには UXAモードを指定
  \end{itemize}
  
\begin{commandlinesmall}
# vi /etc/X11/xorg.conf.d/50-intel.conf
Section ``Device''
  Identifier  ``Card0''
  Driver``intel''
  Option``AccelMethod''  ``uxa''
EndSection
\end{commandlinesmall}

\end{frame}


\subsubsection{ウィンドウマネージャ}
\begin{frame}[containsverbatim]{ウィンドウマネージャ}
  \begin{itemize}
  \item twm が利用可能
  \item icewm、xfce4、lxde などはパッケージの依存関係が解決できないため現状インストールできない
  \item グラフィカルログインマネージャの xdm、lightdm もパッケージの依存関係が解決できないため現状インストールできない
  \end{itemize}
  
\begin{commandlinesmall}
# apt-get install twm xterm eterm
\end{commandlinesmall}

  \begin{itemize}
  \item root ユーザで startx を実行すると twm が起動する
  \end{itemize}

\begin{commandlinesmall}
# startx
\end{commandlinesmall}
  
\end{frame}


\section{Debianの開発ツールのセットアップ}
\emtext{Debianの開発ツールのセットアップ}

\subsection{インストールできるパッケージ}
\begin{frame}[containsverbatim]{インストールできるパッケージ}

\begin{itemize}
\item make / bmake
\item gcc-9 / g++-9
\item clang-7
\item dpkg-dev
\item devscripts
\item build-essential
\item debhelper
\item debootstrap
\item subversion
\item vim
\end{itemize}

\end{frame}

\begin{frame}[containsverbatim]{インストールできないパッケージ}

現状、ftp.ports.debian.org にあるバイナリパッケージ群では依存関係が壊れているため以下のパッケージはインストールできない

\begin{itemize}
\item clang-8
\item git
\item emacs / emacs-nox
\end{itemize}

\end{frame}


\section{その他の機能の紹介}
\emtext{その他の機能の紹介}

\subsection{コンテナ環境:Jail}
\begin{frame}[containsverbatim]{コンテナ環境:Jail}

\begin{itemize}
\item FreeBSD の コンテナ型仮想化環境を実行する機能
\item freebsd-utils パッケージにコマンドが入っている
\item コンテナ環境の作成は debootstrap を使う
\item コンテナ環境の操作は jail、jls、jexec を使う
\end{itemize}

\begin{commandlinesmall}
# debootstrap --no-check-gpg sid ./jail_demo_1 \
http://ftp.ports.debian.org/debian-ports/
\end{commandlinesmall}

\begin{commandlinesmall}
# jail -c path=./jail_demo_1 command=/bin/bash
\end{commandlinesmall}

\begin{commandlinesmall}
# jls
JID  IP Address    Hostname  Path
  1  -                       /root/jail/jail_1
# jexec 1 cat /etc/debian_version
\end{commandlinesmall}

\end{frame}
  
\subsection{Linux エミュレーション}
\begin{frame}[containsverbatim]{Linux エミュレーション}
  \begin{itemize}
  \item FreeBSD には Linux バイナリ互換機能があり、linux バイナリを実行可能
  \item debootstrap で Debian 6 squeeze i386 のコンテナを作って動作確認を実施
    \begin{itemize}
    \item エラーになってしまう状況
    \end{itemize}
  \end{itemize}

\begin{commandlinesmall}
# debootstrap --no-check-gpg --arch=i386 squeeze \
./linux_demo_1 http://archive.debian.org/debian/

ELF binary type ``0'' not known.
E: Unable to execute target architecture
\end{commandlinesmall}

\end{frame}

\subsection{完全仮想化環境}
\begin{frame}[containsverbatim]{完全仮想化環境}
  \begin{itemize}
  \item 本家FreeBSDでは動作しているが、kFreeBSD ではパッケージが存在しない
    \begin{itemize}
    \item VirtualBox
    \item bhyve
    \end{itemize}
  \end{itemize}
\end{frame}


\begin{frame}[containsverbatim]{おわりに}
  \begin{itemize}
  \item Debian GNU/kFreeBSD のインストール方法とセットアップ方法を説明しました
  \item 使ってみて疑問や動かないところがありましたら、相談に乗ります
  \end{itemize}
\end{frame}


\section{参考文献}
\emtext{参考文献}

\begin{frame}{参考文献}

\begin{itemize}
\item 「Debian\_GNU/kFreeBSD - Debian Wiki」\\ \url{https://wiki.debian.org/Debian_GNU/kFreeBSD}
\item ``How to get a debian kfreebsd sid'' (2019-08-25)\\ \url{https://lists.debian.org/debian-bsd/2019/08/msg00004.html}
\item 杉本典充(2015).「Debian GNU/kFreeBSDセットアップガイド2015年版」\\ \url{https://tokyodebian-team.pages.debian.net/pdf2015/debianmeetingresume201511-presentation-sugimoto.pdf}
\end{itemize}

\end{frame}

\end{document}


;;; Local Variables: ***
;;; outline-regexp: "\\([ 	]*\\\\\\(documentstyle\\|documentclass\\|emtext\\|section\\|begin{frame}\\)\\*?[ 	]*[[{]\\|[]+\\)" ***
;;; End: ***
