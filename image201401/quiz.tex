%; whizzy-master ../debianmeetingresume201311.tex
% 以上の設定をしているため、このファイルで M-x whizzytex すると、whizzytexが利用できます。
%

\santaku
{PHPのメンテナチームを3つに分割する事が提案されました。分割されたグループの名前で間違っているのはどれ?}
{Debian PHP PECL Maintainers}
{Debian PHP PEAR Maintainers}
{Debian PHP DOCUMENT Maintainers}
{C}
{正しくは''Debian PHP Maintainers''です。以前、Debian PHP Maintainersは、PHP本体のパッケージも、PEARモジュールのパッケージも両方メンテナンスしていました。}

\santaku
{Debianについて、Debian Developer以外の人でも貢献したを讃えましょうということで、作られたサイトは?}
{advocates.debian.org}
{contributors.debian.org}
{superstar.debian.org}
{B}
{Debianに貢献したDebian Developer以外の人のアカウントが\url{http://contributors.debian.org/}にリストアップされるようになりました。なお、貢献についての集計の元は、\url{https://contributors.debian.org/sources/}に掲載されている情報を元に集計しているとの事です。}

\santaku
{2013/12後半頃にs390xアーキテクチャのデフォルトCコンパイラとしてのgccのバージョンが変更されました。どのバージョンになったのでしょう?}
{4.8}
{4.7}
{4.6}
{A}
{2013/12/23現在、powerpc/ia64/sparcアーキテクチャのデフォルトCコンパイラはまだgcc 4.6のようです。Debianの次期バージョンのJessieではgcc 4.6はサポート対象外なので早いところgcc 4.6から脱却する必要があります。}

\santaku
{2014/1に有名なデータベースがパッケージとして追加されました。何というデータベースでしょうか?}
{Maria DB}
{Percona DB}
{GDB}
{A}
{Maria DBは、LAMPシステムで有名なMysql DBの別の実装です。ついにMaria DBキター!今後のMysql依存のDebianのパッケージの動向が気になるこの頃です。}


