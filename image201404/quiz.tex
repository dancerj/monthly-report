%; whizzy-master ../debianmeetingresume201311.tex
% 以上の設定をしているため、このファイルで M-x whizzytex すると、whizzytexが利用できます。
%

\santaku
{2015年のDebconf15の開催国はどこになったでしょう?}
{ボスニア・ヘルツェゴビナ}
{ウクライナ}
{ドイツ}
{C}
{2015年はドイツだそうです。ちなみに2014年のDebconf14は8/23-31でUSAのPortland,Oregon
で開催予定です。}

\santaku
{DPL選挙が行われました。2014年のDPLは誰になったでしょうか?}
{Lucas Nussbaum}
{Neil McGovern}
{Rapha\"{e}l Hertzog}
{A}
{2014年のDPLの候補者は2名で、Lucas Nussbaumさん、Neil McGovernさんの2名
でした。2名とも自薦とのことです。選挙の結果、
Lucas Nussbaum(lucus)さんの圧勝だったようです。ちなみに、
Rapha\"{e}l Hertzogさんは、The Debian 
Administrator\'s handbookの作者、他にも偉業がたくさん。}

\santaku
{clang3.4によるDebianパッケージの再構築が行われました。結果何\%のパッケージが成功したでしょうか?}
{90\%}
{50\%}
{10\%}
{A}
{\url {http://clang.debian.net/}がclangによるDebianパッケージ再構築のポータルサイトです。毎年、clangのバージョンを上げて、全Debianパッケージを再構築した結果が載ります。2014/1の結果としては再構築対象のパッケージの数が2013/1より大幅に増えているにもかかわらず、構築失敗に終わったパケージ数が変わらなかったという非常に良い結果となっています。}

\santaku
{beagle boardシリーズという非常に人気のあるARMの実験ボードにバンドルされるOSの将来の見通しについて、beagle boardの創立者がどのOSにする予定と発言したか?}
{Gentoo}
{Debianっしょ!Debian}
{Andoroid OS}
{B}
{\url{http://opensource.com/life/14/3/interview-jason-kridner-beagleboard}
にて、将来Debianにするという発言があります。ところで、beagle boardシリーズは未だにOMAPを
使い続けるのかが興味津々ではあります。}

\santaku
{激論の末、2014/3中旬頃にDebianのca-certificatesパッケージから消えたroot証明書があります。それはなんでしょう?}
{RapidSSLのroot証明書}
{CAcertのroot証明書}
{Verisignのroot証明書}
{B}
{議論のサマリは、LWNの記事\url{https://lwn.net/Articles/590879/}が判りやすいです。また、CAcertって何?という方は、第71回東京エリアDebian勉強会(2010年12月開催)\url{http://tokyodebian.alioth.debian.org/2010-12.html}に掲載されている勉強会資料がお勧めです。}

\santaku
{Jessieにてデスクトップ環境を選択した際に導入される、デフォルトコミュニケーションツールの候補について議論がされていります。以下のどれでしょう?}
{Empathy}
{licq}
{jitsi}
{C}
{Debianでは、デフォルトのコミュニケーションツールとして、ほぼ完全にRTC/VoIPをサポートすることが望ましいとされたため、こちらに向いているツールとして今までのEmpathyよりもjitsiが向いているのでは?ということから議論が開始されました。}

\santaku
{2014/4/2にDebianにOTRチームというOTRソフトをパッケージ化するグループが結成されました。ところでOTRってなんの略?}
{Owa-Tte-Ru}
{OpticalTRacking}
{Off-the-Record}
{C}
{Off-the-Recordとは、暗号化技術を使ってインフラ提供者にすらメッセージのやりとりの内容を見せない(記録させない)メッセージサービスを目指したものです\url{https://www.otr.im/}。}

\santaku
{2014/4/16にてDebian squeezeのサポート期間が伸びる宣言がDSAによりアナウンスされました。結局いつになった?}
{2015/2}
{2016/2}
{2016/5}
{B}
{Long Term Support(LTS)だそうです\url{https://lists.debian.org/debian-security-announce/2014/msg00082.html}。予定では、2014/5/31頃にサポート終了するはずでしたので、2年弱の延長となります。ただサポート延長されるのは、Debian squeezeの全部のパッケージではないので、サポートされないパッケージを十分にお使いの皆様は、Debian wheezyへアップグレードすることをお勧めしておきます。}


