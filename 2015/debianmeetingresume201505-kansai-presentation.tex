\documentclass[cjk,dvipdfmx,10pt,compress,%
hyperref={bookmarks=true,bookmarksnumbered=true,bookmarksopen=false,%
colorlinks=false,%
pdftitle={第 98 回 関西 Debian 勉強会},%
pdfauthor={倉敷・のがた・佐々木・かわだ},%
%pdfinstitute={関西 Debian 勉強会},%
pdfsubject={資料},%
}]{beamer}

\title{第 98 回 関西 Debian 勉強会}
\subtitle{$\sim$発表資料$\sim$}
\author[かわだ てつたろう]{{\large\bf 倉敷・のがた・佐々木・かわだ}}
\institute[Debian JP]{{\normalsize\tt 関西 Debian 勉強会}}
\date{{\small 2015 年 5 月 24 日}}

%\usepackage{amsmath}
%\usepackage{amssymb}
\usepackage{graphicx}
\usepackage{moreverb}
\usepackage[varg]{txfonts}
\AtBeginDvi{\special{pdf:tounicode EUC-UCS2}}
\usetheme{Kyoto}
\def\museincludegraphics{%
  \begingroup
  \catcode`\|=0
  \catcode`\\=12
  \catcode`\#=12
  \includegraphics[width=0.9\textwidth]}
%\renewcommand{\familydefault}{\sfdefault}
%\renewcommand{\kanjifamilydefault}{\sfdefault}
\begin{document}
\settitleslide
\begin{frame}
\titlepage
\end{frame}
\setdefaultslide

\begin{frame}[fragile]
  \frametitle{Disclaimer}
  \begin{itemize}
  \item 疑問、質問、ツッコミ、茶々、\alert{大歓迎}
  \item その場でインタラクティブにどうぞ
  \item ハッシュタグ \#kansaidebian
  \end{itemize}
\end{frame}

\begin{frame}[fragile]
\frametitle{Agenda}

\tableofcontents

\end{frame}

\section{最近の Debian 関係のイベント}

\takahashi[40]{最近の Debian\\関係のイベント}

\begin{frame}[fragile]
  \frametitle{第97回関西Debian勉強会}
  \begin{itemize}
  \item 日時: 4月26日(日)
  \item 場所: 京都大学
  \end{itemize}
  \begin{block}{内容}
    \begin{itemize}
    \item Jessie リリースパーティ
    \end{itemize}
  \end{block}
\end{frame}

\begin{frame}[fragile]
  \frametitle{第126回東京エリアDebian勉強会}
  \begin{itemize}
  \item 日時: 5月23日(土)
  \item 場所: 株式会社スクウェア・エニックス セミナールーム
  \end{itemize}
  \begin{block}{内容}
    \begin{itemize}
    \item 「自然言語処理チーム(pkg-nlp-ja)とパッケージ」
    \end{itemize}
  \end{block}
\end{frame}

\begin{frame}[fragile]
  \frametitle{Debian Project}
  \begin{itemize}
  \item 
  \end{itemize}
\end{frame}

\takahashi[50]{そんな\\こんなで}
\takahashi[120]{次}

\section{事前課題発表}

\takahashi[50]{事前課題}

\begin{frame}[fragile]
  \frametitle{事前課題}
  \begin{block}{今回の事前課題}
    \begin{description}
    \item[事前課題1]
      Jessieを使ってみて嵌ったこと、使おうとしているが気になること、を教えてください。
    \end{description}
  \end{block}
\end{frame}

\takahashi[50]{事前課題\\発表}

\begin{frame}
  \frametitle{ Yukiharu YABUKI }
  \begin{enumerate}
  \item 玄箱をwheezyからjessieにアップグレード
  \end{enumerate}
\end{frame}

\begin{frame}
  \frametitle{ むんくさん }
  \begin{enumerate}
  \item  debian-users ML で流れているインストールトラブルを見て首を傾げながら、
    情報収集が大事そうだなぁと思っています。
  \end{enumerate}
\end{frame}

\begin{frame}
  \frametitle{ t3rkwd }
  \begin{enumerate}
  \item nginxの設定。
  \end{enumerate}
\end{frame}

\begin{frame}
  \frametitle{ 川江 浩 }
  \begin{enumerate}
  \item sysmtclの使い出がイマイチかな?
  \end{enumerate}
\end{frame}

\begin{frame}
  \frametitle{ Yosuke OTSUKI }
  \begin{enumerate}
  \item 転職を気にオープンソースのプロジェクトに参加してみたいと思っています。
  \end{enumerate}
\end{frame}

\takahashi[50]{そんな\\こんなで}
\takahashi[120]{次}

\section{Jessie落穂拾い}
\takahashi[30]{Jessie落穂拾い\\by\\かわだてつたろう}

\takahashi[50]{そんな\\こんなで}
\takahashi[120]{次}

\section{今後の予定}
\begin{frame}[fragile]
\frametitle{今後の予定}

\begin{block}{第99回関西Debian勉強会}
  \begin{itemize}
  \item 日時: 6月28日(日)
  \item 場所: 福島区民センター 304号
  \end{itemize}
\end{block}

\begin{block}{第127回東京エリアDebian勉強会}
  \begin{itemize}
  \item 日時: 6月20日(土)
  \item 場所: 株式会社スクウェア・エニックス セミナールーム
  \item 内容: 未定
  \end{itemize}
\end{block}

\end{frame}

\takahashi[50]{  }

\end{document}
%%% Local Variables:
%%% mode: japanese-latex
%%% TeX-master: t
%%% End:
