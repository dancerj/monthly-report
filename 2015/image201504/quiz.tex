%; whizzy-master ../debianmeetingresume201311.tex
% 以上の設定をしているため、このファイルで M-x whizzytex すると、whizzytexが利用できます。
%

\santaku
{2015/2/17にて、DPLのLucas Nussbaumにより立て直しの協力募集が行われたチームは?}
{DSA team}
{partners team}
{東京エリアDebian勉強会team}
{B}
{Debian Projectには、Debian パートナーズプログラム(https://www.debian.org/partners/)という制度があります。こちらを統括しているチームとしてpartners teamがあるのですが、昨今、こちらの構成員の時間が取れない状態らしく、対応が滞っているという状況のようです。なお、DSAはThe Debian System Administratorsというチームのことでバリバリ活動されてます。東京エリアDebian勉強会teamは当勉強会幹事の頭の中の妄想上のチームですが、頑張ってるぜ!}

\santaku
{2015/2/23に、Debianがクレジットに載せられた映画があることがindenti.caのzakの投稿で判りました。何と言う名前の映画?}
{ミレニアム ドラゴン・タトゥーの女}
{Toy Story}
{Citizenfour}
{C}
{Citizenfourというエドワード・スノーデンを題材にしたドキュメンタリー映画のクレジットのSpecial ThanksとしてDebianが入りました。imdbにもcompanyとして登録されました。Citizenfourは2015年アカデミー賞のうち長編ドキュメンタリー賞を受賞しました(http://www.imdb.com/company/co0504449/, http://www.huffingtonpost.jp/tatsuhei-morozumi/citizenfour\_b\_6741756.html)ギャガ株式会社配給で将来日本でも公開されるとのことですので、将来クレジットを確認すると良いかもしれません}

\santaku
{2015/3/31にJessieのリリース目標の日がアナウンスされました。いつでしょう?}
{4/1}
{4/18}
{4/25}
{C}
{遂に4/25がJessieのリリース目標の日取りとなりました。ただ、最悪のケースとして、Jessieにリリースに支障があるような非常に重大な問題が見つかり修正が間に合わない場合は、ずれる可能性があるとは記載されています。現在uddを参照すると、キーパッケージだけでも47個ぐらいRCバグが残ったままのようですので、Fix! Fix! なお、4/25に東京でリリースパーティーが企画されています。詳細はconnpassのDebian JPグループを見て下さいませ。}

\santaku
{2015/4/16のlucas最後のbit from DPLが流れました。こちらに記載されていたDebianからパッケージをリリースできるようにするための法的対策を検討中のソフトウェアは以下のどれ?}
{libdvdcss}
{OTR software}
{Unreal Engine4}
{A}
{なんと、libdvdcssと、ZFSが検討中らしい。もちろん、ダメになる可能性は十分にあるので、状況を静観したいと思います。libdvdcssはそのコード本体がクラッキング行為の塊なので、万一debianに入った場合、日本でdebianをミラーしても大丈夫かどうかは今後の議論ですね}

\santaku
{Debianにて開発者の属性の不均衡を正す事を目的として活動する公式チームがDebian Projectに新設したと4/16にlucasによりアナウンスがありました。なんという名前のチーム?}
{Debian Release team}
{Debian Outreach team}
{Technical Comittie}
{B}
{現在、是正すべき属性のターゲットとしては性別分布を目標としているようです。現在、Debian公式開発者の性別の割合は男性に大きく偏ってしまっています。Debianは多様性を重視するプロジェクトです。将来、性別の不均衡が是正されると良いですね。}

\santaku
{2015/4/15に選出された新DPLは誰?}
{Mehdi Dogguy}
{Gergely Nagy}
{Neil McGovern}
{C}
{Neil McGovernさんはDebian ProjectでRelease Teamで活動されるなど多数の貢献をされている方です(所信表明は:https://www.debian.org/vote/2014/platforms/neilm )。ちなみに、2015のDebian JP Project会長は岩松さんが選出されました。おめでとうございます。}






