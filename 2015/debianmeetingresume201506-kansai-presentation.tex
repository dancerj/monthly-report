\documentclass[cjk,dvipdfmx,10pt,compress,%
hyperref={bookmarks=true,bookmarksnumbered=true,bookmarksopen=false,%
  colorlinks=false,%
  pdftitle={第 99 回 関西 Debian 勉強会},%
  pdfauthor={倉敷・のがた・佐々木・かわだ},%
  % pdfinstitute={関西 Debian 勉強会},%
  pdfsubject={資料},%
}]{beamer}

\title{第 99 回 関西 Debian 勉強会}
\subtitle{$\sim$発表資料$\sim$}
\author[%
倉敷, のがた, 佐々木, かわだ]{%
  {\large{\textbf{倉敷・のがた・佐々木・かわだ}}}}
\institute[Debian JP]{\normalsize{関西 Debian 勉強会}}
\date{{\small{2015 年 6 月 28 日}}}

%\usepackage{amsmath}
%\usepackage{amssymb}
\usepackage{graphicx}
\usepackage{moreverb}
\usepackage[varg]{txfonts}
\AtBeginDvi{\special{pdf:tounicode EUC-UCS2}}
\usetheme{Kyoto}
%\renewcommand{\familydefault}{\sfdefault}
%\renewcommand{\kanjifamilydefault}{\sfdefault}
\begin{document}
\settitleslide
\begin{frame}
  \titlepage
\end{frame}
\setdefaultslide

\begin{frame}[fragile]
  \frametitle{Disclaimer}
  \begin{itemize}
  \item 疑問、質問、ツッコミ、茶々、\alert{大歓迎}
  \item その場でインタラクティブにどうぞ
  \item ハッシュタグ \texttt{\#kansaidebian}
  \end{itemize}
\end{frame}

\begin{frame}[fragile]
\frametitle{Agenda}

\tableofcontents

\end{frame}

\section{最近の Debian 関係のイベント}

\takahashi[40]{最近の Debian\\関係のイベント}

\begin{frame}[fragile]
  \frametitle{第98回関西Debian勉強会}
  \begin{itemize}
  \item 日時: 5月24日(日)
  \item 場所: 福島区民センター
  \end{itemize}
  \begin{block}{内容}
    \begin{itemize}
    \item Jessie 落穂拾い
    \item 雑談:
      \begin{itemize}
      \item Reproducable binary package by 矢吹さん
      \item Java packaging 四方山話
      \end{itemize}
    \end{itemize}
  \end{block}
\end{frame}

\begin{frame}[fragile]
  \frametitle{第126回東京エリアDebian勉強会}
  \begin{itemize}
  \item 日時: 6月20日(土)
  \item 場所: 株式会社スクウェア・エニックス セミナールーム
  \end{itemize}
  \begin{block}{内容}
    \begin{itemize}
    \item 「Debian と脆弱性対策」
    \end{itemize}
  \end{block}
\end{frame}


\begin{frame}[fragile]
  \frametitle{OSC 2015 Hokkaido}
  % \begin{columns}[totalwidth=\textwidth]
  %   \begin{column}{.65\textwidth}
      \begin{block}{Debian 勉強会出張版@OSC 2015 Hokkaido}
        \begin{itemize}
        \item%
          ブース: 実機展示, ステッカー配布
        \item %
          セミナー: 「8,9,10: Jessie, Stretch, Buster」by やまねさん
        \item %
          杉本さん(dictoss) より, 勉強会 ML にレポートがありました.
        \end{itemize}
      \end{block}
  %   \end{column}
  %   \begin{column}{.3\textwidth}
  %     \includegraphics[height=\textheight]{./image201506/OSC2015do.png}
  %   \end{column}
  % \end{columns}
\end{frame}


\begin{frame}[fragile]
  \frametitle{最近の Debian Project}
  \begin{itemize}
  \item %
    Debian 8.1: Jessie Point Release.
  \item devian-devel %
    \begin{itemize}
    \item%
      \texttt{[CTTE \#750135]}: Aptitude Project Maintainer \\
      @see \texttt{https://bugs.debian.org/cgi-bin/bugreport.cgi?bug=750135}
    \end{itemize}
  \end{itemize}
\end{frame}

\takahashi[50]{そんな\\こんなで}
\takahashi[120]{次}

\section{事前課題発表}

\takahashi[50]{事前課題}

\begin{frame}[fragile]
  \frametitle{事前課題}
  \begin{block}{今回の事前課題}
    \begin{description}
    \item[事前課題1]
      Jessieを使ってみて嵌ったこと、使おうとしているが気になること、を教えてください。
    \end{description}
  \end{block}
\end{frame}

\takahashi[50]{事前課題\\発表}


\begin{frame}
\frametitle{川江 浩}
\begin{itemize}
\item %
  Jessie は快適です.
\item %
  最近のハマリポイント
  \begin{itemize}
  \item %
    Debian GNU/Hurd を KVM に入れようとしています. インストール後に起動すると, kernel panic をおこして上がってきません. 同じ症状の方, いらっしゃいませんか?
  \item %
    Solaris 8 を KVM に入れようとじたばたしております.
  \end{itemize}
\end{itemize}
\end{frame}
\begin{frame}
\frametitle{lurdan}
\begin{itemize}
\item 後程 LT します.
\end{itemize}
\end{frame}
\begin{frame}
\frametitle{佐々木洋平}
\begin{itemize}
\item %
  Jessie の LXC 環境を, 親が wheezy の時に作ると LXC コンテナには systemd が入ってしまって起動しない. chroot して sysvinit-core を入れる必要があった. からい.
\end{itemize}
\end{frame}
\begin{frame}
\frametitle{gdevmjc}
\begin{itemize}
\item 新規にゲットした Let's Note CF-N9 に Jessie をインストールしています. 今日これからハマリ所を探そうかと.
\item 今の所順調です. ...残念ながら(笑)
\end{itemize}
\end{frame}
\begin{frame}
\frametitle{t3rkwd}
\begin{itemize}
\item %
  Jessie のハマり所
  \begin{itemize}
  \item %
    \texttt{nginx} の設定ファイルの解釈が厳格(?)になったので, 設定修正が必要だった.
  \end{itemize}
\end{itemize}
\end{frame}
\begin{frame}
\frametitle{Yosuke OTSUKI}
\begin{itemize}
\item %
  Wheezy から Jessie に上げてみた
  \begin{enumerate}
  \item %
    wifi の firmware バイナリが消えました. 再インストールしたら大丈夫でしたが.
  \item %
    アップグレードしたら, 抜いた筈の apache が再度インストールされていました. 何故でしょう?
    \begin{itemize}
    \item %
      gnome-core に依存して, gnome-user-share がインストールされているから, では?(佐々木)
    \end{itemize}
  \end{enumerate}
\end{itemize}
\end{frame}
\begin{frame}
\frametitle{大林一平}
\begin{itemize}
\item %
  現在は Stretch を使っていますが, Jessie からアップグレードしました
\item %
  CF-RZ4 を使っていますが, ハードウェアが新しいので X が上がらなかったので, Stretch にしました.
  今なら jessie-backports の Intel VGA ドライバを使えば良いのかもしれませんが, 当時(リリース直後)は backports が無かったので....
\end{itemize}
\end{frame}

\takahashi[50]{そんな\\こんなで}
\takahashi[120]{次}

\takahashi[50]{そんな\\こんなで}
\takahashi[120]{次}

\section{今後の予定}
\begin{frame}[fragile]
\frametitle{今後の予定}

\begin{block}{第100回関西Debian勉強会 - OSC 2015 Kansai@Kyoto}
  \begin{itemize}
  \item 日時: 8月7,8日(金, 土)
  \item 場所: KRP
  \end{itemize}
\end{block}

\begin{block}{第127回東京エリアDebian勉強会}
  \begin{itemize}
  \item 日時: 7月20日(土)…?
  \item 場所: 株式会社スクウェア・エニックス セミナールーム
  \item 内容: 未定(Debian パッケージング道場?)
  \end{itemize}
\end{block}

\end{frame}

\takahashi[50]{  }

\end{document}
%%% Local Variables:
%%% mode: japanese-latex
%%% TeX-master: t
%%% End:
