\begin{prework}{ 野島 }
  \begin{enumerate}
  \item Q.hack time に何をしますか?\\
    A. DDTSSやら、xmrisパッケージング化で!
  \end{enumerate}
\end{prework}

\begin{prework}{ roger  }
  \begin{enumerate}
  \item Q.hack timeに何をしますか?\\
    A. BTSバグの確認など
  \item (オプション)Q.本勉強会をどこでお知りになりましたか?\\
    A. twitter
    \end{enumerate}
\end{prework}

\begin{prework}{ yus4ku }
  \begin{enumerate}
  \item Q.hack timeに何をしますか?\\
    A. パッケージング。前回の勉強会の続き。
  \item (オプション)Q.本勉強会をどこでお知りになりましたか?\\
    A. ML, debian-devel\@d.o.j
  \end{enumerate}
\end{prework}

\begin{prework}{ ktaka }
  \begin{enumerate}
  \item Q.hack timeに何をしますか?\\
    A. jessieのディスクレスPXEブート用のイメージを作成してみようと思います。あるいはコンテナ関連。
  \item (オプション)Q.本勉強会をどこでお知りになりましたか?\\
    A. dots
  \end{enumerate}
\end{prework}

\begin{prework}{ knok }
  \begin{enumerate}
  \item Q.hack timeに何をしますか?\\
    A. 自由ソフトウェアによる動画配信の手段を模索する
  \item (オプション)Q.本勉強会をどこでお知りになりましたか?\\
    A. ML
  \end{enumerate}
\end{prework}

\begin{prework}{ dictoss }
  \begin{enumerate}
  \item Q.hack timeに何をしますか?\\
    A.xl2tpdパッケージの動作確認、\\
     kfreebsd関連の情報収集
  \item (オプション)Q.本勉強会をどこでお知りになりましたか?\\
    A. メーリングリスト
  \end{enumerate}
\end{prework}

\begin{prework}{ yy\_y\_ja\_jp }
  \begin{enumerate}
  \item Q.hack timeに何をしますか?\\
    A. DDTSS\\
    http://ddtp.debian.net/ddtss/index.cgi/ja
  \end{enumerate}
\end{prework}

\begin{prework}{ henrich }
  \begin{enumerate}
  \item Q.hack timeに何をしますか?\\
    A. gitの使い方を学ぼうと思います。
  \end{enumerate}
\end{prework}

\begin{prework}{ koedoyoshida }
  \begin{enumerate}
  \item Q.hack timeに何をしますか?\\
    A. 未定
  \item (オプション)Q.本勉強会をどこでお知りになりましたか?\\
    A. ML
  \end{enumerate}
\end{prework}

\begin{prework}{ issei }
  \begin{enumerate}
  \item Q.hack timeに何をしますか?\\
    A. grubについて調べる(UEFImodeでのブートがtestingでのみうまくいったが、実際よく分からなかったので)
  \end{enumerate}
\end{prework}


