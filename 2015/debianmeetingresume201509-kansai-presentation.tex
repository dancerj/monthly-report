\documentclass[cjk,dvipdfmx,10pt,compress,%
hyperref={bookmarks=true,bookmarksnumbered=true,bookmarksopen=false,%
colorlinks=false,%
pdftitle={第 102 回 関西 Debian 勉強会},%
pdfauthor={倉敷・のがた・佐々木・かわだ},%
%pdfinstitute={関西 Debian 勉強会},%
pdfsubject={資料},%
}]{beamer}

\title{第 102 回 関西 Debian 勉強会}
\subtitle{$\sim$発表資料$\sim$}
\author[かわだ てつたろう]{{\large\bf 倉敷・のがた・佐々木・かわだ}}
\institute[Debian JP]{{\normalsize\tt 関西 Debian 勉強会}}
\date{{\small 2015 年 9 月 27 日}}

%\usepackage{amsmath}
%\usepackage{amssymb}
\usepackage{graphicx}
\usepackage{moreverb}
\usepackage[varg]{txfonts}
\AtBeginDvi{\special{pdf:tounicode EUC-UCS2}}
\usetheme{Kyoto}
\def\museincludegraphics{%
  \begingroup
  \catcode`\|=0
  \catcode`\\=12
  \catcode`\#=12
  \includegraphics[width=0.9\textwidth]}
%\renewcommand{\familydefault}{\sfdefault}
%\renewcommand{\kanjifamilydefault}{\sfdefault}
\begin{document}
\settitleslide
\begin{frame}
\titlepage
\end{frame}
\setdefaultslide

\begin{frame}[fragile]
  \frametitle{Disclaimer}
  \begin{itemize}
  \item 疑問、質問、ツッコミ、茶々、\alert{大歓迎}
  \item その場でインタラクティブにどうぞ
  \item ハッシュタグ \#kansaidebian
  \end{itemize}
\end{frame}

\begin{frame}[fragile]
\frametitle{Agenda}

\tableofcontents

\end{frame}

\section{最近の Debian 関係のイベント}

\takahashi[40]{最近の Debian\\関係のイベント}

\begin{frame}[fragile]
  \frametitle{第101回関西Debian勉強会}
  \begin{itemize}
  \item 日時: 8月23日(日)
  \item 場所: 福島区民センター
  \end{itemize}
  \begin{block}{内容}
    \begin{itemize}
    \item 「wiki:Subkeys」
    \end{itemize}
  \end{block}
\end{frame}

\begin{frame}[fragile]
  \frametitle{第130回東京エリアDebian勉強会}
  \begin{itemize}
  \item 日時: 9月12日(土)
  \item 場所: イベント&コミュニティスペース dots.
  \end{itemize}
  \begin{block}{内容}
    \begin{itemize}
    \item 第3回 Debianパッケージング道場
    \end{itemize}
  \end{block}
\end{frame}

\begin{frame}[fragile]
  \frametitle{Debian Project}
  \begin{itemize}
  \item Bits from Perl maintainers
  \item Summary of the Debian CD BoF at DC15
  \item Summary of the DebConf firmware discussion
  \item an abandon of the pursuit of LSB compatibility for Debian
  \item system upgrade by systemd
  \end{itemize}
\end{frame}

\takahashi[50]{そんな\\こんなで}
\takahashi[120]{次}

\takahashi[50]{事前課題}

\begin{frame}[fragile]
  \frametitle{事前課題}
  \begin{block}{今回の事前課題}
    \begin{description}
    \item[事前課題1]
      Debconf15のセッションの中で気になったものを教えてください。
    \item[事前課題2]
      そのセッションのVideoを見てきて下さい。
    \end{description}
  \end{block}
\end{frame}

\takahashi[50]{事前課題\\発表}

\begin{frame}
  \frametitle{ Yamada Yohei (山田 洋平) }
  \begin{enumerate}
  \item Linux\_kernel\_BoF.webm
  \item 時間が無いです。
  \end{enumerate}
\end{frame}

\begin{frame}
  \frametitle{ むんくさん }
  \begin{enumerate}
  \item DNS in Debian
  \item はい、見ました。
  \end{enumerate}
\end{frame}

\begin{frame}
  \frametitle{ ItSANgo }
  \begin{enumerate}
  \item Debian\_Package\_Infrastructure\_walk\_through が気になりました。
  \item 見たのですが、何が話されているのか理解できませんでしたorz

    Lets\_get\_ready\_to\_Go の方はコードが出てくるのでなんとなくわかりましたorz
  \end{enumerate}
\end{frame}

\begin{frame}
  \frametitle{ Kozo Nishida }
  \begin{enumerate}
  \item Automatic\_packaging.webm
  \item はい
  \end{enumerate}
\end{frame}

\begin{frame}
  \frametitle{ t3rkwd }
  \begin{enumerate}
  \item Firmware - a hard or soft problem?
  \item サマリを読んでから探してみました。が、ありませんでした。
  \end{enumerate}
\end{frame}

\begin{frame}
  \frametitle{ Yosuke OTSUKI }
  \begin{enumerate}
  \item Workshop の Multiarch/Crossbuild/Bootstrap/Toolchain minisprint
  \item debconf の動画サーバー重いです。 you tube にも上がっているセッションがあり重宝しています。 余談ですが debconf の video だと思ってみていたら defcon の video 20分ぐらい見てました。
  \end{enumerate}
\end{frame}

\begin{frame}
  \frametitle{ 川江 浩 }
  \begin{enumerate}
  \item GNUHurd
  \item OK.
  \end{enumerate}
\end{frame}

\takahashi[50]{そんな\\こんなで}
\takahashi[120]{次}

\section{ドイツ、ハイデルベルクで開催されたDebconf15へいってきました}
\takahashi[30]{ドイツ、ハイデルベルクで開催されたDebconf15へいってきました\\by\\矢吹 幸治}

\takahashi[50]{そんな\\こんなで}
\takahashi[120]{次}

\section{今後の予定}
\begin{frame}[fragile]
\frametitle{今後の予定}

\begin{block}{第103回関西Debian勉強会}
  \begin{itemize}
  \item 日時: 11月7日(土)
  \item 場所: 関西オープンソース2015
  \end{itemize}
\end{block}

\begin{block}{第131回東京エリアDebian勉強会}
  \begin{itemize}
  \item 日時: 10月17日(土)?
  \item 場所: 未定
  \end{itemize}
\end{block}

\end{frame}

\takahashi[50]{  }

\end{document}
%%% Local Variables:
%%% mode: japanese-latex
%%% TeX-master: t
%%% End:
