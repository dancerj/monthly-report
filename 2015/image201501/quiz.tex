%; whizzy-master ../debianmeetingresume201311.tex
% 以上の設定をしているため、このファイルで M-x whizzytex すると、whizzytexが利用できます。
%

\santaku
{12/29に、Debianのサービスと連携できるchromium向けの拡張機能がリリースされました。何と連携できるのでしょうか?}
{tracker.debian.org}
{sources.debian.net}
{www.debian.or.jp}
{B}
{zackさんのDebconf14のプレゼンを見てインスパイアされたとのことで、Raphael Geissertさんが作ったchromium用(chromeでも動く)のオンラインソースコードエディタを実現する拡張機能です。Debianパッケージのソースコードを直接オンラインで閲覧できるsources.debian.net上のコードをこの拡張機能を使ってシームレスに編集可能になります。zackさんの示した深刻な問題(第117回東京エリアDebian勉強会資料参照)を解決する為の画期的なアイデアの1つです!マヂすばらしい!自由ソフトウェアラブ、Debianラブな人は、早速、chromiumへNautilusからこの拡張機能を投げ込んで試すべし!}

\santaku
{2014/11/11にRapha\"el Hertzogさんにより提案されたDEP14の内容は以下のどれ?}
{gitパッケージングリポジトリの使い方について}
{controlファイルにDebian-Homepageフィールドを入れる件}
{debian/upstream/metadataを用意する件}
{A}
{パッケージングリポジトリのブランチとタグにベンダー情報(例:debian/*,ubuntu/*)やらバージョン情報の入れ方などを含む、諸々のルールの提案。詳しくは、http://dep.debian.net/deps/dep14/参照。なお、Debian-Homepageフィールドの件はDEP13。debian/upstream/metadataの件はDEP12。}

\santaku
{2015/1/8 Marvel社よりDebian ARM用パッケージビルドマシンの寄付がありました。さて何台?}
{7台}
{8台}
{10台}
{B}
{Marvel社が提供したサーバは MV78460 SoCベースのボード8つを提供とのこと。なお、MV78460は、ARM v7コア4基を搭載するMarvel社開発のSoC。参考:http://www.marvell.com/embedded-processors/armada-xp/}

\santaku
{2015/1/10にwheezyがアップデートされました。さてバージョンは次のどれ。}
{7.6}
{7.7}
{7.8}
{C}
{wheezyをお使いの方々は早速アップグレードしましょう。}

\santaku
{2015/1/1にある長さ未満のDebian keyringの消去が完了したそうです。ある長さとは?}
{2048}
{4096}
{8192}
{A}
{アナウンスと、消された該当のキー一覧はdebian-devel-announceの2015/1/1の記事に記載があります。https://lists.debian.org/debian-devel-announce/2015/01/msg00000.html}

\santaku
{2014/12/6にcdn.debian.netのレコードの示す先を、とあるFQDNのCNAMEとしたい旨の問い合わせがdebian-devel MLにありました。どこに向ける?}
{ftp.debian.or.jp}
{ftp.jp.debian.org}
{http.debian.net}
{C}
{cdn.debian.netを提供しているArakiさんからの問い合わせ。squeeze以上で搭載されているapt ver 0.7.21ではhttpのredirectに対応しているので、利用者に最も近いパッケージダウンロード先を教えてくれる機構をhttp.debian.netに一本化したいとのこと。特に反対する人は居ない模様です。}

\santaku
{2015/1/9にdebian-devel MLにて、一部パッケージのdescriptionが長すぎる事の議論がありました。最も長いdescriptionの行を持つパッケージは?}
{firmware-linux-nonfree}
{texlive-latex-extra}
{irssi-scripts}
{B}
{2015/1/9現在、unstable/mainでは、最も長い行数の順でいくと、1位 texlive-latex-extra(実に1935行!!)、2位 texlive-fonts-extra(437行)、3位 irssi-scripts(350行)。unstable/non-freeでは、1位 firmware-linux-nonfree(251行)です。}


