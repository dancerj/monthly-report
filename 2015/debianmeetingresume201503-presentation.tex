%; whizzy paragraph -pdf xpdf -latex ./whizzypdfptex.sh
%; whizzy-paragraph "^\\\\begin{frame}\\|\\\\emtext"
% latex beamer presentation.
% platex, latex-beamer でコンパイルすることを想定。 

%     Tokyo Debian Meeting resources
%     Copyright (C) 2012 Junichi Uekawa

%     This program is free software; you can redistribute it and/or modify
%     it under the terms of the GNU General Public License as published by
%     the Free Software Foundation; either version 2 of the License, or
%     (at your option) any later version.

%     This program is distributed in the hope that it will be useful,
%     but WITHOUT ANY WARRANTY; without even the implied warreanty of
%     MERCHANTABILITY or FITNESS FOR A PARTICULAR PURPOSE.  See the
%     GNU General Public License for more details.

%     You should have received a copy of the GNU General Public License
%     along with this program; if not, write to the Free Software
%     Foundation, Inc., 51 Franklin St, Fifth Floor, Boston, MA  02110-1301 USA

\documentclass[cjk,dvipdfmx,12pt]{beamer}
\usetheme{Tokyo}
\usepackage{monthlypresentation}

%  preview (shell-command (concat "evince " (replace-regexp-in-string "tex$" "pdf"(buffer-file-name)) "&")) 
%  presentation (shell-command (concat "xpdf -fullscreen " (replace-regexp-in-string "tex$" "pdf"(buffer-file-name)) "&"))
%  presentation (shell-command (concat "evince " (replace-regexp-in-string "tex$" "pdf"(buffer-file-name)) "&"))

%http://www.naney.org/diki/dk/hyperref.html
%日本語EUC系環境の時
\AtBeginDvi{\special{pdf:tounicode EUC-UCS2}}
%シフトJIS系環境の時
%\AtBeginDvi{\special{pdf:tounicode 90ms-RKSJ-UCS2}}

\newenvironment{commandlinesmall}%
{\VerbatimEnvironment
  \begin{Sbox}\begin{minipage}{1.0\hsize}\begin{fontsize}{8}{8} \begin{BVerbatim}}%
{\end{BVerbatim}\end{fontsize}\end{minipage}\end{Sbox}
  \setlength{\fboxsep}{8pt}
% start on a new paragraph

\vspace{6pt}% skip before
\fcolorbox{dancerdarkblue}{dancerlightblue}{\TheSbox}

\vspace{6pt}% skip after
}
%end of commandlinesmall

\title{東京エリアDebian勉強会}
\subtitle{第124回 2015年3月度}
\author{野島貴英}
\date{2015年3月7日}
\logo{\includegraphics[width=8cm]{image200607/openlogo-light.eps}}

\begin{document}

\begin{frame}
\titlepage{}
\end{frame}

\begin{frame}{設営準備にご協力ください。}
会場設営よろしくおねがいします。
\end{frame}

\begin{frame}{Agenda}
 \begin{minipage}[t]{0.45\hsize}
  \begin{itemize}
   \item 注意事項
	 \begin{itemize}
	  \item 写真はセミナールーム内のみ可です。
          \item 出入りは自由でないので、もし外出したい方は、野島まで一声くださいませ。
	 \end{itemize}
   \item 事前課題発表
  \end{itemize}
 \end{minipage} 
 \begin{minipage}[t]{0.45\hsize}
  \begin{itemize}
   \item 最近あったDebian関連のイベント報告
	 \begin{itemize}
	 \item 第123回 東京エリアDebian勉強会
         \item OSC 2015 Tokyo/spring出展
	 \end{itemize}
   \item Debian Trivia Quiz
   \item Raspberry Pi 2 Model B に Debian Jessie / armhf をインストールする
   \item 今後のイベント
   \item 今日の宴会場所
  \end{itemize}
 \end{minipage}
\end{frame}

\section{事前課題}
\emtext{事前課題}
{\footnotesize
\begin{prework}{ $BLnEg(B }
  \begin{enumerate}
  \item Q.hack time$B$K2?$r$7$^$9$+!)(B\\
    A. DDTSS$B$7$^$C$9(B!\\
    http://ddtp.debian.net/ddtss/index.cgi/ja
  \item ($B%*%W%7%g%s(B)Q.$B2?$K$D$$$FJ9$-$?$$!?;22C<T$HOC$r$7$?$$$G$9$+!)(B\\
    A. $B;22C<T$H5;=Q$NOC$r$7$?$$!*(B
  \end{enumerate}
\end{prework}

\begin{prework}{ nametake }
  \begin{enumerate}
  \item Q.hack time$B$K2?$r$7$^$9$+!)(B\\
    A. Debian$B3+H/4D6-%;%C%H%"%C%W(B
  \item ($B%*%W%7%g%s(B)Q.$B$I$3$G:#2s$NJY6/2q$N3+:E$rCN$j$^$7$?$+!)(B\\
    A. $B$=$NB>!#(B
  \end{enumerate}
\end{prework}

\begin{prework}{ NOKUBI Takatsugu }
  \begin{enumerate}
  \item Q.hack time$B$K2?$r$7$^$9$+!)(B\\
    A. OpenCV$B!"(BKAKASHI$B!"(BBlender
  \item ($B%*%W%7%g%s(B)Q.$B$I$3$G:#2s$NJY6/2q$N3+:E$rCN$j$^$7$?$+!)(B\\
    A. $B$=$NB>(B
  \item ($B%*%W%7%g%s(B)Q.$B2?$K$D$$$FJ9$-$?$$!?;22C<T$HOC$r$7$?$$$G$9$+!)(B\\
    A. key sign
  \end{enumerate}
\end{prework}

\begin{prework}{ alohaug }
  \begin{enumerate}
  \item Q.hack time$B$K2?$r$7$^$9$+!)(B\\
    A. $B%/%j!<%s%k!<%`4D6-$G(BPGP$B80:n@.!"(BGnuk$BIuF~(B \& dictoss$B$5$s80=pL>(B
  \item ($B%*%W%7%g%s(B)Q.$B$I$3$G:#2s$NJY6/2q$N3+:E$rCN$j$^$7$?$+!)(B\\
    A. twitter (@tokyodebian)
  \item ($B%*%W%7%g%s(B)Q.$B2?$K$D$$$FJ9$-$?$$!?;22C<T$HOC$r$7$?$$$G$9$+!)(B\\
    A. PGP$B804IM}$N$"$k$"$k%M%?<}=8!#(B
  \end{enumerate}
\end{prework}

\begin{prework}{ Roger Shimizu }
  \begin{enumerate}
  \item Q.hack time$B$K2?$r$7$^$9$+!)(B\\
    A. $BL$Dj(B
  \item ($B%*%W%7%g%s(B)Q.$B$I$3$G:#2s$NJY6/2q$N3+:E$rCN$j$^$7$?$+!)(B\\
    A. $B$=$NB>(B
  \item ($B%*%W%7%g%s(B)Q.$B2?$K$D$$$FJ9$-$?$$!?;22C<T$HOC$r$7$?$$$G$9$+!)(B\\
    A. GPG$B%-!<%5%$%s$,9T$o$l$k$G$7$g$&$+!#(B
  \end{enumerate}
\end{prework}

\begin{prework}{ ryo\_s }
  \begin{enumerate}
  \item Q.hack time$B$K2?$r$7$^$9$+!)(B\\
    A. $B=q@RFI$_(B
  \item ($B%*%W%7%g%s(B)Q.$B$I$3$G:#2s$NJY6/2q$N3+:E$rCN$j$^$7$?$+!)(B\\
    A. $BM'C#$dCN$j9g$$$+$iD>@\(B
  \end{enumerate}
\end{prework}

\begin{prework}{ dictoss }
  \begin{enumerate}
  \item Q.hack time$B$K2?$r$7$^$9$+!)(B\\
    A. rasbbery pi 2$B$K(BDebian$B$r%$%s%9%H!<%k$9$k2<D4$Y!#(B
  \item ($B%*%W%7%g%s(B)Q.$B$I$3$G:#2s$NJY6/2q$N3+:E$rCN$j$^$7$?$+!)(B\\
    A. Debian JP$B$N%a!<%j%s%0%j%9%H(B
  \item ($B%*%W%7%g%s(B)Q.$B2?$K$D$$$FJ9$-$?$$!?;22C<T$HOC$r$7$?$$$G$9$+!)(B\\
    A. CPU$B$N%]!<%F%#%s%0$O2?$rCN$C$F$$$kI,MW$,$"$k$N$+65$($F2<$5$$!#(B
  \end{enumerate}
\end{prework}

\begin{prework}{ myokoym }
  \begin{enumerate}
  \item Q.hack time$B$K2?$r$7$^$9$+!)(B\\
    A. deb$B%Q%1!<%8$N:n@.<j=g$r0l$+$i3X$S$?$$$H;W$$$^$9!#(B
  \item ($B%*%W%7%g%s(B)Q.$B$I$3$G:#2s$NJY6/2q$N3+:E$rCN$j$^$7$?$+!)(B\\
    A. Debian JP$B$N%a!<%j%s%0%j%9%H(B
  \item ($B%*%W%7%g%s(B)Q.$B2?$K$D$$$FJ9$-$?$$!?;22C<T$HOC$r$7$?$$$G$9$+!)(B\\
    A. deb$B%Q%C%1!<%8$N:n@.$d8x3+$K$D$$$F(B
  \end{enumerate}
\end{prework}

\begin{prework}{ yy\_y\_ja\_jp }
  \begin{enumerate}
  \item Q.hack time$B$K2?$r$7$^$9$+!)(B\\
    A. DDTSS 
  \item ($B%*%W%7%g%s(B)Q.$B$I$3$G:#2s$NJY6/2q$N3+:E$rCN$j$^$7$?$+!)(B\\
    A. $B$=$NB>(B
  \item ($B%*%W%7%g%s(B)Q.$B2?$K$D$$$FJ9$-$?$$!?;22C<T$HOC$r$7$?$$$G$9$+!)(B\\
    A. DDTSS$B$N%l%S%e!<$N$*4j$$(B
  \end{enumerate}
\end{prework}



}

\section{イベント報告}
\emtext{イベント報告}

\begin{frame}{第123回東京エリアDebian勉強会}

\begin{itemize}
\item 場所はスクウェア・エニックスさんのセミナルームをお借りしての開催でした。
\item 参加者は9名でした。
\item セミナ内容は杉本さんによる「Debian GNU/kFreeBSDにおけるJail構築を試してみた」でした。
\item LTは、今井さんによる「Gnukと私」でした。
\item 残りの時間でhack timeを行い、成果発表をしました。
\item 「世界のやまちゃん 新宿花園店」で、久々に宴会をやりました。
\end{itemize} 
  
\end{frame}

\begin{frame}{第123回東京エリアDebian勉強会(つづき)}

  今回セミナ発表は、Debian GNU/kFreeBSD上にJail環境でkFreeBSDや、FreeBSDを稼働させるお話でした。また、linux-i386のシステムを、FreeBSDのLinuxバイナリ互換機能で動作させる試みについても発表がありました。

  LTは、発表者の今井さんがGnuk Tokenを使うに至った動機と、Gnuk Tokenの現状に関する発表でした。Gnuk Token現役利用者の発表は珍しいため非常に興味深い発表でした。
  
\end{frame}

\begin{frame}{OSC 2015 Tokyo/spring出展}

  毎年のことですが、今年も2/28(土)にて、OSC 2015 Tokyo/springが開かれました。

  場所はいつもの明星大学 日野キャンパスでした。

  セミナは、岩松さんにより「Debian Updates」が行われました。20名程度の方に来ていただけたとのことでした。
  
\end{frame}

\begin{frame}{OSC 2015 Tokyo/spring出展(つづき)}

  今回の出展は、
  
  \begin{itemize}
  \item 店員が2名が立ってピッタリなサイズのブースとなりました。(いつもより狭い)
  \item 64bit ARMのボードである、Hikey Board(\url{https://www.96boards.org/products/hikey/})がDebian稼働状態で展示されました。
 \item Gnuk Tokenの展示も行いました。
  \end{itemize}
  
\end{frame}

\section{Debian Trivia Quiz}
\emtext{Debian Trivia Quiz}
\begin{frame}{Debian Trivia Quiz}

  Debian の常識、もちろん知ってますよね?
知らないなんて恥ずかしくて、知らないとは言えないあんなことやこんなこと、
みんなで確認してみましょう。

今回の出題範囲は\url{debian-devel-announce@lists.debian.org},
\url{debian-news@lists.debian.org} に投稿された
内容などからです。

\end{frame}

\subsection{問題}

%; whizzy-master ../debianmeetingresume201311.tex
% 以上の設定をしているため、このファイルで M-x whizzytex すると、whizzytexが利用できます。
%

\santaku
{2015/3/3にDebConf15のアナウンスが行われました。スポンサードな参加の登録期限は何時まででしょう?}
{2015/3/7}
{2015/3/15}
{2015/3/29}
{C}
{宿泊代と食事がDebConf開催側持ちとなるスポンサードな参加登録の〆切は3/29です!〆切の時刻はUTCなのか、ドイツのローカルタイムか、細かい事がわからないので、参加を検討されている方は〆切に対して十分に日程の余裕をもって登録される事をおすすめします。DebConf15は、ドイツ ハイデルバーグで2015/8/15-22の開催です。DebCampは、8/9-8/14となります。}

\santaku
{2015/3/4にDPLの今年の選出についてアナウンスがありました。DPLの立候補〆切は何時?}
{2015/3/4}
{2015/3/9}
{2015/4/1}
{B}
{毎年恒例のDPL選挙です。DPLの立候補〆切は3/9で、選挙は4/1から行われます。今年は誰が立候補するのでしょうか?また、同時に、Debian JP Projectについても、2015年の会長の立候補者募集が行われています。}

\santaku
{2015/3/1にて、cdn.debian.netがその役割を終え、あるFQDNを指すだけになるということがアナウンスされました。このFQDNは以下のどれ?}
{ftp.debian.org}
{http.debian.net}
{sources.debain.net}
{B}
{cdn.debian.netは、aptによるDebianパッケージの取得先について、DNSクエリの発行された場所に基づいて、ユーザに最も近いパッケージリポジトリのサーバのIPアドレスをDNSのレコードで返却する仕組みです。Arakiさんによって2007年頃にて設計、構築、運用されておりました。2010年にaptがHTTP リダイレクトをサポートできるようになったため、cdn.debian.netの機能をHTTPのリダイレクト機能で実現するhttp.debian.netが稼働を開始しました。今回、遂に、cdn.debian.netを引退させるアナウンスが行われたという状況です。もし、source.listにcdn.debian.netのFQDNを指定している場合は、http.debian.netに変えましょう。}




\section{Raspberry Pi 2 Model B に Debian Jessie / armhf をインストールする}
\emtext{Raspberry Pi 2 Model B に Debian Jessie / armhf をインストールする}

\section{今後のイベント}
\emtext{今後のイベント}
\begin{frame}{今後のイベント}
\begin{itemize}
 \item 関西エリアDebian勉強会
 \item 東京エリアDebian勉強会 
\end{itemize}
\end{frame}

\section{今日の宴会場所}
\emtext{今日の宴会場所}
\begin{frame}{今日の宴会場所}
未定
\end{frame}

\end{document}

;;; Local Variables: ***
;;; outline-regexp: "\\([ 	]*\\\\\\(documentstyle\\|documentclass\\|emtext\\|section\\|begin{frame}\\)\\*?[ 	]*[[{]\\|[]+\\)" ***
;;; End: ***
