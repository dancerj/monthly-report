\documentclass[cjk,dvipdfmx,10pt,compress,%
hyperref={bookmarks=true,bookmarksnumbered=true,bookmarksopen=false,%
colorlinks=false,%
pdftitle={第 101 回 関西 Debian 勉強会},%
pdfauthor={倉敷・のがた・佐々木・かわだ},%
%pdfinstitute={関西 Debian 勉強会},%
pdfsubject={資料},%
}]{beamer}

\title{第 101 回 関西 Debian 勉強会}
\subtitle{$\sim$発表資料$\sim$}
\author[かわだ てつたろう]{{\large\bf 倉敷・のがた・佐々木・かわだ}}
\institute[Debian JP]{{\normalsize\tt 関西 Debian 勉強会}}
\date{{\small 2015 年 8 月 23 日}}

%\usepackage{amsmath}
%\usepackage{amssymb}
\usepackage{graphicx}
\usepackage{moreverb}
\usepackage[varg]{txfonts}
\AtBeginDvi{\special{pdf:tounicode EUC-UCS2}}
\usetheme{Kyoto}
\def\museincludegraphics{%
  \begingroup
  \catcode`\|=0
  \catcode`\\=12
  \catcode`\#=12
  \includegraphics[width=0.9\textwidth]}
%\renewcommand{\familydefault}{\sfdefault}
%\renewcommand{\kanjifamilydefault}{\sfdefault}
\begin{document}
\settitleslide
\begin{frame}
\titlepage
\end{frame}
\setdefaultslide

\begin{frame}[fragile]
  \frametitle{Disclaimer}
  \begin{itemize}
  \item 疑問、質問、ツッコミ、茶々、\alert{大歓迎}
  \item その場でインタラクティブにどうぞ
  \item ハッシュタグ \#kansaidebian
  \end{itemize}
\end{frame}

\begin{frame}[fragile]
\frametitle{Agenda}

\tableofcontents

\end{frame}

\section{最近の Debian 関係のイベント}

\takahashi[40]{最近の Debian\\関係のイベント}

\begin{frame}[fragile]
  \frametitle{第100回関西Debian勉強会 OSC Kansai@Kyoto 2015}
  \begin{itemize}
  \item 日時: 8月8日(土)
  \item 場所: 京都リサーチパーク
  \end{itemize}
  \begin{block}{内容}
    \begin{itemize}
    \item 「Debian Updates (Jessie, Stretch, Buster)」
    \item ブース展示
    \end{itemize}
  \end{block}
\end{frame}

\begin{frame}[fragile]
  \frametitle{第129回東京エリアDebian勉強会}
  \begin{itemize}
  \item 日時: 9月22日(土)
  \item 場所: 株式会社スクウェア・エニックス セミナールーム
  \end{itemize}
  \begin{block}{内容}
    \begin{itemize}
    \item 「APT1.1 超☆牛さんパワー炸裂!」
    \end{itemize}
  \end{block}
\end{frame}

\begin{frame}[fragile]
  \frametitle{Debian Project}
  \begin{itemize}
  \item Debian turns 22!
  \item debconf15
  \item Bits from the Release Team: GCC 5 as default, transitions thereof
  \item Sparc removal
  \item Debian Installer Stretch Alpha 2 release
  \item The Debian Administrator's Handbook
  \end{itemize}
\end{frame}

\takahashi[50]{そんな\\こんなで}
\takahashi[120]{次}

\section{事前課題発表}

\takahashi[50]{事前課題}

\begin{frame}[fragile]
  \frametitle{事前課題}
  今回の事前課題はありませんでした。
\end{frame}

\takahashi[50]{事前課題\\発表}

\begin{frame}
  \frametitle{ Yamada Yohei (山田 洋平) }
  \begin{enumerate}
  \item ノートPCが死亡
  \item マウスカーソルを変えてみた
  \end{enumerate}
\end{frame}

\begin{frame}
  \frametitle{ むんくさん }
  \begin{enumerate}
  \item ラップトップの環境構築中
  \item Let's noteのホーイルパッドの設定をした
  \end{enumerate}
\end{frame}

\begin{frame}
  \frametitle{ Yosuke OTSUKI }
  \begin{enumerate}
  \item 東京からAC アダプタを持って帰ってきた
  \item armのクロスコンパイル環境
  \end{enumerate}
\end{frame}

\begin{frame}
  \frametitle{ nogajun }
  \begin{enumerate}
  \item 途中から抜けてきました
  \end{enumerate}
\end{frame}

\begin{frame}
  \frametitle{ t3rkwd }
  \begin{enumerate}
  \item OSC出展
  \item このあと話します
  \end{enumerate}
\end{frame}

\begin{frame}
  \frametitle{ lurdan }
  \begin{enumerate}
  \item 趣味のパッケージをメンテ
  \item WindowsでDebianを
  \end{enumerate}
\end{frame}

\begin{frame}
  \frametitle{ 川江 浩 }
  \begin{enumerate}
  \item kvm上のWindows8.1が10になってくれない
  \end{enumerate}
\end{frame}

\begin{frame}
  \frametitle{ 宮原 健吾 }
  \begin{enumerate}
  \item 今年からDebianを使い始めた
  \item 1985年ぐらいからUnixを使っている
  \item Volumio
  \end{enumerate}
\end{frame}

\takahashi[50]{そんな\\こんなで}
\takahashi[120]{次}

\section{wiki:Subkeys}
\takahashi[30]{wiki:Subkeys\\by\\かわだてつたろう}

\takahashi[50]{そんな\\こんなで}
\takahashi[120]{次}

\section{今後の予定}
\begin{frame}[fragile]
\frametitle{今後の予定}

\begin{block}{第102回関西Debian勉強会}
  \begin{itemize}
  \item 日時: 9月27日(日)
  \item 場所: 福島区民センター 304号
  \end{itemize}
\end{block}

\begin{block}{第130回東京エリアDebian勉強会}
  \begin{itemize}
  \item 日時: 9月12日(土)
  \item 場所: イベント&コミュニティスペースdots.
  \item 内容: 第3回 Debianパッケージング道場
  \end{itemize}
\end{block}

\end{frame}

\takahashi[50]{  }

\end{document}
%%% Local Variables:
%%% mode: japanese-latex
%%% TeX-master: t
%%% End:
