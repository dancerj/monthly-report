%; whizzy-master ../debianmeetingresume201311.tex
% 以上の設定をしているため、このファイルで M-x whizzytex すると、whizzytexが利用できます。
%

\santaku
{Debian Project関係者のPodCastのサイトが公開されました。以下のどれ?}
{www.debian.org}
{www.debianandstuff.com}
{www.debian.or.jp}
{B}
{英語です。記念すべき第1回目は「MoinMoin Vs. MediaWiki」であり、パーソナリティーはAsheesh Laroiaさんと、Sam Erbsさんとなります。収録はDebConf14中に収録したそうです。}

\santaku
{2014/10/27のDPNにAda initiativeから寄付のアナウンスの件が載っていました。Ada initiativeって何?}
{オープンなテクノロジに関して女性活躍の支援をする団体}
{プログラミング言語Adaの普及促進をする団体}
{Adaさんの政治後援会}
{A}
{オープンなテクノロジに関して女性活躍の支援が活発になってきました。Debianでは、Debian Womenというプロジェクトがあります。Gnome Foundationが2006年にFree \& Open Source Software Outreach Program(略してOPW)を開始したのがきっかけで、Debian Projectもこちらの動きに賛同している状況です。}

\santaku
{2014/10/27にDebianのwhoisコマンドが入れ替わりました。特徴はどれ?}
{サイズが小さくなった}
{DFSGに準拠した}
{作者独自の調査によりIANAの情報より正確になった}
{C}
{今までBSD由来のwhoisコマンドがDebianで使われてきましたが、この度Marcoさん作のwhoisコマンドに変わりました。Marcoさんは1年を費やして、IANAよりも正確に情報を得られるサーバ群を独自の調査で探し当て、こちらで対応するようにしたそうです。なお、Marcoさんのwhoisコマンドは全Linuxディストリビューションの標準のwhoisコマンドになるそうです。}

\santaku
{2014/10/15時点で、Freexianと契約したDebianのLTSのスポンサーは全部で何社?}
{14社}
{13社}
{12社}
{A}
{\url{http://raphaelhertzog.com/2014/10/15/freexians-second-report-about-debian-long-term-support/}に掲載されています。こちらのスポンサーのお陰で、LTSを担当できるDebian開発者らのフルタイムのうち、25\%の時間を割く事ができるようになったとのことです。古いDebianを使っている会社さんは是非スポンサーになってくださいませ。}

\santaku
{2014/10/27のDPNにてDebian Multimediaの進捗状況報告がありました。libav6:11で搭載された新しい機能は次のうちどれ。}
{libx265-encoder}
{libx265-decoder}
{libx264-encoder}
{A}
{libav6:11はjessie搭載予定のMultimedia用codecライブラリです。遂にx265のエンコーダが搭載されたようです。x265は、ワンセグなどで使われている高性能なコーデックのH.264/MPEG-4 AVCの後継であるH.265の互換実装となります。H.265はH.264の2倍の圧縮率を誇るとのことで、Jessieでの動画鑑賞が楽しみですね。}

\santaku
{2014/11/5にてFreezeが行われました。この時残っているRC bugは何個だったでしょう?}
{200個}
{310個}
{400個}
{B}
{Freeze時に310個しかRC bugがなかったのは、昨今のFreezeではなかったほどの快挙だそうです。さあ、RC bug潰しまくりましょう!}

\santaku
{2014/10/27にて初めてJessieベースのDebianEduがリリースされました。DebianEduはDebianの用語ではどのしくみに分類されるでしょうか?}
{Derivative}
{Blend}
{PureBlend}
{C}
{PureBlendは、特定用途向けのDebianに仕上がるようにインストールを行う場合、controlファイルに必要なパッケージをRecommendsで指定しただけのパッケージを用意することによって、簡単にDebianパッケージ群のみをまとめてインストールできるようにするためのしくみです。詳しくは、「第108回東京エリアDebian勉強会、2014年1月勉強会」(http://tokyodebian.alioth.debian.org/2014-01.html)の資料に詳しいです。}

\santaku
{Jessieから取り除かれる予定のQtのバージョンはいくつでしょう?}
{Qt3}
{Qt4}
{Qt5}
{B}
{Qt4はupstream側で2015年に開発を終了する決定となりました。Jessieに搭載予定のQtのバージョンはQt5となります。}

\santaku
{mainパッケージのDependsフィールドに''package-in-main | packages-non-free''と書いて良いかどうかの決定が2014/10/31にTechnicalCommiteeにより下されました。結論は以下のうちのどれ?}
{状況次第でOKだったり、NGだったり}
{NG}
{OK}
{C}
{例えば、''Depends: unrar-free | rar''というようなパターンがありえます。通常は、mainパッケージで構成されるDebianシステムはDFSG準拠であるべきなので、「non-freeパッケージのみ」に依存するようなパッケージをmain側に作ってはいけないというルールがあります。今回の場合は、mainパッケージのリポジトリ指定時に、non-freeのパッケージが優先して導入されることは無いのでOKとなりました。}

\santaku
{2014/11/9のRelease TeamからDebian 9,Debian 10のコードネームが決まりました。Debian 10のコードネームは次のうちのどれ?}
{Buster}
{Stretch}
{Jessie}
{A}
{ちなみにDebian 9は''Strech''だそうです。}

\santaku
{2014/11/9のRelease Teamのメールにて、arm64, ppc64el, kfreebsdについて、Jessieの公式リリースに含むかどうかの決断が行われました。「含まない」とされたのは次のうちどれ?}
{arm64}
{ppc64el}
{kfreebsd}
{C}
{大変遺憾ながら、kfreebsdは、期日までにJessie公式リリースに必要とされる品質に達しなかったとの事です。ただ、kfreebsdがDebianプロジェクト自体から消えるわけではないので、Jessie公式リリース時期に、条件が揃えば非公式のリリースとして扱う事が可能とのことです。このパターンになったのは、Debian GNU/Hurd 2013が2013年にそのようなリリースを行ったことがあります。}

\santaku
{2014/11/14にて、Debian Medチームから、とあるパッケージをやっとDFSG準拠にすることが出来たとの報告がありました。そのパッケージ名は以下のどれ?}
{abyss}
{arb}
{phylip}
{C}
{upstreamはワシントン大学であるphylipは、bioinfomaticsでは有名なソフトウェアとのことです。しかしながら、ワシントン大学の課しているライセンスは、非営利の研究用途のみに利用を許諾している状態でした。こちらについて長年のDebian Medの交渉により、遂にphylipはDFSG準拠の自由ライセンスにしてもらえたとのことです。また、phylipが原因でnon-freeにせざるを得なかったseaviewというパッケージも無事mainにすることが出来るとのことでした。}
