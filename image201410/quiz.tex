%; whizzy-master ../debianmeetingresume201311.tex
% 以上の設定をしているため、このファイルで M-x whizzytex すると、whizzytexが利用できます。
%

\santaku
{2014/9/14時点でJessieでサポートされると残念ながら「言われなかった」アーキテクチャはどれ}
{amd64}
{powerpc}
{sparc}
{C}
{2014/9/14時点でサポートされると言われたアーキテクチャは:amd64,armel,armhf,i386,kfreebsd-amd64/kfreebsd-i386/mips/mipsel/powerpc/s390x。なお、arm64/ppc64elは好調な頑張りとのことで、この調子が続けば入るかも??という状況。}

\santaku
{2014/11/1時点でJessieでサポートされるかどうかについて依然として懸念のあるものはどれ}
{hurd}
{kFreeBSD}
{s390x}
{B}
{kFreeBSDは頑張って欲しいとのこと。最終ジャッジは2014/11/1に行われる。なお、s390xは9/14では懸念なし、hurdはすでにリリースは無理との判断になっている。}

\santaku
{2014/9/4にtesting入りしたデスクトップ環境は以下のどれ}
{xfce}
{cinnamon}
{unity}
{B}
{cinnamonはGTK+3を利用して作られたデスクトップ環境です。元々はLinux MINT向けにGNOME Shellからforkしたデスクトップシェルでした。開発が続き、デスクトップシェルから発展して遂にデスクトップ環境となりました。}

\santaku
{2014/10/18にリリースされたwheezyのDebianのバージョンはいくら?}
{7.7}
{7.8}
{7.9}
{A}
{7回目のアップデートとなります。早速アップグレードしましょう!新規にインストールする安定版ならDebian7.7からやりましょう!}

\santaku
{OPWの呼びかけがdebian-devel-announceで2014/9/30に行われました。ところでOPWって何の略?}
{One-time PassWord}
{One-Piece-Woven technology}
{Outreach-Program-for-Women}
{C}
{Outreach-Program-for-Women(略してOPW)は、GNOME Foundationが始めた、FOSSのプロジェクトの貢献者にもっと女性を増やそう!という活動となります。FOSSの貢献者の男女構成比は、圧倒的に男の割合が高いといういびつな傾向があるため、こちらを是正しようとする活動です。}

\santaku
{2014/10/14にサービスを近いうちに閉じるよとアナウンスのあったサイトはどれでしょう?}
{githubredir.debian.net}
{tracker.debian.org}
{rtc.debian.org}
{A}
{githubredir.debian.netは、github上にあるupstreamのソースについて、リリースの為にバージョン番号でtag打ちされたバージョンのソースのtarボールに対する直接のリンクを統一的な書式のURLで生成するサービスです。つまり、debian/watchにupstreamのソースの有りかを書きやすくするために使われます。githubが改善され今となっては不要となってしまったということもあり、近いうちに廃止するというアナウンスが行われました。}

\santaku
{2014/10/5にリリースされたDebian Installer Jessie Beta2の変更点はどれ}
{syslinuxまわりで過去の互換の無い大きな変更点が出た。}
{デフォルトのinitがsystemdとなった。}
{GNOMEデスクトップ環境がデフォルトになった。}
{C}
{一度はXfceデスクトップ環境と言われていましたが、accessibilityの完成度の高さには勝てず、再びGNOMEに戻ってきた形となりました。なお、他の選択肢はBeta1の時の変更点となります。}

\santaku
{2014/10/16に投票が開始されました。内容は次のうちどれ?}
{特定のinitとプログラムが依存しても良いか?ダメか?程度次第か?}
{GNOMEをデフォルトにしてよいのか?}
{野島が勉強会幹事をやり続けて良いのか?}
{A}
{Debianの各種initシステムと他アプリケーションの依存についての規定に関する投票となります。投票内容について詳しくはhttps://www.debian.org/vote/2014/vote\_003。ところで、Debianのイベントの幹事は、もちろん、いつ、誰が、どのようにやってもよくてよ?勇者の数を増やせ!}









