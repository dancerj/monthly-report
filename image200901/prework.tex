%; whizzy-master ../debianmeetingresume200901.tex
% 以上の設定をしているため、このファイルで M-x whizzytex すると、whizzytexが利用できます。

\begin{prework}{上川純一}
\preworksection{Debianに関しての、2009年の抱負}

\begin{itemize}
 \item 使っているパッケージをメンテナンス、使ってないパッケージを捨てる
 \item 今止まっている pbuilder / cowdancer / qemubuilder のリグレッションテストサーバを
       復活させる
\end{itemize}

\preworksection{2012のDebian}

この時計なんだけどさ、タッチスクリーンとプロジェクタがついていて、それで
動くんだぜ。
おっと、あまり頭をうごかさないでくれ、こいつが敵として認識して攻撃してし
 まう。

\end{prework}

\begin{prework}{山本 浩之}
\preworksection{Debianに関しての、2009年の抱負}

メンテナンスしているパッケージを地味に増やしていく。

\preworksection{2012のDebian}

順調に squeeze のリリースも遅れてますな。

\end{prework}


\begin{prework}{岩松 信洋}
\preworksection{Debianに関しての、2009年の抱負}
\begin{itemize}
\item DD になる。
\item 他のサブプロジェクトにも積極的に参加する。
\item SH の開発を継続する。
\end{itemize}

\preworksection{2012のDebian}

まだ目で操作するのは慣れないな。メガネで操作するのはつらいぜ。

\end{prework}

\begin{prework}{小林 儀匡}
\preworksection{Debianに関しての、2009年の抱負}
\begin{itemize}
\item DD になる。
\item upstreamへのパッチの積極的な投稿を続ける。
\end{itemize}

\preworksection{2012のDebian}

おまえみたいに久し振りに会う奴の名前や好きな話題がとっさに出てこないから、
その辺りの情報を瞬時にこの超小型イヤホンに送ってもらって\footnote{元ネタ: 星新一『ささやき』。}……いや何でもない。
いたって普通に、生活で使うデバイス全般で動かしているよ。

それよりのび太、おまえ、噂によれば最近、
Debianの乗った、猫耳のついたロボットを開発しているそうじゃないか。

\end{prework}

\begin{prework}{キタハラ}
\preworksection{Debianに関しての、2009年の抱負}

会社で消滅した debian で動作しているサーバを復活させる。

\preworksection{2012のDebian}

相変わらず調子はいいよ。でも、これ Android だよ。

(画面は自宅サーバのリモートコンソールだったのだ! ちゃんちゃん!!)

\end{prework}

\begin{prework}{まえだこうへい}
\preworksection{Debianに関しての、2009年の抱負}

\begin{itemize}
\item ヨメを連れてスペインに行く。(行きたい。)
\item スペイン前までにヨメを洗脳する。(わら
\end{itemize}

\preworksection{2012年のDebian}
時計なんだけどさ、電波時計のボタンと見せかけて、半径 10m 以内に Debian 信者がいるか分かるんだぜ。
\end{prework}

 
\begin{prework}{あけど}
\preworksection{Debianに関しての、2009年の抱負}

\begin{itemize}
 \item (挫折中の)翻訳の査読を進める
 \item 管理しているDebianな公開サーバがあるのですが、
       これを仮想サーバにサービス無停止で移行しようかと思っています。
       また、この移行に関する資料を作成・公開したいと思います。       

\end{itemize}

\preworksection{2012年のDebian}

これ、ぱっと見ると電子辞書なんだけどこうすると動画とか見れるんだな、
実は携帯端末で自宅サーバとVPN接続しててこれだけで大抵の仕事ができたりす
 るんだ。
そんでこの端末とか自宅のルータ&サーバは全部Debianなのよ

\end{prework}

\begin{prework}{じつかた}

\preworksection{2009年のDebian抱負}

\begin{itemize}
 \item パッケージ作成の勉強を進める
 \item Debianサーバ導入(仕事)
\end{itemize}

\preworksection{2012年のDebian}

 「この携帯、Debianで動いてるんだよ。」

\end{prework}


% この上の部分に以下の内容を挿入する。
% \begin{prework}{名前}
% \preworksection{XXX}
% \preworksection{YYY}
% \end{prework}
%
