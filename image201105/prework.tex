%; whizzy-master ../debianmeetingresume201101.tex
% 以上の設定をしているため、このファイルで M-x whizzytex すると、whizzytexが利用できます。



\begin{prework}{ キタハラ }

Debian限定だと思いつかない・・・。
(お題の意図を読み違えているのかも)
apt-getをhttpで実行するとウェブサービスと言える?
\end{prework}

\begin{prework}{ MATOHARA }

Debian使いとしてウェブサービスに期待すること.
最近は少なくなりましたが,IE 必須のサービス等の環境依存のサービスをやめ
 て欲しいです.
最近だとSilverlight 必須のサービスでMoonlight で動きそうで動かないといっ
 たことがありました.
\url{http://live6.channel.ne.jp/world_ipv6/}
\end{prework}

\begin{prework}{ taitioooo }

情報に対する課金がなくなること。

\end{prework}

\begin{prework}{ 野島 貴英 }

\begin{itemize}
\item jslinuxという強力なエミュレータも出たので、ブラウザで動くDebian
 experimental環境とかブラウザで動くGnomeのお試し環境とかを提供するウェブ
 サービスとか素敵かも。
こもきっとウェブサービス!(なんか空気読めてない回答な気もするけど...)

\item USBに書き込めばdebian環境がそのままブートできるようなイメージをつくって
 くれるウェブサービスが良さそうな気も...例えば、パッケージ一覧にチェック
 入れて、sidとかにチェック入れると、USBメモリにそのまま書き込めばその仕
 様でdebian sidがブートできるようなカスタムイメージを作ってくれるとか。

\item チェックボックスとセレクタだけで、preceedファイル生成してくれるウェブサー
 ビスもいいかも...大量のインストール時とかよさそう。
(もう言いたい放題ですね...)
\end{itemize}











\end{prework}

\begin{prework}{ 岩松 信洋 }
\begin{itemize}
\item 全世界のWebサーバを提供するOSがDebianになること。
\item 分散コンパイルサーバとか欲しい。
\end{itemize}


\end{prework}

\begin{prework}{ 日比野 啓 }

Webサービスもできれば機械処理しやすいものが良い。
あと、クラウド上でのAPIを提供しているようなサービスに、関数型言語に対す
 るサポートが増えてほしい。

\end{prework}

\begin{prework}{ dictoss(杉本 典充) }

CPUとあるdebパッケージを選択すると、そのCPU向けに最大限の最適化したパッ
 ケージと依存するパッケージを再ビルドしてくれるサービス。
\end{prework}

\begin{prework}{ kazken3 }

翻訳をたまにしているので、ディストリビューション間横どおしでの翻訳関連情
 報を提供するサイトがあればいいなと思うことがあります。

#課題とは少しズレているかも知れませんが、
#個人向けのウェブサービスには食傷気味というところもあるので。


\end{prework}

\begin{prework}{ まえだこうへい }

Debianシステムで作った環境との相互互換性。
例えば、最近GAE/Pythonをよく使うので、作ったシステムを
 GAE/Python <-> →Debianシステムのどちらでも(ほとんど変更なしで)動かせると
 便利ですね。
すぐ始めるのにクラウドサービスを利用して作ったけど将来はDebianで動かした
 い、逆に今は政治的な理由で外に出せないDebianシステムを将来は自分の管理
 から外れるので手離れをよくするためにクラウドサービスに簡単に移行できる、
 など。
\end{prework}

\begin{prework}{ yamamoto }

そうですね。
今の所導入を検討しているのは、パーソナルストレージサービスぐらいですかね。
あらゆる所で自分のデータが自分で共有できれば、それで十分な感じです。
\end{prework}
