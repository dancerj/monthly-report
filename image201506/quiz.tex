%; whizzy-master ../debianmeetingresume201311.tex
% 以上の設定をしているため、このファイルで M-x whizzytex すると、whizzytexが利用できます。
%

\santaku
{Debian 8.1 がリリースされました。いつだったでしょうか?}
{2015/6/6}
{2015/6/13}
{2015/6/20}
{A}
{いくつかの脆弱性対策や、バグフィックスが行われたパッケージが取り込まれました。Debian 8(Jessie)をインストールしたばかりの人は、早速アップグレードしましょう!}

\santaku
{2015/6/10にて、unstable版のソースパッケージの数はいくつになったでしょうか?}
{21,000}
{22,000}
{23,000}
{B}
{遂に22,000を超えたそうです。バイナリパッケージの数は45,542との事。益々増えていくようです。}

\santaku
{AutomaticDebugPackagesの提案とは何?}
{大統一Debianの岩松さんのデバッグパッケージの件を実施する}
{自動でデバッグ出来るようにする}
{-dbgパッケージを止め、.ddebパッケージを作る}
{C}
{今まで、デバッグシンボルは-dbgパッケージで配布されていました。しかしながら、こちらの-dbgパッケージは他のデバッグとは何ら関係のないパッケージと一緒にmirrorされるため、-dbgパッケージの利用者がとても少ないにもかかわらず、mirror先の資源をその分消費してしまいます。今回の提案は、デバッグシンボル.ddebというパッケージにしてしまい、こちらのパッケージについてはmirror先も減らす(しない?)という事を検討するものです。}




