%; whizzy-master ../debianmeetingresume201311.tex
% 以上の設定をしているため、このファイルで M-x whizzytex すると、whizzytexが利用できます。
%

\santaku
{6/22にて、backportのチームが、特定の条件を満たすパッケージをごっそり消したのは、どのbackports?}
{squeeze-backports}
{wheezy-backports}
{jessie-backports}
{B}
{jessieで利用できないパッケージを、wheezy-backportsからごっそり消したとのことです。backportsに含まれるどのパッケージがどうなっているか?どうして欲しいか?については、freezeの期間とfreeze後のわずかな期間の間に、backport担当からbackportsチームに自発的にタイムリーに相談して来て欲しいとの勧告も行われました。}

\santaku
{7/7にて、sidでは、特定バージョンのGCCとlibstdc++でコンパイル・動作出来るようにして欲しい旨のアナウンスが流れました。どの組み合わせ?}
{gcc 6/libstdc++6}
{gcc 4/libstdc++5}
{gcc 5/libstdc++6}
{C}
{まずは、Debian sidでは、gcc 5/libstdc++6でコンパイル・動作出来るようにパッケージメンテナの方は修正対応をして欲しい旨のアナウンスがありました。今回、ABIベースでも変更になったり、C++11に対応となったりで影響が諸々発生します。また、この影響で、GFortran側もmodule 14へ移行となるので、GFortranを使っているパッケージメンテナも対応が必要とのことです。}

\santaku
{7/8にて、複数のupstreamから提供されているlibav*群のライブラリについて、提供元のupstreamを変更するとの連絡がありました。どのupstreamに変更となったのでしょうか?}
{FFmpeg}
{libav.org}
{VideoLAN}
{A}
{libav*というマルチメディアのデータを扱うライブラリなのですが、一旦libav.orgが提供しているものに変更となったのですが、またFFmpegが提供しているものに戻ってきた状況です。議論のサマリは https://wiki.debian.org/Debate/libav-provider/ffmpeg}

\santaku
{DPLのNeil MacGovernがredditに開いた「DPLだけど、何か質問ある?」というスレで、DPLにとっても凄いと思うディストリビューションとしてあげられてたものはどれ?}
{当然Debianっしょ!}
{ArchLinux}
{ubuntu}
{B}
{スレの住人の質問に、DPLがArchLinuxが凄いと答えていました。wikiの充実ぶりがとにかく素晴らしいとのこと。}

\santaku
{7/20にdgitの新しいバージョンのものがリリースされました。どのバージョンになった?}
{0.1}
{0.3}
{1.0}
{C}
{Debianアーカイブをgitで操作できるツールのdgitが1.0がリリースされたとのことです。早速debian sidに収録されています。dgit clone package名とすると、https://git.dgit.debian.org/ で管理されているものが手元にcloneされます。}

\santaku
{7/21に Debian Installer Stretch Alpha 1がリリースされました。変更点は以下のどれ?}
{UEFIブートを搭載}
{ネットワークIFがMACアドレスになる}
{インストール時のUIがtextモードからgraphicalモードになった}
{C}
{Strechで利用されるであろう、インストーラプログラムのα1がリリースされました。もちろん、Strechがリリースされたわけではないので注意。変更点は数々あり、デフォルトのCPUアーキテクチャがamd64になったりした。詳しくはdebian-devel-announceを参照}
