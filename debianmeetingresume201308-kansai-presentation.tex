\documentclass[cjk,dvipdfmx,10pt,compress,%
hyperref={bookmarks=true,bookmarksnumbered=true,bookmarksopen=false,%
colorlinks=false,%
pdftitle={第 75 回 関西 Debian 勉強会},%
pdfauthor={倉敷・のがた・佐々木・かわだ・八津尾},%
%pdfinstitute={関西 Debian 勉強会},%
pdfsubject={資料},%
}]{beamer}

\title{第 75 回 関西 Debian 勉強会}
\subtitle{$\sim$発表資料$\sim$}
\author[かわだ てつたろう]{{\large\bf 倉敷・のがた・佐々木・かわだ・八津尾}}
\institute[Debian JP]{{\normalsize\tt 関西 Debian 勉強会}}
\date{{\small 2013 年 8 月 25 日}}

%\usepackage{amsmath}
%\usepackage{amssymb}
\usepackage{graphicx}
\usepackage{moreverb}
\usepackage[varg]{txfonts}
\AtBeginDvi{\special{pdf:tounicode EUC-UCS2}}
\usetheme{Kyoto}
\def\museincludegraphics{%
  \begingroup
  \catcode`\|=0
  \catcode`\\=12
  \catcode`\#=12
  \includegraphics[width=0.9\textwidth]}
%\renewcommand{\familydefault}{\sfdefault}
%\renewcommand{\kanjifamilydefault}{\sfdefault}
\begin{document}
\settitleslide
\begin{frame}
\titlepage
\end{frame}
\setdefaultslide

\begin{frame}[fragile]
\frametitle{Agenda}

\tableofcontents

\end{frame}

\section{最近の Debian 関係のイベント}

\takahashi[40]{最近の Debian\\関係のイベント}

\begin{frame}[fragile]
  \frametitle{大統一 Debian 勉強会 2013}
  \begin{itemize}
  \item 日時: 6 月 29 日(土)
  \item 場所: 日本大学 駿河台キャンパス
  \end{itemize}
  \begin{block}{内容}
    \begin{itemize}
    \item セッション
    \item アンカンファレンス
    \item ハック部屋
    \end{itemize}
  \end{block}
\end{frame}

\begin{frame}[fragile]
  \frametitle{第 72 回関西 Debian 勉強会}
  \begin{itemize}
  \item 日時: 5 月 26 日(日)
  \item 場所: グランフロント大阪 ナレッジサロン
  \end{itemize}
  \begin{block}{内容}
    \begin{itemize}
    \item 「Debian の歩き方」
    \item 「Debian と Ubuntu の違いを知ろう」
    \end{itemize}
  \end{block}
\end{frame}

\begin{frame}[fragile]
  \frametitle{第 74 回関西 Debian 勉強会}
  \begin{itemize}
  \item 日時: 8 月 3 日(土)
  \item 場所: 京都リサーチパーク
  \end{itemize}
  \begin{block}{内容}
    \begin{itemize}
    \item 「Debian 7.0 の実情/今後の開発について」
    \item ブース展示
    \end{itemize}
  \end{block}
\end{frame}

\begin{frame}[fragile]
  \frametitle{第 102 回東京エリア Debian 勉強会}
  \begin{itemize}
  \item 日時: 7 月 20 日(土)
  \item 場所: あんさんぶる荻窪
  \end{itemize}
  \begin{block}{内容}
    \begin{itemize}
    \item 「Debian linux kernel / armmp フレーバー」
    \item 「Raspberry Pi を使ってみた」
    \item 「月刊 Debhelper」
    \end{itemize}
  \end{block}
\end{frame}

\begin{frame}[fragile]
  \frametitle{第 103 回東京エリア Debian 勉強会}
  \begin{itemize}
  \item 日時: 8 月 17 日(土)
  \item 場所: あんさんぶる荻窪
  \end{itemize}
  \begin{block}{内容}
    \begin{itemize}
    \item 「OpenVPN を使ってみた」
    \item 「Debian勉強会の資料のePUB化を試みた」
    \end{itemize}
  \end{block}
\end{frame}


\begin{frame}[fragile]
  \frametitle{DebConf 13}
  \begin{itemize}
  \item スイス、ヴォーマルキュ
  \item 8 月 11 日 から 18 日
  \item 8 月 16 日 Debian 生誕 20 周年
  \end{itemize}
\end{frame}

\takahashi[50]{そんな\\こんなで}
\takahashi[120]{次}

\section{事前課題発表}

\takahashi[50]{事前課題}

\begin{frame}[fragile]
  \frametitle{事前課題}
  \begin{block}{今回の事前課題}
    \begin{description}
    \item[事前課題1]
      構成管理システムをどのように活用(している/したい)かを記述してください
    \end{description}
  \end{block}
\end{frame}

\takahashi[50]{事前課題\\発表}

\begin{frame}
  \frametitle{ 川江 }
  \begin{enumerate}
  \item 取りあえず、sshd、bind、iptables、interfaces等々の設定の更新。synatpicの自動起動と自動アップデート。
  \item sshで家のサーバにログイン(Wheezyマシン)できるようにしときます。
  \end{enumerate}
\end{frame}

\begin{frame}
  \frametitle{ かわにし }
  初参加です。

  構成管理システム自体良くわかりませんが、ぱっと思いついたのはサーバの構成でリソースをどれだけ使っていて、どうすれば改善されるかのヒントなどがでると初心者としては助かるなと思いました。
\end{frame}

\begin{frame}
  \frametitle{ 山城の国の住人 久保博 }
  今まではサーバー管理はいくつか経験していたものの、今に至るまで構成管理システムを活用していません。

  これから仮想化環境が立てられるようなハードウェアを買って、仮想環境をたくさん飼えるようになりたいです。
\end{frame}

\begin{frame}
  \frametitle{ かわだてつたろう }
  使ったことはありません。

  \begin{itemize}
  \item デプロイ先と同じ環境を容易に作りテストに使いたい。
  \item 設定の備忘録代わりにして環境移行(ハード交換、レンタルサーバ変更)のコストを下げられるとよいな。
  \end{itemize}
\end{frame}

\begin{frame}
  \frametitle{ おくの }
  まだ使ったことがありません。ホスティングサーバ屋ですが、構成管理システム使ったら楽になれるならぜひ使いたいです。
\end{frame}

\begin{frame}
  \frametitle{ 佐々木洋平 }
  構成管理のスクリプトはとくに使ったことはありません。50台ぐらいに対して clusterssh を使ってログインして、一括設定していたりします。あとは git で deploy する、とか。

  簡単にできたら良いな、とか思います。
\end{frame}

\begin{frame}
  \frametitle{ kino }
  Q: 構成管理システムをどのように活用したいか

  A: 現状使えていないのですが希望としては、
  \begin{itemize}
  \item 開発サーバーとライブサーバーを同じ設定にして素早く立ちあげたい。
  \item Debianのアップグレードテスト環境を作りたい。
  \item 専用サーバーの上に仮想環境を作って、Vagrant+Puppetみたいな使い方をしたい。
  \end{itemize}
\end{frame}

\begin{frame}
  \frametitle{ lurdan }
  小規模なデータセンターの管理に使っています。

  現状のメリット
  \begin{itemize}
  \item 共通設定について、一度対処したら忘れることができる
  \item イメージのクローンより、イレギュラー要望に対処しやすい
  \end{itemize}

  現状の課題
  \begin{itemize}
  \item 構成結果のテスト方法が確立できていない (serverspec と cucumber-nagios で模索中)
  \item 共通モジュールとしてくくり出す内容の線引きが悩ましい
  \end{itemize}
\end{frame}

\takahashi[50]{そんな\\こんなで}
\takahashi[120]{次}

\section{puppet による構成管理の実践}
\takahashi[30]{puppet による構成管理の実践\\by\\倉敷 悟}

\takahashi[50]{そんな\\こんなで}
\takahashi[120]{次}

\section{今後の予定}
\begin{frame}[fragile]
\frametitle{今後の予定 (1)}

\begin{block}{第 76 回関西 Debian 勉強会}
  \begin{itemize}
  \item 日時: 9 月 22 日(日)
  \item 会場: 港区民センター
  \item 内容: 未定
  \end{itemize}
\end{block}

\begin{block}{第 104 回東京エリア Debian 勉強会}
  \begin{itemize}
  \item 日時: 9 月 21 日
  \item 会場、内容: 未定
  \end{itemize}
\end{block}

\end{frame}

\takahashi[50]{  }

\end{document}
%%% Local Variables:
%%% mode: japanese-latex
%%% TeX-master: t
%%% End:
