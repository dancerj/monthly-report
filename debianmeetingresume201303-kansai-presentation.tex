\documentclass[cjk,dvipdfmx,10pt,compress,%
hyperref={bookmarks=true,bookmarksnumbered=true,bookmarksopen=false,%
colorlinks=false,%
pdftitle={第 70 回 関西 Debian 勉強会},%
pdfauthor={倉敷・のがた・佐々木・かわだ・八津尾},%
%pdfinstitute={関西 Debian 勉強会},%
pdfsubject={資料},%
}]{beamer}

\title{第 70 回 関西 Debian 勉強会}
\subtitle{$\sim$発表資料$\sim$}
\author[かわだ てつたろう]{{\large\bf 倉敷・のがた・佐々木・かわだ・八津尾}}
\institute[Debian JP]{{\normalsize\tt 関西 Debian 勉強会}}
\date{{\small 2013 年 3 月 24 日}}

%\usepackage{amsmath}
%\usepackage{amssymb}
\usepackage{graphicx}
\usepackage{moreverb}
\usepackage[varg]{txfonts}
\AtBeginDvi{\special{pdf:tounicode EUC-UCS2}}
\usetheme{Kyoto}
\def\museincludegraphics{%
  \begingroup
  \catcode`\|=0
  \catcode`\\=12
  \catcode`\#=12
  \includegraphics[width=0.9\textwidth]}
%\renewcommand{\familydefault}{\sfdefault}
%\renewcommand{\kanjifamilydefault}{\sfdefault}
\begin{document}
\settitleslide
\begin{frame}
\titlepage
\end{frame}
\setdefaultslide

\begin{frame}[fragile]
\frametitle{Agenda}

\tableofcontents

\end{frame}

\section{最近の Debian 関係のイベント}

\takahashi[40]{最近の Debian\\関係のイベント}

\begin{frame}[fragile]
  \frametitle{第 69 回関西 Debian 勉強会}
  \begin{itemize}
  \item 日時: 2 月 24 日(日)
  \item 場所: GREE 大阪オフィスセミナールーム
  \end{itemize}
  \begin{block}{内容}
    \begin{itemize}
    \item 「Preseeding Debian」
    \item 「Ruby In Wheezy」
    \end{itemize}
  \end{block}
\end{frame}

\begin{frame}[fragile]
  \frametitle{オープンソースカンファレンス 2013 Tokushima}
  \begin{itemize}
  \item 日時: 3 月 9 日(土)
  \item 場所: とくぎんトモニプラザ(徳島県青少年センター)
  \end{itemize}
  \begin{block}{ブース展示物}
    \begin{itemize}
    \item RaspberryPi と MK802 の展示
    \item あんどきゅめんてっどでびあんの展示
    \item Debian のインフォグラフィック日本語版の展示と配布
    \item Debianステッカーの配布
    \end{itemize}
  \end{block}
\end{frame}

\begin{frame}[fragile]
  \frametitle{第 98 回 東京エリア Debian 勉強会}
  \begin{itemize}
  \item 日時: 3 月 16 日(土)
  \item 場所: ミラクルリナックス株式会社
  \end{itemize}
  \begin{block}{内容}
    \begin{itemize}
    \item 「ldapvi \& python-ldap で stress-free life」
    \item 「月刊 Debhelper」
    \item 「gdb の python 拡張」
    \end{itemize}
  \end{block}
\end{frame}

\takahashi[50]{そんな\\こんなで}
\takahashi[120]{次}

\section{事前課題発表}

\takahashi[50]{事前課題}

\begin{frame}[fragile]
  \frametitle{事前課題}
  \begin{block}{今回の事前課題}
    \begin{description}
    \item[事前課題1] GNOME3 にどんな拡張機能があれば便利か、考えてみて、教えてください。

    \item[事前課題2] wheezy(sid) で GNOME3 環境を用意してきてください。
    
    \item[事前課題3] 「Large deployment of GNOME from the administrator's perspective」
    を読んで気になる点、わからない点を教えてください。

    \url{http://fr2012.mini.debconf.org/slides/LargeGnomeDeployment.pdf}

    \end{description}
  \end{block}
\end{frame}

\takahashi[50]{事前課題\\発表}

\begin{frame}
  \frametitle{kazuhito\_m}
  \begin{enumerate}
  \item Gnomeというか、Nautilsに対してかもしれませんが…

    ディレクトリ表示ペインの上か下に「CurrentDirでコマンド打てる入力粋」がほしいなと。すでにあるのでしょうか?
  \item 一応入れてきました。しかし…仮想機だからか「GnomeShellっぽいやつ」が全滅しています。

    世のスクショの画面と同じにならない…一応「GnomeClassic」じゃなく「Gnome」にはなって居るのですが…。 
  \item すみません、読み中です。(会場であればお話を)
  \end{enumerate}
\end{frame}

\begin{frame}
  \frametitle{yyatsuo}
  \begin{enumerate}
  \item  (できるのかもしれませんが…) キーボードで window を操作したい 
  \item 用意しました。が、ログインマネージャは slim です
  \item わからないことだらけだったので皆様に教えていただきたく。
  \end{enumerate}
\end{frame}

\begin{frame}
  \frametitle{山下尊也 (1/2)}
  \begin{enumerate}
  \item[1]
    \begin{itemize}
    \item shellshape
    \item Remove User Name
    \item TaskBar
    \item Advanced Settings in UserMenu
    \item Alternative Status Menu
    \item Axe Menu
    \item User Themes
    \item Impatience
    \item Frippery Move Clock
  \end{itemize}
  すでに拡張機能入れすぎて、自分が何で満足しているのか分からなくなっているという...

  タイル型の拡張で良いのがあれば良いですね。まぁ面倒なんで、いつも最大表示にしちゃいますけど。

  後、タスクバーのアイテムが動かせないのも許せなかったです。なんで、時間が真ん中なの!!とか最初思ったので。
  \end{enumerate}
\end{frame}

\begin{frame}
  \frametitle{山下尊也 (2/2)}
  \begin{enumerate}
  \item[2] これはインストール済み。
  \item[3] 特になしです。というか、まだ読めてない...orz 
  \end{enumerate}
\end{frame}

\begin{frame}
  \frametitle{山城の国の住人 久保博 }
  \begin{enumerate}
  \item 時間毎のタイプ頻度とかマウスの移動距離とかをグラフ化する。何時頃に何をしていたか思い出すヒントに。
  \item はい、 wheezy の GNOME3 環境を用意します。
  \item これから何とか
  \end{enumerate}
\end{frame}

\begin{frame}
  \frametitle{よしだともひろ}
  \begin{enumerate}
  \item 思いつきませんでした。
  \item 入っています。
  \item すみません、当日読むことになりそう。
  \end{enumerate}
\end{frame}

\begin{frame}
  \frametitle{ikuya}
  \begin{enumerate}
  \item 質問者なのでパスで。。
  \item 用意しました。持っていきます。
  \item 21ページ目のNTPはNFSの間違いではないでしょうか?
  \end{enumerate}
\end{frame}

\begin{frame}
  \frametitle{lurdan}
  \begin{enumerate}
  \item 余分な装飾と常駐プロセスをまとめて無効化できる的な何か、あるいはタイル(ry
  \item ゴメンね?
  \item 特には。参考になる+1 という感じです。
  \end{enumerate}
\end{frame}

\begin{frame}
  \frametitle{水野源}
  無回答
\end{frame}

\begin{frame}
  \frametitle{かわだてつたろう}
  \begin{enumerate}
  \item マウスジェスチャー。
  \item はい。
  \item 設定の具体例がもっと欲しい。

  設定を変更するにはどこをさわればよいかは理解したけれど、設定を変更するとどういうことが出来るようになるかまでは理解できていない。

  Jessie で変わるところ(systemd、JavaScript での設定)も気になります。
  \end{enumerate}
\end{frame}

\begin{frame}
  \frametitle{0xBCD1BC92}
  \begin{enumerate}
  \item Gnome Calendar と Google Calendar の同期
  \item はい
  \item (a) Configuring system connections の Internal と External の前提条件がよくわかりません。例示の具体的なネットワーク図があるとわかりやすいのでは?

    (b) PolicyKit の説明が、ConsoleKitと一緒なので、もうちょい増補してほしいなあ。 
  \end{enumerate}
\end{frame}

\begin{frame}
  \frametitle{mkamotsu}
  \begin{enumerate}
  \item 無回答
  \item 無回答
  \item gnome-keyringの仕組みを全く理解してませんでしたが、上記資料のおかげでGNOME環境でなくても利用する方法がわかりました。

    (Debian-specific)の部分はアップストリームでどうなっているのか、またどういった理由で手を加えているのかが気になります。
  \end{enumerate}
\end{frame}

\begin{frame}
  \frametitle{佐々木洋平}
  とりあえず。課題はあとで。
\end{frame}

\takahashi[50]{そんな\\こんなで}
\takahashi[120]{次}

\section{Ubuntu と GNOME Shell と私}
\takahashi[30]{Ubuntu と\\ GNOME Shell と\\私\\by\\あわしろいくや}

\takahashi[50]{そんな\\こんなで}
\takahashi[120]{次}

\section{Large Deployment of GNOME from the Administrator's Perspective}
\takahashi[30]{Large Deployment of GNOME from the Administrator's Perspective\\by\\八津尾}

\takahashi[50]{そんな\\こんなで}
\takahashi[120]{次}

\section{今後の予定}
\begin{frame}[fragile]
\frametitle{今後の予定 (1)}

\begin{block}{第 71 回関西 Debian 勉強会}
  \begin{itemize}
  \item 日時: 4 月 28 日(日)
  \item 会場: 福島区民センター 301号室
  \item 内容: 未定
  \end{itemize}
\end{block}

\end{frame}

\begin{frame}[fragile]
\frametitle{今後の予定 (2)}

\begin{block}{第 99 回東京エリア Debian 勉強会}
  \begin{itemize}
  \item 日時: 4 月 20 日(土)
  \item 会場: 未定
  \item 内容: 未定
  \end{itemize}
\end{block}

\begin{block}{第03回 福岡 Debian 勉強会}
  \begin{itemize}
  \item 日時: 3 月 28 日(木) 19:00〜
  \item 会場: 天神セントラルプレイス 2F OnRamp内
  \item 内容: 「Wheezyで搭載されるカーネルの新機能とかとか」など
  \end{itemize}
\end{block}

\end{frame}

\begin{frame}[fragile]
\frametitle{今後の予定 (3)}

\begin{block}{大統一 Debian 勉強会}
  \begin{itemize}
  \item 日時: 6 月 29 日(土)
  \item 会場: 東京 日本大学 駿河台キャンパス
  \item 4 月上旬サイト公開予定
  \item \url{http://gum.debian.or.jp/}
  \end{itemize}
\end{block}

\end{frame}


\takahashi[50]{  }


\end{document}
%%% Local Variables:
%%% mode: japanese-latex
%%% TeX-master: t
%%% End:
