\documentclass[cjk,dvipdfmx,10pt,compress,%
hyperref={bookmarks=true,bookmarksnumbered=true,bookmarksopen=false,%
colorlinks=false,%
pdftitle={第 81 回 関西 Debian 勉強会},%
pdfauthor={倉敷・のがた・佐々木・かわだ・八津尾},%
%pdfinstitute={関西 Debian 勉強会},%
pdfsubject={資料},%
}]{beamer}

\title{第 80 回 関西 Debian 勉強会}
\subtitle{$\sim$発表資料$\sim$}
\author[かわだ てつたろう]{{\large\bf 倉敷・のがた・佐々木・かわだ・八津尾}}
\institute[Debian JP]{{\normalsize\tt 関西 Debian 勉強会}}
\date{{\small 2014 年 2 月 23 日}}

%\usepackage{amsmath}
%\usepackage{amssymb}
\usepackage{graphicx}
\usepackage{moreverb}
\usepackage[varg]{txfonts}
\AtBeginDvi{\special{pdf:tounicode EUC-UCS2}}
\usetheme{Kyoto}
\def\museincludegraphics{%
  \begingroup
  \catcode`\|=0
  \catcode`\\=12
  \catcode`\#=12
  \includegraphics[width=0.9\textwidth]}
%\renewcommand{\familydefault}{\sfdefault}
%\renewcommand{\kanjifamilydefault}{\sfdefault}
\begin{document}
\settitleslide
\begin{frame}
\titlepage
\end{frame}
\setdefaultslide

\begin{frame}[fragile]
\frametitle{Agenda}

\tableofcontents

\end{frame}

\section{最近の Debian 関係のイベント}

\takahashi[40]{最近の Debian\\関係のイベント}

\begin{frame}[fragile]
  \frametitle{第80回関西Debian勉強会}
  \begin{itemize}
  \item 日時: 1月26日(日)
  \item 場所: 福島区民センター
  \end{itemize}
  \begin{block}{内容}
    \begin{itemize}
    \item 「LT: jenkins + jenkins-debian-glue + freight で野良リポジトリ作った」
    \item 「LT: さくらVPSでIPv6設定」
    \item 「もくもくの会」
    \end{itemize}
  \end{block}
\end{frame}

\takahashi[50]{そんな\\こんなで}
\takahashi[120]{次}

\section{事前課題発表}

\takahashi[50]{事前課題}

\begin{frame}[fragile]
  \frametitle{事前課題}
  \begin{block}{今回の事前課題}
    \begin{description}
    \item[事前課題1]
      もくもくの会で行なう作業、質問などの課題を用意して教えてください。
    \item[事前課題2]
      前回(第80回)の勉強会に参加された方は、前回の作業や課題がその後
      どうなったか結果を教えてください。
    \item[事前課題3]
      LT 歓迎です。何かお話したい方はタイトルを下さい。
    \end{description}
  \end{block}
\end{frame}

\takahashi[50]{事前課題\\発表}

\begin{frame}
  \frametitle{ 西山和広 }
  \begin{enumerate}
  \item 作りかけの rd2markdown を進めたいです。
  \item ubuntu 13.04 のサーバーがあったのは 13.10 に上げましたが、Debian squeeze のサーバーはまだ squeeze のままです。
  \item あまり資料を用意できないので docker のデモ程度の話を予定しています。
  \end{enumerate}
\end{frame}

\begin{frame}
  \frametitle{ 木下 }
  \begin{itemize}
  \item Debian7 on PANDABOARDの調査・研究 (WiFiモジュール(USB)の接続、GPUデバイスドライバの有効化)
  \item 結果にたどり着いていません。まだまだ現在進行形です。
  \item プライベートクラウドの実情(OpenStack、CloudStack、Eucalyptusの現状等)
  \end{itemize}
\end{frame}

\begin{frame}
  \frametitle{ 佐々木洋平 }
  \begin{enumerate}
  \item pkg-ruby-extras で幾つか作業関連 (jekyll、tDiary、tilt)
  \item jenkins-debian-glue. piuparts の動作が微妙...。
  \item sytemd-sys を install してしまったので、その辺話せる、かも。
  \end{enumerate}
\end{frame}

\begin{frame}
  \frametitle{ 川江 }
  \begin{enumerate}
  \item
    \begin{itemize}
    \item とりあえず、KVMについて知っている方がいればいいなぁ
    \item DNSの調整は終わりました。ただ、IPSから「NTPリフレクション攻撃」についてのメールが来ていたので、DNSサーバの「設定」ミスだけが、ネットワークの速度の『遅延』の原因ではなかったようです。後、VMでWebサーバを立ててHTML5のサイトを立ち上げようとしているのですが、諸事情によりできてません。
    \end{itemize}
  \end{enumerate}
\end{frame}

\begin{frame}
  \frametitle{ 大北剛史 }
    \begin{itemize}
    \item 前回初参加させてもらいました。もくもくの会用に、ノートブックを1台購入しましたが、debianをインストールしようと今奮闘中です。
    \item debian.jpにて、略語の解説を発見。ITP=Intent to Packageを理解。
    \end{itemize}
\end{frame}

\begin{frame}
  \frametitle{ lurdan }
  \begin{enumerate}
  \item mmp のパッケージングを仕上げたいところです
  \item weblate の環境作成ですが放置してます。余裕があれば今回も python-social-auth のパッケージ作成から続きをやりたいところですが。
  \item 余裕なっしん
  \end{enumerate}
\end{frame}

\begin{frame}
  \frametitle{ Hideaki Oose }
    \begin{itemize}
    \item 無回答
    \end{itemize}
\end{frame}

\takahashi[50]{そんな\\こんなで}
\takahashi[120]{次}

\section{ライトニングトーク}
\takahashi[30]{ライトニングトーク}

\takahashi[50]{そんな\\こんなで}
\takahashi[120]{次}

\section{もくもくの会}
\takahashi[30]{もくもくの会}

\takahashi[50]{そんな\\こんなで}
\takahashi[120]{次}

\section{今後の予定}
\begin{frame}[fragile]
\frametitle{今後の予定}

\begin{block}{第82回関西Debian勉強会}
  \begin{itemize}
  \item 日時: 3月23日(日) 13:30 -
  \item 場所: 福島区民センター
  \end{itemize}
\end{block}

\end{frame}

\takahashi[50]{  }

\end{document}
%%% Local Variables:
%%% mode: japanese-latex
%%% TeX-master: t
%%% End:
