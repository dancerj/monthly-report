%; whizzy-master ../debianmeetingresume201311.tex
% 以上の設定をしているため、このファイルで M-x whizzytex すると、whizzytexが利用できます。
%

\santaku
{FSFがDebian Projectへ案内をしてきた、自由ソフトウェアのみの元で動かすことの出来るハードウェアについてのデータベースは次のうちどれ?}
{h-node.org}
{wiki.debian.org/Hardware}
{openbenchmarking.org}
{A}
{日本語の本件のニュースはsourceforge.jpの記事参照:http://sourceforge.jp/magazine/14/09/11/062900 。FSFはmainリポジトリのみのパッケージで構成されるDebianは自由ソフトウェアとみているとのこと。}

\santaku
{Debconf14の参加人数は結局何人?}
{900人}
{300人}
{1000人}
{B}
{300人とのことです。参考:Debconf13は290人、Debconf12は176人、Debconf11は335人でした。}

\santaku
{8/17にbuilddにて使われるアーカイブがどこからもアクセスできるようになりました。urlはどれ?}
{ftp.debian.or.jp/debian/}
{ftp.jp.debian.org/debian/}
{incoming.debian.org/debian-buildd/}
{C}
{今までは、どこからもアクセスできたわけではなかったようです。}

\santaku
{2014/8/19に登録商標としてDebianロゴが正式に登録されたそうです。どこの国の登録商標でしょうか?}
{米国}
{日本}
{スイス}
{A}
{United States Patent and Trademark Officeになります(つまり米国。)登録されたDebianロゴのデザインは http://tdr.uspto.gov/search.action?sn=86037470 からたどると閲覧できます。}

\santaku
{2014/8/24のBitFromDPLによれば、Debian Projectは仮想通貨による寄付をはじめて受け付けたそうです。具体的には何という仮想通貨でしょう?}
{Greeコイン}
{Crysta}
{BitCoin}
{C}
{Debian ProjectはBitCoinをそのまま受け付けることが出来るシステムを持たないため、その場限りの方法で受け取ったとのことです。今後、こういった仮想通貨での寄付の受け取りと取り扱いについて意見がほしいとのことです。}

\santaku
{検索エンジンのDuckDuckGOより、収入が入ったとのことです。2014/8/24現在、月当たりのDuckDuckGOからの平均収入は月額いくらでしょう?}
{\$10}
{\$152}
{\$1400}
{B}
{Debianパッケージに含まれるブラウザにデフォルトで登録されている検索エンジンの候補としてDuckDuckGOが搭載されていることによる収入となります。DuckDuckGOはプライバシーに配慮した検索エンジンです。最近は、iphone のsafariブラウザにあらかじめ登録される検索エンジンの候補としても上がり有名になりつつあります。URLはhttps://duckduckgo.com/}

\santaku
{2014/8/27にDebian archiveに搭載された2つの新しいアーキテクチャは、arm64以外には以下のどれ?}
{sparc}
{mips}
{ppc64el}
{C}
{ 64 bit powerpcのlittle endianモードのポーティングとのことです。すでに存在するppc64はbig endianのバイナリのポートティングとなります。}

\santaku
{2014/8/31にて、arm64ポートのDebian開発用に、無償のARM64用のコンパイラ・デバッガ等の開発キットの提供が行われたようです。製品名は以下のどれ?}
{Microsoft Visual Studio}
{IAR Embedded Workbench}
{DS-5 Development Studio}
{C}
{ Debian Editionとのことです。アナウンスによれば、ダウンロードリンクは http://ds.arm.com/debian/ からダウンロード可能とのことですが、日本からはダウンロードが現在出来ない模様です。残念!もちろんですが、この開発キットは無償ではあるものの自由ソフトウェアではないので誤解なきよう。}

\santaku
{ Debian keyringからある大きさ以上の秘密鍵長を持たないキーが2014/12/31以降で削除される事についてのリマインドのアナウンスが流れていました。ある大きさとは以下のどれ?}
{ 512bit }
{ 2048bit }
{ 4096bit }
{B}
{ キーサインにつかうgpgの鍵も2048bit以上にしましょう!}

\santaku
{ 2014/9/17にDebian Policy が改定されました。改定後のバージョンはいくつ?}
{ 3.9.5.0 }
{ 3.9.6.0 }
{ 4.0.0.0 }
{B}
{ パッケージ開発をする前に、新しいDebian Policyの変更差分は読んでおきましょう。}
