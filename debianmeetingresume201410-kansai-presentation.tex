\documentclass[cjk,dvipdfmx,10pt,compress,%
hyperref={bookmarks=true,bookmarksnumbered=true,bookmarksopen=false,%
colorlinks=false,%
pdftitle={第 89 回 関西 Debian 勉強会},%
pdfauthor={倉敷・のがた・佐々木・かわだ・八津尾},%
%pdfinstitute={関西 Debian 勉強会},%
pdfsubject={資料},%
}]{beamer}

\title{第 89 回 関西 Debian 勉強会}
\subtitle{$\sim$発表資料$\sim$}
\author[かわだ てつたろう]{{\large\bf 倉敷・のがた・佐々木・かわだ・八津尾}}
\institute[Debian JP]{{\normalsize\tt 関西 Debian 勉強会}}
\date{{\small 2014 年 10 月 26 日}}

%\usepackage{amsmath}
%\usepackage{amssymb}
\usepackage{graphicx}
\usepackage{moreverb}
\usepackage[varg]{txfonts}
\AtBeginDvi{\special{pdf:tounicode EUC-UCS2}}
\usetheme{Kyoto}
\def\museincludegraphics{%
  \begingroup
  \catcode`\|=0
  \catcode`\\=12
  \catcode`\#=12
  \includegraphics[width=0.9\textwidth]}
%\renewcommand{\familydefault}{\sfdefault}
%\renewcommand{\kanjifamilydefault}{\sfdefault}
\begin{document}
\settitleslide
\begin{frame}
\titlepage
\end{frame}
\setdefaultslide

\begin{frame}[fragile]
  \frametitle{Disclaimer}
  \begin{itemize}
  \item 疑問、質問、ツッコミ、茶々、\alert{大歓迎}
  \item その場でインタラクティブにどうぞ
  \item ハッシュタグ \#kansaidebian
\end{itemize}
\end{frame}

\begin{frame}[fragile]
\frametitle{Agenda}

\tableofcontents

\end{frame}

\section{最近の Debian 関係のイベント}

\takahashi[40]{最近の Debian\\関係のイベント}

\begin{frame}[fragile]
  \frametitle{第88回関西Debian勉強会}
  \begin{itemize}
  \item 日時: 9月28日(日)
  \item 場所: 福島区民センター
  \end{itemize}
  \begin{block}{内容}
    \begin{itemize}
    \item もくもくの会
    \end{itemize}
  \end{block}
\end{frame}

\begin{frame}[fragile]
  \frametitle{第118回東京エリアDebian勉強会/オープンソースカンファレンス2014 Tokyo/Fall}
  \begin{itemize}
  \item 日時: 10月18日(土)
  \item 場所: OSC 2014 Tokyo/Fall 会場(明星大学 日野キャンパス 28号館)
  \end{itemize}
  \begin{block}{内容}
    \begin{itemize}
    \item 「Debian update」
    \item ブース展示
    \end{itemize}
  \end{block}
\end{frame}

\begin{frame}[fragile]
  \frametitle{第119回東京エリアDebian勉強会x関東 LibreOffice オフxJessieインストーラテスト会、2014年10月 2nd勉強会}
  \begin{itemize}
  \item 日時: 10月25日(土)
  \item 場所: 株式会社スクウェア・エニックス セミナールーム
  \end{itemize}
  \begin{block}{内容}
    \begin{itemize}
    \item 「Debian x LibreOffice」
    \item 「もくもくの会」
    \end{itemize}
  \end{block}
\end{frame}

\begin{frame}[fragile]
  \frametitle{Debian Project}
  \begin{itemize}
  \item Re-Proposal - preserve freedom of choice of init systems
  \item \#66801 debootstrap: cant install systemd instead of sysvinit
  \item apt-cudf
  \item apt-get purge chromium
  \end{itemize}
\end{frame}

\takahashi[50]{そんな\\こんなで}
\takahashi[120]{次}

\section{事前課題発表}

\takahashi[50]{事前課題}

\begin{frame}[fragile]
  \frametitle{事前課題}
  \begin{block}{今回の事前課題}
    \begin{description}
    \item[事前課題1]
      もくもくの会で行なう作業、質問などの課題を用意して教えてください。
    \item[事前課題2]
      前回(第88回)の勉強会に参加された方は、前回の作業や課題がその後どう
      なったか結果を教えてください。
    \item[事前課題3]
      LT(ライトニングトーク) 歓迎です。何かお話したい方はタイトルを下さい。
    \end{description}
  \end{block}
\end{frame}

\begin{frame}
  \frametitle{ kozo2 }
\end{frame}

\begin{frame}
  \frametitle{ yyatsuo }
\end{frame}

\begin{frame}
  \frametitle{ かわだてつたろう }
  \begin{enumerate}
  \item オレオレバックポートパッケージを作る。
  \end{enumerate}
\end{frame}

\begin{frame}
  \frametitle{ 山城の国の住人 久保博 }

  \begin{enumerate}
  \item 前回宿題の続き。

    バッケージ引き取りの相談

    パッケージ作成方法のおさらい

  \item 前回宿題はまったく進捗なしです

  \end{enumerate}
\end{frame}

\begin{frame}
  \frametitle{ 佐々木洋平 }
  \begin{enumerate}
  \item pkg-ruby-extras 関連の更新。...とtDiary。あと ibus-skk のデバッグ?
  \item 同上。
  \item systemd 無しで jessie を使うには、みたいなネタをできたら良いな、と思います。
  \end{enumerate}
\end{frame}

\begin{frame}
  \frametitle{ lurdan }
  ちょっと顔を出すだけに、なる、かも
\end{frame}

\begin{frame}
  \frametitle{ 川江 }
  \begin{enumerate}
  \item JavaScriptによるサイトのビルド。systemdとsysvinitの関係について
  \item 進捗は悪いです。
  \end{enumerate}
\end{frame}

\takahashi[50]{事前課題\\発表}


\takahashi[50]{そんな\\こんなで}
\takahashi[120]{次}

\section{もくもくの会}
\takahashi[30]{もくもくの会}

\takahashi[50]{そんな\\こんなで}
\takahashi[120]{次}

\section{今後の予定}
\begin{frame}[fragile]
\frametitle{今後の予定}

\begin{block}{第90回関西Debian勉強会@関西オープンソース2014}
  \begin{itemize}
  \item 日時: 11月8日(土)
  \item 場所: 大阪南港ATC ITM棟 10F
  \item 内容: 「Debian 8 "jessie" frozen」10F サロン 13:00〜
  \end{itemize}
\end{block}

\begin{block}{第120回東京エリアDebian勉強会}
  \begin{itemize}
  \item 日時: 11月15日(土)
  \item 場所: 未定
  \item 内容: 未定
  \end{itemize}
\end{block}

\end{frame}

\takahashi[50]{  }

\end{document}
%%% Local Variables:
%%% mode: japanese-latex
%%% TeX-master: t
%%% End:
