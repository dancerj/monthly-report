%; whizzy document
% latex beamer presentation.
% platex, latex-beamer でコンパイルすることを想定。 
%     Tokyo Debian Meeting resources
%     Copyright (C) 2006 Junichi Uekawa

%     This program is free software; you can redistribute it and/or modify
%     it under the terms of the GNU General Public License as published by
%     the Free Software Foundation; either version 2 of the License, or
%     (at your option) any later version.

%     This program is distributed in the hope that it will be useful,
%     but WITHOUT ANY WARRANTY; without even the implied warranty of
%     MERCHANTABILITY or FITNESS FOR A PARTICULAR PURPOSE.  See the
%     GNU General Public License for more details.

%     You should have received a copy of the GNU General Public License
%     along with this program; if not, write to the Free Software
%     Foundation, Inc., 51 Franklin St, Fifth Floor, Boston, MA  02110-1301 USA


\documentclass[cjk,dvipdfmx]{beamer}
\usetheme{Warsaw}
%  preview (shell-command "xpdf debianmeetingresume200601-presentation.pdf&")
%  presentation (shell-command "xpdf -fullscreen debianmeetingresume200601-presentation.pdf&")

%http://www.naney.org/diki/dk/hyperref.html
%日本語EUC系環境の時
\AtBeginDvi{\special{pdf:tounicode EUC-UCS2}}
%シフトJIS系環境の時
%\AtBeginDvi{\special{pdf:tounicode 90ms-RKSJ-UCS2}}


\title[Debian 勉強会クイズ問題]{Debian勉強会クイズ}
\subtitle{2006年1月21日版}
\author{上川}
\date{2006年1月21日}

% 三択問題用
\newcounter{santakucounter}
\newcommand{\santaku}[5]{%
\addtocounter{santakucounter}{1}
\frame{\frametitle{問題\arabic{santakucounter}. #1}
%問題\arabic{santakucounter}. #1
\begin{itemize}
\item □ A #2\\
\item □ B #3\\
\item □ C #4\\
\end{itemize}
}
\frame{\frametitle{問題\arabic{santakucounter}. #1}
%問題\arabic{santakucounter}. #1
\begin{itemize}
\item □ A #2\\
\item □ B #3\\
\item □ C #4\\
\end{itemize}
\vfill{}
#5
}
}


\begin{document}
\frame{\titlepage{}}

\section{DWNQuiz}
%% debianmeetingresume200601.texから以下コピー
\subsection{2005年50号}
\url{http://www.debian.org/News/weekly/2005/50/}
にある2005年12月13日版です。

\santaku
{www.skolelinux.orgについて提案されていないことはどれか}
{バグトラッキングシステムを共有する}
{関係者のblogをplanetでアグリゲートしたりする}
{こっそりと人を誘拐してメンバーを増やす}
{C}

\santaku
{Branden RobinsonがTuxjournalでのインタビューで、Debianの成功に貢献した
ものとしてあげたのは}
{自由なライセンスを強調する人達とソフトウェアの品質を強調する人達がそれ
ぞれ貢献できてきたこと}
{Rubyをがんがんつかってコードを書いた事}% 彼はpythonユーザです。
{お互いに仲のわるい開発者たちが足をひっぱりあいながらお互いを潰しあって
いたこと}
{A}

\santaku
{GPLでリリースされているゲームボーイ用のエミュレータはmainにいれてもよい
のか}
{フリーのゲームを開発しているグループがあるため、mainにいれてもよい}
{ゲームは商用のゲームしか存在しないためcontribにいれる必要がある}
{エミュレーションという不純な動作はnon-freeにあるべきだ}
{A}

\santaku
{パッケージを実行用のパッケージ {\it pkg} と データ用のパッケージ{\it pkg}-data に分割する場合の確認項目についてBill Allombertは投稿し
た。そこで説明していなかったのは}
{パッケージ名は pkg と pkg-data にしてほしい}
{pkg-data は architecture: all にしてほしい}
{pkg-dataのサイズは5MBを越えていることが望ましい}
{C}

\santaku
{tetexの設定ファイルについてFrankの提案したのは?}
{/usr/share/texmfにデフォルトがあり、/etc/texmf にアドミニストレータの設
定があり、HOME/texmf に各ユーザの設定がある構成}
{Debianメンテナが一番偉いのでユーザの設定を無視して、世界統一の設定にすること}
{Debian menu システムの設定システムが優秀なので、それをそのまま採用する
こと}
{A}



\subsection{2005年51号}
\url{http://www.debian.org/News/weekly/2005/50/}
にある2005年12月20日版です。

\santaku
{debianforum.deは開始何年たったか}
{3年}
{4年}
{5年}
{B}

\santaku
{Jaldhar H. Vyas はインドでは通信コストが高いため、雑誌に付録DVDとしてDebianを
付けたいと提案しました。ただ、複数枚はコストがかかるので、DVD1 枚におさめたいと
説明しました。Joerg Jaspertの回答は}
{Cebitなどの展示会で利用するためにすでに作成したことがあるので結構簡単だよ、と回答した}
{そんなものつくることがおこがましい}
{ソースだけだったら1枚でもいけるかも}
{A}

\santaku
{lsb向けの起動スクリプトの利用方法について検討していた際に、エラーが発生した
場合にコマンド自体のエラーが画面に出力されて表示がみだれた。
この問題に対して提案された解決策は}
{エラーは/dev/nullへ}
{エラーなんておきないようにする}
{エラーなどをsyslogに送信してみる}
{C}

\santaku
{dpkg-sig を利用してDebianパッケージに電子署名を追加することができる。
最近dpkg-sig の署名を含むDebianパッケージがDebian archiveにアップロードできなくなってい
た。その理由は}
{dpkgにそんな機能拡張はしてはいけないという主義主張の問題}
{予期しない原因でjennifferのチェックが厳しすぎたため}
{実はdpkg-sigなんてものはなかった}
{B}

\santaku
{TexLiveパッケージのライセンスでJoerg Jaspertがおかしいと指摘したのは}
{Liveという名前がダメだ}
{texは時代遅れです}
{DFSGという謎のライセンスを利用している部分が存在した}
{C}

\subsection{2005年52号}
\url{http://www.debian.org/News/weekly/2005/52/}
にある12月27日版です。

\santaku
{Norbert Tretkovskiは、backports.orgで何がおきたと発表したのか。}
{backports.orgがetchに対応した}
{backports.orgのメンテナンスをあきらめた}
{backports.orgがsargeに対応した}
{C}

\santaku
{http://wiki.debian.org/StatusOfUnstable は何を説明してくれるページか}%\url{http://wiki.debian.org/StatusOfUnstable}
{現状のunstableで何がおきているのかをまとめている wikiページ}
{unstableであるということはどういうのかといういことを熱く語るスレッド}
{今どういうことがunstableになりえるのかということを解説しているページ}
{A}

\santaku
{Kevin Lockeが発表したpowermgmtプロジェクトは何をするものか}
{Debianの中での共通の電源管理用のインフラを提供することを目標とする}
{ハックに必要な栄養の補給方法について検討することを目標とする}
{権力をいかに分配するのかということについて検討することを目標とする}
{A}


\subsection{2006年1号}
\url{http://www.debian.org/News/weekly/2006/01/}
にある1月3日版です。

\santaku
{Debian パッケージを圧縮しなおすことで一番小さくできたのはどの圧縮ソフト
ウェアか}
{gzip}
{bzip2}
{7-zip}
{C}

\santaku
{apt-torrentは何をするものか}
{apt風のインタフェースでbittorrentを利用できるツール}
{aptのパッケージダウンロードをbit-torrent経由で実行するためのツール}
{海流予測用ツール}
{B}

\santaku
{vim-tinyは何をするものか}
{nviのかわりにデフォルトにするためにつくられた}
{ただvimを小さくしてみました}
{vimの機能はむだなものが多いので、普通いらないだろうというものだけにして
みた}
{A}


\santaku
{Lars Wirzeniusの提案したのは何か}
{Debianの品質改善のための提案}
{Debianのパッケージ削減のための提案}
{Debianの利用方法の改善のための提案}
{A}

\subsection{2006年2号}
\url{http://www.debian.org/News/weekly/2006/02/}
にある1月10日版です。

\santaku
{Technical Committeeに参加した新メンバーは誰か}
{Steve Langasek, Anthony Towns, と Andreas Barth }
{Wichert Akkerman, Jason Gunthorpe, と Guy Maor }
{Branden Robinson, Kenshi Muto, と Goto Masanori}
{A}


\santaku
{カーネルに存在していたnon-free firmware blobについては現状どうなってい
るか}
{進展がない}
{ライセンスを変更することで全て対処した}
{request firmware というフレームワークによりユーザ空間に移動した}
{C}

\santaku
{apt-get updateでgpgエラーが発生した、これは何か}
{Debianのアーカイブに侵入されたため}
{Debian アーカイブ署名キーが毎年かわるため、2006年用のものに変更する必要があっ
た}
{gpgはもうサポートされていない}
{B}

\subsection{2006年3号}
\url{http://www.debian.org/News/weekly/2006/03/}
にある1月18日版です。

\santaku
{coldfireとはどういうCPUか}
{ヒートシンクに液体冷却を採用したCPUの総称}
{実はCPUではない}
{組み込み用のCPUで、最近はMMUのあるバージョンもある、m68kと一部バイナリ
互換}
{C}

\santaku
{Anthony Townsが宣言したアーカイブの分割とはどういうものか}
{Debian Projectは今後i386のみを配布することに決定した。}
{メインアーカイブはi386, x86\_64, ppcのみにして、その他はscc.debian.org
で提供する}
{メインアーカイブはi386のみにして、それ以外は別のミラーサーバ名で提供す
るようなインフラを準備する}
{C}

\santaku
{協調メンテナンスをするためのフレームワークとして重要だとRaphael Hertzog
が提案したのは何か}
{SVNで管理している協調管理のソースの進捗状況をトラッキングするためのツール}
{仲よくするための定期的な宴会}
{説明書をしっかり書く事}
{A}



\end{document}