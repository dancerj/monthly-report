%; whizzy paragraph -pdf xpdf -latex ./whizzypdfptex.sh
%; whizzy-paragraph "^\\\\begin{frame}\\|\\\\emtext"
% latex beamer presentation.
% platex, latex-beamer $B$G%3%s%Q%$%k$9$k$3$H$rA[Dj!#(B 

%     Tokyo Debian Meeting resources
%     Copyright (C) 2012 Junichi Uekawa

%     This program is free software; you can redistribute it and/or modify
%     it under the terms of the GNU General Public License as published by
%     the Free Software Foundation; either version 2 of the License, or
%     (at your option) any later version.

%     This program is distributed in the hope that it will be useful,
%     but WITHOUT ANY WARRANTY; without even the implied warreanty of
%     MERCHANTABILITY or FITNESS FOR A PARTICULAR PURPOSE.  See the
%     GNU General Public License for more details.
%     You should have received a copy of the GNU General Public License
%     along with this program; if not, write to the Free Software
%     Foundation, Inc., 51 Franklin St, Fifth Floor, Boston, MA  02110-1301 USA

\documentclass[cjk,dvipdfmx,12pt]{beamer}
\usetheme{Tokyo}
\usepackage{monthlypresentation}

%  preview (shell-command (concat "evince " (replace-regexp-in-string "tex$" "pdf"(buffer-file-name)) "&")) 
%  presentation (shell-command (concat "xpdf -fullscreen " (replace-regexp-in-string "tex$" "pdf"(buffer-file-name)) "&"))
%  presentation (shell-command (concat "evince " (replace-regexp-in-string "tex$" "pdf"(buffer-file-name)) "&"))

%http://www.naney.org/diki/dk/hyperref.html
%$BF|K\8l(BEUC$B7O4D6-$N;~(B
\AtBeginDvi{\special{pdf:tounicode EUC-UCS2}}
%$B%7%U%H(BJIS$B7O4D6-$N;~(B
%\AtBeginDvi{\special{pdf:tounicode 90ms-RKSJ-UCS2}}

\newenvironment{commandlinesmall}%
{\VerbatimEnvironment
  \begin{Sbox}\begin{minipage}{1.0\hsize}\begin{fontsize}{8}{8} \begin{BVerbatim}}%
{\end{BVerbatim}\end{fontsize}\end{minipage}\end{Sbox}
  \setlength{\fboxsep}{8pt}
% start on a new paragraph

\vspace{6pt}% skip before
\fcolorbox{dancerdarkblue}{dancerlightblue}{\TheSbox}

\vspace{6pt}% skip after
}
%end of commandlinesmall

\title{Tokyo Debian Meeting}
\subtitle{The 138th 2016/3}
\author{Takahide Nojima}
\date{2016/3/5}
\logo{\includegraphics[width=8cm]{image200607/openlogo-light.eps}}

\begin{document}

\begin{frame}
\titlepage{}
\end{frame}

\begin{frame}{Agenda}
 \begin{minipage}[t]{0.45\hsize}
  \begin{itemize}
   \item Questionnaire Results
   \item Report of Debian related events in Japan.
	 \begin{itemize}
	 \item The 136th Tokyo Debian Meeting.
	 \item The 137th Tokyo Debian Meeting in OSC 2016 Tokyo/Spring
	 \end{itemize}
  \end{itemize}
 \end{minipage} 
 \begin{minipage}[t]{0.45\hsize}
  \begin{itemize}
   \item Session 1: How to become a Debian Developer
   \item Session 2: About tilegx
   \item Session 3: Use Some Debian Infrastructure
   \item Study hours.
   \item Where do we go for a drink?
  \end{itemize}
 \end{minipage}
\end{frame}

\section{Questionnaire Results}
\emtext{Questionnaire Results}
{\footnotesize
 \begin{prework}{ uchan }
  \begin{enumerate}
  \item Q.What do you plan to do in study hour?\\
    A.deb パッケージ作成のお勉強 (Learning how to make debian package.)
  \end{enumerate}
\end{prework}

\begin{prework}{ inaba\_kazuhiko }
  \begin{enumerate}
  \item Q.What do you plan to do in study hour?\\
    A. よろしくお願いいたします。よくわかりませんが、何かしたいかな。(Nice to meet you,forks! I don't have a plan at the moment, however ,I will find out.)
  \end{enumerate}
\end{prework}

\begin{prework}{ それがし某 }
  (It pronumces like So-Re-Ga-Shi-Bow. It means ``Mr. Nobody'' ,and it sounds like pun in classic manner of Japanese language.)
  \begin{enumerate}
  \item Q. What do you plan to do in study hour?\\
    A. debianでwineのビルド(Try to build WINE ,which is windows api emulator, on Debian.)
  \end{enumerate}
\end{prework}

\begin{prework}{ henrich }
  \begin{enumerate}
  \item Q. What do you plan to do in study hour?\\
    A. 何かしらのパッケージングを続けようかと思います ( I make some packages.)
  \end{enumerate}
\end{prework}

\begin{prework}{ kenhys }
  \begin{enumerate}
  \item Q. What do you plan to do in study hour?\\
    A. パッケージのメンテナンス ( I will do maintenance of my packages. )
  \end{enumerate}
\end{prework}

\begin{prework}{ YoshinoriSato }
  \begin{enumerate}
  \item Q. What do you plan to do in study hour?\\
    A. 未定 (I don't have a plan at the moment. )
  \end{enumerate}
\end{prework}

\begin{prework}{ knok }
  \begin{enumerate}
  \item Q. What do you plan to do in study hour?\\
    A. パッケージのバグ修正 ( Fix bugs of packages. )
  \end{enumerate}
\end{prework}

\begin{prework}{ wskoka }
  \begin{enumerate}
  \item Q. What do you plan to do in study hour?\\
    A. tilegxへの移植 ( Porting debian to tilegx CPU. )
  \end{enumerate}
\end{prework}

\begin{prework}{ roger }
  \begin{enumerate}
  \item Q. What do you plan to do in study hour?\\
    A. D-I に screen/tux ( Making D-I package of Screen/tux. )
  \end{enumerate}
\end{prework}

\begin{prework}{ dictoss }
  \begin{enumerate}
  \item Q. What do you plan to do in study hour?\\
    A. Kfeebsd porting and test
  \end{enumerate}
\end{prework}

\begin{prework}{ koedoyoshida }
 (Ko-Edo Yoshida. Ko-Edo is alter name of Kawagoe city of Saitama Prefecture.Yoshida is popular family name.)
  \begin{enumerate}
  \item Q. What do you plan to do in study hour?\\
    A. minidebconf準備 (Prepareing to hold Debian mini-conf in Japan.)
  \end{enumerate}
\end{prework}

\begin{prework}{ tyamadajp }
  \begin{enumerate}
  \item Q.What do you plan to do in study hour?\\
    A. パッケージを作ってsponsor uploadをお願いする所までやります ( Making some packages and  ask for a Debian Developer to upload them to new queue.)
  \end{enumerate}
\end{prework}

\begin{prework}{ yy\_y\_ja\_jp }
  \begin{enumerate}
  \item Q.What do you plan to do in study hour?\\
    A. DDTSS
  \end{enumerate}
\end{prework}

\begin{prework}{ Charles Plessy }
  \begin{enumerate}
  \item Q.What do you plan to do in study hour?\\
    A. Packaging, packaging, packaging !
  \end{enumerate}
\end{prework}

\begin{prework}{ 野島 (Nojima)}
  \begin{enumerate}
  \item Q.What do you plan to do in study hour?\\
    A. Planing ``Debian Onsen.'' ``Debian Onsen'' is to lodge together at Hot-Spring inn for hacking Debian ,probably, between all night.Off cause, we plan to enjoy to take good Japanese meals, and take the bath of hot spring at that place.
  \end{enumerate}
\end{prework}


}

\section{Reports of Debian related events in Japan}
\emtext{Reports of Debian related events in Japan}

\begin{frame}{The 136th Tokyo Debian Meeting}

\begin{itemize}
\item We hold meeting at freee CO, LTD.
\item We had total 12 paticipants, including 3 official Debian Developers.
\item Sessions were,
  \begin{enumerate}
  \item About the power management in Debian box.
  \item Story about ITP/RFS of libhinawa.
  \end{enumerate}

\item We spent study hours in remains of meeting, last we talked about the results in study hours.
\end{itemize} 
\end{frame}

\begin{frame}{The 136th Tokyo Debian Meeting(Continue)}

  Seminor ``About the power management in Debian box'' was presented by Mr. Iwamatsu (Official Debian Developer.) He talked about how to manage power consumption of Debian box , and also introduction among functionarity of power management of Intel CPUs and Linux Kernel. He described the usage of powertop/tlp package of Debian.
  
\end{frame}

\begin{frame}{The 136th Tokyo Debian Meeting(Continue)}

  Another seminor ``Story about ITP/RFS of libhinawa.'' was presented by Takaswie. Takaswie is upstream of libhinawa,which is the next-gen library for controling audio console via fire-wire.

  He is also one of developers for ALSA , Linux Sound Subsystem.He talked about,
  \begin{itemize}
  \item motivation of making libhinawa,
  \item how to improve legacy FFADOs.
  \item make language bindings of libhinawa to other programing language,
    using GObject-Introspection
  \item story of making packages of libhinawa and puting them new queue of Debian.
  \end{itemize}
\end{frame}
\begin{frame}{The 136th Tokyo Debian Meeting(Continue)}

  His effort of development is so impressive.
  He worked so hard that ,almost during 3 months, he only did sleep,work for pay,eat and development of ALSA/libhinawa until late midnight , every days.

\end{frame}

\begin{frame}{The 136th Tokyo Debian Meeting(Continue)}

 Summarize the results of study hours, as below,

\begin{itemize}
\item iwamatsu \\
  \begin{itemize}
   \item uploaded fonts-hanazono-20160201
   \item uploaded neotoma 1.7.3
   \item sent patch to qemu for being enable tilegx.
   \item gave some advice about porting to tilegx.
   \item I found that the site gathering data of powertop is cool!
  \end{itemize}
\item mkouhei\\
  \begin{itemize}
  \item uploaded shortuuid 0.4.3-1
  \item uploaded django-shortuuidfield 0.1.3-1
  \item commited the scripts , which installs debian sid to vitualbox on OS X, to github.
  \item checked why updating yrmcds on my box is fail.
  \item joined Poeto-san to debian-users$B!w(Bd.o.j
  \item signing keys.
  \end{itemize}
\end{itemize}
\end{frame}

\begin{frame}{The 136th Tokyo Debian Meeting(Continue)}

\begin{itemize}
\item wskoka
  \begin{itemize}
   \item Maintained the debian ports of tilegx (http://133.130.124.236/debian/packages/) 
   \item Set up ssh server.
  \end{itemize}
\item knok\\
  \begin{itemize}
  \item Re-builded the kernel of Orange Pi. Following the instruction,
    \begin{enumerate}
    \item Clone the sources from https://github.com/steffen-g/orangepipc/
    \item apt-get install gcc-arm-linux-gnueabihf libncurses5-dev
    \item /build kernel-prepare
    \item /build kenel-menu (enable NLS modules?)
    \item /build kernel-build
    \end{enumerate}
    I didn't find gcc5 can build the toolchain of sid.I tried to build gcc-4.9-arm using gcc-cross-dev on jessie.
  \item Try to connect wifi using text console of jessie. I tried wicd-ncurses.
  \end{itemize}
\end{itemize}
\end{frame}

\begin{frame}{The 136th Tokyo Debian Meeting(Continue)}

\begin{itemize}
\item kenhys
  \begin{itemize}
   \item I plan to have 10min session about Debian porterbox at next debian meeting.
   \item I developed a patch to fix up build errors on MIPS machine. And I tried build in schroot environment.(However, that machine is so low performance , building process didn't seems to complete.)
  \end{itemize}
\item rosh\\
  \begin{itemize}
  \item Sining Keys.
  \item I discussed about Bug 814540, whether I should enable kernel module ``TTY\_PRINTK''.
  \item I prepared to make package ``micro-evtd''.
  \end{itemize}
\end{itemize}
\end{frame}

\begin{frame}{The 136th Tokyo Debian Meeting(Continue)}

\begin{itemize}
\item yy\_y\_ja\_JP\\
  \begin{itemize}
  \item DDTSS
  \item Translating webwml to Japanese. I made branch 'yyoshino-for-reivew' on git.debian.or.jp/git/webwml.git.
  \end{itemize}
\item Charles\\
  \begin{itemize}
  \item Signed keys
  \item updated libbio-samtools-perl (not uploaded yet)
  \item reviewed BTS entries in mime-support
  \end{itemize}
\end{itemize}
\end{frame}

\begin{frame}{The 136th Tokyo Debian Meeting(Continue)}

\begin{itemize}
\item dictoss\\
  \begin{itemize}
    \item introduced some documents/made partitions, for installing debian to VAIO Pro.
    \item consult the source of xserver-xorg-video-intel. try to find difference  the source with correctly working ones and not working. I found some difference with some arguments of ioctl among them.
    \item re-building kfreebsd with debug options.
    \item Singing keys.
  \end{itemize}
\item takaswie
  \begin{itemize}
  \item discuss about so complex of usb protocol stacks with Nojima-san.
  \item make the patch to improve API among user-defined control of alsa-lib.See. https://github.com/takaswie/alsa-lib/tree/user-control
  \end{itemize}
\end{itemize}
\end{frame}

\begin{frame}{The 136th Tokyo Debian Meeting(Continue)}

\begin{itemize}
\item nojima\\
  discussed about how USB emulates EXPRESS BUS Cards with takaswie-san. We get an ideas how USB use debug port in easiest way.
\end{itemize}

\end{frame}

\section{How to become a Debian Developer}
\emtext{How to become a Debian Developer}

\section{About tilegx}
\emtext{About tilegx}

\section{Use Some Debian Infrastructure}
\emtext{Use Some Debian Infrastructure}

\section{Study hours}
\emtext{Study hours}

\begin{frame}{Please write your result}

  Please write your results of study hours, to following urls,
\url{https://debianmeeting.titanpad.com/10}\\
Up to 18:30pm.

\end{frame}
  
\section{Events related to Debian in Japan}
\emtext{Events related to Debian in Japan}

\begin{frame}{Events related to Debian in Japan}

  \begin{itemize}
  \item ``Debian Onsen'' (we choose the schedule whether 4/23 or 4/30)
  \item Kansai Debian Meeting.
  \end{itemize}

\end{frame}

\section{Where do we go for a drink?}
\emtext{Where do we go for a drink?}

\begin{frame}{Where do we go for a drink?}
 Does anyone know good Japanese restaurants?
\end{frame}

\end{document}

;;; Local Variables: ***
;;; outline-regexp: "\\([ 	]*\\\\\\(documentstyle\\|documentclass\\|emtext\\|section\\|begin{frame}\\)\\*?[ 	]*[[{]\\|[]+\\)" ***
;;; End: ***
