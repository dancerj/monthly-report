%; whizzy-master ../debianmeetingresume201301.tex
% 以上の設定をしているため、このファイルで M-x whizzytex すると、
% whizzytexが利用できます

\begin{prework}{ キタハラ }

\preworksection{2013年の勉強会で発表したい内容を教えてください}

 思い付かない。

\preworksection{2015年ではDebianがどうなっているかを大胆に予想してください}

 表向きはあまり変わっていない気がする。

\end{prework}

\begin{prework}{ dictoss(杉本 典充) }

\preworksection{2013年の勉強会で発表したい内容を教えてください}

 upstreamのtar.gzを複数のdebパッケージに分けて作成する方法を勉強したい。

\preworksection{2015年ではDebianがどうなっているかを大胆に予想してください}

 Linuxディストリビューションの中で最も使われるパッケージ形式がdebになり、
「debianがインストールされていないなんて!」というのが普通になる。

\end{prework}

\begin{prework}{ koedoyoshida }

\preworksection{2013年の勉強会で発表したい内容を教えてください}

 発表したいというか聞きたい内容としては、
最近聞く、Debianのマルチアーキ対応とかでしょうか。

\preworksection{2015年ではDebianがどうなっているかを大胆に予想してください}
\begin{enumerate}
\item UbuntuPhoneをカスタムしてDebianPhoneにしようとする人が現れるが...\\
「DFSGとバイナリFirmwareの戦いは始まったばかりだ!...続く」
\item jessieがリリースされる。
\item Debconf Japanが開催される。
\end{enumerate}
\end{prework}

\begin{prework}{ dai }

\preworksection{2013年の勉強会で発表したい内容を教えてください}

 DDになるまでの修行期間やNMプロセスについてと、DDになってからの作業。もっ
とも出張に重ならないとなかなか出席できないので実際に発表できる機会があ
るかは微妙なところです。

\preworksection{2015年ではDebianがどうなっているかを大胆に予想してください}

Wheezy+1が滞りなくリリースされていた。

\end{prework}

\begin{prework}{ yamamoto }

\preworksection{2015年ではDebianがどうなっているかを大胆に予想してください}

\begin{itemize}
\item debian-ports が(準?)公式プロジェクトに格上げされて、移植関連インフラが強化されている。
\item Linux カーネルがサポートしているアーキテクチャなら全部、移植プロジェクトが始まっている。どっかの企業が新しいアーキテクチャを発表したら、「とりあえず Debian に入れとく?」みたいになっているといいなー。
\item kFreeBSD も軌道に乗っている。
\item Hurd は、、、たぶん相変わらず。
\end{itemize}

\end{prework}

\begin{prework}{ henrich }

\preworksection{2013年の勉強会で発表したい内容を教えてください}

\begin{itemize}
\item なぜFreeであることが重要なのか、という価値の再確認について\\
 Debianはフリーなオペレーティングシステムを作り上げることを目指していま
す。多くの場合フリーであることというのを無条件で善であるとだけ考えてい
ると思いますが、フリーであることによってどのような価値があるのか、とい
うことを再定義して議論してみたいです。
\item Debianのリリースワークフローの改善案\\
Debianの安定版リリースについては「岩のように安定している、が古い」というのが定番です。これをどのように捉え、さらに改善していくかを議論してみたいと思っています。
\end{itemize}

\preworksection{2015年ではDebianがどうなっているかを大胆に予想してください}

正直分かりません。

\end{prework}

\begin{prework}{ まえだこうへい }

\preworksection{2013年の勉強会で発表したい内容を教えてください}

仕事もprivateもDebianパッケージもPython関係ばかりなのでpython絡みで何か。

\preworksection{2015年ではDebianがどうなっているかを大胆に予想してください}

娘は4歳になった。最近自分のPCが欲しいというので、もう使っていない白MacBookと
Debianのインストール用のCDイメージ入れたUSBメモリを渡した。「ありがとう」と言って、
娘は当たり前のようにDebianのインストールを始めた。

という日が来るに違いない。
\end{prework}

\begin{prework}{上川純一}

\preworksection{2013年の勉強会で発表したい内容を教えてください}

Debianからがんばれば一眼レフデジカメを制御できるっぽいので、それについて
話をまとめてみたい。

\preworksection{2015年ではDebianがどうなっているかを大胆に予想してください}

2013年、タブレット型パソコンの普及によりラップトップは少量しか生産されな
 い高級品となってしまった。

2014年、メガネ型インタフェースにより携帯電話が駆逐されてしまった。

2015年、我々の自由ソフトウェア運動は大きく後退していた。OSの起動の自由の
ための戦いは苦戦を強いられている。自由にOSを選べるのはサーバインスタンス
提供業者の提供する VPS 仮想マシンの上でのみ。タブレット型パソコンの起動シー
ケンスに割り込むのに必要な暗号鍵が解読されているハードウェアはオープンソー
ス開発者の地下組織では高値で取引されるようになっていた。

\end{prework}

\begin{prework}{ 野島 貴英 }

\preworksection{2013年の勉強会で発表したい内容を教えてください}

 qemuネタと、NookColorにDebianをネイティブインストールする件を発表したいな。
Gnome-shellなネタもいいな。ARネタ1本やってみたいな。
(い、一応全部Debianに絡ませる予定)

\preworksection{2015年ではDebianがどうなっているかを大胆に予想してください}

 電車の中で、ARメガネでwaylandなDesktop環境を写しつつ、Debian unstableで
Hackしている毎日に違いない。俺のタブレット端末は全部Debianをネイティブで
入れてあるに決まっている。3Dプリンタ、激安プリント基盤作成サービスも普及
してるし、世の中に出回り過ぎて余ってしまった高度なスマホの部品もデータシート
付きの通販で手に入るから、今度の日曜は同人ハードウェアなタブレットPCを、
自分で作ってDebianを入れてみることにしよう。おっと、もちろん設計図は
当然githubでGPLで公開しとかなきゃね。

\end{prework}
