%; whizzy chapter
% -initex iniptex -latex platex -format platex -bibtex jbibtex -fmt fmt
% 以上 whizzytex を使用する場合の設定。

%     Tokyo Debian Meeting resources
%     Copyright (C) 2011 Junichi Uekawa
%     Copyright (C) 2011 Nobuhiro Iwamatsu

%     This program is free software; you can redistribute it and/or modify
%     it under the terms of the GNU General Public License as published by
%     the Free Software Foundation; either version 2 of the License, or
%     (at your option) any later version.

%     This program is distributed in the hope that it will be useful,
%     but WITHOUT ANY WARRANTY; without even the implied warranty of
%     MERCHANTABILITY or FITNESS FOR A PARTICULAR PURPOSE.  See the
%     GNU General Public License for more details.

%     You should have received a copy of the GNU General Public License
%     along with this program; if not, write to the Free Software
%     Foundation, Inc., 51 Franklin St, Fifth Floor, Boston, MA  02110-1301 USA

%  preview (shell-command (concat "evince " (replace-regexp-in-string "tex$" "pdf"(buffer-file-name)) "&"))
% 画像ファイルを処理するためにはebbを利用してboundingboxを作成。
%(shell-command "cd image201109; ebb *.png")

%%ここからヘッダ開始。

\documentclass[b5paper,10pt,fleqn]{jsarticle}
\usepackage{multicol}
\usepackage{fancybox}
\usepackage{wallpaper}

\pagestyle{empty}

\setlength{\oddsidemargin}{-20mm}
\setlength{\topmargin}{-30mm}
\setlength{\textwidth}{200mm}
\setlength{\textheight}{450mm}

\newcommand{\commanddescription}[2]{%
■\textgt{#1}
\\\hspace{10mm}\texttt{#2}}

\begin{document}

%\CenterWallPaper{1}{image200502/openlogo-nd.eps}

% Header
{\Large Debian dpkg/APT/Aptitude Cheat Sheet} Debian 7.x (wheezy) 対応
%\begin{flushright}Tokyo Area Debian Meeting\end{flushright}

\hrule height 0.5mm

\begin{multicols}{2}

\begin{center}
{\Large \shadowbox{dpkg}}
\end{center}

\commanddescription
{パッケージをインストールする}
{dpkg -i パッケージファイル名}

\commanddescription
{パッケージをアンインストールする}
{dpkg -r パッケージ名}

\commanddescription
{インストール済みパッケージ一覧を表示する}
{dpkg -l}

\commanddescription
{インストール済みパッケージが含むファイル一覧を表示する}
{dpkg -L パッケージ名}

\commanddescription
{ファイルを含むインストール済みパッケージを検索する}
{dpkg -S ファイル名}

\commanddescription
{インストール済みパッケージ一覧データをバックアップする}
{dpkg -{}-get-selections}

%\commanddescription
%{パッケージを更新しないように固定(hold)する}
%{echo パッケージ名 hold | dpkg -{}-set-selections}

%\commanddescription
%{パッケージの固定(hold)を解除する}
%{echo パッケージ名 install | dpkg -{}-set-selections}

\commanddescription
{インストールするパッケージのアーキテクチャを表示する}
{dpkg -{}-print-architecture}

\commanddescription
{インストール可能なアーキテクチャをリストに追加する}
{dpkg -{}-add-architecture アーキテクチャ名}

\commanddescription
{インストール可能なアーキテクチャをリストから削除する}
{dpkg -{}-remove-architecture アーキテクチャ名}

\commanddescription
{追加のアーキテクチャ名表示する}
{dpkg -{}-print-foreign-architectures}

\begin{center}
{\Large \shadowbox{APT}}
\end{center}

\commanddescription
{リポジトリから最新のパッケージ情報を取得する}
{apt-get update}

\commanddescription
{システムをアップデートする}
{apt-get upgrade}

\commanddescription
{システムをアップデートする (最重要パッケージを優先)}
{apt-get dist-upgrade}

\commanddescription
{パッケージをその依存関係とともにインストールする}
{apt-get install パッケージ名}

\commanddescription
{設定ファイルを残してパッケージを削除する}
{apt-get remove パッケージ名}

\commanddescription
{設定ファイルと共にパッケージを削除する}
{apt-get purge パッケージ名}

\commanddescription
{ソースパッケージを取得する}
{apt-get source ソースパッケージ名 または バイナリパッケージ名}

\commanddescription
{パッケージの構築時に依存するパッケージをインストールする}
{apt-get build-dep パッケージ名}

\commanddescription
{ローカルリポジトリを掃除し、パッケージファイルを全削除する}
{apt-get clean}

\commanddescription
{ローカルリポジトリを掃除し、不要なパッケージファイルのみ削除する}
{apt-get autoclean}

\commanddescription
{他のパッケージにまったく依存していない自動的にインストールされたパッケージを削除する}
{apt-get autoremove}

\commanddescription
{バイナリパッケージをカレントディレクトリにダウンロードする}
{apt-get download パッケージ名}

\commanddescription
{指定のパッケージの変更履歴をダウンロードして表示する}
{apt-get changelog パッケージ名} 

\commanddescription
{パッケージの情報を表示する}
{apt-cache show パッケージ名}

\commanddescription
{正規表現とマッチするパッケージを検索する}
{apt-cache search 正規表現}

\commanddescription
{パッケージのソースパッケージ情報を表示する}
{apt-cache showsrc パッケージ名}

\commanddescription
{パッケージの依存情報を表示する}
{apt-cache depends パッケージ名}

\commanddescription
{パッケージの被依存情報を表示する}
{apt-cache rdepends パッケージ名}

\commanddescription
{パッケージのインストール候補を表示する}
{apt-cache policy パッケージ名}

\commanddescription
{パッケージを更新しないように固定(hold)する}
{apt-mark hold パッケージ名}

\commanddescription
{パッケージの固定(hold)を解除する}
{apt-mark unhold パッケージ名}

\commanddescription
{固定(hold)されているパッケージを表示する}
{apt-mark showhold}

\begin{center}
{\Large \shadowbox{Aptitude}}
\end{center}

\commanddescription
{リポジトリから最新のパッケージ情報を取得する}
{aptitude update}

\commanddescription
{システムをアップデートする}
{aptitude safe-upgrade}

\commanddescription
{システムをアップデートする (最重要パッケージを優先)}
{aptitude full-upgrade}

\commanddescription
{パッケージをその依存関係とともにインストールする}
{aptitude install パッケージ名}

\commanddescription
{設定ファイルを残してパッケージを削除する}
{aptitude remove パッケージ名}

\commanddescription
{設定ファイルと共にパッケージを削除する}
{aptitude purge パッケージ名}

\commanddescription
{ローカルリポジトリを掃除し、パッケージファイルを全削除する}
{aptitude clean}

\commanddescription
{ローカルリポジトリを掃除し、不要なパッケージファイルのみ削除する}
{aptitude autoclean}

\end{multicols}

%Footer
\hrule height 0.5mm
LICENSE: GNU GPL Version 2 / Copyright 2011 Debian Hack Cafe / Debian JP Project

\end{document}
