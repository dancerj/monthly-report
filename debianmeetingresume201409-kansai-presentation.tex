\documentclass[cjk,dvipdfmx,10pt,compress,%
hyperref={bookmarks=true,bookmarksnumbered=true,bookmarksopen=false,%
colorlinks=false,%
pdftitle={第 88 回 関西 Debian 勉強会},%
pdfauthor={倉敷・のがた・佐々木・かわだ・八津尾},%
%pdfinstitute={関西 Debian 勉強会},%
pdfsubject={資料},%
}]{beamer}

\title{第 88 回 関西 Debian 勉強会}
\subtitle{$\sim$発表資料$\sim$}
\author[かわだ てつたろう]{{\large\bf 倉敷・のがた・佐々木・かわだ・八津尾}}
\institute[Debian JP]{{\normalsize\tt 関西 Debian 勉強会}}
\date{{\small 2014 年 9 月 28 日}}

%\usepackage{amsmath}
%\usepackage{amssymb}
\usepackage{graphicx}
\usepackage{moreverb}
\usepackage[varg]{txfonts}
\AtBeginDvi{\special{pdf:tounicode EUC-UCS2}}
\usetheme{Kyoto}
\def\museincludegraphics{%
  \begingroup
  \catcode`\|=0
  \catcode`\\=12
  \catcode`\#=12
  \includegraphics[width=0.9\textwidth]}
%\renewcommand{\familydefault}{\sfdefault}
%\renewcommand{\kanjifamilydefault}{\sfdefault}
\begin{document}
\settitleslide
\begin{frame}
\titlepage
\end{frame}
\setdefaultslide

\begin{frame}[fragile]
  \frametitle{Disclaimer}
  \begin{itemize}
  \item 疑問、質問、ツッコミ、茶々、\alert{大歓迎}
  \item その場でインタラクティブにどうぞ
  \item ハッシュタグ \#kansaidebian
\end{itemize}
\end{frame}

\begin{frame}[fragile]
\frametitle{Agenda}

\tableofcontents

\end{frame}

\section{最近の Debian 関係のイベント}

\takahashi[40]{最近の Debian\\関係のイベント}

\begin{frame}[fragile]
  \frametitle{第87回関西Debian勉強会}
  \begin{itemize}
  \item 日時: 8月24日(日)
  \item 場所: 港区民センター
  \end{itemize}
  \begin{block}{内容}
    \begin{itemize}
    \item もくもくの会
    \end{itemize}
  \end{block}
\end{frame}

\begin{frame}[fragile]
  \frametitle{第117回東京エリアDebian勉強会}
  \begin{itemize}
  \item 日時: 9月27日(土)
  \item 場所: 株式会社スクウェア・エニックス セミナールーム
  \end{itemize}
  \begin{block}{内容}
    \begin{itemize}
    \item 「Debconf14のビデオ紹介」
    \item もくもくの会
    \end{itemize}
  \end{block}
\end{frame}

\begin{frame}[fragile]
  \frametitle{Debian Project}
  \begin{itemize}
  \item DebConf14
  \item John Paul Adrian Glaubitzさん来日
  \item ``shellshock'' bugs
  \end{itemize}
\end{frame}

\takahashi[50]{そんな\\こんなで}
\takahashi[120]{次}

\section{事前課題発表}

\takahashi[50]{事前課題}

\begin{frame}[fragile]
  \frametitle{事前課題}
  \begin{block}{今回の事前課題}
    \begin{description}
    \item[事前課題1]
      もくもくの会で行なう作業、質問などの課題を用意して教えてください。
    \item[事前課題2]
      前回(第87回)の勉強会に参加された方は、前回の作業や課題がその後どう
      なったか結果を教えてください。
    \item[事前課題3]
      LT(ライトニングトーク) 歓迎です。何かお話したい方はタイトルを下さい。
    \end{description}
  \end{block}
\end{frame}

\begin{frame}
  \frametitle{ 木下 }
  \begin{enumerate}
  \item
    \begin{enumerate}
    \item プライベートクラウドの調査・研究
      \begin{itemize}
      \item Openstackの研究
      \item Eucalyptusの研究/実験
      \end{itemize}
    \item グリッドコンピューティング関連の調査・研究
      \begin{itemize}
      \item GlobusToolkitで何ができる?

        →AndroidOS等のJavaVMのコンパイルで使えたら嬉しいかも。
      \end{itemize}
    \item Debian7 on PANDABOARDの調査・研究
      \begin{itemize}
      \item WiFiモジュール(On Board:TI製)の有効化
      \item GPUデバイスドライバの有効化
      \end{itemize}
    \end{enumerate}
  \item ※欠席だった為割愛
  \end{enumerate}
\end{frame}

\begin{frame}
  \frametitle{ かわだてつたろう }
後で。。。
\end{frame}

\begin{frame}
  \frametitle{ 佐々木洋平 }
tDiary...
\end{frame}

\begin{frame}
  \frametitle{ murase\_{}syuka }
  \begin{itemize}
  \item mrubyのパッケージ更新
  \item blenderのビルド
    \begin{itemize}
    \item openimageioでboostがエラー
    \end{itemize}
  \item emacs弄る
  \item etc
  \end{itemize}
\end{frame}

\begin{frame}
  \frametitle{ Mitsutoshi NAKANO $<$bkbin005@rinku.zaq.ne.jp$>$ }
  \begin{enumerate}
  \item 以下のうち、いずれか。
    \begin{enumerate}
    \item Debian のパッケージビルドの方法についての勉強
    \item canna-yubinを拡張し、FreeWnnに対応させる。
    \item しばらくNetにアクセスできない可能性が出てきたので、関係各所への連絡の文案を考える。
    \end{enumerate}
  \item
    \begin{enumerate}
    \item tamago の upstream の準備

      tamagoの大元の開発者である戸村さんと連絡が取れました。

      戸村さんより

      \begin{quote}
      tamago については、勤務先の業務の一部として開発したことに
      なるはずなので、それ関連の手続きがどうなっているかなども
      確認しようとしているところでした。
      \end{quote}

      とのことで、権利関係の確認をしていただいております。

    \item Debian のパッケージビルドの方法についての勉強

      滞っています。
    \end{enumerate}
  \end{enumerate}
\end{frame}

\begin{frame}
  \frametitle{ 山城の国の住人 久保博 }
とあるパッケージの不具合修正パッチが投げっぱなしになっている件について、ご相談させてください。
\end{frame}

\begin{frame}
  \frametitle{ 川江 }
  \begin{enumerate}
  \item emacsを使った、JavascriptとCSSのコーティング。
  \item 苦戦中。ただし、やっとコードの全体像が読めてきた感じ(なぜに、fireFox、IE、Safari、Chromeは書式が違うのだろうか?もう少し「統一」してくれ!)
  \item なし
  \end{enumerate}
\end{frame}

\takahashi[50]{事前課題\\発表}


\takahashi[50]{そんな\\こんなで}
\takahashi[120]{次}

\section{もくもくの会}
\takahashi[30]{もくもくの会}

\takahashi[50]{そんな\\こんなで}
\takahashi[120]{次}

\section{今後の予定}
\begin{frame}[fragile]
\frametitle{今後の予定}

\begin{block}{第89回関西Debian勉強会}
  \begin{itemize}
  \item 日時: 10月26日(日) 13:30 -
  \item 場所: 港区民センター
  \end{itemize}
\end{block}

\begin{block}{第118回東京エリアDebian勉強会}
  \begin{itemize}
  \item 日時: 10月18日(土)
  \item 場所: 未定
  \item 内容: 未定
  \end{itemize}
\end{block}

\end{frame}

\takahashi[50]{  }

\end{document}
%%% Local Variables:
%%% mode: japanese-latex
%%% TeX-master: t
%%% End:
