\documentclass[cjk,dvipdfmx,10pt,compress,%
hyperref={bookmarks=true,bookmarksnumbered=true,bookmarksopen=false,%
colorlinks=false,%
pdftitle={第 72 回 関西 Debian 勉強会},%
pdfauthor={倉敷・のがた・佐々木・かわだ・八津尾},%
%pdfinstitute={関西 Debian 勉強会},%
pdfsubject={資料},%
}]{beamer}

\title{第 72 回 関西 Debian 勉強会}
\subtitle{$\sim$発表資料$\sim$}
\author[かわだ てつたろう]{{\large\bf 倉敷・のがた・佐々木・かわだ・八津尾}}
\institute[Debian JP]{{\normalsize\tt 関西 Debian 勉強会}}
\date{{\small 2013 年 5 月 26 日}}

%\usepackage{amsmath}
%\usepackage{amssymb}
\usepackage{graphicx}
\usepackage{moreverb}
\usepackage[varg]{txfonts}
\AtBeginDvi{\special{pdf:tounicode EUC-UCS2}}
\usetheme{Kyoto}
\def\museincludegraphics{%
  \begingroup
  \catcode`\|=0
  \catcode`\\=12
  \catcode`\#=12
  \includegraphics[width=0.9\textwidth]}
%\renewcommand{\familydefault}{\sfdefault}
%\renewcommand{\kanjifamilydefault}{\sfdefault}
\begin{document}
\settitleslide
\begin{frame}
\titlepage
\end{frame}
\setdefaultslide

\takahashi[40]{はじめに}

\takahashi[40]{会場提供\\ANNAI 様}

\begin{frame}[fragile]
\frametitle{Agenda}

\tableofcontents

\end{frame}

\section{最近の Debian 関係のイベント}

\takahashi[40]{最近の Debian\\関係のイベント}

\begin{frame}[fragile]
  \frametitle{第 71 回関西 Debian 勉強会}
  \begin{itemize}
  \item 日時: 4 月 28 日(日)
  \item 場所: 福島区民センター
  \end{itemize}
  \begin{block}{内容}
    \begin{itemize}
    \item 「クラウド初心者が AWS に Debian をのっけて翻訳サービスの試行に挑戦してみた」
    \item 「リリースノートを読んでみよう。」
    \end{itemize}
  \end{block}
\end{frame}

\begin{frame}[fragile]
  \frametitle{第 100 回 東京エリア Debian 勉強会}
  \begin{itemize}
  \item 日時: 5 月 11 日(土)
  \item 場所: 渋谷ファーストプレイス
  \end{itemize}
  \begin{block}{内容}
    \begin{itemize}
    \item アンカンファレンス形式によるセッション
    \end{itemize}
  \end{block}
\end{frame}

\begin{frame}[fragile]
  \frametitle{Debian Project}
  \begin{itemize}
  \item 5 月 4 日 Debian 7.0 「Wheezy」リリース
  \item 6 月 15 日 Wheezy ポイントリリース予定
  \item Jessie の開発
  \end{itemize}
\end{frame}

\takahashi[50]{そんな\\こんなで}
\takahashi[120]{次}

\section{事前課題発表}

\takahashi[50]{事前課題}

\begin{frame}[fragile]
  \frametitle{事前課題}
  \begin{block}{今回の事前課題}
    \begin{description}
    \item[事前課題1]
      DebianとUbuntuの違いを挙げてみてください
    \item[事前課題2]
      Debianを使っていて/使い始めて/使おうとして、困った/ていることを、
      1件以上書いてください
    \end{description}
  \end{block}
\end{frame}

\takahashi[50]{事前課題\\発表}

\begin{frame}
  \frametitle{ 榎真治 }
  \begin{enumerate}
  \item 自由なオペレーティングシステムをめざすDebianと自由を重視しつつユーザーの利便性も重視しているUbuntuというミッションの違いがあるという理解です。

    また、完全なコミュニティベースのDebianとCanonicalがコミュニティの中で特別な地位をしめるUbuntuというプロジェクト体制の違いもあると思います。

    そのほかにも、リリースポリシーなども違います。

  \item メール環境WindowsのBeckyからSylpheedへ移行する際に、フォルダ単位でしかインポートできず、メールの数が多いとインポートがエラーになることがあり、手こずっています。

    スケジュールソフトで、サイボウズLiveと同期できるような方法がないかを探しています。
  \end{enumerate}
\end{frame}

\begin{frame}\frametitle{ 山城の国の住人 久保博 }
  \begin{enumerate}
  \item
    \begin{itemize}
    \item 開発元の団体が違う
    \item リリースサイクルやサポートのライフサイクルが違う
    \end{itemize}
  \item 
    \begin{itemize}
    \item Debian wheezy のマシン二台で、マイクの音声が拾えなくて困っています。
    \item デスクトップ PC で grub (GRUB2) で Windows XP と Debian wheezy との dual boot にしていたのですが、いつの間にか Windows XP が起動しなくなりました。
    \item  GNOME3 に戸惑っています。まだ慣れません。
    \end{itemize}
  \end{enumerate}
\end{frame}

\begin{frame}\frametitle{ Takubo.Morio }
  \begin{enumerate}
  \item 
    \begin{itemize}
    \item Debianの開発はコミュニティ。
    \item Ubuntuの開発は企業(Canonical)主体。
    \end{itemize}
  \end{enumerate}
\end{frame}

\begin{frame}\frametitle{ スペンス }
  今のところDebianは普段使っていないけど、検討したいと思って、勉強会に参加させていただきたいです。
\end{frame}

\begin{frame}\frametitle{ おくの }
  \begin{enumerate}
  \item デフォルトのデスクトップ環境が違う

    個人的にはUbuntuのUnityは割と好きです。

  \item GNOME3にまだ慣れていないので慣れるまで大変です。
  \end{enumerate}
\end{frame}

\begin{frame}\frametitle{ murase\_{}syuka }
  \begin{enumerate}
  \item 
    \begin{itemize}
    \item gnome3 / unity
    \item stable / tested(unstable)
    \end{itemize}
  \item 
    \begin{itemize}
    \item VBox上のdebian64bitからdebian32bit環境への引越しする場合

      パッケージ構成が同じなら

      /etc, /home/user以下をcopyで問題なし?
    \item debianで使いやすい無料dnsサービスって?

      DynDNSが有料化されたので
    \item GNU/kfreebsdは運用可能なほど安定している?

      squeezeのときはインストールが成功しなかったので
    \item nVidia配布ドライバの行儀の良いインストール方法

      kernelの更新のたび?ドライバの再インストールしてたので
    \end{itemize}
  \end{enumerate}
\end{frame}

\begin{frame}\frametitle{ 末廣 雅利 }
  \begin{enumerate}
  \item Ubuntu を使ったことないのでよく分かってないのですが、
    \begin{itemize}
    \item リリースサイクルが違う
    \item Ubuntu の方が導入の敷居が低く、使いやすい(イメージがある)
    \end{itemize}
  \item 
    \begin{itemize}
    \item Ruby のライブラリ

      Ruby のライブラリを Debian パッケージで入れるのか、自力で入れるのかで困ったことがありました。

      今は、rbenv + Bundler で済ませてしまっています。
    \item USBメモリからのインストールに挫折

      etch か lenny の頃、USBメモリからインストールしようと四苦八苦したが成功せず、
      結局、最小の CD を焼いてネットワークインストールしたことがあります。
    \end{itemize}
  \end{enumerate}
\end{frame}

\begin{frame}\frametitle{ Sachy }
  \begin{enumerate}
  \item リリースの頻度が違う。開発している団体が違う。
  \item Debian Ubuntu共通のことですが、古いパッケージが欲しい時に手に入れられないことが多い。
  \end{enumerate}
\end{frame}

\begin{frame}\frametitle{ kozo2 }
  \begin{enumerate}
  \item 同じpackage名でもbinaryが違う,必ずしも互換性があるわけではない
  \item netinst.isoで有線アダプタが無いマシンで無線ネットワークの検出に失敗した

    netinst.isoでうまくbootできなかった
  \end{enumerate}
\end{frame}

\begin{frame}\frametitle{ 西山和広 }
  \begin{enumerate}
  \item Debian はユニバーサルオペレーティングシステムだけど Ubuntu は Linux のみで対応アーキテクチャも限定されている。

    Debian はすべての公式パッケージがセキュリティアップデートなどのサポート対象だが、Ubuntu は universe などサポート対象外のパッケージが多い。

    Debian はリリースが不定期だが、Ubuntu は定期的にリリースされている。
  \item zabbix が wheezy に入らなかった。
  \end{enumerate}
\end{frame}

\begin{frame}\frametitle{ lurdan }
  \begin{enumerate}
  \item
    \begin{itemize}
    \item 独裁者 (と所有企業) の有無
    \item リリースサイクル
    \item 用途毎にバリエーションを作って使い捨てるか、単一の配布で多くの用途をカバーしようとするか、の方針
    \end{itemize}
  \item 表面化してないけど結構古いままのパッケージが最近目につくかも?
  \end{enumerate}
\end{frame}

\begin{frame}\frametitle{ かわだてつたろう }
  \begin{enumerate}
  \item
    \begin{itemize}
    \item Debian の main と Ubuntu の main。
    \item DFSG と Ubuntu のライセンス。
    \item PPA の有無。
    \end{itemize}
  \item Debian だけではないですが、ディストリビューションとどのように付き合っていくか。

    最新バージョンがパッケージに無い場合やソフトウェア個別のパッケージ管理システムと相性がよくない場合など。
  \end{enumerate}
\end{frame}

\begin{frame}\frametitle{ 川江 }
  \begin{enumerate}
  \item Debianはユーザーインターフェイスがシンプル。Ubuntuはユーザーインターフェイスが凝ってる( Unity etc.)
  \item 最新のハードや、希少のハードに対応してない事、また、Debian で走らすための「情報」が乏しい。
  \end{enumerate}
\end{frame}

\begin{frame}\frametitle{ kyoko.oh }

(無回答)

\end{frame}

\begin{frame}\frametitle{ kino (1/2)}
  \begin{enumerate}
  \item
    \begin{itemize}
    \item Debianはコミュニティベース、UbuntuはCanonical社主導
    \item サポートポリシー。

      Debianは全パッケージサポート、UbuntuはmultiverseやuniverseはUbuntuチームとしてサポートされない。

    \item リリースサイクル

      Debian: メジャーバージョン1世代+1年間

      Ubuntu: 6か月サイクル 9か月間サポート+LTS 5年サポート

    \item 個別のパッケージの設定も微妙に違う。
    \end{itemize}
  \end{enumerate}
\end{frame}

\begin{frame}\frametitle{ kino (2/2)}
  \begin{enumerate}
    \setcounter{enumi}{1}
  \item
    \begin{itemize}
    \item 日本語で質問できるサイトってどこかにありますか?
    \item Ubuntuでの情報を流用しやすくするために、どこを読み替えればよいかがわかるとうれしい。
    \item アップグレードできる利点と引き換えに、サポート期間が短いのが結構つらい。2世代サポートが欲しいです。
    \item アップグレード時に意図して使っていないが依存で入ってくるパッケージのconfファイルを更新するかと聞かれてどう対処するか迷う時があります。

      アップグレード時のベストプラクティスを知りたいです。
    \item APT-Pinning 機能は素晴らしいのですが、理解が難しいのでハンズオンを希望します。
    \item 2chWikiの信頼出来る部分をdebian.or.jpの見やすい所に置けないでしょうか。
    \item Webアプリ系だとパッケージインストールすると逆にメンテナンスが難しくなることがあるので、素人アイデアとして、ファイルの分割をせずアップストリームtarの解凍そのままをパッケージに出来ないものでしょうか。
    \end{itemize}
  \end{enumerate}
\end{frame}

\begin{frame}\frametitle{ 佐々木洋平 }

(無回答)

\end{frame}

\takahashi[50]{そんな\\こんなで}
\takahashi[120]{次}

\section{Debian と Ubuntu の違いを知ろう}
\takahashi[30]{Debian と Ubuntu の\\違いを知ろう\\by\\西田}

\takahashi[50]{そんな\\こんなで}
\takahashi[120]{次}

\section{Debian の歩き方}
\takahashi[30]{Debian の歩き方\\by\\佐々木、倉敷、他}

\takahashi[50]{そんな\\こんなで}
\takahashi[120]{次}

\section{今後の予定}
\begin{frame}[fragile]
\frametitle{今後の予定 (1)}

\begin{block}{大統一 Debian 勉強会}
  \begin{itemize}
  \item 日時: 6 月 29 日(土)
  \item 会場: 東京 日本大学 駿河台キャンパス
  \item 公式サイト \url{http://gum.debian.or.jp/}
  \item ライトニングトーク、GPG キーサインパーティ募集中
  \item 懇親会受付も近日案内予定
  \end{itemize}
\end{block}
\end{frame}

\begin{frame}[fragile]
\frametitle{今後の予定 (2)}

\begin{block}{第 74 回関西 Debian 勉強会}
  \begin{itemize}
  \item 日時: 7 月 28 日(日)
  \item 会場: 福島区民センター 304号室
  \item 内容: 未定
  \end{itemize}
\end{block}

\begin{block}{第 75 回関西 Debian 勉強会}
  \begin{itemize}
  \item 日時: 8 月 3 日(土)
  \item オープンソースカンファレンス 2013 Kansai@Kyoto
  \end{itemize}
\end{block}

\begin{block}{第 102 回東京エリア Debian 勉強会}
  \begin{itemize}
  \item 日時: 7 月 某日
  \item 会場、内容: 未定
  \end{itemize}
\end{block}

\end{frame}

\takahashi[50]{  }


\end{document}
%%% Local Variables:
%%% mode: japanese-latex
%%% TeX-master: t
%%% End:
